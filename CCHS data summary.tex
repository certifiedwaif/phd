\documentclass{article}
\begin{document}
\section{Data source}
The primary data source is the Canadian Community Health Survey Cycle 1.2, focusing on Mental Health and Well-being. The survey was conducted between
2002-05-01 to 2002-12-31. The data set, questionnaire and sampling
metholodology is extensively documented in documents available from the 
Statistics Canada web site 
\begin{verbatim}
http://www23.statcan.gc.ca/imdb/p2SV.pl?Function=getSurvey&SurvId=1632&InstaId=5285&SDDS=5015
\end{verbatim}
 and in the data dictionary PDFs in the cchs2002/Documentation 
 subdirectory.

The CCHS uses a complex survey design to perform their surveys, which
is detailed in the documentation. The weighting variable they calculate
is stored in the variable WTSB\_M.

This was merged with a secondary data source, where additional questions
relevant to bipolar disorder were asked. We have both of these individual
data sets, and the resulting merged data set.

\section{File location}
The files are all stored in the directories cchs and cchs2002 under the
directory /dskh/nobackup/markg on verona.

\section{Data Files}
\begin{tabular}{lp{6cm}]}
Filename & Description \\
\hline
Cchs1-2.sav coping Aug 25 2013.sav
& The Coping data set with the additional variables, in SPSS format.
\\
HS.txt
& The raw CCHS data file
\\
CCHS\_SAS/data.sas7bdat
& The SAS data file for the data set formed by joining the CCHS data to the Coping data
\\
CCHS\_SAS/formats.sas7bcat
& The SAS data file for the data set formed by joining the CCHS data to the Coping data \\
\hline
\end{tabular}

\section{Code}
\begin{tabular}{lp{6cm}]}
Filename & Description \\
\hline
cchs02.sas
& SAS code which sets up library names
\\
2010-03-25 LOW MOOD ANALYSES.sas
& SAS code which does data cleaning and actually produces the tables used in the draft research paper.
This code takes account of the weighting of the data given in the WTSB\_M variable.
\\
\hline
\end{tabular}

\section{Results}
The file cchs\_tables.rtf contains the tables of results that were 
generated when the 2010-03-25 LOW MOOD ANALYSES.sas code was run in SAS.

\section{Analysis}
Analysis was begun in R, but quickly abandoned as I realised that there
were many steps of data cleaning and calculating new variables that had
already been done in the SAS code we were sent which were hard to match.

Analysis was performed in SAS, using the SAS code provided by Dr Nancy 
Low and the data provided to us. The tables produced matched the tables 
in the draft research paper that Dr Nancy Low and her group have written,
indicating that their analysis was performed correctly.

\end{document}