\chapter{Zero-inflated models formula sheet}
\section{Univariate model} 

\subsection{Model}
\begin{align*}
	\vx_i   & \sim \text{Poisson}(\lambda)                    \\
	\lambda & \sim \text{Gamma}(\alpha_\lambda,\beta_\lambda) \\
	\rho    & \sim \text{Beta}(\alpha_\rho, \beta_\rho)       \\
	\vr_i   & \sim \text{Binomial}(\vp_i)                     
\end{align*}

\subsection{Log-likelihood}
\begin{align*}
\log p(\vy) &= \quad \log p(\vy | \lambda, \vr) + \log p(\vr | \rho) + \log p(\rho) + \log p(\lambda) \\
&= \quad (\vr_i \vx_i) \log \lambda - \log \Gamma(\vr_i \vx_i + 1) + \vr_i \log(\rho) + (1 - \vr_i) \log(1 - \rho) \\
& \quad + (\alpha_\rho - 1) \log (\rho) + (\beta_\rho - 1) \log(1 - \rho) - \log \text{Beta} (\alpha_\rho, \beta_\rho) \\
& \quad + \alpha_\lambda \log \beta_\lambda - \log \Gamma(\alpha_\lambda) + (\alpha_\lambda - 1) \log \lambda - \beta_\lambda \lambda
\end{align*}

\subsection{Mean field update equations}
We are now in a position to calculate the mean field update equations for the factorised
variational approximation for the univariate zero-inflated model.

First, we find the approximation for $\lambda$. Assuming that $q(r_i) \sim \text{Bernoulli}{(p_i)}$,
% Mean field update for q(\lambda)
\[
\begin{array}{rl}
	q^*(\lambda)
	  & \propto                                                            
	\lambda^{\alpha_\lambda+\vone^T\vy - 1} 
	\exp\left\{ 
	\bE_{-q(\lambda)} \left[
	-(\beta_\lambda + \vone^T\vr) \lambda 
	\right] 
	\right\} 
	\\ [0.5ex]
	  &                                                                    
	\propto \lambda^{\alpha_\lambda+\vone^T\vy - 1} \exp{\left \{-(\beta_\lambda + \vone^T\vp)\lambda \right \} } 
	\\
	  & = \text{Gamma}{(\alpha_\lambda+\vone^T\vy, \beta_\lambda+\vone^T\vp)}, 
\end{array}
\]

\noindent Next, turning our attention fo $\rho$,
% Mean field update for q(\rho)
\[
\begin{array}{rl}
	\log{q^*(\rho)} 
	  &                                                                 
	\propto \left\{ 
	\bE_{-q(\rho)}\left[ 
	\vone^T\vr \log{(\rho)} 
	+ (n - \vone^T\vr) \log{(1 - \rho)} 
	\right] 
	+ \alpha_\rho \log{(\rho)} 
	+ \beta_\rho \log{(1 - \rho)} 
	\right\} 
	\\ [0.5ex]
	  &                                                                 
	\propto \exp \left( 
	(\vone^T\vp + \alpha_\rho) \log{(\rho)} 
	+ (n - \vone^T\vp + \beta_\rho) \log{(1 - \rho)} 
	\right) 
	\\ [0.5ex]
	  & = \Beta(\alpha_\rho + \vone^T\vp, \beta_\rho + n - \vone^T\vp), 
\end{array}
\]
\noindent and for $\vr_i, i=1, \ldots, n$,
\[
\begin{array}{rl}
	\ds \log{q^*(r_i)} & \propto -\bE_{q(\lambda)} [\lambda ] r_i + y_i \log{(r_i)} + r_i \bE_{q(\rho)} \left[\log{\left(\frac{\rho}{1 - \rho}\right)}\right]          \\[0.5ex]
	                   & \ds = -r_i \frac{\alpha_{q(\lambda)}}{\beta_{q(\lambda)}} + y_i \log{(r_i)} + r_i \left(\psi(\alpha_{q(\rho)}) - \psi(\beta_{q(\rho)})\right) \\ [0.5ex]%
	                   & \ds = \text{Bernoulli}(p_i)                                                                                                                   
\end{array}
\]
\noindent where
\[
\begin{array}{rl}
	\ds p_i                                               
	= \frac{\exp{(\eta_i)}}{\I(y_i = 0) + \exp{(\eta_i)}}  
	= \text{expit}(\eta_i), \quad \mbox{(when $y_i = 0$)} 
\end{array}
\]

\noindent and $\eta_i = - \alpha_{q(\lambda)}/\beta_{q(\lambda)} + \psi(\alpha_{q(\rho)}) - \psi(\beta_{q(\rho)})$.
%
%Note that if $y_i = 0$ then
%$y_i \log{(r_i)} = \log{(y_i^{r_i})} = \log{(0^0)} = \log{(1)} = 0$. Dr John %Ormerod hit
%upon the idea of side-stepping this problem by writing the q-likelihood as
%
%$$
%q^*(r_i) \propto r_i^{y_i} \exp{(r_i \eta_i)}
%$$
%
%
%Now, either $y_i = 0$ or $y_i \ne 0$.
%
%So
%
%$$
%q^*(r_i) = \text{Bernoulli}(p_i)
%$$
%
% Why do you care about these when you have the algorithm?
%\noindent So the mean field update equations are \ldots
%
%$$
%\begin{array}{ll}
%p_i &= \ds \frac{\exp{(\eta_i)}}{I(y_i = 0) + \exp{(\eta_i)}} %\\
%&= \text{expit}(\eta_i)
%\end{array}
%\begin{array}{cl}
%\alpha_{q(\rho)} &\leftarrow \alpha_\rho + \vone^T\vp \\
%\beta_{q(\rho)} &\leftarrow \beta_\rho + n - \vone^T\vp \\
%\eta &\leftarrow -\frac{\alpha_{q(\lambda)}}{\beta_{q(\lambda)}} + \psi(\alpha_{q(\rho)}) - \psi(\beta_{q(\rho)}) \\
%\vp_{q(\vr_0)} &\leftarrow \expit{(\eta)} \\
%\alpha_{q(\lambda)} &\leftarrow \alpha_\lambda+ \vone^T\vy \\
%\beta_{q(\lambda)} &\leftarrow \beta_\lambda+ \vone^T\vp
%\end{array}
%$$
%
% The formatting of this entire section is horrible.
% It should be organised into subterms, as you will for the multivariate
% model.

% Include derivations for mean field updates at the end?

\subsection{Deriving the Variational Lower bound}
The lower bound $\log{\underline{p}(\vtheta)}$ is
$$
\bE_q[\log{p(\vy, \vtheta)} - \log{q(\vtheta)}]
$$

\noindent where
\begin{align*}
	q(\vtheta) &= q(\lambda) q(\rho) \prod_{i=1}^n q(r_i), \\
	q(\lambda) &\sim \text{Gamma}{(\alpha_{q(\lambda)}, \beta_{q(\lambda))}}, \\
	q(\rho) &\sim \text{Beta}(\alpha_{q(\rho)}, \beta_{q(\rho)}) \text{ and } \\
	q(r_i) &= 1 \text{ if } y_i \ne 0,\text{ and } p_i \text{ if } y_i = 0,
\end{align*}
where $p_i$ is calculated for each iteration as specified in Algorithm \ref{alg:algorithm_one}.

The lower bound can be calculated directly to be
\[
	\begin{array}{rl}
		\E_q \left\{ \log{p(\vy, \vr, \lambda, \rho)} - \log{q(\vr, \lambda, \rho)} \right\} & = T_1 + T_2 \\
	\end{array}
\]

\noindent where
\[
	\begin{array}{rl}
		T_1 & \ds =                                                                                                                          
		\quad \alpha_\lambda \log{(\beta_\lambda)} + (\alpha_\lambda - 1) \bE_q[\log{(\lambda)}] - \beta_\lambda \bE_q [\lambda] - \log\Gamma(\alpha_\lambda) \\
		    & \ds \quad + \sum_{i=1}^n ( -\bE_q [\lambda] \bE_q [r_i] + \bE_q[y_i \log{(\lambda r_i)}] - \log\Gamma(y_i+1)) \quad \mbox{and} 
		\\ [1ex]
		T_2 & = \quad \E_q[r_i] \bE_q[\log{(\rho)}] + \bE_q[(1 - r_i)] \bE_q[\log{(1 - \rho)}]                                                     
		- \bE_q[\log{q(r)}] 
		- \bE_q[\log{q(\lambda)}] 
		- \bE_q[ \log{q(\rho)}]
	\end{array}
\]

\noindent with 
\[
	\begin{array}{rl}
	\E[\lambda] &= \alpha_{q(\lambda)}/\beta_{q(\lambda)}, \\
		\E [\log{\lambda}] & = -\{ \alpha_{q(\lambda)} - \log{(\beta_{q(\lambda)})} + \log{\Gamma(\alpha_{q(\lambda)})} + (1 - \alpha_{q(\lambda)}) \psi{(\alpha_{q(\lambda)})} \},        \\
		\E[\log{\rho}]     & = - [ \quad \log{\text{Beta}(\alpha_{q(\rho)}, \beta_{q(\rho)})} - (\alpha_{q(\rho)} - 1) \psi{(\alpha_{q(\rho)})} - (\beta_{q(\rho)} - 1)\psi{(\beta_{q(\rho)})}  \\
		                    & \qquad + (\alpha_{q(\rho)} + \beta_{q(\rho)} - 2)\psi{(\alpha_{q(\rho)} + \beta_{q(\rho)})} ],                                                               
		\\
		\E[r_i]            & =                                                                                                                                                             
		\begin{cases}
		1                   & \text{if } y_i \ne 0                                                                                                                                          \\
		p_i                 & \text{if } y_i = 0,                                                                                                                                           \\
		\end{cases}
		\\
		-\E_q[\log{q(r)}]  & = \sum_{i=1}^n \I(y_i = 0) \log{(p_i)}                                                                                                                        \\ [0.5ex]
		-\E_q[\log{q(\lambda)}] 
		                    & = \alpha_{q(\lambda)} - \log{(\beta_{q(\lambda)})} + \log{\Gamma{(\alpha_{q(\lambda)})}} + (1 - \alpha_{q(\lambda)}) \psi{(\alpha_{q(\lambda)})}, \text{ and }            \\ [0.5ex]
		\E_q[\log{q(\rho)}] & = - \{ \quad \log{(\text{Beta}(\alpha_{q(\rho)}, \beta_{q(\rho)})} - (\alpha_{q(\rho)} - 1) \psi{(\alpha_{q(\rho)})} - (\beta_{q(\rho)} - 1)\psi{(\beta_{q(\rho)})} \\ [0.5ex]
		                    & \qquad + (\alpha_{q(\rho)} + \beta_{q(\rho)} - 2)\psi{(\alpha_{q(\rho)} + \beta_{q(\rho)})} \}.                                                                \\ [0.5ex]
	\end{array}
\]

\section{Multivariate model}

The variational approximation of the univariate zero-inflated Poisson model extends naturally to the
multivariate case.

\subsection{Model}

\subsection{Log-likelihood}
The full log-likelihood is then
\begin{align*}
	\log p(\vy, \vtheta) & = \quad \vy^\top \mR(\mC \vnu) - \vr^\top \exp{(\mC \vnu)} - \vone^\top \log \Gamma(\vy + \vone)                                 \\
	                     & \quad + \vr^\top \log{(\rho)} + (\vone - \vr)^\top \log{(\vone - \rho)}                                                          \\
	                     & \quad + \frac{\nu}{2} \log |\mPsi| - \left( \frac{\nu + p + 1}{2} \right) \log{|\mSigma|} - \frac{1}{2} \tr (\mPsi \mSigma^{-1}) \\
	                     & \quad - \frac{1}{2} \log{|\sigma^2_\vbeta \mI_p|} - \frac{1}{2} \vbeta^\top \sigma_\vbeta^{-2} \mI_p \vbeta                      \\
	                     & \quad - \frac{1}{2} \log |\mG| - \frac{1}{2} \vu^\top \mG^{-1} \vu.                                                                
\end{align*}

\subsection{Mean field updates}
The optimal approximation for $\vmu$ is
\[
	\begin{array}{rl}
		q(\vmu) & \propto \exp \{ \E_{-q(\vnu)} \log p(\vy, \vtheta) \}.
	\end{array}
\]

\noindent The optimal approximation for $\vr$ is
\[
	\begin{array}{rl}
		q(\vr) & \propto \exp{\{\E_{-q(\vr)}y^\top\mR(\mC\vmu) - \vr^\top\exp{(\mC\vnu)}- \frac{1}{2} \vnu^\top \text{diag}(\sigma_{\vu}^2)^{-1} \vnu\}} 
	\end{array}
\]
where
\[
	\begin{array}{rl}
		  & \E_{-q(\vr)} [\vy^\top\mR(C\vnu) - \vr^\top\exp{(\mC\vnu)}- \frac{1}{2} \vnu^\top \text{diag}(\sigma_{\vu}^2)^{-1} \vnu]                                                 \\
		= & \vy^\top\mR\mC \vmu - \vp^\top \exp{\{\mC \vmu +  \frac{1}{2} \text{diag}(\mC \mLambda \mC^\top)\}} -  \frac{1}{2} \vmu^\top \hat{\mD} \vmu - \frac{1}{2} \text{tr}(\mLambda \hat{\mD} )) 
	\end{array}
\]

Recall that
$\E \vu^\top \mA^{-1} \vu = \E \tr(\vu^\top \mA^{-1} \vu) = \tr [\E (\vu \vu^\top) \mA^{-1}] = \vmu^\top \mA^{-1} \vmu + \tr{(\mA^{-1} \mSigma)}$,
and that $\mG = \text{Cov}(\vu) = \text{blockdiag}_{1 \leq i \leq m} \mSigma = \mI_m \otimes \mSigma.$ Then the
optimal approximation for $q^*(\mSigma)$ is

\begin{align*}
	q^*(\mSigma) & \propto \exp \left \{ -\frac{1}{2} \tr (\mPsi \mSigma^{-1}) - \frac{1}{2} \E \vu^\top \mG^{-1} \vu               
	- \frac{\nu + p + 1}{2} \log |\mSigma| - \frac{1}{2} \log |\mG| \right \} \\
	             & =\exp \left \{ -\frac{1}{2} \tr \left[\mPsi + \sum_{i=1}^m (\vmu_i \vmu_i^\top + \mLambda_i) \right] \mSigma^{-1}
	- \frac{\nu + p + m + 1}{2} \log |\mSigma| \right \}.
\end{align*}

\noindent Hence, $q^*(\mSigma) = \text{Inverse Wishart} \left( \mPsi + \sum_{i=1}^m (\vmu_i \vmu_i^\top + \mLambda_i), \nu + m \right)$.


\subsection{Calculation of the Variational Lower bound}
% Where are the priors for \vbeta and \vu
	
The variational lower bound is equal to $\bE_q[\log{p(\vy, \vtheta)} - \log{q(\vtheta)}] = T_1 + T_2 + T_3$,
where
% This is the new T_1
$$
\begin{array}{rl}
	T_1 & = \quad \E_q[\log{p(\vy, \vnu)} - \log{q(\vnu)}]                                                                                                                                                  \\
	    & = \quad \vy \mP \mC \vmu - \vp^\top \exp{\left[ \mC \vmu + \frac{1}{2} \text{diag} (\mC \mLambda \mC^\top) \right]} - \vone^\top\log \Gamma{(\vy + \vone)}                                               \\
	    & \quad + \frac{p + m}{2} (1 + \log{2 \pi}) + \frac{1}{2} \log{|\mLambda|},                                                                                                                                \\
	T_2 & = \quad \E_q \left[ \log p(\mSigma_{\vu \vu}) - \log q(\mSigma_{\vu \vu}) \right]                                                                                                                 \\
	    & = \quad \E_q \big[ v/2(\log |\Psi| - \log |\Psi + \vmu_\vu \vmu_\vu^\top + \mLambda_{\vu \vu}|) + \frac{1}{2} \log 2 + \frac{1}{2} \log|\mSigma_{\vu \vu}| + \log \Gamma_{p+1}(v/2) \\
	    & \quad - \log \Gamma_{p}(v/2) + \frac{1}{2} \tr((\vmu_{\vu} \vmu_{\vu}^\top + \mLambda_{\vu \vu}) \mSigma_{\vu \vu}^{-1}) \big]                                                                                                  \\
	    & = \quad v/2\big(\log |\Psi| - \log |\Psi + \vmu_\vu \vmu_\vu^\top + \mLambda_{\vu \vu}|\big) + \frac{1}{2} \log 2 + \frac{1}{2} \E_q \log |\mSigma_{\vu \vu}| + \log \Gamma_{p+1}(v/2) \\
	    & \quad - \log \Gamma_{p}(v/2) + \frac{1}{2} \tr\big(\mI_m + \Psi(\Psi+ \vmu_\vu \vmu_\vu^\top + \mLambda_{\vu \vu})^{-1}/(v + p + 2)\big), \text{ and }                                                                                       \\
	T_3 & = - \vp^\top \log \vp - (\vone - \vp)^\top \log (\vone - \vp) - \log \Beta (\alpha_\rho, \beta_\rho) + \log \Beta (\alpha_q, \beta_q),
\end{array}
$$
	
\noindent with $\E_q \big[ \log |\mSigma_{\vu \vu}| \big] = m \log 2 + \log \left | \Psi + \vmu_\vu \vmu_\vu^\top + \mLambda_{\vu \vu} \right | + \sum_{i=1}^m \Psi \left ( \frac{v - i + 1}{2} \right )$

\section{New writing on the piecewise polynomial idea}
The exponential function grows very fast for moderate arguments, which can lead to floating point overflow
and underflow issues. We instead define a piecewise function
\[
	\begin{cases}
		e^{r_{ij}}, r_{ij} < t \\
		a r_{ij}^2 + b r_{ij} + c, r_{ij} \geq t
	\end{cases}
\]
and then choose the co-efficients $a$, $b$ and $c$ such that the function, first and second derivatives all
agree at $r_{ij} = t$. This allows us to form the following system of equations

\[
	\begin{array}{lllll}
		e^t &= &a t^2 &+ \quad b t &+ \quad c \\
		e^t &= &&\quad 2a t &+ \quad b \\
		e^t &= &&&\quad 2a \\
	\end{array}
\]
Then solving for $a$, $b$ and $c$, we obtain
\begin{align*}
	a &= e^t / 2 \\
	b &= e^t - t e^t = (1 - t) e^t \\
	c &= e^t - at^2 - bt = e^t - e^t/2 t^2 - (1 - t) e^t t = [1 - t^2/2 - (1 - t) t] e^t
\end{align*}

\section{Appendix - Algebraic derivations of conditional likelihoods}

The joint likelihood is:

$$
p(\vy, \vr, \lambda, \rho) = \frac{b^a \lambda^{a - 1} \exp{(-b \lambda)}}{\Gamma{(a)}} \prod_{i=1}^n \frac{\exp{(-\lambda r_i)} (\lambda r_i)^{y_i}}{y_i !} \rho^{r_i} (1 - \rho)^{1 - r_i}.
$$

%\begin{align*}
%p(\lambda|\vy, \vr, \rho) &= \frac{\prod_{i=1}^n p(y_i | \lambda, r_i) p(\lambda) p(r_i|\rho) p(\rho)}{\int \prod_{i=1}^n p(y_i | \lambda, r_i) p(\lambda) p(r_i|\rho) p(\rho)) d \lambda}.
%\end{align*}
%
%Concentrating for now on the denominator in this expression, we re-arrange and collect
%like terms to obtain
%$$
%\prod_{i=1}^n \rho^r_i (1 - \rho)^{1 - r_i} \frac{r_i^{y_i}}{y_i !}
%    \int \frac{b^a \lambda^{a - 1} \exp{(-b \lambda)} \lambda^{y_i} \exp{(-\lambda r_i)}}{\Gamma{(a)}} d \lambda
%$$
%
%The integral in this expression is
%\begin{align*}
%& \int \frac{b^a \lambda^{(a + y_i) - 1} \exp{(-\lambda(b + r_i))}}{\Gamma{(a + y_i)}} d \lambda \frac{\Gamma{(a+ y_i)}}{\Gamma{(a)} b^{-y_i}} \\
%=& \frac{\Gamma{(a+ y_i)}}{\Gamma{(a)} b^{-y_i}}.
%\end{align*}
%
%Collecting the multiplicands in the integral over $\lambda$ together, we obtain
%$$
%\prod_{i=1}^n \rho^{r_i} (1 - \rho)^{1 - r_i} \frac{r_i^{y_i}}{y_i!}
%    \int \frac{b^{na + \sum_{i=1}^n y_i} \lambda^{(na + \sum_{i=1}^n y_i) - 1} \exp{(-\lambda(nb + \sum_{i=1}^n r_i))}}{\Gamma{(na + \sum_{i=1}^n y_i)}} d \lambda
%    \frac{\Gamma{(na + \sum_{i=1}^n y_i)}}{\Gamma{(na)} b^{-\sum_{i=1}^n y_i}}.
%$$

%By cancelling like terms in the numerator and denominator of the full likelihood we arrive 
%at

$$
\beta_{\lambda}^{\alpha_\lambda+\vone^T\vy} \lambda^{(\alpha_\lambda + \vone^T\vy) - 1} \exp{\left(-(\beta_\lambda + \vone^T\vr) \lambda \right)}
$$

$$
\frac{\rho^{\vone^T\vr} (1 - \rho)^{\vone^T(\vone - \vr)}}{\Beta{\alpha_\rho + \vone^T \vr, \beta_\rho + \vone^T(\vone - \vr)}}
$$

$$
\begin{array}{ll}
	p(r_i | \text{rest}) & \ds \propto \frac{(\lambda r_i)^{y_i} \exp{(-\lambda r_i)}}{y_i !} \rho^{r_i} (1 - \rho)^{1 - r_i} \\
	                     & \ds \propto r_i^{y_i} (e^{-\lambda})^{r_i} \rho^{r_i} (1 - \rho)^{1 - r_i}                         
\end{array}
$$

We make use of the fact that if $y_i = 0$, $r_i = 0$, and if $y_i \ne 0$,
$r_i = 1$. So $y_i^{r_i} = I(x \ne 0)$ and hence the likelihood can be re-written as

$$
\begin{array}{ll}
	p(r_i | \text{rest}) & \ds = \frac{(e^{-\lambda + \logit{(\rho)}})^{r_i}}{I(y_i = 0) + (e^{-\lambda + \logit{(\rho)}})^{r_i}} 
\end{array}
$$
