\documentclass{article}[12pt]

\addtolength{\oddsidemargin}{-.75in}%
\addtolength{\evensidemargin}{-.75in}%
\addtolength{\textwidth}{1.5in}%
\addtolength{\textheight}{1.3in}%
\addtolength{\topmargin}{-.8in}%
\addtolength{\marginparpush}{-.75in}%
\setlength\parindent{0pt}
% \setlength{\bibsep}{0pt plus 0.3ex}

\usepackage[authoryear]{natbib}
\usepackage{graphicx}
\usepackage{algorithm,algorithmic}

\title{Variational approximations for zero-inflated Semiparametric Regression
			 models}
\author{Mark Greenaway}
% include.tex

% \newcommand{\expit}[1]{\text{expit} #1}
% \newcommand{\logit}[1]{\text{logit} #1}

\def \R {{\mathbb{R}}}
\def \vbeta {{\boldsymbol \beta}}
\def \vnu {{\bf \nu}}
\def \vy {{\bf y}}
\def \vx {{\bf x}}
\def \vu {{\bf u}}
\def \vr {{\bf r}}
\def \vp {{\bf p}}
\def\vectorfontone{\bf}
\def\vone{{\bf 1}}
\def\vzero{{\bf 0}}
\def \vmu {{\boldsymbol \mu}}
\def \vnu {{\bf \nu}}
\def \vmuqbeta {{\vmu_{q(\vbeta)}}}
\def \vmubeta {{\vmu_{\vbeta}}}
\def \Sigmaqbeta {{\Sigma_{q(\vbeta)}}}
\def \Sigmabeta {{\Sigma_{\vbeta}}}
\def \va {{\bf a}}
\def \vtheta {{\bf \theta}}
\def \mX {{\bf X}}
\def \mZ {{\bf Z}}
\def \mR {{\bf R}}
\def \mC {{\bf C}}
\def \mI {{\bf I}}
\def \mLambda {{\boldsymbol \Lambda}}
\def \mSigma {{\boldsymbol \Sigma}}
\def \B {{\text{B}}}

\def\ds{{\displaystyle}}

\def\diag{{\mbox{diag}}}
\def\bbE{\mathbb{E}}


\usepackage{latexsym,amssymb,amsmath,amsfonts}
%\usepackage{tabularx}
\usepackage{theorem}
\usepackage{verbatim,array,multicol,palatino}
\usepackage{graphicx}
\usepackage{graphics}
\usepackage{fancyhdr}
\usepackage{algorithm,algorithmic}
\usepackage{url}
%\usepackage[all]{xy}



\def\approxdist{\stackrel{{\tiny \mbox{approx.}}}{\sim}}
\def\smhalf{\textstyle{\frac{1}{2}}}
\def\vxnew{\vx_{\mbox{{\tiny new}}}}
\def\bib{\vskip12pt\par\noindent\hangindent=1 true cm\hangafter=1}
\def\jump{\vskip3mm\noindent}
\def\etal{{\em et al.}}
\def\etahat{{\widehat\eta}}
\def\thick#1{\hbox{\rlap{$#1$}\kern0.25pt\rlap{$#1$}\kern0.25pt$#1$}}
\def\smbbeta{{\thick{\scriptstyle{\beta}}}}
\def\smbtheta{{\thick{\scriptstyle{\theta}}}}
\def\smbu{{\thick{\scriptstyle{\rm u}}}}
\def\smbzero{{\thick{\scriptstyle{0}}}}
\def\boxit#1{\begin{center}\fbox{#1}\end{center}}
\def\lboxit#1{\vbox{\hrule\hbox{\vrule\kern6pt
      \vbox{\kern6pt#1\kern6pt}\kern6pt\vrule}\hrule}}
\def\thickboxit#1{\vbox{{\hrule height 1mm}\hbox{{\vrule width 1mm}\kern6pt
          \vbox{\kern6pt#1\kern6pt}\kern6pt{\vrule width 1mm}}
               {\hrule height 1mm}}}


%\sloppy
%\usepackage{geometry}
%\geometry{verbose,a4paper,tmargin=20mm,bmargin=20mm,lmargin=40mm,rmargin=20mm}


%%%%%%%%%%%%%%%%%%%%%%%%%%%%%%%%%%%%%%%%%%%%%%%%%%%%%%%%%%%%%%%%%%%%%%%%%%%%%%%%
%
% Some convenience definitions
%
% \bf      -> vector
% \sf      -> matrix
% \mathcal -> sets or statistical
% \mathbb  -> fields or statistical
%
%%%%%%%%%%%%%%%%%%%%%%%%%%%%%%%%%%%%%%%%%%%%%%%%%%%%%%%%%%%%%%%%%%%%%%%%%%%%%%%%

% Sets or statistical values
\def\sI{{\mathcal I}}                            % Current Index set
\def\sJ{{\mathcal J}}                            % Select Index set
\def\sL{{\mathcal L}}                            % Likelihood
\def\sl{{\ell}}                                  % Log-likelihood
\def\sN{{\mathcal N}}                            
\def\sS{{\mathcal S}}                            
\def\sP{{\mathcal P}}                            
\def\sQ{{\mathcal Q}}                            
\def\sB{{\mathcal B}}                            
\def\sD{{\mathcal D}}                            
\def\sT{{\mathcal T}}
\def\sE{{\mathcal E}}                            
\def\sF{{\mathcal F}}                            
\def\sC{{\mathcal C}}                            
\def\sO{{\mathcal O}}                            
\def\sH{{\mathcal H}} 
\def\sR{{\mathcal R}}                            
\def\sJ{{\mathcal J}}                            
\def\sCP{{\mathcal CP}}                            
\def\sX{{\mathcal X}}                            
\def\sA{{\mathcal A}} 
\def\sZ{{\mathcal Z}}                            
\def\sM{{\mathcal M}}                            
\def\sK{{\mathcal K}}     
\def\sG{{\mathcal G}}                         
\def\sY{{\mathcal Y}}                         
\def\sU{{\mathcal U}}  


\def\sIG{{\mathcal IG}}                            


\def\cD{{\sf D}}
\def\cH{{\sf H}}
\def\cI{{\sf I}}

% Vectors
\def\vectorfontone{\bf}
\def\vectorfonttwo{\boldsymbol}
\def\va{{\vectorfontone a}}                      %
\def\vb{{\vectorfontone b}}                      %
\def\vc{{\vectorfontone c}}                      %
\def\vd{{\vectorfontone d}}                      %
\def\ve{{\vectorfontone e}}                      %
\def\vf{{\vectorfontone f}}                      %
\def\vg{{\vectorfontone g}}                      %
\def\vh{{\vectorfontone h}}                      %
\def\vi{{\vectorfontone i}}                      %
\def\vj{{\vectorfontone j}}                      %
\def\vk{{\vectorfontone k}}                      %
\def\vl{{\vectorfontone l}}                      %
\def\vm{{\vectorfontone m}}                      % number of basis functions
\def\vn{{\vectorfontone n}}                      % number of training samples
\def\vo{{\vectorfontone o}}                      %
\def\vp{{\vectorfontone p}}                      % number of unpenalized coefficients
\def\vq{{\vectorfontone q}}                      % number of penalized coefficients
\def\vr{{\vectorfontone r}}                      %
\def\vs{{\vectorfontone s}}                      %
\def\vt{{\vectorfontone t}}                      %
\def\vu{{\vectorfontone u}}                      % Penalized coefficients
\def\vv{{\vectorfontone v}}                      %
\def\vw{{\vectorfontone w}}                      %
\def\vx{{\vectorfontone x}}                      % Covariates/Predictors
\def\vy{{\vectorfontone y}}                      % Targets/Labels
\def\vz{{\vectorfontone z}}                      %

\def\vone{{\vectorfontone 1}}
\def\vzero{{\vectorfontone 0}}

\def\valpha{{\vectorfonttwo \alpha}}             %
\def\vbeta{{\vectorfonttwo \beta}}               % Unpenalized coefficients
\def\vgamma{{\vectorfonttwo \gamma}}             %
\def\vdelta{{\vectorfonttwo \delta}}             %
\def\vepsilon{{\vectorfonttwo \epsilon}}         %
\def\vvarepsilon{{\vectorfonttwo \varepsilon}}   % Vector of errors
\def\vzeta{{\vectorfonttwo \zeta}}               %
\def\veta{{\vectorfonttwo \eta}}                 % Vector of natural parameters
\def\vtheta{{\vectorfonttwo \theta}}             % Vector of combined coefficients
\def\vvartheta{{\vectorfonttwo \vartheta}}       %
\def\viota{{\vectorfonttwo \iota}}               %
\def\vkappa{{\vectorfonttwo \kappa}}             %
\def\vlambda{{\vectorfonttwo \lambda}}           % Vector of smoothing parameters
\def\vmu{{\vectorfonttwo \mu}}                   % Vector of means
\def\vnu{{\vectorfonttwo \nu}}                   %
\def\vxi{{\vectorfonttwo \xi}}                   %
\def\vpi{{\vectorfonttwo \pi}}                   %
\def\vvarpi{{\vectorfonttwo \varpi}}             %
\def\vrho{{\vectorfonttwo \rho}}                 %
\def\vvarrho{{\vectorfonttwo \varrho}}           %
\def\vsigma{{\vectorfonttwo \sigma}}             %
\def\vvarsigma{{\vectorfonttwo \varsigma}}       %
\def\vtau{{\vectorfonttwo \tau}}                 %
\def\vupsilon{{\vectorfonttwo \upsilon}}         %
\def\vphi{{\vectorfonttwo \phi}}                 %
\def\vvarphi{{\vectorfonttwo \varphi}}           %
\def\vchi{{\vectorfonttwo \chi}}                 %
\def\vpsi{{\vectorfonttwo \psi}}                 %
\def\vomega{{\vectorfonttwo \omega}}             %


% Matrices
%\def\matrixfontone{\sf}
%\def\matrixfonttwo{\sf}
\def\matrixfontone{\bf}
\def\matrixfonttwo{\boldsymbol}
\def\mA{{\matrixfontone A}}                      %
\def\mB{{\matrixfontone B}}                      %
\def\mC{{\matrixfontone C}}                      % Combined Design Matrix
\def\mD{{\matrixfontone D}}                      % Penalty Matrix for \vu_J
\def\mE{{\matrixfontone E}}                      %
\def\mF{{\matrixfontone F}}                      %
\def\mG{{\matrixfontone G}}                      % Penalty Matrix for \vu
\def\mH{{\matrixfontone H}}                      %
\def\mI{{\matrixfontone I}}                      % Identity Matrix
\def\mJ{{\matrixfontone J}}                      %
\def\mK{{\matrixfontone K}}                      %
\def\mL{{\matrixfontone L}}                      % Lower bound
\def\mM{{\matrixfontone M}}                      %
\def\mN{{\matrixfontone N}}                      %
\def\mO{{\matrixfontone O}}                      %
\def\mP{{\matrixfontone P}}                      %
\def\mQ{{\matrixfontone Q}}                      %
\def\mR{{\matrixfontone R}}                      %
\def\mS{{\matrixfontone S}}                      %
\def\mT{{\matrixfontone T}}                      %
\def\mU{{\matrixfontone U}}                      % Upper bound
\def\mV{{\matrixfontone V}}                      %
\def\mW{{\matrixfontone W}}                      % Variance Matrix i.e. diag(b'')
\def\mX{{\matrixfontone X}}                      % Unpenalized Design Matrix/Nullspace Matrix
\def\mY{{\matrixfontone Y}}                      %
\def\mZ{{\matrixfontone Z}}                      % Penalized Design Matrix/Kernel Space Matrix

\def\mGamma{{\matrixfonttwo \Gamma}}             %
\def\mDelta{{\matrixfonttwo \Delta}}             %
\def\mTheta{{\matrixfonttwo \Theta}}             %
\def\mLambda{{\matrixfonttwo \Lambda}}           % Penalty Matrix for \vnu
\def\mXi{{\matrixfonttwo \Xi}}                   %
\def\mPi{{\matrixfonttwo \Pi}}                   %
\def\mSigma{{\matrixfonttwo \Sigma}}             %
\def\mUpsilon{{\matrixfonttwo \Upsilon}}         %
\def\mPhi{{\matrixfonttwo \Phi}}                 %
\def\mOmega{{\matrixfonttwo \Omega}}             %
\def\mPsi{{\matrixfonttwo \Psi}}                 %

\def\mone{{\matrixfontone 1}}
\def\mzero{{\matrixfontone 0}}

% Fields or Statistical
\def\bE{{\mathbb E}}                             % Expectation
\def\bP{{\mathbb P}}                             % Probability
\def\bR{{\mathbb R}}                             % Reals
\def\bI{{\mathbb I}}                             % Reals
\def\bV{{\mathbb V}}                             % Reals

\def\vX{{\vectorfontone X}}                      % Targets/Labels
\def\vY{{\vectorfontone Y}}                      % Targets/Labels
\def\vZ{{\vectorfontone Z}}                      %

% Other
\def\etal{{\em et al.}}
\def\ds{\displaystyle}
\def\d{\partial}
\def\diag{\text{diag}}
%\def\span{\text{span}}
\def\blockdiag{\text{blockdiag}}
\def\tr{\text{tr}}
\def\RSS{\text{RSS}}
\def\df{\text{df}}
\def\GCV{\text{GCV}}
\def\AIC{\text{AIC}}
\def\MLC{\text{MLC}}
\def\mAIC{\text{mAIC}}
\def\cAIC{\text{cAIC}}
\def\rank{\text{rank}}
\def\MASE{\text{MASE}}
\def\SMSE{\text{SASE}}
\def\sign{\text{sign}}
\def\card{\text{card}}
\def\notexp{\text{notexp}}
\def\ASE{\text{ASE}}
\def\ML{\text{ML}}
\def\nullity{\text{nullity}}

\def\logexpit{\text{logexpit}}
\def\logit{\mbox{logit}}
\def\dg{\mbox{dg}}

\def\Bern{\mbox{Bernoulli}}
\def\sBernoulli{\mbox{Bernoulli}}
\def\sGamma{\mbox{Gamma}}
\def\sInvN{\mbox{Inv}\sN}
\def\sNegBin{\sN\sB}

\def\dGamma{\mbox{Gamma}}
\def\dInvGam{\mbox{Inv}\Gamma}

\def\Cov{\mbox{Cov}}
\def\Mgf{\mbox{Mgf}}

\def\mis{{mis}} 
\def\obs{{obs}}

\def\argmax{\operatornamewithlimits{\text{argmax}}}
\def\argmin{\operatornamewithlimits{\text{argmin}}}
\def\argsup{\operatornamewithlimits{\text{argsup}}}
\def\arginf{\operatornamewithlimits{\text{arginf}}}


\def\minimize{\operatornamewithlimits{\text{minimize}}}
\def\maximize{\operatornamewithlimits{\text{maximize}}}
\def\suchthat{\text{such that}}


\def\relstack#1#2{\mathop{#1}\limits_{#2}}
\def\sfrac#1#2{{\textstyle{\frac{#1}{#2}}}}


\def\comment#1{
\vspace{0.5cm}
\noindent \begin{tabular}{|p{14cm}|}  
\hline #1 \\ 
\hline 
\end{tabular}
\vspace{0.5cm}
}


\def\mytext#1{\begin{tabular}{p{13cm}}#1\end{tabular}}
\def\mytextB#1{\begin{tabular}{p{7.5cm}}#1\end{tabular}}
\def\mytextC#1{\begin{tabular}{p{12cm}}#1\end{tabular}}

\def\jump{\vskip3mm\noindent}

\def\KL{\text{KL}}
\def\N{\text{N}}
\def\Var{\text{Var}}

\def \E {\mathbb{E}}
\def \BigO {\text{O}}
\def \IG {\text{IG}}
\def \Beta {\text{Beta}}


\begin{document}
\maketitle

Abstract:

Keywords: Approximate Bayesian inference ; mixed model ; Markov chain Monte Carlo ; Stan ; 
					penalized splines .

\section{Introduction}
\label{sec:introduction}

Count data with a large number of zero counts arises in many areas of application, such as data arising from
physical activity studies, insurance claims, hospital visits or defects in manufacturing processes. These
models have been used for many applications, including defects in manufacturing in \cite{lambert1992},
horticulture in \cite{BIOM:BIOM1030} and \cite{Hall2000}, length of stay data from hospital admissions in
\cite{BIMJ:BIMJ200390024}, psychology in \cite{JOFP:rethink}, pharmaceutical studies in \cite{Min01042005},
traffic accidents on roadways in \cite{Shankar1997829} and longitudinal studies in
\cite{LeeWangScottYauMcLachlan2006}.

In this paper, we build upon the earlier work on Bayesian zero-inflated models of \cite{Ghosh20061360} and
\cite{VatsaWilson2014}. While simple forms of these models are easy to fit with maximum likelihood techniques,
more general models incorporating random effects, splines and missing data typically have no closed form
solutions. Fitting these models is typically done with Monte Carlo Markov Chain techniques, but these can be
slow and prone to convergence problems. We use Variational Bayes to fit close approximations to these models
using a deterministic algorithm which converges much more quickly.

In Section \ref{sec:introduction} we introduce the model of interest. In Section \ref{sec:methodology} we
provide a framework for our approach incorporating regression modelling and random effects. In Section
\ref{sec:algorithms} we present algorithms for fitting these models. In Section \ref{sec:results} we show how
our approach offers computational advantages over existing approaches. In Section \ref{sec:application} we
show an application of our method to physical activity data. Appendices contain details of our variational
lower bound derivation.

\section{Methodology}
\label{sec:methodology}

In this section we present a VB approach to a Bayesian zero-inflated Poisson model for count data with extra
zeroes. After introducing Bayesian zero-inflated models and VB methodology we derive the VB factorised
approximation to the full Bayesian model.

\subsection{Variational Bayesian inference}

Semiparametric mean field Variational Bayes is an approximate Bayesian inference method as detailed in
\cite{ormerod10} and \cite{RohdeWand2015}. The essential idea is that the log likelihood can be rewritten as

\begin{align*}
\log p(\vy) &= \log p(\vy) \int q(\vtheta) d \vtheta \\
&= \int q(\vtheta) \log \{ \frac{p(\vtheta, \vy) / q(\vtheta)}{p(\vtheta | \vy) / q(\vtheta)} \} d \vtheta \\
&= \int q(\vtheta) \log \frac{p(\vy, \vtheta)}{q(\vtheta)} d \vtheta +
	 \text{KL} \{ {q(\vtheta) || p(\vtheta|\vy)} \} % \\
%& \geq \int q(\vtheta) \log \frac{p(\vy, \vtheta)}{q(\vtheta)} d \vtheta
\end{align*}

which is the sum of the the KL divergence between the true posterior and an approximation and a variational
lower bound.

The approximation is fit by iteratively minimising the Kullback-Leibler divergence between the true posterior
and an approximating distribution, and thus maximising the variational lower bound as defined in Equation
\ref{eq:lower_bound_defn}.

\begin{equation}
\label{eq:lower_bound_defn}
\int q(\vtheta) \log \frac{p(\vy, \vtheta)}{q(\vtheta)} d \vtheta
\end{equation}

Thus the optimal approximation $q^*(\vtheta)$ is

$$
q^*(\vtheta) = \argmin_{q \in Q} \text{KL} \{ {q(\vtheta) || p(\vtheta|\vy)} \}.
$$

A common approach is to assume an approximation in a factorised form

$$q(\vtheta) = \Pi_{i=1}^M q(\theta_i).$$

The variational lower bound is maximised iteratively. On each iteration, the value of each parameter in the
model is calculated as the expectation of the full likelihood relative to the other parameters in the model,
which is referred to as the mean field update:

$$q_i^*(\theta_i) \propto \exp{\{ \bE_{-q(\theta_i)} \log p(\vy, \vtheta) \}}$$

This is done for each parameter in the model in turn until the variational lower bound's increase is
negligible and convergence is achieved.

\section{The zero-inflated Poisson regression model}

Variational approximations are well-suited to accelerating the fit of Bayesian zero-inflated models to data.
Typically zero-inflated models arise in applications where we wish to build multivariate regression models. To
be able to construct multivariate models with as much generality as possible, we specify the full model as a
General Design Bayesian Generalized Linear Mixed Model, as in \citep{zhao06}. This allows us to incorporate
within-subject correlation, measurement error, missing data and smoothing splines (as in \cite{Wand2008}) in
our models.

% Idea: We can use an approximation of the from q(\beta, \u, \Sigma) q(\rho) \Product q(r_i)
% and use GVA on q(\beta, \u, \Sigma) and mean field updates on \rho and r_i

\subsection{Model and data}

The $j$th predictor/response pair for the $i$th group is denoted by $(\vx_{ij}, \vy_{ij}), 1 \leq j \leq n_i,
1 \leq i \leq m$, where the $\vx_{ij}$ are unrestricted, while the $\vy_{ij}$ are nonnegative integers.

For each $1 \leq i \leq m$, define the $n_i \times 1$ vectors $\vy_{ij} = [\vy_{i 1}, \ldots, \vy_{i
n_i}]^\top$ and $\vone_i = [1, \ldots, 1]^\top$, where the first of these vectors is the response. It is
reasonable to assume that the vectors $\vy_1, \ldots, \vy_m$ are independent of each other.

Let $\mR = \diag{(\vr)}$, $\mC = [\mX \mZ]$ and $\vnu = [\vbeta^\top \vu^\top]^\top$. Consider the model

$$
\begin{array}{rl}
\log{p(\vy|\vr, \vbeta, \vu)} &= \vy^\top \mR (\mC\vnu) - \vr^\top \exp{(\mC\vnu)} - \vone^\top \log{\Gamma{(\vy + \vone)}}, \\
\mbox{ and }
r_i &\sim \text{Bernoulli}(\rho), 1 \leq i \leq n \\
\end{array}
$$

with priors

$$ 
\begin{array}{rl}
\log{p(\mSigma_{\vu \vu})} &= \text{Inverse Wishart}(\mPsi, v),\\
p(\rho) &\propto 1 \\
\mbox{ and } \vnu|\sigma_\vu^2 &\sim \mbox{N}(\vzero, \sigma_\vu^2 \mI)\\
\end{array}
$$

where $\mX$ is $n \times p$, $\mZ$ is $n \times mb$ and $\mSigma_{\vu \vu}$ and $\mPsi$ are $b \times b$.

\subsection{Approximation}
Let $\vr_0 = \{ r_i : y_i = 0 \}$.
We assume an approximation of the form
$$
q(\vr_0, \vnu, \sigma_{\vu}^2, \rho) = q(\vnu) q(\mSigma_{\vu \vu}) q(\rho) q(\vr_0) \\
$$

where 
$q(\sigma_{\vu}^2) = \mbox{Inverse Wishart}\left(\mPsi + \sum_{i=1}^m \vmu_i \vmu_i^\top + \mLambda_{\vu_i \vu_i}, v + m + 
p\right)$ \mbox{and } $q(r_i) = \Bernoulli{(p_i)}$

with
$$
p_i = \expit\left[ \psi{(\alpha_{q(\rho)})} - \psi{(\beta_{q(\rho)})} - \exp{(c_i^\top\vmu + \half c_i^\top \mLambda c_i)} \right]
$$

\text{when} $\vy_i = 0$.

%$\propto \exp{\left \{-r_i \bE_{-r_i} [\exp{(c_i^\top\vnu)}] + r_i [\psi(\alpha_\rho) - \psi(\beta_\rho)] \right \} }.\\$

The optimal approximation for $\vr$ is
$$
\begin{array}{rl}
q(\vr) &\propto \exp \left [ \bE_{-q(\vr)}\vy^\top\mR(\mC\vmu) - \vr^\top\exp{(\mC\vnu)}-\half \vnu^\top \mSigma_{\vu \vu} \vnu \right ] \\ [1ex]
	&= \exp{ \left\{ \vy^\top\mR\mC \vmu - \vr^\top \exp{[\mC \vmu + \half \text{diag}(\mC \mLambda \mC^\top)]} - \half \vmu^\top \mD \vmu - \half \text{tr}(\mLambda \mD ) \right\} }
\end{array}
$$

where $\mD = \left[ (\mPsi + \sum_{i=1}^m \vmu_i \vmu_i^\top + \mLambda_{\vu_i\vu_i}) / (v - p - 1) \right]^{-1}$. 

This is close in form to a Poisson regression model. Poisson regression models with normal priors have no
closed form for their mean field updates due to non- conjugacy, but can be fit using Gaussian Variational
Approximation, as in \citep{ormerod09}. The model can be fit using Algorithm \ref{alg:algorithm_one} below.

\begin{algorithm}
\caption[Algorithm 1]{Iterative scheme for obtaining the parameters in the
optimal densities $q^*(\vmu, \mLambda)$, $q^*(\mSigma_{\vu \vu})$ and $q^*(\rho)$}
\label{alg:algorithm_one}
\begin{algorithmic}
\REQUIRE{$\alpha_{q(\rho)} \leftarrow \alpha_\rho + \vone^\top\vp, p_{q(\mSigma_{\vu \vu})} \leftarrow p + 1$} \\[1ex]
\WHILE{the increase in $\log{\underline{p}}(\vy;q)$ is significant}
% \vmu, \mLambda
\STATE Optimise $\vmu$ and $\mLambda$ using $\vy, \mC, \vp$ and $\mSigma_{\vu \vu}$ \\[1ex]
% \vp
\STATE $\beta_{q(\rho)} \leftarrow \beta_\rho + n - \vone^\top\vp$ \\[1ex]
\STATE $\eta \leftarrow -\exp \left [ \mC \vmu + \half \diag{(\mC\mLambda\mC^\top)} \right ] + \psi{(\alpha_{q{(\rho)}})} - \psi{(\beta_{q{(\rho)}})}$ \\[1ex]
\STATE $\vp_{q(\vr_0)} \leftarrow \expit{(\eta)}$ \\[1ex]
% \mSigma_{\vu \vu}
\STATE $\mPsi_{q(\mSigma_{\vu \vu})} \leftarrow \Psi + \sum_{i=1}^m \vmu_i \vmu_i^\top + \mLambda_{{\vu}_i}$ \\[1ex]
\STATE $\mSigma_{\vu\vu} \leftarrow [\mPsi_{q(\mSigma_{\vu \vu})}/(v - d - 1)]^{-1}$
\ENDWHILE
\end{algorithmic}
\end{algorithm}

\section{Numerical optimisation strategies}
\label{sec:algorithms}

In this section, we compare the accuracy, stability and speed of a number of algorithms for fitting the
variational approximation to our model.

\subsection{Laplace-Variational Approximation}

Laplace's method of approximation uses the second order Taylor expansion of the full log likelihood around the
mode to find a Gaussian approximation to the full posterior. This can then be optimised using Newton-Raphson
style iterations. The algorithm is very quick to execute, but the resulting approximate posterior
distributions are not as accurate as those produced by the other algorithms considered in this article.

% NR
% Detail the function and its derivatives

Taylor expanding the full log likelihood once around the mode yields the following function

\begin{align*}
\log \underline{p}(\vmu, \mLambda; \vy) = \vy^\top\mP\mC\vmu - \vp^\top\exp \left (\mC \vmu \right ) - \half \vmu^\top \mSigma^{-1} \vmu.
\end{align*}

This can be optimised using a Newton-Raphson style algorithm where

\begin{align*}
\frac{\partial \log p(\vmu, \mLambda; \vy)}{\partial \vmu} &\approx \mP \mC (\vy - \exp{(\mC \vmu)}) - \mSigma^{-1} \vmu \text{ and} \\
\frac{\partial \log p(\vmu, \mLambda; \vy)}{\partial \mLambda} &\approx - \mC^\top \text{diag}(\vp e^{(\mC \vmu)}) \mC - \mSigma^{-1}.
\end{align*}

The steps of the algorithm are shown in Algorithm \ref{alg:laplace_alg}.

\begin{algorithm}
\caption{Laplace scheme for optimising $\log \underline{p}(\vmu, \mLambda; \vy)$}
\label{alg:laplace_alg}
\begin{algorithmic}
% Fit \vmu, \mLambda using Laplace approximation
\WHILE{the increase in $\log \underline{p}(\vmu, \mLambda; \vy)$ is significant}
% \vmu, \mLambda
\STATE $\mLambda \leftarrow \left [\mP \mC^\top \text{diag}(\exp{(\mC \vmu)}) \mC + \mSigma^{-1} \right ]^{-1}$ \\ [1ex] 
\STATE $\vmu \leftarrow \vmu + \mLambda \left [ \frac{\partial \log p(\vmu, \mLambda; \vy)}{\partial \vmu} \right ]$ \\ [1ex]
\ENDWHILE
\end{algorithmic}
\end{algorithm}

\subsection{Optimising the GVA lower bound}

% Detail techniques used for fitting models.

Assuming $q(\vnu) \sim N(\vmu, \mLambda)$, the Gaussian variational lower bound in Algorithm
\ref{alg:algorithm_one} can be optimised using a variety of algorithms. The variational lower bound is not
necessarily unimodal, leading to potential difficulty in optimising to the global maximum. However, for fixed
$\vp$ and $\mSigma$, the variational lower bound is log-concave, and so standard optimisation methods such as
L-BFGS-B as described in, for example, \cite{Liu1989}, work well. This led to an extremely accurate
approximation of the true posterior at the expense of some additional computational effort.

\subsubsection{GVA, ($\mLambda = \mR \mR^\top$)}

The first variant of the Gaussian Variational Approximation algorithm optimises the Gaussian variational lower
bound of the log likelihood with respect to $\vmu$ and the Cholesky decomposition $\mR$ of $\mLambda$, that
is, $\mLambda = \mR \mR^\top$. This algorithm trades the computational complexity of numerically evaluating an
integral for greatly increased accuracy in the approximating posterior distribution. The resulting function is
below and can be optimised with L-BFGS-B.

% Detail the function and its derivatives
\begin{align*}
\log \underline{p}(\vmu, \mLambda; \vy) &= \quad \vy^\top\mP \mC \vmu - \vp^\top \exp(\mC \vmu + \half \text{diag}(\mC \mLambda \mC^\top)) - \half \vmu^\top \mSigma^{-1} \vmu - \half \tr{(\mSigma^{-1} \mLambda)} + \log{|\mR|} \\
&\quad - \tfrac{p}{2} \log{(2 \pi)} + \half \log{|\mSigma^{-1}|} + \tfrac{p}{2} \log{(2 \pi)} + \tfrac{p}{2} \\
\end{align*}

using the derivatives

\begin{align*}
\frac{\partial \log \underline{p}(\vmu, \mLambda; \vy)}{\partial \vmu} &= \mP \mC (\vy - \mC^\top \exp(\mC \vmu + \half \text{diag}{(\mC \mLambda \mC^\top)})) - \mSigma^{-1} \vmu \text{ and}\\
\frac{\partial \log \underline{p}(\vmu, \mLambda; \vy)}{\partial \mLambda} &= \left [\mLambda^{-1} - \mP \mC^\top \exp(\mC \vmu + \half \text{diag}(\mC \mLambda \mC^\top)) \mP \mC) - \mSigma^{-1} \right ] \mR.
\end{align*}

\subsubsection{GVA, $\mLambda = \left (\mR \mR^\top \right )^{-1}$}

The second variant of the Gaussian Variational Approximation algorithm is similiar to the first, but instead
of optimising the Gaussian variational lower bound with respect to $\vmu$ and the Cholesky factor $\mR$ of
$\mLambda$, we instead optimise the Cholesky factor of the inverse of $\mLambda$ i.e. $\mLambda = (\mR
\mR^\top)^{-1}$.

This new choice of parameterisation allows us to calculate $\exp(\mC \vmu + \half \text{diag}(\mC \mLambda
\mC^\top))$ and its derivatives by solving the linear system $\mR \va = \mC_{i}, i=1, \ldots, n$ for $\va$ and
then calculating $\va^\top\va$ rather than calculating $\text{diag}(\mC \mLambda \mC^\top)$ directly.

By interchanging the fixed and random effects in $\mC = [\mX \mZ]$ to $\mC = [\mZ \mX]$, and re- ordering the
dimensions of $\vmu, \mLambda$ and $\mSigma$ in the same manner, the independence between the blocks relating
to the random effects in $\mZ$ induce sparsity in the Cholesky factor $\mR$ of $\mLambda^{-1}$. Thus the
Gaussian $q(\vnu) \sim N(\vmu, \mLambda)$ can be optimised over a space of dimension $\half p (p + 1) + pq +
\half q (q + 1)$ rather than dimension $\half (p + mq) (p + mq + 1)$ as in the dense parameterisation. This
leads to subtantial performance gains when m is large, as is typically the case in problems of practical
importance such as clinical trials or the application presented in Section \ref{sec:application}.

The Gaussian variational lower bound in this parameterisation is

\begin{align*}
\log \underline{p}(\vmu, \mLambda; \vy) &= \quad \vy\mP\mC \vmu - \vp^\top \exp(\mC \vmu + \half \text{diag}(\mC \mLambda \mC^\top)) - \half \vmu^\top \mSigma^{-1} \vmu - \half \tr{(\mSigma^{-1} \mLambda)} \\
&\quad- \tfrac{p}{2} \log{(2 \pi)} + \half \log{|\mSigma^{-1}|} + \tfrac{p}{2} \log{(2 \pi)} + \tfrac{p}{2} - \log{|\mR|}
\end{align*}

The derivative with respect to $\vmu$ is the same as that in the GVA algorithm, but as the parameterisation of
$\mLambda$ has changed, the  derivative with respect to $\mLambda$ becomes

\begin{align*}
\frac{\partial \log \underline{p}(\vmu, \mLambda; \vy)}{\partial \mLambda}
&= \hphantom{-}(\mLambda^{-1} + \mH)(-\mLambda \mR \mLambda) \\
&= -(\mI + \mH\mLambda)\mR\mLambda \\
&= - (\mR\mLambda + \mH\mLambda\mR\mLambda)
\end{align*} 

where $\mH = (\mP \mC)^\top \text{diag}(\exp(\mC \vmu + \half \mC \mLambda \mC^\top)) \mP \mC - \mSigma^{-1}$.


\subsubsection{GVA fixed point}

% Fixed point update of \mLambda

This variant of the algorithm uses Newton-Raphson-like fixed point updates on the Gaussian variational lower
bound. This algorithm is fast, but unstable. The steps are presented in Algorithm \ref{alg:algorithm_nr} where

\begin{align*}
\frac{\partial \log \underline{p}(\vmu, \mLambda; \vy)}{\partial \vmu} &= \quad \mC^\top\vp \left [\vy - \mC\exp(\mC \vmu + \half \text{diag}(\mC \mLambda \mC^\top)) \right ] - \mSigma^{-1} \vmu \text{ and}\\
\frac{\partial \log \underline{p}(\vmu, \mLambda; \vy)}{\partial \mLambda} &= -\mC^\top \text{diag}(\vp^\top \exp(\mC \vmu +\half \text{diag}(\mC \mLambda \mC^\top))) - \mSigma^{-1}.
\end{align*}

\begin{algorithm}
\caption[Algorithm GVA NR]{Iterative scheme for obtaining optimal $\vmu$ and $\mLambda$
given $\vy$, $\mC$ and $\vp$}
\label{alg:algorithm_nr}
\begin{algorithmic}
% Fit \vmu, \mLambda using Laplace approximation
\WHILE{the increase in $\log{\underline{p}}(\vmu, \mLambda; \vy)$ is significant}
% \vmu, \mLambda
\STATE $\mLambda \leftarrow \left [ \mP^\top \mC^\top \exp(\mC \vmu + \half \text{diag}(\mC \mLambda \mC^\top)) \mC \mP \right ]^{-1}$ \\ [1ex]
\STATE $\vmu \leftarrow \vmu + \mLambda \left [ \frac{\partial \log \underline{p}(\vmu, \mLambda; \vy)}{\partial \vmu} \right ]$
\ENDWHILE
\end{algorithmic}
\end{algorithm}

% Splines

\section{Numerical results}
\label{sec:results}

The accuracy of each model fitting algorithm presented in Section \ref{sec:algorithms} was assessed by
comparing the approximating distribution of each parameter with the posterior distribution of Monte Carlo
Markov Chain samples of that parameter. 1 million Monte Carlo Markov Chain samples were generated using Stan.
The accuracy of examples of random intercept, random slope and spline models were evaluated using this method.

\subsection{Simulated data}

For each of these simulations, the model using was as presented in Section \ref{sec:methodology}.

Several common application scenarios were simulated and their accuracy evaluated. A random intercept model was
simulated with $\vbeta = (2, 1)^\top$, $\rho = 0.5$, $m = 20$, $n_i = 10$ and $b = 1$. The results are
presented in Table \ref{tab:accuracy_int}. A random slope model was simulated with $\vbeta = (2, 1)^\top$,
$\rho = 0.5$, $m = 20$, $n_i = 10$ and $b = 2$. The results are presented in Table \ref{tab:accuracy_slope}.
Spline model was fit to a data set generated from the function $3 + 3 \sin{(\pi x)}$ on the interval $[-1,
1]$. The resulting model fits are presented in Figure \ref{fig:spline}.

The stability of the algorithms was confirmed by running them on 10,000 different data sets that were randomly
generated after having initialised the random number generator with different seeds.

Median accuracy of the algorithms was assessed by running them on 100 randomly generated data sets. The
results are presented in Figure \ref{fig:median_accuracy_intercept} and Figure
\ref{fig:median_accuracy_slope}.

% Figure: Median accuracy graph intercept
\begin{figure}
\caption{Median accuracy of random intercept}
\label{fig:median_accuracy_intercept}
\includegraphics[width=100mm, height=100mm]{code/results/median_accuracy_combined_intercept.pdf}
\end{figure}

% Figure: Median accuracy graph slope
\begin{figure}
\caption{Median accuracy of slope}
\label{fig:median_accuracy_slope}
\includegraphics[width=120mm, height=120mm]{code/results/median_accuracy_combined_slope.pdf}
\end{figure}

% Table of accuracy results - intercept model
\begin{table}
\caption{Table of accuracy - Random intercept model}
\label{tab:accuracy_int}
\begin{tabular}{|l|rrrr|}
\hline
& Laplace's Method & GVA $(\mLambda = \mR \mR^\top)$ & GVA2 $(\mLambda = (\mR \mR^\top)^{-1})$ & GVA FP\\
\hline
$\vbeta_1$ & $83\%$ & $91\%$ & $91\%$ & $91\%$ \\ 
$\vbeta_2$ & $77\%$ & $99\%$ & $99\%$ & $99\%$ \\ 
Mean of $\vu$ & $81\%$ & $95\%$ & $95\%$ & $95\%$ \\
$\sigma^2_{\vu_1}$ & $63.0\%$ & $63.4\%$ & $63.4\%$ & $63.4\%$ \\ 
$\rho$ & $98\%$ & $97\%$ & $97\%$ & $97\%$ \\ 
\hline
\end{tabular}
\end{table}

\begin{table}
\caption{Table of accuracy - Random slope model}
\label{tab:accuracy_slope}
\begin{tabular}{|l|rrrr|}
\hline
& Laplace's Method & GVA $(\mLambda = \mR \mR^\top)$ & GVA $(\mLambda = (\mR \mR^\top)^{-1})$ & GVA FP\\
\hline
$\vbeta_1$   &66\%&88\%&89\%&88\%\\
$\vbeta_2$   &69\%&88\%&90\%&88\%\\
Mean of $\vu$    &72\%&91\%&91\%&91\%\\
$\sigma^2_{\vu_1}$ &72\%&71\%&71\%&69\%\\
$\sigma^2_{\vu_2}$ &72\%&71\%&71\%&70\%\\
$\rho$ &91\%&90\%&90\%&90\%\\
\hline
\end{tabular}
\end{table}

% \begin{table}
% \caption{Table of accuracy - Splines}
% \label{tab:accuracy_spline}
% \begin{tabular}{|l|l|}
% \hline
% Approximation & Accuracy \\
% \hline
% Laplace's Method & 0.969 \\
% GVA & 0.969 \\
% GVA2 & 0.969 \\
% GVA NR & 0.969 \\
% \hline
% \end{tabular}
% \end{table}

\begin{figure}
\label{fig:spline}
\caption{Comparison of VB and MCMC spline fits with the true function}
\includegraphics[width=100mm, height=100mm]{code/results/accuracy_plots_spline_gva2.pdf}
\end{figure}

% Graphs - exactly what sort of graphs do we need?
% Median accuracy
% Increase in lower bound
% MCMC posterior, with approximating posterior for at least one or two of the
% key parameters, such as, say, vbeta[2]

\subsection{Stability results}

The numerical stability of each fitting algorithm in Section \ref{sec:algorithms} was assessed by initialising
each algorithm from a range of different starting points. Errors due to numerical instability and the fitted
$\vmu$ were recorded for each starting point.

A data set of 100 individuals in ten groups (m=10) was generated from a model with a fixed intercept
and slope, and a random intercept. $\vmu$ was initialised from a grid of points on the interval
$[-4.5, 5]$ for intercept and slope. The error counts are presented in Table
\ref{tab:stability_results}.

\begin{table}
\caption{Count of numerical errors for each algorithm during stability tests}
\label{tab:stability_results}
\begin{tabular}{|l|r|}
\hline
Algorithm & Error count \\
\hline
Laplace's algorithm & 12 \\
GVA & 1,771 \\
GVA2 & 537 \\
GVA FP & 992 \\
\hline
\end{tabular}
\end{table}

\section{Application}
\label{sec:application}

% TODO: You need to describe the data set and the model.

The GVA2 algorithm was used to fit a random intercept model to the Roaches data set provided by Andrew Gelman.
The results are presented in Table \ref{tab:application_roaches}.

%       lci  uci
% 1  3.179 3.157 3.201
% 2 -0.046 -0.053 -0.039
% 3 -0.420 -0.434 -0.406
% 1 -0.976 -1.015 -0.936
% 2 -0.309 -0.323 -0.295
% 3 -0.947 -0.963 -0.930
% 4 -2.129 -2.384 -1.874
% 5 -3.230 -3.490 -2.970
% 6 -3.099 -3.404 -2.794
% 7 -1.290 -1.326 -1.255
% 8 -0.956 -0.991 -0.921
% 9 -2.404 -2.600 -2.209
% 10 -1.076 -1.123 -1.029
% 11 -1.079 -1.107 -1.052
% 12 -1.681 -1.737 -1.624

%> round(cbind(fit1$vmu, lci, uci), 3)
% fit1$a_rho
% [1] 377.2375
% > fit1$b_rho
% [1] 152.7625

\begin{table}
\caption{Table of results - Roaches}
\label{tab:application_roaches}
\begin{tabular}{|l|rrrr|}
\hline
Covariate & Posterior Mean & Lower 95\% CI & Upper 95\% CI & Accuracy \\
\hline
Intercept & 3.18 & 3.16 & 3.20 & 90\% \\
Time & -0.05 & -0.05 & -0.04 & 97\% \\
Time:Treatment & -0.43 & -0.43 & -0.41 & 93\% \\
Random intercept & -1.60 & -1.71 & -1.49 & 90\% \\
$\sigma^2_{\vu_1}$ & 0.58 & 0.57 & 0.57 & 57\% \\
$\rho$ & 0.71 & 0.67 & 0.75 & 88\% \\
\hline
\end{tabular}
\end{table}

\begin{figure}
	\caption{Accuracy graphs for roach model}
	\label{fig:accuracy_roach}
	\centering
	% \includepdf[width=75mm,height=75mm,pages={1,2,3,16},nup=2x2]{code/results/accuracy_plots_application_gva2.pdf}
	\begin{tabular}{@{}c@{\hspace{.5cm}}c@{}}
		\includegraphics[page=1,width=.45\textwidth]{code/results/accuracy_plots_application_gva2.pdf} & 
		\includegraphics[page=2,width=.45\textwidth]{code/results/accuracy_plots_application_gva2.pdf} \\[.5cm]
		\includegraphics[page=3,width=.45\textwidth]{code/results/accuracy_plots_application_gva2.pdf} &
		\includegraphics[page=16,width=.45\textwidth]{code/results/accuracy_plots_application_gva2.pdf} \\[.5cm]
	\end{tabular}
\end{figure}

% \begin{figure}
% \caption{Accuracy graph for fixed intercept of roach model}
% \label{fig:accuracy_roach_intercept}
% \includegraphics[width=100mm, height=100mm]{code/results/application_fixed_intercept.pdf}
% \end{figure}

% \begin{figure}
% \caption{Accuracy graph for fixed slope of roach model}
% \label{fig:accuracy_roach_slope}
% \includegraphics[width=100mm, height=100mm]{code/results/application_fixed_slope_1.pdf}
% \end{figure}

% \begin{figure}
% \caption{Accuracy graph for $\sigma_u^2$ of roach model}
% \label{fig:accuracy_roach_sigmau2}
% \includegraphics[width=100mm, height=100mm]{code/results/application_fixed_slope_2.pdf}
% \end{figure}

% \begin{figure}
% \caption{Accuracy graph for $\rho$ of roach model}
% \label{fig:accuracy_roach_rho}
% \includegraphics[width=100mm, height=100mm]{code/results/application_variance.pdf}
% \end{figure}

To assess the speed of each approach, a test case was constructed of a random slope model with $m=50$ groups,
each containing $n_i = 100$ individuals. A model was then fit to this data set ten times using each algorithm,
and the results averaged. They are presented in Table \ref{tab:application_slope_speed}.

\begin{table}
\caption{Table of results - Speed}
\label{tab:application_slope_speed}
\begin{tabular}{|l|rr|}
\hline
Algorithm & Mean (seconds) & Standard deviation (seconds) \\
\hline
Laplace's method & 18.79 s & 0.07 s \\
GVA & 76.18 s & 1.24 s \\
GVA2 & 27.55 s & 0.66 s \\
GVA FP & 4.83 s & 0.07 s \\
\hline
\end{tabular}
\end{table}

\section{Discussion}


\newpage
\section{Appendix} 
% TODO: Mean field updates?
\subsection{Calculation of the Variational Lower bound}
% Where are the priors for \vbeta and \vu

The variational lower bound is equal to $\bE_q[\log{p(\vy, \vtheta)} - \log{q(\vtheta)}] = T_1 + T_2 + T_3$,
where

% This is the new T_1
$$
\begin{array}{rl}
T_1 &= \quad \bE_q[\log{p(\vy, \vnu)} - \log{q(\vnu)}] \\
&= \quad \vy \mP \mC \vmu - \vp^\top \exp{\left[ \mC \vmu + \half \text{diag} (\mC \mLambda \mC^\top) \right]} - \vone^\top\log \Gamma{(\vy + \vone)}\\
& \quad + \frac{p + m}{2} (1 + \log{2 \pi}) + \half \log{|\mLambda|}, \\
T_2 &= \quad \bE_q \left[ \log p(\mSigma_{\vu \vu}) - \log q(\mSigma_{\vu \vu}) \right] \\
&= \quad \bE_q \big[ v/2(\log |\Psi| - \log |\Psi + \vmu_\vu \vmu_\vu^\top + \mLambda_{\vu \vu}|) + \half \log 2 + \half \log|\mSigma_{\vu \vu}| + \log \Gamma_{p+1}(v/2) - \log \Gamma_{p}(v/2)\\
&\quad + \half \tr((\vmu_{\vu} \vmu_{\vu}^\top + \mLambda_{\vu \vu}) \mSigma_{\vu \vu}^{-1}) \big] \\
&= \quad v/2\big(\log |\Psi| - \log |\Psi + \vmu_\vu \vmu_\vu^\top + \mLambda_{\vu \vu}|\big) + \half \log 2 + \half \bE_q \log |\mSigma_{\vu \vu}| + \log \Gamma_{p+1}(v/2) - \log \Gamma_{p}(v/2) \\
&\quad + \half \tr\big(\mI_m + \Psi(\Psi+ \vmu_\vu \vmu_\vu^\top + \mLambda_{\vu \vu})^{-1}/(v + p + 2)\big) \\
T_3 &= - \vp^\top \log \vp - (\vone - \vp)^\top \log (\vone - \vp) - \log \Beta (\alpha_\rho, \beta_\rho) + \log \Beta (\alpha_q, \beta_q)
\end{array}
$$

with $\bE_q \big[ \log |\mSigma_{\vu \vu}| \big] = m \log 2 + \log \left | \Psi + \vmu_\vu \vmu_\vu^\top + \mLambda_{\vu \vu} \right | + \sum_{i=1}^m \Psi \left ( \frac{v - i + 1}{2} \right )$

\subsection{Numerical stability of fitting algorithms with respect to starting point}

% TODO: Generate images using local_solutions.R and place here

\bibliographystyle{elsarticle-harv}
\bibliography{Chapter_1_zero_inflated_models}

\end{document}
