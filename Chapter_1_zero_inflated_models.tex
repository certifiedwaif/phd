\documentclass{amsart}
% include.tex
\newcommand{\Bernoulli}[1]{\text{Bernoulli} \left( #1 \right)}
\newcommand{\mydigamma}[1]{\psi \left( #1 \right)}
%\newcommand{\diag}[1]{\text{diag}\left( #1 \right)}
\newcommand{\tr}[1]{\text{tr}\left( #1 \right)}
\newcommand{\Poisson}[1]{\text{Poisson} \left( #1 \right)}
\def \half {\frac{1}{2}}
\def \R {\mathbb{R}}
\def \vbeta {\vec{\beta}}
\def \vy {\vec{y}}
\def \vmu {\vec{\mu}}
\def \vmuqbeta {\vmu_{q(\vbeta)}}
\def \vmubeta {\vmu_{\vbeta}}
\def \Sigmaqbeta {\Sigma_{q(\vbeta)}}
\def \Sigmabeta {\Sigma_{\vbeta}}
\def \va {\vec{a}}
\def \vtheta {\vec{\theta}}
\def \mX {\vec{X}}

\def\ds{{\displaystyle}}

\def\diag{{\mbox{diag}}}

\begin{document}
% First, simplest zero-inflated count model to consider.
Count data with a large number of zero counts arises in many areas of
application, such as data arising from physical activity studies, 
insurance claims, hospital visits or defects in manufacturing processes.

\section{A factorised variational approximation to the univariate zero-inflated Poisson model}

\noindent Consider $X_1, X_2, \ldots, X_n$ where $X_i = R_i Y_i$, $Y_i \sim \Poisson{\lambda}$ independent of
$R_i \text{indep.} \sim \Bernoulli{\rho}$. Let $\rho \sim \Unif{0, 1}$, 
$\lambda \sim \Gamma(a, b)$ and $r_i | \rho \sim \Bernoulli{\rho}$, $1 
\leq i \leq n$. Then the joint likelihood is:

$$
p(\vx, \vr, \lambda, \rho) = \frac{b^a \lambda^{a - 1} \exp{(-b \lambda)}}{\Gamma{(a)}} \prod_{i=1}^n \frac{\exp{(-\lambda r_i)} (\lambda r_i)^{x_i}}{x_i !} \rho^{r_i} (1 - \rho)^{1 - r_i}.
$$

\noindent We calculate the full conditionals for $\lambda$, $\rho$ and $\vr$.

\subsection{Full conditional for $\lambda$}
First, we calculate the full conditional for $\lambda$.

\begin{align*}
p(\lambda|\vx, \vr, \rho) &= \frac{\prod_{i=1}^n p(x_i | \lambda, r_i) p(\lambda) p(r_i|\rho) p(\rho)}{\int \prod_{i=1}^n p(x_i | \lambda, r_i) p(\lambda) p(r_i|\rho) p(\rho)) d \lambda}.
\end{align*}

Concentrating for now on the denominator in this expression, we re-arrange and collect
like terms to obtain
$$
\prod_{i=1}^n \rho^r_i (1 - \rho)^{1 - r_i} \frac{r_i^{x_i}}{x_i !}
	\int \frac{b^a \lambda^{a - 1} \exp{(-b \lambda)} \lambda^{x_i} \exp{(-\lambda r_i)}}{\Gamma{(a)}} d \lambda
$$

The integral in this expression is
\begin{align*}
& \int \frac{b^a \lambda^{(a + x_i) - 1} \exp{(-\lambda(b + r_i))}}{\Gamma{(a + x_i)}} d \lambda \frac{\Gamma{(a+ x_i)}}{\Gamma{(a)} b^{-x_i}} \\
=& \frac{\Gamma{(a+ x_i)}}{\Gamma{(a)} b^{-x_i}}.
\end{align*}

Collecting the multiplicands in the integral over $\lambda$ together, we obtain
$$
\prod_{i=1}^n \rho^{r_i} (1 - \rho)^{1 - r_i} \frac{r_i^{x_i}}{x_i!}
	\int \frac{b^{na + \sum_{i=1}^n x_i} \lambda^{(na + \sum_{i=1}^n x_i) - 1} \exp{(-\lambda(nb + \sum_{i=1}^n r_i))}}{\Gamma{(na + \sum_{i=1}^n x_i)}} d \lambda
	\frac{\Gamma{(na + \sum_{i=1}^n x_i)}}{\Gamma{(na)} b^{-\sum_{i=1}^n x_i}}.
$$

By cancelling like terms in the numerator and denominator of the full likelihood we arrive 
at the functional form of the full conditional, which is proportional to

$$
b^{a+\sum_{i=1}^n x_i} \lambda^{(a + \sum_{i=1}^n x_i) - 1} \exp{\left(-(b + \sum_{i=1}^n r_i) \lambda \right)}
$$

\noindent which is clearly seen to be a Gamma distribution. Thus
$$
\lambda | \vx, \vr, \rho \sim \myGamma{a + \sum_{i=1}^n x_i, b + \sum_{i=1}^n r_i}.
$$

\subsection{Full conditional for $\rho$}
We now turn our attention to the full conditional distribution for $\rho$. Following the same
technique as above, we first concentrate on the denominator in the full conditional:
\begin{align*}
p(\vx, \vr, \lambda) &= \int \frac{b^a \lambda^{a - 1} \exp{(-b \lambda)}}{\Gamma{(a)}} \prod_{i=1}^n \frac{\exp{(-\lambda r_i)} (\lambda r_i)^{x_i}}{x_i !} \rho^{r_i} (1 - \rho)^{1 - r_i} d \rho \\
&= \left(\frac{b^a \lambda^{a - 1} \exp{(-b \lambda)}}{\Gamma{(a)}} \prod_{i=1}^n \frac{\exp{(-\lambda r_i)} (\lambda r_i)^{x_i}}{x_i !} \right) \int \rho^{\sum_{i=1}^n r_i} (1 - \rho)^{n - \sum_{i=1}^n r_i} d \rho
\end{align*}

This integral on the right hand side is a Beta integral with the parameters
$a = \sum_{i=1}^n r_i + 1$ and $b = (n - \sum_{i=1}^n r_i) + 1$. All of the terms in the numerator and 
denominator of the full conditional cancel except for
$$
\frac{\rho^{\sum_{i=1}^n r_i} (1 - \rho)^{n - \sum_{i=1}^n r_i}}{\Beta{\sum_{i=1}^n r_i + 1, n - \sum_{i=1}^n r_i + 1}}
$$

and thus we see that $\rho | \vx, \vr, \lambda \sim \Beta{\sum_{i=1}^n r_i + 1, (n - \sum_{i=1}^n r_i) + 1}$.

\subsection{Full conditional for $r_i$}
\begin{align*}
p(r_i | \text{rest}) \propto \frac{(\lambda r_i)^{x_i} \exp{(-\lambda r_i)}}{x_i !} \rho^{r_i} (1 - \rho)^{1 - r_i} \\
\propto r_i^{x_i} (e^{-\lambda})^{r_i} \rho^{r_i} (1 - \rho)^{1 - r_i}
\end{align*}.

Once again, we rely upon the fact that if $x_i = 0$, $r_i = 0$, and if $x_i \ne 0$,
$r_i = 1$. So $x_i^{r_i} = I(x \ne 0)$ and hence the likelihood can be re-written as

\begin{align*}
p(r_i | \text{rest}) &= \frac{(e^{-\lambda + \logit{\rho}})^{r_i}}{I(x_i = 0) + (e^{-\lambda + \logit{\rho}})^{r_i}} \\
&= \Bernoulli{\text{expit}(\eta_i)}
\end{align*}.

% Step Two: Assume q(r_i) = Bernoulli(\rho_i), 1 \leq i \leq n for some known \rho_i. Find the
% variational Bayes updates of the q-densities q(\lambda) and q(\rho) corresponding to the
% factorisation
% q(\vr, \lambda, \rho) = q(\lambda) q(\rho) \sum_{i=1}^n q(r_i)

We assume a factorised approximation of the form

$$
q(\theta) = q(\lambda) q(\rho) \prod_{i=1}^n q(p_i)
$$

where $q(\lambda)$ is a Gamma distribution, $q(\rho)$ is a Beta distribution and
$q(p_i)$ are Bernoulli distributions.

By taking the expectation of each full conditional with respect to 

\subsection{Mean field update equations}
We are now in a position to calculate the mean field update equations for the factorised
variational approximation.

Assuming that $q^*(r_i) \sim \Bernoulli{p_i}$,

\begin{align*}
q^*(\lambda) &\propto \lambda^{a+\sum_{i=1}^n x_i - 1} \exp{\left[ E_{-q(\lambda)} \{-(b + \sum_{i=1}^n r_i) \lambda \} \right]} \\
&\propto \lambda^{a+\sum_{i=1}^n x_i - 1} \exp{\{-(b + \sum_{i=1}^n p_i)\lambda \} } \\
&= \myGamma{a+\sum_{i=1}^n x_i, b + \sum_{i=1}^n p_i}
\end{align*}

and

\begin{align*}
q^*(\rho) &\propto \exp{\{ E_{-q(\rho)} \left[ \sum_{i=1}^n r_i \log{(\rho)} + \sum_{i=1}^n (1- r_i) \log{(1 - \rho)} \right] } \\
&\propto \exp{ \left( \sum_{i=1}^n p_i \log{\rho} + \sum_{i=1}^n (1 - p_i) \log{(1 - \rho)} \right) } \\
&= \Beta{1+ \sum_{i=1} p_i, n - \sum_{i=1}^n p_i + 1}
\end{align*}

Turning our attention now to $r_i$, we first note that the relevant terms of the
log-likelihood are

$$
-\lambda r_i + x_i \log{r_i} + r_i \log{\left(\frac{\rho}{1 - \rho}\right)}
$$

Taking expectations with respect to $\lambda$ and $\rho$ we have

$$
-r_i \frac{\alpha_\lambda^*}{\beta_\lambda^*} + x_i \log{(r_i)} + r_i (\Psi(\alpha_\rho^*) - \Psi(\beta_\rho^*))
$$

Note that if $x_i = 0$ then
$x_i \log{(r_i)} = \log{(x_i^{r_i})} = \log{(0^0)} = \log{(1)} = 0$. Dr John Ormerod hit
upon the idea of side-stepping this problem by writing the q-likelihood as

$$
q_{r_i}^*(r_i) \propto r_i^{x_i} \exp{(r_i \eta_i)}
$$

where $\eta_i = - \frac{\alpha_\lambda^*}{\beta_\lambda^*} + \Psi(\alpha_\rho^*) - \Psi(\beta_\rho^*)$.

Now, either $x_i = 0$ or $x_i \ne 0$.

\begin{align*}
p_i &= \frac{\exp{(\eta_i)}}{I(x_i = 0) + \exp{(\eta_i)}} \\
&= \text{expit}(\eta_i)
\end{align*}

\subsection{Lower bound}
The lower bound $\log{p(x)}$ is
$$
	E_q[\log{p(x, \theta)} - \log{q(\theta)}]
$$

where $q(\theta) = q(\lambda) q(\rho) \prod_{i=1}^n q(r_i)$,
$q(\lambda) \sim \text{Gamma}{(a_\lambda^*, b\lambda^*)}$,
$q(\rho) \sim \text{Beta}(a_\rho^*, b_\rho^*)$ and
$q(r_i) = 1$ if $x_i \ne 0$, and $p_i$ if $x_i = 0$, where $p_i$ is
calculated for each iteration as specified above.

In this case, the lower bound can be calculated directly to be

\begin{align*}
E_q \log{p(x, r, \lambda, \rho)} - E_q \log{q(r, \lambda, \rho)} &= a \log{b} + (a-1) E\left[\log{\lambda}\right] - b E\left[\lambda\right] - \log{\Gamma{(a)}} \\
& + \sum_{i=1}^n \left(-E \left[ \lambda \right] E(r_i) + E \left[x_i \log{\lambda r_i}\right] - \log{x_i!} + E \left[r_i\right] E\left[\log(\rho)\right] + E \left[(1-r_i)\right] E\left[\log{1 - \rho}\right] \right) \\
& - \log{q(r)} - \log{q(\lambda)} - \log{q(\rho)}
\end{align*}

where
$$
E [\log{\lambda}] = -\{ a_\lambda^* - \log{b_\lambda^*} + \log{\gamma(a_\lambda^*)} + (1 - a_\lambda^*) \Psi{a_\lambda^*} \} \\
$$
$$
E [\lambda] = \frac{a_\lambda^*}{b_\lambda^*} \\
$$
$$
E[\log{\rho}] = - \{ \log{B(a_\rho^*, b_\rho^*)} - (a_\rho^* - 1) \psi{a_\rho^*} - (b_\rho^* - 1)\Psi{b_\rho^*} + (a_\rho^* + b_\rho^* - 2)\Psi{a_\rho^* + b_\rho^*} \} \\
$$

By symmetry, we see that $(1 - \rho)$ has the same entropy as $\rho$,
and so $E [\log{(\rho)}] = E [\log{(1 - \rho)}]$.
$$
E[r_i] = 
	\begin{cases}
	1 & \text{if } x_i \ne 0 \\
	p_i & \text{if } x_i = 0 \\
	\end{cases}
$$

$$
-E_q \log{q(r)} = \sum_{i=1}^n I(x_i = 0) \log{(p_i)}
$$

$$
-E_q \log{q(\lambda)} = q_q^* - \log{(b_q^*)} + \log{\Gamma{(a_q^*)}} + (1 - a_q^*) \Psi{(a_q^*)}
$$

$$
E[\log{q(\rho)}] = - \{ \log{(B(a_q^*, b_q^*)} - (\alpha_q^* - 1) \psi{(a_q^*)} - (\beta_q^* - 1)\Psi{(\beta_q^*)} + (\alpha_q^* + \beta_q^* - 2)\Psi{(\alpha_q^* + \beta_q^*)} \}
$$

\section{Extending the zero-inflated Poisson model to a regression model}
The above univariate model demonstrates that variational approximations are well-suited
to accelerating the fit of Bayesian zero-inflated models to data. We now extend the above
model to a regression model by replacing the single parameter $\lambda$ with a set of
regression coefficients $\beta$. We replace the Gamma prior on $\lambda$ with a multivariate
normal prior $p(\vbeta) \sim N(\mu_\beta, \Sigma_\beta)$.

Using the formulation of GLMs from \ref{Wand02}

$$
p(y|X, \vbeta) = \exp{\left[ \vy^T(\log{\vr} + X\vbeta) - \vr \exp{(X\vbeta)} - \log{(\vy!)} \right]}
$$

\subsection{Mean field updates}
As the prior for $\vbeta$ is not conjugate with the Poisson distribution, $\vbeta$ must be
updated by Newton-Raphson updates on the lower bound.

The likelihood for $r_i$ is proportional to
$$
-\vr \exp{\left(X\vbeta\right)} + \vr \log{(\rho)} + (1 - \vr) \log{(1 - \rho)}
$$

Taking the expectation with respect to $\vbeta$

$$
\mathbb{E_{\vbeta}}\left[ e^{X \vbeta} \right] = % FIXME: What's the best we to evaluate this?
$$

\subsection{Lower bound}

\end{document}
