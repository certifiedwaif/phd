\documentclass{amsart}[12pt]
% \documentclass[times, doublespace]{anzsauth}

\addtolength{\oddsidemargin}{-.75in}%
\addtolength{\evensidemargin}{-.75in}%
\addtolength{\textwidth}{1.5in}%
\addtolength{\textheight}{1.3in}%
\addtolength{\topmargin}{-.8in}%
\addtolength{\marginparpush}{-.75in}%
% \setlength\parindent{0pt}
% \setlength{\bibsep}{0pt plus 0.3ex}

% \usepackage[authoryear]{natbib}
\usepackage{graphicx}
\usepackage{algorithm,algorithmic}

\usepackage{graphicx}
\usepackage{verbatim}
\usepackage{subfigure}
\usepackage{amsfonts}
\usepackage{color}
\usepackage{natbib}

% include.tex

% \newcommand{\expit}[1]{\text{expit} #1}
% \newcommand{\logit}[1]{\text{logit} #1}

\def \R {{\mathbb{R}}}
\def \vbeta {{\boldsymbol \beta}}
\def \vnu {{\bf \nu}}
\def \vy {{\bf y}}
\def \vx {{\bf x}}
\def \vu {{\bf u}}
\def \vr {{\bf r}}
\def \vp {{\bf p}}
\def\vectorfontone{\bf}
\def\vone{{\bf 1}}
\def\vzero{{\bf 0}}
\def \vmu {{\boldsymbol \mu}}
\def \vnu {{\bf \nu}}
\def \vmuqbeta {{\vmu_{q(\vbeta)}}}
\def \vmubeta {{\vmu_{\vbeta}}}
\def \Sigmaqbeta {{\Sigma_{q(\vbeta)}}}
\def \Sigmabeta {{\Sigma_{\vbeta}}}
\def \va {{\bf a}}
\def \vtheta {{\bf \theta}}
\def \mX {{\bf X}}
\def \mZ {{\bf Z}}
\def \mR {{\bf R}}
\def \mC {{\bf C}}
\def \mI {{\bf I}}
\def \mLambda {{\boldsymbol \Lambda}}
\def \mSigma {{\boldsymbol \Sigma}}
\def \B {{\text{B}}}

\def\ds{{\displaystyle}}

\def\diag{{\mbox{diag}}}
\def\bbE{\mathbb{E}}


\usepackage{latexsym,amssymb,amsmath,amsfonts}
%\usepackage{tabularx}
\usepackage{theorem}
\usepackage{verbatim,array,multicol,palatino}
\usepackage{graphicx}
\usepackage{graphics}
\usepackage{fancyhdr}
\usepackage{algorithm,algorithmic}
\usepackage{url}
%\usepackage[all]{xy}



\def\approxdist{\stackrel{{\tiny \mbox{approx.}}}{\sim}}
\def\smhalf{\textstyle{\frac{1}{2}}}
\def\vxnew{\vx_{\mbox{{\tiny new}}}}
\def\bib{\vskip12pt\par\noindent\hangindent=1 true cm\hangafter=1}
\def\jump{\vskip3mm\noindent}
\def\etal{{\em et al.}}
\def\etahat{{\widehat\eta}}
\def\thick#1{\hbox{\rlap{$#1$}\kern0.25pt\rlap{$#1$}\kern0.25pt$#1$}}
\def\smbbeta{{\thick{\scriptstyle{\beta}}}}
\def\smbtheta{{\thick{\scriptstyle{\theta}}}}
\def\smbu{{\thick{\scriptstyle{\rm u}}}}
\def\smbzero{{\thick{\scriptstyle{0}}}}
\def\boxit#1{\begin{center}\fbox{#1}\end{center}}
\def\lboxit#1{\vbox{\hrule\hbox{\vrule\kern6pt
      \vbox{\kern6pt#1\kern6pt}\kern6pt\vrule}\hrule}}
\def\thickboxit#1{\vbox{{\hrule height 1mm}\hbox{{\vrule width 1mm}\kern6pt
          \vbox{\kern6pt#1\kern6pt}\kern6pt{\vrule width 1mm}}
               {\hrule height 1mm}}}


%\sloppy
%\usepackage{geometry}
%\geometry{verbose,a4paper,tmargin=20mm,bmargin=20mm,lmargin=40mm,rmargin=20mm}


%%%%%%%%%%%%%%%%%%%%%%%%%%%%%%%%%%%%%%%%%%%%%%%%%%%%%%%%%%%%%%%%%%%%%%%%%%%%%%%%
%
% Some convenience definitions
%
% \bf      -> vector
% \sf      -> matrix
% \mathcal -> sets or statistical
% \mathbb  -> fields or statistical
%
%%%%%%%%%%%%%%%%%%%%%%%%%%%%%%%%%%%%%%%%%%%%%%%%%%%%%%%%%%%%%%%%%%%%%%%%%%%%%%%%

% Sets or statistical values
\def\sI{{\mathcal I}}                            % Current Index set
\def\sJ{{\mathcal J}}                            % Select Index set
\def\sL{{\mathcal L}}                            % Likelihood
\def\sl{{\ell}}                                  % Log-likelihood
\def\sN{{\mathcal N}}                            
\def\sS{{\mathcal S}}                            
\def\sP{{\mathcal P}}                            
\def\sQ{{\mathcal Q}}                            
\def\sB{{\mathcal B}}                            
\def\sD{{\mathcal D}}                            
\def\sT{{\mathcal T}}
\def\sE{{\mathcal E}}                            
\def\sF{{\mathcal F}}                            
\def\sC{{\mathcal C}}                            
\def\sO{{\mathcal O}}                            
\def\sH{{\mathcal H}} 
\def\sR{{\mathcal R}}                            
\def\sJ{{\mathcal J}}                            
\def\sCP{{\mathcal CP}}                            
\def\sX{{\mathcal X}}                            
\def\sA{{\mathcal A}} 
\def\sZ{{\mathcal Z}}                            
\def\sM{{\mathcal M}}                            
\def\sK{{\mathcal K}}     
\def\sG{{\mathcal G}}                         
\def\sY{{\mathcal Y}}                         
\def\sU{{\mathcal U}}  


\def\sIG{{\mathcal IG}}                            


\def\cD{{\sf D}}
\def\cH{{\sf H}}
\def\cI{{\sf I}}

% Vectors
\def\vectorfontone{\bf}
\def\vectorfonttwo{\boldsymbol}
\def\va{{\vectorfontone a}}                      %
\def\vb{{\vectorfontone b}}                      %
\def\vc{{\vectorfontone c}}                      %
\def\vd{{\vectorfontone d}}                      %
\def\ve{{\vectorfontone e}}                      %
\def\vf{{\vectorfontone f}}                      %
\def\vg{{\vectorfontone g}}                      %
\def\vh{{\vectorfontone h}}                      %
\def\vi{{\vectorfontone i}}                      %
\def\vj{{\vectorfontone j}}                      %
\def\vk{{\vectorfontone k}}                      %
\def\vl{{\vectorfontone l}}                      %
\def\vm{{\vectorfontone m}}                      % number of basis functions
\def\vn{{\vectorfontone n}}                      % number of training samples
\def\vo{{\vectorfontone o}}                      %
\def\vp{{\vectorfontone p}}                      % number of unpenalized coefficients
\def\vq{{\vectorfontone q}}                      % number of penalized coefficients
\def\vr{{\vectorfontone r}}                      %
\def\vs{{\vectorfontone s}}                      %
\def\vt{{\vectorfontone t}}                      %
\def\vu{{\vectorfontone u}}                      % Penalized coefficients
\def\vv{{\vectorfontone v}}                      %
\def\vw{{\vectorfontone w}}                      %
\def\vx{{\vectorfontone x}}                      % Covariates/Predictors
\def\vy{{\vectorfontone y}}                      % Targets/Labels
\def\vz{{\vectorfontone z}}                      %

\def\vone{{\vectorfontone 1}}
\def\vzero{{\vectorfontone 0}}

\def\valpha{{\vectorfonttwo \alpha}}             %
\def\vbeta{{\vectorfonttwo \beta}}               % Unpenalized coefficients
\def\vgamma{{\vectorfonttwo \gamma}}             %
\def\vdelta{{\vectorfonttwo \delta}}             %
\def\vepsilon{{\vectorfonttwo \epsilon}}         %
\def\vvarepsilon{{\vectorfonttwo \varepsilon}}   % Vector of errors
\def\vzeta{{\vectorfonttwo \zeta}}               %
\def\veta{{\vectorfonttwo \eta}}                 % Vector of natural parameters
\def\vtheta{{\vectorfonttwo \theta}}             % Vector of combined coefficients
\def\vvartheta{{\vectorfonttwo \vartheta}}       %
\def\viota{{\vectorfonttwo \iota}}               %
\def\vkappa{{\vectorfonttwo \kappa}}             %
\def\vlambda{{\vectorfonttwo \lambda}}           % Vector of smoothing parameters
\def\vmu{{\vectorfonttwo \mu}}                   % Vector of means
\def\vnu{{\vectorfonttwo \nu}}                   %
\def\vxi{{\vectorfonttwo \xi}}                   %
\def\vpi{{\vectorfonttwo \pi}}                   %
\def\vvarpi{{\vectorfonttwo \varpi}}             %
\def\vrho{{\vectorfonttwo \rho}}                 %
\def\vvarrho{{\vectorfonttwo \varrho}}           %
\def\vsigma{{\vectorfonttwo \sigma}}             %
\def\vvarsigma{{\vectorfonttwo \varsigma}}       %
\def\vtau{{\vectorfonttwo \tau}}                 %
\def\vupsilon{{\vectorfonttwo \upsilon}}         %
\def\vphi{{\vectorfonttwo \phi}}                 %
\def\vvarphi{{\vectorfonttwo \varphi}}           %
\def\vchi{{\vectorfonttwo \chi}}                 %
\def\vpsi{{\vectorfonttwo \psi}}                 %
\def\vomega{{\vectorfonttwo \omega}}             %


% Matrices
%\def\matrixfontone{\sf}
%\def\matrixfonttwo{\sf}
\def\matrixfontone{\bf}
\def\matrixfonttwo{\boldsymbol}
\def\mA{{\matrixfontone A}}                      %
\def\mB{{\matrixfontone B}}                      %
\def\mC{{\matrixfontone C}}                      % Combined Design Matrix
\def\mD{{\matrixfontone D}}                      % Penalty Matrix for \vu_J
\def\mE{{\matrixfontone E}}                      %
\def\mF{{\matrixfontone F}}                      %
\def\mG{{\matrixfontone G}}                      % Penalty Matrix for \vu
\def\mH{{\matrixfontone H}}                      %
\def\mI{{\matrixfontone I}}                      % Identity Matrix
\def\mJ{{\matrixfontone J}}                      %
\def\mK{{\matrixfontone K}}                      %
\def\mL{{\matrixfontone L}}                      % Lower bound
\def\mM{{\matrixfontone M}}                      %
\def\mN{{\matrixfontone N}}                      %
\def\mO{{\matrixfontone O}}                      %
\def\mP{{\matrixfontone P}}                      %
\def\mQ{{\matrixfontone Q}}                      %
\def\mR{{\matrixfontone R}}                      %
\def\mS{{\matrixfontone S}}                      %
\def\mT{{\matrixfontone T}}                      %
\def\mU{{\matrixfontone U}}                      % Upper bound
\def\mV{{\matrixfontone V}}                      %
\def\mW{{\matrixfontone W}}                      % Variance Matrix i.e. diag(b'')
\def\mX{{\matrixfontone X}}                      % Unpenalized Design Matrix/Nullspace Matrix
\def\mY{{\matrixfontone Y}}                      %
\def\mZ{{\matrixfontone Z}}                      % Penalized Design Matrix/Kernel Space Matrix

\def\mGamma{{\matrixfonttwo \Gamma}}             %
\def\mDelta{{\matrixfonttwo \Delta}}             %
\def\mTheta{{\matrixfonttwo \Theta}}             %
\def\mLambda{{\matrixfonttwo \Lambda}}           % Penalty Matrix for \vnu
\def\mXi{{\matrixfonttwo \Xi}}                   %
\def\mPi{{\matrixfonttwo \Pi}}                   %
\def\mSigma{{\matrixfonttwo \Sigma}}             %
\def\mUpsilon{{\matrixfonttwo \Upsilon}}         %
\def\mPhi{{\matrixfonttwo \Phi}}                 %
\def\mOmega{{\matrixfonttwo \Omega}}             %
\def\mPsi{{\matrixfonttwo \Psi}}                 %

\def\mone{{\matrixfontone 1}}
\def\mzero{{\matrixfontone 0}}

% Fields or Statistical
\def\bE{{\mathbb E}}                             % Expectation
\def\bP{{\mathbb P}}                             % Probability
\def\bR{{\mathbb R}}                             % Reals
\def\bI{{\mathbb I}}                             % Reals
\def\bV{{\mathbb V}}                             % Reals

\def\vX{{\vectorfontone X}}                      % Targets/Labels
\def\vY{{\vectorfontone Y}}                      % Targets/Labels
\def\vZ{{\vectorfontone Z}}                      %

% Other
\def\etal{{\em et al.}}
\def\ds{\displaystyle}
\def\d{\partial}
\def\diag{\text{diag}}
%\def\span{\text{span}}
\def\blockdiag{\text{blockdiag}}
\def\tr{\text{tr}}
\def\RSS{\text{RSS}}
\def\df{\text{df}}
\def\GCV{\text{GCV}}
\def\AIC{\text{AIC}}
\def\MLC{\text{MLC}}
\def\mAIC{\text{mAIC}}
\def\cAIC{\text{cAIC}}
\def\rank{\text{rank}}
\def\MASE{\text{MASE}}
\def\SMSE{\text{SASE}}
\def\sign{\text{sign}}
\def\card{\text{card}}
\def\notexp{\text{notexp}}
\def\ASE{\text{ASE}}
\def\ML{\text{ML}}
\def\nullity{\text{nullity}}

\def\logexpit{\text{logexpit}}
\def\logit{\mbox{logit}}
\def\dg{\mbox{dg}}

\def\Bern{\mbox{Bernoulli}}
\def\sBernoulli{\mbox{Bernoulli}}
\def\sGamma{\mbox{Gamma}}
\def\sInvN{\mbox{Inv}\sN}
\def\sNegBin{\sN\sB}

\def\dGamma{\mbox{Gamma}}
\def\dInvGam{\mbox{Inv}\Gamma}

\def\Cov{\mbox{Cov}}
\def\Mgf{\mbox{Mgf}}

\def\mis{{mis}} 
\def\obs{{obs}}

\def\argmax{\operatornamewithlimits{\text{argmax}}}
\def\argmin{\operatornamewithlimits{\text{argmin}}}
\def\argsup{\operatornamewithlimits{\text{argsup}}}
\def\arginf{\operatornamewithlimits{\text{arginf}}}


\def\minimize{\operatornamewithlimits{\text{minimize}}}
\def\maximize{\operatornamewithlimits{\text{maximize}}}
\def\suchthat{\text{such that}}


\def\relstack#1#2{\mathop{#1}\limits_{#2}}
\def\sfrac#1#2{{\textstyle{\frac{#1}{#2}}}}


\def\comment#1{
\vspace{0.5cm}
\noindent \begin{tabular}{|p{14cm}|}  
\hline #1 \\ 
\hline 
\end{tabular}
\vspace{0.5cm}
}


\def\mytext#1{\begin{tabular}{p{13cm}}#1\end{tabular}}
\def\mytextB#1{\begin{tabular}{p{7.5cm}}#1\end{tabular}}
\def\mytextC#1{\begin{tabular}{p{12cm}}#1\end{tabular}}

\def\jump{\vskip3mm\noindent}

\def\KL{\text{KL}}
\def\N{\text{N}}
\def\Var{\text{Var}}

\def \E {\mathbb{E}}
\def \BigO {\text{O}}
\def \IG {\text{IG}}
\def \Beta {\text{Beta}}



\newcommand{\mgc}[1]{{\color{blue}#1}}
\newcommand{\joc}[1]{{\color{red}#1}}
 

\begin{document}

\title{Variational Approximation for Zero-Inflated Semiparametric Regression Models}
\author{Mark Greenaway and John T. Ormerod}

\begin{abstract}
	\noindent We consider variational inference for zero--inflated Poisson regression models using a latent
	variable representation. The model is extended to include random effects which allow simple incorporation of
	spline and other modelling structures. Several variational approximations to the resulting set of models are
	presented, including a novel approach based on the inverse covariance matrix rather than the covariance matrix
	of the approximate posterior density for the random effects. This parameterisation improves upon the
	computational cost and numerical stability of previous methods. We demonstrate these approximations on
	simulated and real data sets.
\end{abstract}
 
\maketitle

\noindent Keywords: Approximate Bayesian inference ; mixed model ; Markov chain Monte Carlo ; Stan ; penalized splines.

\joc{
	Comments: 
	\begin{itemize}
		\item Need to organize in terms of a flow of ideas. What are we approximating?
		      		      		      
		\item I believe that we are using a semiparametric mean field variational Bayes approach discussed by Rohde \& Wand (2015).
		      However, I am not sure that we are using the their formalisms. (see page 3-6 of Rohde and Wand 2015).
	\end{itemize}	
}

\section{Introduction}
\label{sec:introduction}

\mgc{This section is too short}

Count data with a large number of zero counts arises in many areas of application, such as data arising from
physical activity studies, insurance claims, hospital visits or defects in manufacturing processes. Zero
inflation is a frequent cause of overdispersion in Poisson data, and not accounting for the extra zeroes
may lead to biased parameter estimates. These models have been used for many applications, including defects
in manufacturing in \citep{lambert1992}, horticulture in \citep{BIOM:BIOM1030} and \citep{Hall2000}, length of
stay data from hospital admissions in \citep{BIMJ:BIMJ200390024}, psychology in \citep{JOFP:rethink},
pharmaceutical studies in \citep{Min01042005}, traffic accidents on roadways in \citep{Shankar1997829} and
longitudinal studies in \citep{LeeWangScottYauMcLachlan2006}.

The strength of this approach derives from modelling the zero and non-zero count data seperately as a mixture
of distributions for the zero and non-zero components, allowing analysis of both the proportion of zeroes in
the data set and the conditions for the transition from zero observations to non-zero observations. When
combined with a multivariate mixed model regression framework, an extremely rich class of models can be fit
allowing a broad range of applications to be addressed. Often the transition from zero to non-zero has a
direct interpretation in the area of application, and is interesting in its' own right.

In this chapter, we build upon the earlier work on Bayesian zero-inflated models of \citep{Ghosh2006} and
\citep{Vatsa2014}. While simple forms of these models are easy to fit with standard maximum likelihood
techniques, more general models incorporating random effects, splines and missing data typically have no
closed form solutions and hence present a greater computational challenge to fit.

Fitting these models is typically done with Monte Carlo Markov Chain techniques, but these can be slow and
prone to convergence problems. We build upon a latent variable representation of these models to allow a 
tractable semiparametric mean field Variational Bayes approximation to be derived. Semiparametric mean field 
Variational Bayes is an approximate Bayesian inference method as detailed in \citep{Ormerod2010} and
\citep{RohdeWand2015}, which allows us to fit close approximations to these models using a deterministic 
algorithm which converges much more quickly.

We allow a flexible regression modelling approach by using a Gaussian Variational Approximation as defined in
\citep{Ormerod2012} on the regression parameters to allow a non-conjugate Gaussian prior to be used, making 
the resulting Gaussian posterior of the regression parameters easy to interpret. We adopt a Mean Field 
Variational Bayes (VB) approach on the other parameters in the model to derive the rest of the approximation.

The focus of this chapter is on developing a method of fitting flexible ZIP regression models accurately, and
showing the advantages of our method to previously presented methods. In Section \ref{sec:model} we define our
model and provide a framework for our approach incorporating regression modelling and random effects. In
Section \ref{sec:vb} we briefly introduce Variational Bayes methodogy. In Section \ref{sec:gaussian} we
focus on fitting the Gaussian part of our model. In Section \ref{sec:param}, we present new parameterisations for
use in these algorithms which offers substantial advantages in accuracy, numerical stability and computational
speed. In Section \ref{sec:results} we perform numerical experiments on simulated data which show how our
approach offers computational advantages over existing approaches. In Section \ref{sec:application} we show an
application of our method to a multi-level longitudinal study of pest control in apartments. Finally, in
Section \ref{sec:discussion} we conclude with a discussion of the results. An appendix contains details of the
derivation of the variational lower bound for our model.

\section{Model}
\label{sec:model}

In this section we present a Bayesian zero-inflated Poisson model for count data with extra zeroes. After
introducing the latent variable representation of Bayesian zero-inflated models, we incorporate regression
modelling and extend the model to a more flexible mixed model approach.

\subsection{Definitions}

Let $p$ be the dimension of the space of fixed effects, $m$ be the number of individuals in the random effects
and $b$ be the block size for each of those individuals.
We use $\vone_p$ and $\vzero_p$ to denote the $p \times 1$ column vector with all entries equal to 1 or 0,
respectively.

$\vy$ by the $n \times 1$ vector. The norm of a column vector $\vv$, defined to be $\sqrt{\vv^\top \vv}$, is 
denoted by $\|\vv\|$. For a $p \times 1$ vector $\va$, we let $\diag{(\va)}$ denote the $p \times p$.

$\vtheta$ is the parameter vector in $\R^p$,
$\mX$ is the design matrix of fixed effects with dimensions $n \times p$,
$\mZ$ is the design matrix of random effects with dimensions $n \times m b$
and $\mC = [ \mX \mZ ]$.

$\vbeta$ is the $p \times 1$ column vector representing the fixed effects,
$\vu$ be the $m b \times 1$ column vector representing the random effects
and $\vnu = [\vbeta^\top, \vu^\top]$.

$\mSigma$, $\mPsi$ and $\mLambda$ are $(p + m b) \times (p + m b)$ matrices.

$\vp$ is the $n \times 1$ column vector of probabilities that each observation in $\vy$ is non-zero.

$\expit(x)$ denotes the function $\tfrac{1}{1 + \exp(-x)}$, the inverse of the logit function.

$p(x)$ is the full probability distribution, $q(x)$ the approximating probability distribution.
and $q^*(x)$ the optimal approximating probability distribution.

$\text{Bernoulli}(\pi)$ denotes the probability distribution $\pi^k (1 - \pi)^{1-k}$
and $\text{Inverse Wishart}(\mPsi, v)$ denotes the probability distribution
$\tfrac{|\mPsi|^\frac{v}{2}}{2^{\frac{vp}{2}} \Gamma_p{(\tfrac{v}{2})}} |\mX|^{-\tfrac{v + p + 1}{2}} \exp{[-\half \tr{(\mPsi \mX^{-1})}]}$ where $\Gamma_p{(x)}$ denotes the multivariate gamma function and
$\tr$ is the trace function.

\subsection{Modelling zero-inflated Poisson data}

\mgc{Re-write this paragraph}

We consider a sample of counts $y_i$, $1 \le i\le n$, where there are excessive number of zeros for a Poisson
model, but the sample is otherwise well--modelled by a Poisson distribution. There are two main
parameterizations for modelling such data. The first approach models the probability of a zero by $\rho$ and
adjusts for counts greater than zero. This model uses the probability distribution
$$
P(Y_i = 0) = \rho, 
\quad \mbox{and} \quad 
P(Y_j = y_i) = \left( \frac{1 - \rho}{1 - e^{-\lambda}} \right) \frac{\lambda^{y_i} e^{-\lambda}} {y_i!},\qquad y_i \ge 1,
$$

% \noindent where $\lambda>0$ models the size of the counts. Here, $\bE(y_i) = ???$ 
% and $\mbox{Var}(y_i) = ???$.

A second approach using latent variables views the data as the product of two data--generating processes, a
Bernoulli process that determines whether the data is zero, and a second process where data is generated from
a Poisson distribution if it is non-zero. This leads to the model
\begin{equation}\label{eq:modelTwo}
	P(Y_i = 0) = \rho + (1 - \rho) e^{-\lambda}, 
	\quad \mbox{and} \quad 
	P(Y_i = y_i) = (1 - \rho) \frac{\lambda^{y_i} e^{-\lambda}} {y_i!},\qquad y_i \ge 1,
\end{equation}

\noindent where $\bE(y_i) = (1 - \rho)\lambda$ 
and $\mbox{Var}(y_i) =  \lambda(1 - \rho)(1 + \rho\lambda)$.

An auxiliary variable representation of this parameterization introduces the auxiliary variables $r_i$ which
equal $1$ when $y_i>0$ and $0$ otherwise. This leads to the specification
$$
P(Y_i=y_i|r_i) = \frac{\exp(-\lambda r_i)(\lambda r_i)^{y_i}}{y_i!} \quad \mbox{and} \quad r_i \sim \mbox{Bernoulli}(1-\rho)
$$

\noindent Marginalising the above likelihood over $r_i$ leads to the likelihood in (\ref{eq:modelTwo}).

We can extend the model naturally to multiple covariates by using a log-link and replacing $\lambda$ with
$\vx_i^\top \vbeta$ where $\vx_i,\vbeta\in\bR^p$ with $\vx_i$ being a vector of observed predictors and
$\vbeta$ is  a vector of regression coefficients. Let $\vr = (r_1,\ldots,r_n)$ 
\begin{equation}%\label{eq:main}
	\begin{array}{rl}
		\log p(\vy|\vr, \vbeta) 
		    & = \vy^\top \mR (\mX\vbeta)                           
		- \vr^\top \exp{(\mX\vbeta)} 
		- \vone^\top \log{\Gamma{(\vy + \vone)}}, \quad \mbox{ and }\\ [1ex]
		r_i & \sim \text{Bernoulli}(1-\rho), \quad 1 \leq i \leq n \\
	\end{array}
\end{equation} 

\noindent where $\mX$ is the $n\times p$ matrix whose $i$th rows equal $\vx_i$.

\subsection{Mixed model architecture}

Typically zero-inflated models arise in applications where we wish to build multivariate regression models. To
be able to construct multivariate models with as much generality as possible, we specify the full model as a
General Design Bayesian Generalized Linear Mixed Model, as in \citep{zhao06}.

The $j$th predictor/response pair for the $i$th group is denoted by $(\vx_{ij}, \vy_{ij}), 1 \leq j \leq n_i, 1 \leq i \leq m$, where $\vx_{ij} \in \R$, and the $\vy_{ij}$ are nonnegative integers.

For each $1 \leq i \leq m$, define the $n_i \times 1$ vectors $\vy_{ij} = [\vy_{i 1}, \ldots, \vy_{i
	n_i}]^\top$ and let $\vone$ define a vector of ones of appropriate length, where the first of these vectors is
the response. It is reasonable to assume that the vectors $\vy_1, \ldots, \vy_m$ are independent of each
other.

\joc{The next paragraph is too abrupt. How did this model come about? Where is the Poisson component? How is this combined with the zero-inflation?
Start with a single observation before using matrix/vector notation.}

The above approach can be extended to a flexible mixed model approach by incorporating the zero indicator
variable $\vr$ into a regression model likelihood -- in this case a Poisson mixed model. This allows us to
incorporate within-subject correlation and smoothing splines, as in \citep{Wand2008}, into our models. When
the indicator $\vr_i = 0$, the likelihood is $1$ for $\vy_i = 0$ and $0$ for all $\vy_i > 0$, and when the
indicator is $\vr_i = 1$, the likelihood is a Poisson regression likelihood for $\vy_i$. $\vr_i$ is a
Bernoulli indicator with probability $\rho$, allowing a proportion of zeroes in the observed data to be
specified.

The log--likelihood for one observation is then
\begin{equation*}
	\begin{array}{rl}
		\log p(y_i | \vx, \vr, \vz) & = y_i r_i (\vx_i^\top \vbeta + \vz_i^\top \vu) - r_i \exp (\vx_i^\top \vbeta + \vz_i^\top \vu) - \log \Gamma (y_i + 1), \\
		r_i                         & \sim \text{Bernoulli}(\rho), 1 \leq i \leq n, \text{ and }                                                              \\
		\rho                        & \sim \mbox{Uniform}(0, 1).                                                                                              \\
	\end{array}
\end{equation*}

\noindent We now extend this to multiple observations. Let $\mR = \diag{(\vr)}$, $\mC = [\mX, \mZ]$ and $\vnu = [\vbeta^\top, \vu^\top]^\top$. Consider the model

\begin{equation}\label{eq:main}
	\begin{array}{rl}
		\log{p(\vy|\vr, \vbeta, \vu)} & = \vy^\top \mR (\mC\vnu) - \vr^\top \exp{(\mC\vnu)} - \vone^\top \log{\Gamma{(\vy + \vone)}}, \quad \mbox{ and } \\ [1ex]
		r_i                           & \sim \text{Bernoulli}(\rho), 1 \leq i \leq n                                                                     \\
	\end{array}
\end{equation}

\noindent \joc{(The prior structure will depend on the structure of the random effects model)}
with priors
\begin{align*}
	\log{p(\mSigma_{\vu \vu})} & = \text{Inverse Wishart}(\mPsi, v),    \\
	\rho                       & \sim \mbox{Uniform}(0, 1),             \\
	\vbeta|\sigma^2_\vbeta     & \sim \N_p(\vzero, \sigma^2_\vbeta \mI) \\
	\mbox{ and } \vu|\mG       & \sim \N_{mb}(\vzero, \mG)              
\end{align*}

\noindent where $\mX$ is $n \times p$, $\mZ$ is $n \times mb$ and $\mSigma_{\vu \vu}$ is $mb \times mb$ and
$\mPsi$ is $b \times b$. The covariance of $\Cov(\vu) \equiv \blockdiag_{1 \leq i \leq m} (\mSigma) \equiv
\mI_m \otimes \mSigma$.

In the random intercept case, $\mSigma = \sigma_u^2 \mI$ while in the random slopes case
$$\mSigma = 
\begin{pmatrix}
	\sigma_{\vu_1}^2                                 & \rho_{\vu_1 \vu_2} \sigma_{\vu_1} \sigma_{\vu_2} \\
	\rho_{\vu_1 \vu_2} \sigma_{\vu_1} \sigma_{\vu_2} & \sigma_{\vu_2}^2                                 
\end{pmatrix}
$$
where $\sigma_{\vu_1}^2$ is the variance of the random intercepts, $\sigma_{\vu_2}^2$ is the variance of the
random slopes and $\rho_{\vu_1 \vu_2}$ is the correlation between the random intercepts and random slopes.

% \joc{Shouldn't we specify the structure of $\mSigma_{\vu \vu})$ later
% which is different for the random intercept, slope and spline cases?}
% \joc{(Perhaps it is wroth spelling out all of the various random effects structures that we will be using. Consider templating from Zhao \etal (2006).))}
\mgc{Need more detail, ala Zhao paper page 38, Example 1 and Example 2. Spline case still to go.}

In the spline case, we use the cubic spline basis
$1$, $x$, $x^3$, $(x - \kappa_1)^3_+$, \ldots, $(x - \kappa_K)^3_+$, where $K$ is the number of knots.
$\mSigma$ is a $K + 2$ banded matrix, where $K$ is the number of knots. Banded matrices
are highly sparse, and matrix operations can be performed on them in $\BigO(n)$ time.

The matrix $\mSigma$ is symmetric, with contents
\[
	\mSigma =
	\begin{pmatrix}
	\sigma^2_{\text{intercept}} & \ldots & & & & \text{symmetric} \\
		\rho_{\text{intercept} x} & \sigma^2_x & \ldots\\
		\rho_{\text{intercept} x^2} & \rho_{x x^2} & \sigma^2_{x^2} & \ldots \\
		\rho_{\text{intercept} x^3} & \rho_{x x^3} & \rho_{x^2 x^3} & \sigma^2_{x^3} & \ldots \\
		0 & \rho_{x (x - \kappa_1)^3_+} & \rho_{x^2 (x-\kappa_1)_+^3} & \rho_{x^3 (x - \kappa_1)_+^3} & \sigma^2_{(x - \kappa_1)_+^3} & \ldots \\
		0 & 0 & \rho_{x^2 (x-\kappa_2)_+^3} & \rho_{x^3 (x-\kappa_2)_+^3} & \rho_{(x-\kappa_1)_+^3 (x-\kappa_2)_+^3} &\sigma^2_{(x - \kappa_2)_+^3} \\
		0 & 0 & 0 & \rho_{x^3 (x - \kappa_3)_+^3} \\
	\end{pmatrix}
\]



\section{Variational Bayes}
\label{sec:vb}

% General Design Bayesian Generalized Linear Mixed Model, as in \citep{zhao06}. This allows us to incorporate
% within-subject correlation, and smoothing splines (as in \citep{Wand2008}) in our models.


% Idea: We can use an approximation of the from q(\beta, \u, \Sigma) q(\rho) \Product q(r_i)
% and use GVA on q(\beta, \u, \Sigma) and mean field updates on \rho and r_i

\subsection{Semiparametric Mean Field Variational Bayes}

\subsubsection{Definitions}

The density function of a random vector $\vu$ is denoted by $p(\vu)$.  The conditional density function of a
random vector $\vu$ given $\vv$ is denoted by $p(\vu|\vv)$. Consider a generic Bayesian model with parameter
vector $\vtheta \in \Theta$. Throughout this section we assume that $\vy$ and $\vtheta$ are continuous random
vectors. The KL divergence between the probability distributions $p$ and $q$ is defined as
$$
\KL(q || p) \equiv \int q(\vtheta) \log \left \{ \frac{q(\vtheta)}{p(\vtheta | \vy)} \right \} d \vtheta.
$$

\noindent The approximation is fit by iteratively minimising the Kullback-Leibler divergence between the true
posterior and an approximating distribution. Thus the optimal approximation $q^*(\vtheta)$ is
$$
q^*(\vtheta) = \argmin_{\vxi \in \Xi} \text{KL} \{ {q(\vtheta|\vxi) || p(\vtheta|\vy)} \}.
$$

\noindent A common approach is to assume an approximation in a factorised form
$$q(\vtheta) = \Pi_{i=1}^M q(\theta_i).$$

\noindent The variational lower bound is maximised iteratively. On each iteration, the value of each parameter
in the model is calculated as the expectation of the full likelihood relative to the other parameters in the
model, which is referred to as the mean field update:
$$q_i^*(\theta_i) \propto \exp{\{ \bE_{-q(\theta_i)} \log p(\vy, \vtheta) \}}$$

\noindent This is done for each parameter in the model in turn until the variational lower bound's increase is
negligible and convergence is achieved.

This approach works well for classes of models where all of the parameters are conjugate. For more
general classes of models, mean field updates are not analytically tractable and general gradient-based
optimisation methods must be used, as for the Gaussian Variational Approximation (see \citep{ormerod09}) used
in this paper. These methods are generally difficult to apply in practice, as the problems can involve the
optimisation of many parameters over high-dimensional, constrained spaces whose constraints cannot be simply
expressed.

Much of Bayesian inference is based on the posterior distribution of a model's parameters given observed data 
defined by $p(\vtheta|\vy) = p(\vy|\vtheta)p(\vtheta)/p(\vy)$ where $\vy$ is a vector of observed data,
$\vtheta$ are the model parameters $p(\vy|\vtheta)$ is the model distribution, $p(\vtheta)$ is a prior 
distribution on $\vtheta$ and $p(\vy)=\int p(\vy|\vtheta)p(\vtheta)d\vtheta$. Here the integral is performed
over the domain of $\vtheta$. If a subset of $\vtheta$ are discrete random variables then the integral over
these parameters is replaced with a combinatorial sum over all possible values of these discrete random 
variables.

\noindent The calculation of the posterior distribution is usually either computationally intractable or no
closed from exists for the posterior distribution and approximation is required. Variational approximation is
a  class of methods for dealing with this problem by transforming the problem into an optimization problem.
Most, but not all variational approximations are based on minimising the Kullback-Leibler divergence between
the true posterior $p(\vtheta|\vy)$ and an approximating distribution $q(\vtheta)$, sometimes called a
$q$-density. Suppose that $q(\vtheta)$ is parameterised by $\vxi$ and write $q(\vtheta)\equiv
q(\vtheta;\vxi)$. We attempt to solve
$$
\ds q^*(\vtheta) = \argmin_{\vxi \in \Xi} \text{KL} \{ {q(\vtheta;\vxi) || p(\vtheta|\vy)} \}.
$$

\noindent Noting that the Kullback-Leibler divergence between two distributions is zero if and only if
the two distributions are equal almost everywhere the above problem is no easier than the original problem
of finding $p(\vtheta|\vy)$. This problem is mitigated by imposing structure on the form of $q$. Two
strategies for doing this are:
\begin{enumerate}
	\item[(A)] Specifying the parametric form of $q$; or 
	\item[(B)] choosing $q$ to be of the form $q(\vtheta) = \prod_{i=1}^M q(\theta_i)$,
\end{enumerate}

\noindent where it can be shown (see Ormerod \& Wand, 2010, for example) that the optimal form of the
$q_i$'s are of the form
\begin{equation}\label{eq:consistency}
	q_i^*(\theta_i) \propto \exp{\{ \bE_{-q(\theta_i)} \log p(\vy, \vtheta) \}},  \quad 1\le i\le M,
\end{equation}

\noindent where the above set of equations, sometimes called consistency conditions, need to be 
satisfied simultaneously. It can be shown that by calculating $q_i^*(\theta_i)$ for a particular
$i$ with the remaining $q_j^*(\theta_j)$, $j\ne i$ fixed, results in a monotonic decrease in the 
Kullback-Leibler divergence. Note that a combination of (A) and (B) can be used which has recently
been formalized by Rohde \& Wand (2015). 

\noindent It can easily be shown that
$$
\ds \log p(\vy) = \int q(\vtheta;\vxi)\frac{p(\vy|\vtheta)p(\vtheta)}{q(\vtheta;\vxi)} d\vtheta + \text{KL}(q(\vtheta;\vxi)||p(\vtheta|\vy)).
$$

\noindent As the Kullback-Leibler divergence is strictly positive, the first term on the right hand side
is a lower bound on the marginal log-likelihood which we will define by
$$
\ds \log \underline{p}(\vy;\vxi) \equiv \int q(\vtheta;\vxi)\frac{p(\vy|\vtheta)p(\vtheta)}{q(\vtheta;\vxi)} d\vtheta
$$

\noindent and maximizing $\log \underline{p}(\vy;\vxi)$ with respect to $\vxi$ is eqivalent to minimizing
$\text{KL}(q(\vtheta;\vxi)||p(\vtheta|\vy))$.

%This approach works well for classes of models where all of the parameters are conjugate. For more general
%classes of models, mean field updates are not analytically tractable and general gradient-based optimisation
%methods must be used, as for the Gaussian Variational Approximation (see \citep{ormerod09}) used in this paper.
%These methods are generally difficult to apply in practice, as the problems can involve the optimisation of
%many parameters over high-dimensional, constrained spaces whose constraints cannot be simply expressed.

\subsection{Approximation}

Our variational approximation for the model is of the form 
$$
q(\vr_0, \vnu, \sigma_{\vu}^2, \rho) = q(\vnu)q(\rho) q(\vr_0)q(\mSigma_{\vu \vu})  \\
$$

\noindent 
where we define $\vr_0 = \{ r_i : y_i = 0 \}$,
$q(\sigma_{\vu}^2) = \mbox{Inverse Wishart}\left(\mPsi + \sum_{i=1}^m (\vmu_i \vmu_i^\top + \mLambda_{\vu_i \vu_i}), v + m \right)$ \mbox{and } $q(r_i) = \Bernoulli{(p_i)}$ with
$$
p_i = \expit\left[ \psi{(\alpha_{q(\rho)})} - \psi{(\beta_{q(\rho)})} - \exp{(c_i^\top\vmu + \half c_i^\top \mLambda c_i)} \right]
$$

\noindent 
\text{when} $\vy_i = 0$.

%$\propto \exp{\left \{-r_i \bE_{-r_i} [\exp{(c_i^\top\vnu)}] + r_i [\psi(\alpha_\rho) - \psi(\beta_\rho)] \right \} }.\\$

\noindent The optimal approximation \joc{(reword: the ``optimal approximation'' might be called the true posterior)} for $\vr$ is
$$
\begin{array}{rl}
	q(\vr) & \propto \exp \left [ \bE_{-q(\vr)}\vy^\top\mR(\mC\vmu) - \vr^\top\exp{(\mC\vnu)}-\half \vnu^\top \mSigma_{\vu \vu} \vnu \right ]                                                  \\ [1ex]
	       & = \exp{ \left\{ \vy^\top\mR\mC \vmu - \vr^\top \exp{[\mC \vmu + \half \text{diag}(\mC \mLambda \mC^\top)]} - \half \vmu^\top \mD \vmu - \half \text{tr}(\mLambda \mD ) \right\} } 
\end{array}
$$

\noindent where $\mD = \left[ (\mPsi + \sum_{i=1}^m \vmu_i \vmu_i^\top + \mLambda_{\vu_i\vu_i}) / (v - p - 1) \right]^{-1}$. 

This is close in form to a Poisson regression model. Poisson regression models with normal priors
have no closed form for their mean field updates due to non- conjugacy, but can be fit using Gaussian
Variational Approximation, as in \citep{ormerod09}. The model can be fit using Algorithm
\ref{alg:algorithm_one} below.

\begin{algorithm}
	\caption[Algorithm 1]{Iterative scheme for obtaining the parameters in the
		optimal densities $q^*(\vmu, \mLambda)$, $q^*(\mSigma_{\vu \vu})$ and $q^*(\rho)$}
	\label{alg:algorithm_one}
	\begin{algorithmic}
		\REQUIRE{$\alpha_{q(\rho)} \leftarrow \alpha_\rho + \vone^\top\vp, p_{q(\mSigma_{\vu \vu})} \leftarrow p + 1$} \\[1ex]
		\WHILE{the increase in $\log{\underline{p}}(\vy;q)$ is significant}
		% \vmu, \mLambda
		\STATE Optimise $\vmu$ and $\mLambda$ using $\vy, \mC, \vp$ and $\mSigma_{\vu \vu}$ \\[1ex]
		% \vp
		\STATE $\beta_{q(\rho)} \leftarrow \beta_\rho + n - \vone^\top\vp$ \\[1ex]
		\STATE $\eta \leftarrow -\exp \left [ \mC \vmu + \half \diag{(\mC\mLambda\mC^\top)} \right ] + \psi{(\alpha_{q{(\rho)}})} - \psi{(\beta_{q{(\rho)}})}$ \\[1ex]
			\STATE $\vp_{q(\vr_0)} \leftarrow \expit{(\eta)}$ \\[1ex]
			% \mSigma_{\vu \vu}
			\STATE $\mPsi_{q(\mSigma_{\vu \vu})} \leftarrow \Psi + \sum_{i=1}^m (\vmu_i \vmu_i^\top + \mLambda_{{\vu}_i {\vu}_i})$ \\[1ex]
			\STATE $\mSigma_{\vu\vu} \leftarrow [\mPsi_{q(\mSigma_{\vu \vu})}/(v - d - 1)]^{-1}$
			\ENDWHILE
		\end{algorithmic}
	\end{algorithm}
			
	\section{Optimising the Gaussian Part of the Model}
	\label{sec:gaussian}
		
	In this section, we compare the accuracy, stability and speed of a number of algorithms for fitting the
	Gaussian component of our model, $q(\vmu, \mLambda)$ in Algorithm \ref{alg:algorithm_one}.  We compare four
	approaches for accuracy, computational complexity and stability.
	
	\subsection{Laplace-Variational Approximation}
		
	Laplace's method of approximation uses the second order Taylor expansion of the full log likelihood around
	the mode to find a Gaussian approximation to the full posterior.
	
	% The algorithm is very quick to execute, but the resulting approximate posterior
	% distributions are not as accurate as those produced by the other algorithms considered in this article.
		
	% NR
	% Detail the function and its derivatives
		
	Taylor expanding the full log likelihood once around the mode yields the following approximation to 
	the	log-likelihood:
	\begin{align*}
		\log \underline{p}(\vmu, \mLambda; \vy) = \vy^\top\mP\mC\vmu - \vp^\top\exp \left (\mC \vmu \right ) - \half \vmu^\top \mSigma^{-1} \vmu. 
	\end{align*}
		
	\noindent This can be iteratively optimised with respect to $\vmu$ and $\mLambda$ using a Newton-Raphson style
	algorithm, with the derivatives for $\vmu$ and $\mLambda$ given below.
	\mgc{Are the derivatives better placed in the appendices?}
	\begin{align*}
		\frac{\partial \log p(\vmu, \mLambda; \vy)}{\partial \vmu}     & \approx \mP \mC (\vy - \exp{(\mC \vmu)}) - \mSigma^{-1} \vmu \text{ and} \\
		\frac{\partial \log p(\vmu, \mLambda; \vy)}{\partial \mLambda} & \approx - \mC^\top \text{diag}(\vp e^{(\mC \vmu)}) \mC - \mSigma^{-1}.   
	\end{align*}
		
	\noindent The steps of the algorithm are shown in Algorithm \ref{alg:laplace_alg}.
		
	\begin{algorithm}
		\caption{Laplace scheme for optimising $\log \underline{p}(\vmu, \mLambda; \vy)$}
		\label{alg:laplace_alg}
		\begin{algorithmic}
			\REQUIRE $\frac{\partial \log p(\vmu, \mLambda; \vy)}{\partial \mLambda} \approx - \mC^\top \text{diag}(\vp e^{(\mC \vmu)}) \mC - \mSigma^{-1}$.
			% Fit \vmu, \mLambda using Laplace approximation
			\WHILE{the increase in $\log \underline{p}(\vmu, \mLambda; \vy)$ is significant}
			% \vmu, \mLambda
			\STATE $\mLambda \leftarrow \left [\mP \mC^\top \text{diag}(\exp{(\mC \vmu)}) \mC + \mSigma^{-1} \right ]^{-1}$ \\ [1ex] 
			\STATE $\vmu \leftarrow \vmu + \mLambda \left [ \frac{\partial \log p(\vmu, \mLambda; \vy)}{\partial \vmu} \right ]$ \\ [1ex]
			\ENDWHILE
		\end{algorithmic}
	\end{algorithm}
		
	\subsection{Gaussian Variational Approximation}
		
	% Detail techniques used for fitting models.
		
	The full variational likelihood for Generalised Linear Mixed models is computationally difficult to compute,
	requiring the evaluating of a high dimensional integral. However, \citep{ormerod09} devised an accurate
	approximation to the full variational likelihood, the Gaussian Variational Lower Bound, which only requires the 
	evaluation of a substantially simpler univariate function.
	
	To optimise the Gaussian component of the model in each iteration of \ref{alg:algorithm_one}, the Gaussian variational lower bound must to be optimised over the set of $\vmu$ and $\mLambda$
	values, and $\mLambda$ in particular is both of high dimension $(p + mb)^2$ and constrained to be
	semi-positive definite. Thus care must be taken in the parameterisation in order to optimise
	$\mLambda$ effectively.
	
	\subsubsection{Covariance parameterisation}

	Assuming $q(\vnu)$ is a multivariate normal distribution $\N(\vmu, \mLambda)$, the Gaussian variational lower
	bound in Algorithm \ref{alg:algorithm_one} can be optimised using a variety of algorithms. The variational
	lower bound is not necessarily unimodal if $\vp$ and $\mLambda$ are free to vary, leading to potential
	difficulty in optimising to the global maximum. However, for fixed $\vp$ and $\mSigma$, the variational
	lower bound is log-concave with respect to $\vmu$ and $\mLambda$, and so standard optimisation methods such
	as L-BFGS-B as described in, for example, \citep{Liu1989}, work well. This leads to an extremely accurate
	approximation of the true posterior at the expense of some additional computational effort.
		
	We fit the Gaussian part of our approximation $q(\vbeta, \vu) \sim \N(\vmu, \mLambda)$ by maximising 
	the variational lower bound	
	\begin{align*}
		\log \underline{p}(\vmu, \mLambda; \vy) & = \quad \vy^\top\mP \mC \vmu - \vp^\top \exp[\mC \vmu + \half \text{diag}(\mC \mLambda \mC^\top)] - \half \vmu^\top \mSigma^{-1} \vmu - \half \tr{(\mSigma^{-1} \mLambda)} + \log{|\mR|} \\
		                                        & \quad - \tfrac{p}{2} \log{(2 \pi)} + \half \log{|\mSigma^{-1}|} + \tfrac{p}{2} \log{(2 \pi)} + \tfrac{p}{2}                                                                              
	\end{align*}
	\noindent with respect to $\vmu, \mLambda$, keeping $\vp$, $\mSigma$, and $\rho$ fixed.
		
	The first variant of the Gaussian Variational Approximation algorithm that we present optimises the
	Gaussian variational lower bound of the log likelihood with respect to $\vmu$ and the Cholesky decomposition
	$\mR$ of   $\mLambda$, that is, $\mLambda = \mR \mR^\top$. The resulting function
	% This algorithm trades the computational complexity of
	% numerically evaluating an integral for greatly increased accuracy in the approximating posterior
	% distribution. 
	\begin{align*}
		\log \underline{p}(\vmu, \mLambda; \vy) & = \quad \vy^\top\mP \mC \vmu - \vp^\top \exp[\mC \vmu + \half \text{diag}(\mC \mLambda \mC^\top)] - \half \vmu^\top \mSigma^{-1} \vmu - \half \tr{(\mSigma^{-1} \mLambda)} + \log{|\mR|} \\
		                                        & \quad - \tfrac{p}{2} \log{(2 \pi)} + \half \log{|\mSigma^{-1}|} + \tfrac{p}{2} \log{(2 \pi)} + \tfrac{p}{2}                                                                              
	\end{align*}
	can be optimised with L-BFGS-B using the derivatives
	\begin{align*}
		\frac{\partial \log \underline{p}(\vmu, \mLambda; \vy)}{\partial \vmu}     & = \mP \mC (\vy - \mC^\top \exp(\mC \vmu + \half \text{diag}{(\mC \mLambda \mC^\top)})) - \mSigma^{-1} \vmu \text{ and}                \\
		\frac{\partial \log \underline{p}(\vmu, \mLambda; \vy)}{\partial \mLambda} & = \left [\mLambda^{-1} - \mP \mC^\top \exp(\mC \vmu + \half \text{diag}(\mC \mLambda \mC^\top)) \mP \mC) - \mSigma^{-1} \right ] \mR. 
	\end{align*}

	\begin{figure}[p]
		\caption{\tiny Covariance matrix -- Fixed effects before random effects}
		\label{fig:covfixedrandom}
		\includegraphics[scale=.25]{mX_mZ_mLambda.pdf}
	\end{figure}
	
	\begin{figure}[p]
		\caption{\tiny Covariance matrix -- Random effects before fixed effects}
		\label{fig:covrandomfixed}
		\includegraphics[scale=.25]{mZ_mX_mLambda.pdf}
	\end{figure}
						      				      			      			      			      	
	\begin{figure}[p]
		\caption{\tiny Cholesky factor -- Fixed effects before random effects}
		\label{fig:cholfixedrandom}
		\includegraphics[scale=.25]{mX_mZ_cholesky.pdf}
	\end{figure}
	
	\begin{figure}[p]
		\caption{\tiny Cholesky factor -- Random effects before fixed effects}
		\label{fig:cholrandomfixed}
		\includegraphics[scale=.25]{mZ_mX_cholesky.pdf}
	\end{figure}
	
	\subsubsection{Precision parameterisation}
		
	\noindent The second variant of the Gaussian Variational Approximation algorithm is similiar to the first, but
	instead of optimising the Gaussian variational lower bound with respect to $\vmu$ and the Cholesky factor
	$\mR$ of $\mLambda$, we instead optimise the Cholesky factor of the inverse of $\mLambda$ i.e. $\mLambda =
	(\mR \mR^\top)^{-1}$.

	The Gaussian variational lower bound in this parameterisation is
	\begin{align*}
		\log \underline{p}(\vmu, \mLambda; \vy) & = \quad \vy\mP\mC \vmu - \vp^\top \exp(\mC \vmu + \half \text{diag}(\mC \mLambda \mC^\top)) - \half \vmu^\top \mSigma^{-1} \vmu - \half \tr{(\mSigma^{-1} \mLambda)} \\
		                                        & \quad- \tfrac{p}{2} \log{(2 \pi)} + \half \log{|\mSigma^{-1}|} + \tfrac{p}{2} \log{(2 \pi)} + \tfrac{p}{2} - \log{|\mR|}                                             
	\end{align*}
		
	\noindent The derivative with respect to $\vmu$ is the same as that in the first variant of the algorithm, but 
	as the parameterisation of $\mLambda$ has changed, the  derivative with respect to $\mLambda$ becomes
	\begin{align*}
		\frac{\partial \log \underline{p}(\vmu, \mLambda; \vy)}{\partial \mLambda}
		  & = \hphantom{-}(\mLambda^{-1} + \mH)(-\mLambda \mR \mLambda) \\
		  & = -(\mI + \mH\mLambda)\mR\mLambda                           \\
		  & = - (\mR\mLambda + \mH\mLambda\mR\mLambda)                  
	\end{align*} 
		
	\noindent where $\mH = (\mP \mC)^\top \text{diag}(\exp(\mC \vmu + \half \mC \mLambda \mC^\top)) \mP \mC - \mSigma^{-1}$.
		
	\subsubsection{GVA fixed point}
		
	% Fixed point update of \mLambda
		
	This variant of the algorithm uses Newton-Raphson-like fixed point updates on the Gaussian variational lower
	bound. We optimise the same variational lower bound as in the covariance parameterisation above, using the
	derivatives below. The steps are presented in Algorithm \ref{alg:algorithm_nr} where	
	\begin{align*}
		\frac{\partial \log \underline{p}(\vmu, \mLambda; \vy)}{\partial \vmu}     & = \quad \mC^\top\vp \left [\vy - \mC\exp(\mC \vmu + \half \text{diag}(\mC \mLambda \mC^\top)) \right ] - \mSigma^{-1} \vmu \text{ and} \\
		\frac{\partial \log \underline{p}(\vmu, \mLambda; \vy)}{\partial \mLambda} & = -\mC^\top \text{diag}(\vp^\top \exp(\mC \vmu +\half \text{diag}(\mC \mLambda \mC^\top))) - \mSigma^{-1}.                             
	\end{align*}
	As this algorithm involves a simple Newton-Raphson style update step, it is computationally simple to
	implement, but potentially unstable as there is no adaptation of step size, as in L-BFGS-B.

	\begin{algorithm}
		\caption[Algorithm GVA NR]{Iterative scheme for obtaining optimal $\vmu$ and $\mLambda$
			given $\vy$, $\mC$ and $\vp$}
		\label{alg:algorithm_nr}
		\begin{algorithmic}
			\REQUIRE $\frac{\partial \log \underline{p}(\vmu, \mLambda; \vy)}{\partial \vmu} = \mP \mC (\vy - \mC^\top \exp(\mC \vmu + \half \text{diag}{(\mC \mLambda \mC^\top)})) - \mSigma^{-1} \vmu$.
			% Fit \vmu, \mLambda using Laplace approximation
			\WHILE{the increase in $\log{\underline{p}}(\vmu, \mLambda; \vy)$ is significant}
			% \vmu, \mLambda
			\STATE $\mLambda \leftarrow \left [ \mP^\top \mC^\top \exp(\mC \vmu + \half \text{diag}(\mC \mLambda \mC^\top)) \mC \mP \right ]^{-1}$ \\ [1ex]
			\STATE $\vmu \leftarrow \vmu + \mLambda \left [ \frac{\partial \log \underline{p}(\vmu, \mLambda; \vy)}{\partial \vmu} \right ]$
			\ENDWHILE
		\end{algorithmic}
	\end{algorithm}
		
	% Splines
		
	\section{Parameterisations and computational cost of Gaussian Variational Approximation approaches}
	\label{sec:param}
	\subsection{Covariance parameterisation of $\mLambda$}

	To ensure symmetry of $\mLambda$, we parameterise the optimisation problem in terms of $\mLambda$'s
	Cholesky factor  $\mLambda = \mR^\top \mR$. We optimise over the space $(\vmu, \overline{\mR})$, where $\vmu
	\in \R^{p + m}b$ and $\overline{\mR}$ is a lower-triangular $(p + mb) \times (p + mb)$ matrix. Then
		
	\begin{equation*}
		\mR_{ij} =
		\begin{cases}
			\exp(\overline{\mR}_{ij}), & i = j             \\
			\overline{\mR}_{ij},       & i > j             \\
			0,                         & \text{otherwise}, 
		\end{cases}
	\end{equation*}
		
	\noindent exponentiating the diagonal to ensure positive-definiteness of $\mR$. We parameterise $\mLambda$
	as $\mLambda = \mR \mR^\top$ so that is is guaranteed to be symmetric, and we only have $p(p-1)/2$ 
	parameters to deal with instead of $p^2$ parameters, some of which are constrained. 

	This parameterisation can lead to numeric overflow when the diagonals of $\overline{\mR}$ become moderately
	large, which can lead to singular matrices when attempting to invert $\mLambda$. We dealt with this by
	defining a piecewise function which is exponential for arguments less than $t$, and quadratic for arguments
	greater than or equal to $t$
	$$
	f(r_{ij}) =
	\begin{cases}
	e^{r_{ij}}, r_{ij} < t \\
	a r_{ij}^2 + b r_{ij} + c, r_{ij} \geq t
	\end{cases}
	$$
	and then choosing the co-efficients $a$, $b$ and $c$ such that the function, first and second derivatives would
	agree at $r_{ij} = t$.

	To find the co-efficients $a$, $b$ and $c$, we solved the system of equations formed by repeatedly 
	differentiating the quadratic at $r_{ij} =  t$ and equating it with $e^t$
	$$
	\begin{array}{lllll}
	e^t &= &a t^2 &+ \quad b t &+ \quad c \\
	e^t &= &&\quad 2a t &+ \quad b \\
	e^t &= &&&\quad 2a \\
	\end{array}
	$$
	to obtain $a = e^t / 2$, $b = (1 - t) e^t$ and $c = [1 - t^2/2 - (1 - t) t] e^t$.

	We also addressed the overflow problem by working with the Cholesky factorisation of $\mLambda^{-1}$
	rather than $\mLambda$, allowing us to solve a system of equations rather than invert and multiply by a
	matrix, which is also faster and more numerically stable. We use knowledge of the regression  model we are
	fitting to specify a sparse matrix structure, greatly reducing the dimension of   the problem and thus
	improving both computational speed and numeric accuracy.

	% \noindent By noticing that the lower rows of the product depend on the higher rows of the Cholesky factor, we
	% observe that by re-ordering the fixed and random effects in $\mLambda$ so that the , we arrive at a covariance structure which is sparse in the first diagonal block. Thus the Cholesky factor of $\mLambda$ that we optimise over is as sparse as possible. This reduces the dimension of the optimisation problem we have to solve from
	% $\BigO(np^2)$ to $\BigO(np)$.
	
	Any symmetric matrix $\mLambda$ can be written as a product of its' Cholesky factors, $\mLambda =
	\mR \mR^\top$ where $\mR$ is lower triangular. $\mR$ is unique if $\mR_{ii} \geq 0$.
	
	\begin{align*}
		&\begin{pmatrix}
		\mR_{11}          & 0                                    & 0                                     \\
		\mR_{21}          & \mR_{22}                             & 0                                     \\
		\mR_{31}          & \mR_{32}                             & \mR_{33}                              
		\end{pmatrix}
		\begin{pmatrix}
		\mR_{11}          & \mR_{21}                             & \mR_{31}                              \\
		0                 & \mR_{22}                             & \mR_{32}                              \\
		0                 & 0                                    & \mR_{33}                              
		\end{pmatrix}
		\\
		=& \begin{pmatrix}
		\mR_{11}^2        &                                      & \text{symmetric}                      \\
		\mR_{21}\mR_{11} & \mR_{21}^2 + \mR_{22}^2 \\
		\mR_{31} \mR_{11} & \mR_{31}\mR_{21} + \mR_{32} \mR_{22} & \mR_{31}^2 + \mR_{32} ^2 + \mR_{33}^2 
		\end{pmatrix}.
	\end{align*}

	\noindent We exploit this structure. By interchanging the fixed and random effects in the design matrix $\mC = [\mX \mZ]$ to $\mC = [\mZ \mX]$, and re-ordering the dimensions of $\vmu, \mLambda$ and $\mSigma$ in the same manner, the independence between the
	blocks relating to the random effects in $\mZ$ induce sparsity in the Cholesky factor $\mR$ of
	$\mLambda^{-1}$, as can be seen in Figures \ref{fig:covfixedrandom}, \ref{fig:covrandomfixed},
	\ref{fig:cholfixedrandom} and \ref{fig:cholrandomfixed}. Thus the Gaussian $q(\vnu) \sim \N(\vmu, \mLambda)$ can be optimised over a space of dimension $\half p (p + 1) + pq + \half q (q + 1)$ rather than dimension
	$\half (p + mq) (p + mq + 1)$ as in the dense parameterisation. This leads to subtantial performance gains
	when $m$ is large, as is typically the case in problems of practical importance such as longitudinal or 
	clinical trials with many subjects or the application presented in Section \ref{sec:application}.
		
	By re-ordering the fixed and random effects in $\mLambda$, we end up with a covariance structure which is 
	sparse in the first diagonal block.

	\subsection{Precision parameterisation}

	We optimise over the space $(\vmu, \overline{\mR})$ as before, but now 
		
	\begin{equation*}
		\mR_{ij} =
		\begin{cases}
			\exp(-\overline{\mR}_{ij}), & i = j             \\
			\overline{\mR}_{ij},        & i > j             \\
			0,                          & \text{otherwise}, 
		\end{cases}
	\end{equation*}
	
	\noindent This new choice of parameterisation allows us to calculate $\half \text{diag}(\mC \mLambda
	\mC^\top)$ by solving the linear systems $\mR \va = \mC_{i}, i=1, \ldots, n$ for   $\va$ and then calculating
	$\va^\top\va$ where $\mC_{i} = $ the $i$th row of $\mC$, rather than calculating $\text{diag}(\mC \mLambda
	\mC^\top)$ directly.
	
	The implementation of these algorithms was not without its' challenges, chiefly numerical issues encountered during testing and verification of the accuracy of the model fitting. Using the exponential function to parameterise the main diagonal coupled with L-BFGS-B's unconstrained line search and \texttt{optim()}'s lack of robustness to infinities lead to many overflow problems which may have been lessened or dealt with entirely by using a function with a less aggressive growth rate, such as an appropriate piecewise quadratic.
	
	The main computational cost is the evaluation of the variational lower bound and its' derivatives. By virtue
	of their dimension, the expressions involving $\mLambda$ dominate the computational cost. The key term is
	$\frac{1}{2} \diag(\mC \mLambda \mC^\top)$. The naive way to calculate this is to ignore the symmetry in
	this expression and simply calculate the product $\mC \mLambda \mC^\top$, which takes $2 n \times (p + m
	b)^2$ floating point operations, and take the diagonal entries of the result. This is obviously wasteful, as
	all of the off--diagonal entries of the resulting product are immediately discarded.
	
	By parameterising $\mLambda$ in terms of its' Cholesky factors and realising that
	
	\[
		\mC \mLambda \mC^\top = \mC \mR \mR^\top \mC^\top
	\]
	
	\noindent and that
	
	\[
		\diag(\mC \mLambda \mC^\top)_{ii} = \mC_{i .} \mR \mR^\top \mC_{i .}^\top, 1 \leq i \leq n
	\]
	
	\noindent we can calculate the products $\mC_{i .} \mR, 1 \leq i \leq n$, using $n \times \frac{1}{2}(p + m
	b)(p + m b   + 1)$ floating point operations, and storing the results of the $i$th product in the $i$th
	element of the   vector $\va$ and then calculate $\diag(\mC \mLambda \mC^\top) = \va^\top \va$.
	
	Moreover, mixed models typically have sparse design matrices, allowing us to encode $\mR$ as a sparse matrix, and	further reduce   this depending on the model. For example, in the random intercept case, only the diagonals of the random effects block need to be non-zero, and hence the above expression can be calculated in
	$\BigO(n)$ floating point operations.
	
	% $\log |\mR|$ can be calculated using only $p + m b$ floating point operations, as $\mR$ is lower triangular.
	
	For the precision parameterisation, we observe that in this parameterisation
	\[
		\diag(\mC \mLambda \mC^\top)_{ii} = \mC_{i .} \mR^{-\top} \mR^{-1} \mC_{i .}^\top, 1 \leq i \leq n,
	\]
	
  \noindent and so by solving $\mR \va = \mC_{i .}^\top$ for $\va$ for all $i$ at a cost of $n \times
	\frac{1}{2} (p + m b) (p + m b + 1)$ floating point operations, and then calculating
	\[
		\diag(\mC \mLambda \mC^\top)_{ii} = \va^\top \va, 1 \leq i \leq n,
	\]
	
	\noindent we can then calculate $\diag(\mC \mLambda \mC^\top)$.
	
	As above, by using our knowledge of the model being fit we can encode $\mR$ sparsely to decrease
	the required   computation still further. In the random intercept model case, the computational cost will drop
	to   $n \times [m + \half{1}{2} p (p + 1) + p \times m b]$.
	
	\section{Numerical results}
	\label{sec:results}
		
	The accuracy of each model fitting algorithms presented in Section \ref{sec:gaussian} was assessed by
	comparing the approximating distribution of each parameter with the posterior distribution of Monte Carlo
	Markov Chain samples of that parameter. 1 million Monte Carlo Markov Chain samples were generated using Stan.
	The accuracy of examples of random intercept, random slope and spline models were evaluated using this method.
		
	\subsection{Simulated data}
		
	For each of these simulations, the model is as presented in Section \ref{sec:model}.
		
	Several common application scenarios were simulated and their accuracy evaluated. A random intercept model was simulated with $\vbeta = (2, 1)^\top$, $\rho = 0.5$, $m = 20$, $n_i = 10$ and $b = 1$. The results are
	presented in Table \ref{tab:accuracy_int}. A random slope model was simulated with $\vbeta = (2, 1)^\top$,
	$\rho = 0.5$, $m = 20$, $n_i = 10$ and $b = 2$. The results are presented in Table \ref{tab:accuracy_slope}.
	Spline model was fit to a data set generated from the function $3 + 3 \sin{(\pi x)}$ on the interval $[-1,
	1]$. The resulting model fits are presented in Figure \ref{fig:spline}.
		
	% The stability of the algorithms was confirmed by running them on 10,000 different data sets that were randomly
	% generated after having initialised the random number generator with different seeds.
		
	Median accuracy of the algorithms was assessed by running them on 100 randomly generated data sets. The	results are presented in Figure \ref{fig:median_accuracy_intercept} and Figure
	\ref{fig:median_accuracy_slope}.
		
	% Figure: Median accuracy graph intercept
	\begin{figure}
		\begin{center}
			\includegraphics[width=0.7\textwidth, height=100mm]{code/results/median_accuracy_combined_intercept.pdf}
			\caption{Median accuracy of random intercept}
			\label{fig:median_accuracy_intercept}
		\end{center}
	\end{figure}
		
	% Figure: Median accuracy graph slope
	\begin{figure}
		\caption{Median accuracy of slope}
		\label{fig:median_accuracy_slope}
		\includegraphics[width=120mm, height=120mm]{code/results/median_accuracy_combined_slope.pdf}
	\end{figure}
		
	% Table of accuracy results - intercept model
	\begin{table}
		\caption{Table of accuracy - Random intercept model}
		\label{tab:accuracy_int}
		\begin{tabular}{|l|rrrr|}
			\hline
			                   & Laplace's Method & GVA $(\mLambda = \mR \mR^\top)$ & GVA NP $(\mLambda = (\mR \mR^\top)^{-1})$ & GVA FP   \\
			\hline
			$\vbeta_1$         & $85\%$           & $90\%$                          & $91\%$                                    & $90\%$   \\ 
			$\vbeta_2$         & $76\%$           & $98\%$                          & $99\%$                                    & $99\%$   \\ 
			Mean of $\vu$      & $81\%$           & $94\%$                          & $94\%$                                    & $94\%$   \\
			$\sigma^2_{\vu_1}$ & $66\%$         & $66\%$                        & $66\%$                                  & $66\%$ \\ 
			$\rho$             & $99\%$           & $99\%$                          & $99\%$                                    & $99\%$   \\ 
			\hline
		\end{tabular}
	\end{table}
		
	\begin{table}
		\caption{Table of accuracy - Random slope model}
		\label{tab:accuracy_slope}
		\begin{tabular}{|l|rrrr|}
			\hline
			                   & Laplace's Method & GVA $(\mLambda = \mR \mR^\top)$ & GVA $(\mLambda = (\mR \mR^\top)^{-1})$ & GVA FP \\
			\hline
			$\vbeta_1$         & 67\%             & 88\%                            & 88\%                                   & 88\%   \\
			$\vbeta_2$         & 70\%             & 89\%                            & 88\%                                   & 89\%   \\
			Mean of $\vu$      & 70\%             & 91\%                            & 91\%                                   & 91\%   \\
			$\sigma^2_{\vu_1}$ & 71\%           & 73\%                          & 73\%                                 & 73\% \\
			$\sigma^2_{\vu_2}$ & 68\%           & 69\%                          & 69\%                                 & 69\% \\
			$\rho$             & 91\%             & 90\%                            & 90\%                                   & 90\%   \\
			\hline
		\end{tabular}
	\end{table}
		
	% \begin{table}
	% \caption{Table of accuracy - Splines}
	% \label{tab:accuracy_spline}
	% \begin{tabular}{|l|l|}
	% \hline
	% Approximation & Accuracy \\
	% \hline
	% Laplace's Method & 0.969 \\
	% GVA & 0.969 \\
	% GVA NP & 0.969 \\
	% GVA NR & 0.969 \\
	% \hline
	% \end{tabular}
	% \end{table}
		
	\begin{figure}
		\label{fig:spline}
		\caption{Comparison of VB and MCMC spline fits with the true function}
		\includegraphics[width=100mm, height=100mm]{code/results/accuracy_plots_spline_gva2.pdf}
	\end{figure}
		
	% Graphs - exactly what sort of graphs do we need?
	% Median accuracy
	% Increase in lower bound
	% MCMC posterior, with approximating posterior for at least one or two of the
	% key parameters, such as, say, vbeta[2]
		
	\subsection{Numerical stability}

	In the process of performing numerical experiments, we discovered that our model fitting software was 
	prone to numeric overflow due to the log link in our model and the exponentiation of the diagonals of the
	Cholesky factors in the covariance parameterisation of the Gaussian Variational Approximation of $\vnu$.

	We dealt with this difficulty by developing a 'safe exponential' parameterisation for the diagonals of the
	Cholesky factors. The parameterisation is exponential up to a threshold $t$, and then quadratic beyond that
	threshold.

	We repeated our numerical experiments with the new parameterisation, varying the threshold within reasonable
	bounds and found that the numerical experiments no longer resulted in overflow, and that the numerical accuracy
	of the approximation was still very good.

	\subsection{Stability results}
		
	The numerical stability of each fitting algorithm in Section \ref{sec:gaussian} was assessed by initialising
	each algorithm from a range of different starting points. Errors due to numerical instability and the fitted
	$\vmu$ were recorded for each starting point.
		
	A data set of 100 individuals in ten groups ($m=10$) was generated from a model with a fixed intercept and
	slope, and a random intercept. $\vmu$ was initialised from a grid of points on the interval $[-4.5, 5]$ for
	intercept and slope, spaced $0.1$ apart. The error counts are presented in Table
	\ref{tab:stability_results}.


	The stability of the GVA algorithm with the parameterisation $\mLambda = (\mR^\top \mR)^{-1}$ depends on the
	threshold chosen for the safe exponential function. When the threshold is set to $2$, the algorithm is stable
	for all starting points within the grid except $6$. When the threshold is set to $\infty$, equivalent to using
	the naive $\exp$ parameterisation, the algorithm encounters numerical errors for every starting point on the 
	grid.
		
	\begin{table}
		\caption{Count of numerical errors for each algorithm during stability tests}
		\label{tab:stability_results}
		\begin{tabular}{|l|r|}
			\hline
			Algorithm           & Error count \\
			\hline
			Laplace's algorithm & 12          \\
			GVA $\mLambda = \mR^\top \mR$                 & 1,306       \\
			GVA $\mLambda = (\mR^\top \mR)^{-1}$              & 6         \\
			GVA NR fixed point              & 992         \\
			\hline
		\end{tabular}
	\end{table}
		
	\section{Application}
	\label{sec:application}

	\subsection{Poisson example}
	The data set used for this example was the police stop example from Chapter 15 of \citep{Gelman2007}.
	The model fit was
	\begin{align*}
		y_{ep} &\sim \text{Poisson}(n_{ep} e^{\beta_0 + \beta_e \text{ethnicity}_e + \alpha_{k} \text{crime} + u_p + \epsilon_{ep}}) \\
		u_p &\sim \N(0, \sigma_u^2) \\
		\epsilon_{ep} &\sim \N(0, \sigma_\epsilon^2)
	\end{align*}
	where $p$ is the $p$-th precinct, and $e$ is the $e$-th ethnicity (blacks, hispanics or whites), and $c$ is
	the $c$-th category of crime (violent crimes, weapons crimes, propery crimes, drug crimes). The random 
	intercepts $u_p$ allow for variation in the base rates of stops across precincts,
	the co-efficients $\beta_j$ measure the effect of ethnicity on the rate of police stops and
	the co-efficients $\alpha_k$ measure the effect of each type of crime on the rate.
	The model finds the relationship between the number of police stops in each precinct and 
	ethnicity	for each type of crime.

	The model was fit using the GVA algorithm with the $\mLambda = (\mR^\top \mR)^{-1}$ parameterisation, using
	the prior $a_\rho = 3$, $b_\rho = 1$ on $\rho$. Accuracy of the approximation was assessed by comparing the
	fitted distribution for each parameter to a kernel density estimate of the parameter's distribution from
	1 million samples from the equivalent model fitted using Stan. The results are presented in Table
	\ref{tab:application_police_stops}.

	% Table of results
	\begin{table}
		\caption{Table of results - Police stops}
		\label{tab:application_police_stops}
		\begin{tabular}{|l|rrrr|}
			\hline
			Covariate          & Posterior Mean & Lower 95\% CI & Upper 95\% CI & Accuracy \\
			\hline
			Intercept [African-Americans] & 4.04 & 4.03 & 4.042 & 82.4\% \\
			$\beta_2$ [hispanics] & -0.445 & -0.445 & -0.445 & 98.1\% \\
			$\beta_3$ [whites] & -1.380 & -1.380 & -1.380 & 98.5\% \\
			$\alpha_1$ [weapons crimes] & 0.580 & 0.580 & 0.580 & 89.2\%     \\
			$\alpha_2$ [property crimes] & -0.191 & -0.191 & -0.191 & 92.2\%     \\
			$\alpha_3$ [drug crimes] & -0.750 & -0.750 & -0.750 & 94.9\%     \\
			Random intercept & 1.321 & -0.190 & 2.202 & 86.7\% \\
			$\sigma^2_{\vu}$ & 2.095 & 1.585 & 3.26 & 40\% \\
			\hline
		\end{tabular}
	\end{table}

	% Table of speeds


	% TODO: You need to describe the data set and the model.
	\subsection{Zero--inflated example -- Cockroaches in Apartments data set from Gelman}
	The model fitting was applied to the cockroach data set from \S 6.7 of \citep{Gelman2007} taken from a study
	on the effect of integrated pest management in controlling cockroach levels in urban apartments. The data
	set contains data on 160 treatment and 104 control apartments, along with the response $y_i$ in each
	apartment of the number of cockroaches caught in a set of traps. The apartments had the traps deployed for
	different numbers of days, referred to as trap days, which was handled by using a log offset
	\citep{Agresti2002}. The predictors in the data set included the pre-treatment roach level, a treatment
	indicator, the time of the observation and an indicator for whether the apartment is in a senior building
	restricted to the elderly.
		
	In the example application presented in this paper, the zero component represents an apartment completely free of roaches, while the non-zero component represents an apartment where roaches have been able to live and reproduce, possibly in spite of pest control treatment aimed at preventing them from doing so.
		
	The GVA algorithm with the $\mLambda = (\mR^\top \mR)^{-1}$ parameterisation was used to fit a random
	intercept model to the Roaches data set provided by Andrew Gelman. The fitted co-efficients and accuracy
	results are presented in Table \ref{tab:application_roaches}.
		
	%       lci  uci
	% 1  3.179 3.157 3.201
	% 2 -0.046 -0.053 -0.039
	% 3 -0.420 -0.434 -0.406
	% 1 -0.976 -1.015 -0.936
	% 2 -0.309 -0.323 -0.295
	% 3 -0.947 -0.963 -0.930
	% 4 -2.129 -2.384 -1.874
	% 5 -3.230 -3.490 -2.970
	% 6 -3.099 -3.404 -2.794
	% 7 -1.290 -1.326 -1.255
	% 8 -0.956 -0.991 -0.921
	% 9 -2.404 -2.600 -2.209
	% 10 -1.076 -1.123 -1.029
	% 11 -1.079 -1.107 -1.052
	% 12 -1.681 -1.737 -1.624
		
	%> round(cbind(fit1$vmu, lci, uci), 3)
	% fit1$a_rho
	% [1] 377.2375
	% > fit1$b_rho
	% [1] 152.7625
		
	\begin{table}
		\caption{Table of results - Roaches}
		\label{tab:application_roaches}
		\begin{tabular}{|l|rrrr|}
			\hline
			Covariate          & Posterior Mean & Lower 95\% CI & Upper 95\% CI & Accuracy \\
			\hline
			Intercept          & 3.18           & 3.16          & 3.20          & 90\%     \\
			Time               & $-$0.05        & $-$0.05       & $-$0.04       & 97\%     \\
			Time:Treatment     & $-$0.43        & $-$0.43       & $-$0.41       & 93\%     \\
			Random intercept   & $-$1.60        & $-$1.71       & $-$1.49       & 90\%     \\
			$\sigma^2_{\vu_1}$ & 0.58           & 0.57          & 0.57          & 57\%     \\
			$\rho$             & 0.71           & 0.67          & 0.75          & 88\%     \\
			\hline
		\end{tabular}
	\end{table}
		
	\begin{figure}
		\caption{Accuracy graphs for roach model}
		\label{fig:accuracy_roach}
		\centering
		% \includepdf[width=75mm,height=75mm,pages={1,2,3,16},nup=2x2]{code/results/accuracy_plots_application_GVA2.pdf}
		\begin{tabular}{@{}c@{\hspace{.5cm}}c@{}}
			\includegraphics[page=1,width=.45\textwidth]{code/results/accuracy_plots_application_GVA_NP.pdf} &   
			\includegraphics[page=2,width=.45\textwidth]{code/results/accuracy_plots_application_GVA_NP.pdf} \\[.5cm]
			\includegraphics[page=3,width=.45\textwidth]{code/results/accuracy_plots_application_GVA_NP.pdf} &   
			\includegraphics[page=16,width=.45\textwidth]{code/results/accuracy_plots_application_GVA_NP.pdf} \\[.5cm]
		\end{tabular}
	\end{figure}
		
	To assess the speed of each approach, a test case was constructed of a random slope model with $m=50$ groups,	each containing $n_i = 100$ individuals. A model was then fit to this data set ten times using each algorithm, and the results averaged. They are presented in Table \ref{tab:application_slope_speed}.
		
	\begin{table}
		\caption{Table of results - Speed}
		\label{tab:application_slope_speed}
		\begin{tabular}{|l|rr|}
			\hline
			Algorithm        & Mean (seconds) & Standard deviation (seconds) \\
			\hline
			Laplace's method & 0.37 s        & 0.07 s                       \\
			GVA              & 2.04 s        & 1.24 s                       \\
			% Why is this slower?
			GVA NP           & 0.38 s        & 0.66 s                       \\
			GVA FP           & 0.05 s         & 0.07 s                       \\
			\hline
		\end{tabular}
	\end{table}
		
	\section{Discussion}
	\label{sec:discussion}
		
	We have described a Variational Bayes approximation to Zero-Inflated Poisson regression models which allows
	such models to be fit with considerable generality. We have also devised and extensively tested a number of
	alternative approaches for fitting such models, and extended one of these alternative approaches with a new
	parameterisation. Using MCMC methods as the gold standard to test against, we have assessed the accuracy and
	computational speed of these algoritms.
		
	The use of Mean Field Variational Bayes allows estimation of Bayesian ZIP models in a fraction of the time taken to fit the same model using even the best MCMC methods available, with only a small loss of accuracy.
	This is of great utility in applications where speed matters, such as model selection or when applied
	statisticians are experimenting with many models, as is typical in practice.
		
	The new parameterisation of Gaussian Variational Approximation using the Cholesky factorisation of the inverse of $\mLambda$ presented in Section \ref{sec:param} provides significant advantages.  It is well known that the inverse of a sparse matrix need not be sparse. Fitting mixed models generally leads to covariance matrices with an arrow head structure. Due to the sparsity of our parameterisation of $\mLambda^{-1}$,  our algorithm for fitting these models leads to an optimisation problem of significantly lower dimension. This allows the model to be fit more quickly, and with greatly improved numerical stability and with no loss of
	accuracy.
	
	Some of the algorithms which we experimented with were found to be very sensitive to their starting points.
	While these algorithms are typically initialised with a starting point as close as possible to the final
	solution, this gives some sense of the stability of each algorithm.
		
	This article presents the essential ideas necessary for a performant implementation implementing model fitting
	for ZIP regression models.%, but the performance would be even better if our algorithm was re-implemented in a
	%compiled language with good numeric libraries such as C++ with Eigen.
	The majority of the performance
	improvements over existing approaches come from avoiding unneccessary matrix inversion, which is a
	computationally expensive and numerically unstable process taking $\BigO(p^3)$ flops, and from constructing and 
	calculating	with sparse matrices. The gains of these approaches, particularly from sparse matrix techniques, 
	can be difficult to fully realise in R without expert knowledge of the underlying implementation and libraries.
		
	Our application of these ideas to Andrew Gelman's data showed that the new parameterisation very effectively
	speeds up fitting zero-inflated mixed models to real world data with a large number of groups, while still
	maintaining excellent accuracy versus an MCMC approach. This demonstrates the applicability of the ideas
	presented within this paper to real world data sets.
		
	\newpage
	\section{Appendix} 
	% TODO: Mean field updates?
	\subsection{Calculation of the Variational Lower bound}
	% Where are the priors for \vbeta and \vu
		
	The variational lower bound is equal to $\bE_q[\log{p(\vy, \vtheta)} - \log{q(\vtheta)}] = T_1 + T_2 + T_3$,
	where
	% This is the new T_1
	$$
	\begin{array}{rl}
		T_1 & = \quad \bE_q[\log{p(\vy, \vnu)} - \log{q(\vnu)}]                                                                                                                                                  \\
		    & = \quad \vy \mP \mC \vmu - \vp^\top \exp{\left[ \mC \vmu + \half \text{diag} (\mC \mLambda \mC^\top) \right]} - \vone^\top\log \Gamma{(\vy + \vone)}                                               \\
		    & \quad + \frac{p + m}{2} (1 + \log{2 \pi}) + \half \log{|\mLambda|},                                                                                                                                \\
		T_2 & = \quad \bE_q \left[ \log p(\mSigma_{\vu \vu}) - \log q(\mSigma_{\vu \vu}) \right]                                                                                                                 \\
		    & = \quad \bE_q \big[ v/2(\log |\Psi| - \log |\Psi + \vmu_\vu \vmu_\vu^\top + \mLambda_{\vu \vu}|) + \half \log 2 + \half \log|\mSigma_{\vu \vu}| + \log \Gamma_{p+1}(v/2) - \log \Gamma_{p}(v/2)    \\
		    & \quad + \half \tr((\vmu_{\vu} \vmu_{\vu}^\top + \mLambda_{\vu \vu}) \mSigma_{\vu \vu}^{-1}) \big]                                                                                                  \\
		    & = \quad v/2\big(\log |\Psi| - \log |\Psi + \vmu_\vu \vmu_\vu^\top + \mLambda_{\vu \vu}|\big) + \half \log 2 + \half \bE_q \log |\mSigma_{\vu \vu}| + \log \Gamma_{p+1}(v/2) - \log \Gamma_{p}(v/2) \\
		    & \quad + \half \tr\big(\mI_m + \Psi(\Psi+ \vmu_\vu \vmu_\vu^\top + \mLambda_{\vu \vu})^{-1}/(v + p + 2)\big)                                                                                        \\
		T_3 & = - \vp^\top \log \vp - (\vone - \vp)^\top \log (\vone - \vp) - \log \Beta (\alpha_\rho, \beta_\rho) + \log \Beta (\alpha_q, \beta_q)                                                              
	\end{array}
	$$
		
	\noindent with $\bE_q \log |\mSigma_{\vu \vu}| = m \log 2 + \log \left | \Psi + \vmu_\vu \vmu_\vu^\top + \mLambda_{\vu \vu} \right | + \sum_{i=1}^m \Psi \left ( \frac{v - i + 1}{2} \right )$
		
	\subsection{Numerical stability of fitting algorithms with respect to starting point}
		
	\mgc{Generate images using local\textunderscore solutions.R and place here}
		
	\bibliographystyle{elsarticle-harv}
	\bibliography{Chapter_1_zero_inflated_models}
			
\end{document}
