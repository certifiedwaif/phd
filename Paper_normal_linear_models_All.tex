\documentclass{article}[12pt]
% \documentclass[times, doublespace]{anzsauth}

\addtolength{\oddsidemargin}{-.75in}%
\addtolength{\evensidemargin}{-.75in}%
\addtolength{\textwidth}{1.5in}%
\addtolength{\textheight}{1.3in}%
\addtolength{\topmargin}{-.8in}%
\addtolength{\marginparpush}{-.75in}%
% \setlength\parindent{0pt}
% \setlength{\bibsep}{0pt plus 0.3ex}

\usepackage{amsthm}
\usepackage[authoryear]{natbib}
\usepackage{graphicx}
\usepackage{algorithm,algorithmic}
\usepackage{cancel}
%\usepackage{mathtools}
\usepackage{algorithm,algorithmic}
\usepackage[inner=2.5cm,outer=1.5cm,bottom=2cm]{geometry}
\usepackage{setspace}
\onehalfspacing
\usepackage{microtype}
\usepackage{color}
\usepackage{relsize}
 

\newtheorem{theorem}{Theorem}[section]
\newtheorem{res}{Result}%[chapter]

 

%% include.tex
\newcommand{\Bernoulli}[1]{\text{Bernoulli} \left( #1 \right)}
\newcommand{\mydigamma}[1]{\psi \left( #1 \right)}
%\newcommand{\diag}[1]{\text{diag}\left( #1 \right)}
\newcommand{\tr}[1]{\text{tr}\left( #1 \right)}
\newcommand{\Poisson}[1]{\text{Poisson} \left( #1 \right)}
\def \half {\frac{1}{2}}
\def \R {\mathbb{R}}
\def \vbeta {\vec{\beta}}
\def \vy {\vec{y}}
\def \vmu {\vec{\mu}}
\def \vmuqbeta {\vmu_{q(\vbeta)}}
\def \vmubeta {\vmu_{\vbeta}}
\def \Sigmaqbeta {\Sigma_{q(\vbeta)}}
\def \Sigmabeta {\Sigma_{\vbeta}}
\def \va {\vec{a}}
\def \vtheta {\vec{\theta}}
\def \mX {\vec{X}}

\def\ds{{\displaystyle}}

\def\diag{{\mbox{diag}}}


\usepackage{latexsym,amssymb,amsmath,amsfonts}
%\usepackage{tabularx}
\usepackage{theorem}
\usepackage{verbatim,array,multicol,palatino}
\usepackage{graphicx}
\usepackage{graphics}
\usepackage{fancyhdr}
\usepackage{algorithm,algorithmic}
\usepackage{url}
%\usepackage[all]{xy}



\def\approxdist{\stackrel{{\tiny \mbox{approx.}}}{\sim}}
\def\smhalf{\textstyle{\frac{1}{2}}}
\def\vxnew{\vx_{\mbox{{\tiny new}}}}
\def\bib{\vskip12pt\par\noindent\hangindent=1 true cm\hangafter=1}
\def\jump{\vskip3mm\noindent}
\def\etal{{\em et al.}}
\def\etahat{{\widehat\eta}}
\def\thick#1{\hbox{\rlap{$#1$}\kern0.25pt\rlap{$#1$}\kern0.25pt$#1$}}
\def\smbbeta{{\thick{\scriptstyle{\beta}}}}
\def\smbtheta{{\thick{\scriptstyle{\theta}}}}
\def\smbu{{\thick{\scriptstyle{\rm u}}}}
\def\smbzero{{\thick{\scriptstyle{0}}}}
\def\boxit#1{\begin{center}\fbox{#1}\end{center}}
\def\lboxit#1{\vbox{\hrule\hbox{\vrule\kern6pt
      \vbox{\kern6pt#1\kern6pt}\kern6pt\vrule}\hrule}}
\def\thickboxit#1{\vbox{{\hrule height 1mm}\hbox{{\vrule width 1mm}\kern6pt
          \vbox{\kern6pt#1\kern6pt}\kern6pt{\vrule width 1mm}}
               {\hrule height 1mm}}}


%\sloppy
%\usepackage{geometry}
%\geometry{verbose,a4paper,tmargin=20mm,bmargin=20mm,lmargin=40mm,rmargin=20mm}


%%%%%%%%%%%%%%%%%%%%%%%%%%%%%%%%%%%%%%%%%%%%%%%%%%%%%%%%%%%%%%%%%%%%%%%%%%%%%%%%
%
% Some convenience definitions
%
% \bf      -> vector
% \sf      -> matrix
% \mathcal -> sets or statistical
% \mathbb  -> fields or statistical
%
%%%%%%%%%%%%%%%%%%%%%%%%%%%%%%%%%%%%%%%%%%%%%%%%%%%%%%%%%%%%%%%%%%%%%%%%%%%%%%%%

% Sets or statistical values
\def\sI{{\mathcal I}}                            % Current Index set
\def\sJ{{\mathcal J}}                            % Select Index set
\def\sL{{\mathcal L}}                            % Likelihood
\def\sl{{\ell}}                                  % Log-likelihood
\def\sN{{\mathcal N}}                            
\def\sS{{\mathcal S}}                            
\def\sP{{\mathcal P}}                            
\def\sQ{{\mathcal Q}}                            
\def\sB{{\mathcal B}}                            
\def\sD{{\mathcal D}}                            
\def\sT{{\mathcal T}}
\def\sE{{\mathcal E}}                            
\def\sF{{\mathcal F}}                            
\def\sC{{\mathcal C}}                            
\def\sO{{\mathcal O}}                            
\def\sH{{\mathcal H}} 
\def\sR{{\mathcal R}}                            
\def\sJ{{\mathcal J}}                            
\def\sCP{{\mathcal CP}}                            
\def\sX{{\mathcal X}}                            
\def\sA{{\mathcal A}} 
\def\sZ{{\mathcal Z}}                            
\def\sM{{\mathcal M}}                            
\def\sK{{\mathcal K}}     
\def\sG{{\mathcal G}}                         
\def\sY{{\mathcal Y}}                         
\def\sU{{\mathcal U}}  


\def\sIG{{\mathcal IG}}                            


\def\cD{{\sf D}}
\def\cH{{\sf H}}
\def\cI{{\sf I}}

% Vectors
\def\vectorfontone{\bf}
\def\vectorfonttwo{\boldsymbol}
\def\va{{\vectorfontone a}}                      %
\def\vb{{\vectorfontone b}}                      %
\def\vc{{\vectorfontone c}}                      %
\def\vd{{\vectorfontone d}}                      %
\def\ve{{\vectorfontone e}}                      %
\def\vf{{\vectorfontone f}}                      %
\def\vg{{\vectorfontone g}}                      %
\def\vh{{\vectorfontone h}}                      %
\def\vi{{\vectorfontone i}}                      %
\def\vj{{\vectorfontone j}}                      %
\def\vk{{\vectorfontone k}}                      %
\def\vl{{\vectorfontone l}}                      %
\def\vm{{\vectorfontone m}}                      % number of basis functions
\def\vn{{\vectorfontone n}}                      % number of training samples
\def\vo{{\vectorfontone o}}                      %
\def\vp{{\vectorfontone p}}                      % number of unpenalized coefficients
\def\vq{{\vectorfontone q}}                      % number of penalized coefficients
\def\vr{{\vectorfontone r}}                      %
\def\vs{{\vectorfontone s}}                      %
\def\vt{{\vectorfontone t}}                      %
\def\vu{{\vectorfontone u}}                      % Penalized coefficients
\def\vv{{\vectorfontone v}}                      %
\def\vw{{\vectorfontone w}}                      %
\def\vx{{\vectorfontone x}}                      % Covariates/Predictors
\def\vy{{\vectorfontone y}}                      % Targets/Labels
\def\vz{{\vectorfontone z}}                      %

\def\vone{{\vectorfontone 1}}
\def\vzero{{\vectorfontone 0}}

\def\valpha{{\vectorfonttwo \alpha}}             %
\def\vbeta{{\vectorfonttwo \beta}}               % Unpenalized coefficients
\def\vgamma{{\vectorfonttwo \gamma}}             %
\def\vdelta{{\vectorfonttwo \delta}}             %
\def\vepsilon{{\vectorfonttwo \epsilon}}         %
\def\vvarepsilon{{\vectorfonttwo \varepsilon}}   % Vector of errors
\def\vzeta{{\vectorfonttwo \zeta}}               %
\def\veta{{\vectorfonttwo \eta}}                 % Vector of natural parameters
\def\vtheta{{\vectorfonttwo \theta}}             % Vector of combined coefficients
\def\vvartheta{{\vectorfonttwo \vartheta}}       %
\def\viota{{\vectorfonttwo \iota}}               %
\def\vkappa{{\vectorfonttwo \kappa}}             %
\def\vlambda{{\vectorfonttwo \lambda}}           % Vector of smoothing parameters
\def\vmu{{\vectorfonttwo \mu}}                   % Vector of means
\def\vnu{{\vectorfonttwo \nu}}                   %
\def\vxi{{\vectorfonttwo \xi}}                   %
\def\vpi{{\vectorfonttwo \pi}}                   %
\def\vvarpi{{\vectorfonttwo \varpi}}             %
\def\vrho{{\vectorfonttwo \rho}}                 %
\def\vvarrho{{\vectorfonttwo \varrho}}           %
\def\vsigma{{\vectorfonttwo \sigma}}             %
\def\vvarsigma{{\vectorfonttwo \varsigma}}       %
\def\vtau{{\vectorfonttwo \tau}}                 %
\def\vupsilon{{\vectorfonttwo \upsilon}}         %
\def\vphi{{\vectorfonttwo \phi}}                 %
\def\vvarphi{{\vectorfonttwo \varphi}}           %
\def\vchi{{\vectorfonttwo \chi}}                 %
\def\vpsi{{\vectorfonttwo \psi}}                 %
\def\vomega{{\vectorfonttwo \omega}}             %


% Matrices
%\def\matrixfontone{\sf}
%\def\matrixfonttwo{\sf}
\def\matrixfontone{\bf}
\def\matrixfonttwo{\boldsymbol}
\def\mA{{\matrixfontone A}}                      %
\def\mB{{\matrixfontone B}}                      %
\def\mC{{\matrixfontone C}}                      % Combined Design Matrix
\def\mD{{\matrixfontone D}}                      % Penalty Matrix for \vu_J
\def\mE{{\matrixfontone E}}                      %
\def\mF{{\matrixfontone F}}                      %
\def\mG{{\matrixfontone G}}                      % Penalty Matrix for \vu
\def\mH{{\matrixfontone H}}                      %
\def\mI{{\matrixfontone I}}                      % Identity Matrix
\def\mJ{{\matrixfontone J}}                      %
\def\mK{{\matrixfontone K}}                      %
\def\mL{{\matrixfontone L}}                      % Lower bound
\def\mM{{\matrixfontone M}}                      %
\def\mN{{\matrixfontone N}}                      %
\def\mO{{\matrixfontone O}}                      %
\def\mP{{\matrixfontone P}}                      %
\def\mQ{{\matrixfontone Q}}                      %
\def\mR{{\matrixfontone R}}                      %
\def\mS{{\matrixfontone S}}                      %
\def\mT{{\matrixfontone T}}                      %
\def\mU{{\matrixfontone U}}                      % Upper bound
\def\mV{{\matrixfontone V}}                      %
\def\mW{{\matrixfontone W}}                      % Variance Matrix i.e. diag(b'')
\def\mX{{\matrixfontone X}}                      % Unpenalized Design Matrix/Nullspace Matrix
\def\mY{{\matrixfontone Y}}                      %
\def\mZ{{\matrixfontone Z}}                      % Penalized Design Matrix/Kernel Space Matrix

\def\mGamma{{\matrixfonttwo \Gamma}}             %
\def\mDelta{{\matrixfonttwo \Delta}}             %
\def\mTheta{{\matrixfonttwo \Theta}}             %
\def\mLambda{{\matrixfonttwo \Lambda}}           % Penalty Matrix for \vnu
\def\mXi{{\matrixfonttwo \Xi}}                   %
\def\mPi{{\matrixfonttwo \Pi}}                   %
\def\mSigma{{\matrixfonttwo \Sigma}}             %
\def\mUpsilon{{\matrixfonttwo \Upsilon}}         %
\def\mPhi{{\matrixfonttwo \Phi}}                 %
\def\mOmega{{\matrixfonttwo \Omega}}             %
\def\mPsi{{\matrixfonttwo \Psi}}                 %

\def\mone{{\matrixfontone 1}}
\def\mzero{{\matrixfontone 0}}

% Fields or Statistical
\def\bE{{\mathbb E}}                             % Expectation
\def\bP{{\mathbb P}}                             % Probability
\def\bR{{\mathbb R}}                             % Reals
\def\bI{{\mathbb I}}                             % Reals
\def\bV{{\mathbb V}}                             % Reals

\def\vX{{\vectorfontone X}}                      % Targets/Labels
\def\vY{{\vectorfontone Y}}                      % Targets/Labels
\def\vZ{{\vectorfontone Z}}                      %

% Other
\def\etal{{\em et al.}}
\def\ds{\displaystyle}
\def\d{\partial}
\def\diag{\text{diag}}
%\def\span{\text{span}}
\def\blockdiag{\text{blockdiag}}
\def\tr{\text{tr}}
\def\RSS{\text{RSS}}
\def\df{\text{df}}
\def\GCV{\text{GCV}}
\def\AIC{\text{AIC}}
\def\MLC{\text{MLC}}
\def\mAIC{\text{mAIC}}
\def\cAIC{\text{cAIC}}
\def\rank{\text{rank}}
\def\MASE{\text{MASE}}
\def\SMSE{\text{SASE}}
\def\sign{\text{sign}}
\def\card{\text{card}}
\def\notexp{\text{notexp}}
\def\ASE{\text{ASE}}
\def\ML{\text{ML}}
\def\nullity{\text{nullity}}

\def\logexpit{\text{logexpit}}
\def\logit{\mbox{logit}}
\def\dg{\mbox{dg}}

\def\Bern{\mbox{Bernoulli}}
\def\sBernoulli{\mbox{Bernoulli}}
\def\sGamma{\mbox{Gamma}}
\def\sInvN{\mbox{Inv}\sN}
\def\sNegBin{\sN\sB}

\def\dGamma{\mbox{Gamma}}
\def\dInvGam{\mbox{Inv}\Gamma}

\def\Cov{\mbox{Cov}}
\def\Mgf{\mbox{Mgf}}

\def\mis{{mis}} 
\def\obs{{obs}}

\def\argmax{\operatornamewithlimits{\text{argmax}}}
\def\argmin{\operatornamewithlimits{\text{argmin}}}
\def\argsup{\operatornamewithlimits{\text{argsup}}}
\def\arginf{\operatornamewithlimits{\text{arginf}}}


\def\minimize{\operatornamewithlimits{\text{minimize}}}
\def\maximize{\operatornamewithlimits{\text{maximize}}}
\def\suchthat{\text{such that}}


\def\relstack#1#2{\mathop{#1}\limits_{#2}}
\def\sfrac#1#2{{\textstyle{\frac{#1}{#2}}}}


\def\comment#1{
\vspace{0.5cm}
\noindent \begin{tabular}{|p{14cm}|}  
\hline #1 \\ 
\hline 
\end{tabular}
\vspace{0.5cm}
}


\def\mytext#1{\begin{tabular}{p{13cm}}#1\end{tabular}}
\def\mytextB#1{\begin{tabular}{p{7.5cm}}#1\end{tabular}}
\def\mytextC#1{\begin{tabular}{p{12cm}}#1\end{tabular}}

\def\jump{\vskip3mm\noindent}

\def\KL{\text{KL}}
\def\N{\text{N}}
\def\Var{\text{Var}}

\def \E {\mathbb{E}}
\def \BigO {\text{O}}
\def \IG {\text{IG}}
\def \Beta {\text{Beta}}



\def\I{\mI}
\def\N{N}
\def\E{{\mathbb E}}
\def\R{{\mathbb R}}

\def\bR{{\mathbb R}}
\def\bC{{\mathbb C}}
\def\bZ{{\mathbb Z}}

\def\Beta{\mbox{Beta}}
\def\BigO{\mbox{O}}

\makeatletter
\DeclareRobustCommand{\prodK}{%
	\mathop{\vphantom{\sum}\mathpalette\bigstar@\relax}\slimits@
}
 
\newcommand{\bigstar@}[2]{%
 	\vcenter{%
 		\sbox\z@{$#1\sum$}%
 		\hbox{\resizebox{.9\dimexpr\ht\z@+\dp\z@}{!}{\mbox{K}}}%
 	}%
}
\makeatother
 
%\newcommand{}{\mathlarger{\operatorname{K}}}

\newcommand{\mgc}[1]{{\color{blue}#1}}
\newcommand{\joc}[1]{{\color{red}#1}}

\begin{document}

%\maketitle


\title{Full Bayesian inference for linear model selection \\ using a generalized $g$-pior}

\author{Mark Greenaway
	%\addressnum{1},   
	and
	John T. Ormerod
	%\addressnum{1}\corrauth}
	%\affiliation{University of Sydney
}
% Specifying address(es) in the manner required by the Journal.
%\address{
%\addressnum{1}  School of Mathematics and Statistics Carslaw Building (F07), University of Sydney, Sydney, NSW 2006, Australia.
%\hspace*{1ex} Email: \texttt{john.ormerod@sydney.edu.au}
%}
% Note that the Journal requires that a paper must begin with a
% "Summary" not an "Abstract".  This is automatically taken care
% of by the anzsauth document style.  So even though the following
% says "\begin{abstract}" the heading "Summary" will appear in
% the processed version.

% Note that "keywords" should not include words and phrases
% that form part of the title of the paper.
% Also keywords (with the exception of proper names and
% certain abbreviations that conventionally appear all in
% capital letters) should *not* be capitalised.
%\keywords{Generalized $g$-prior (X); Approximate Bayesian Inference; Model Section (X); Special functions; model averaging}

% This shows how acknowledgements should appear in the Journal.
% Note that if you are acknowledging financial support, grant
% numbers should *not* usually be specified.
%\ack{....}

\maketitle


\begin{abstract}
\noindent 
We consider linear models with a special case of the prior structure proposed in \cite{Maruyama2011} and numerically analyse the parameter posterior distributions corresponding to this setting.  
We derive Monte Carlo sampling schemes, exact expressions in terms of special functions, and convenient asymptotic approximations of the parameter posterior distributions. 
Implementation issues of Bayesian model averaging under the proposed prior structure is also discussed 
and R software is made available for this purpose. Our methodology is illustrated on several datasets.
\end{abstract}


\section{Introduction}

The problem of variable selection in the context of linear models is a well trodden one. 

\begin{itemize}
	\item Zellnor $g$-prior
	
	\item Spike and slab
	
	\item George and McCulloch 1993
	
	\item Model averaging
	
	\item MCMC
	
	\item Bartlett's paradox and the information paradox. Jeffreys 1961; Berger and Pericchi 2001
	
	\item \cite{Liang2008} different priors on $g$
	
	\item \cite{Maruyama2011} 
	
	\item Bayarri et al. (2012).
	
	\item Li and Clyde (????)
	
	\item Parameter Posterior Distributions
\end{itemize}



In particular
in the Bayesian inferential paradigm the prior structure needs to be exquisitely carefully chosen.

\joc{ 
Bayesian variable selection has a long history (Zellner 1971, sec 10.4; Leamer 1978a,b; Mitchell and 
Beauchamp 1988), the advent of Markov chain Monte Carlo methods catalyzed Bayesian model selection and 
averaging in regression models (George and McCulloch 1993, 1997; Geweke 1996; Smith and Kohn 1996; 
Raftery, Madigan, and Hoeting 1997; Hoeting, Madigan, Raftery, and Volinsky 1999; Clyde and George 2004). 

Prior density choice for Bayesian model selection and model averaging, however, remains an open area
(Berger and Pericchi 2001; Clyde and George 2004). Subjective elicitation of priors for model-specific 
coefficients is often precluded, particularly in high-dimensional model spaces, such as in nonparametric 
regression using spline and wavelet bases. Thus, it is often necessary to resort to specification of
priors using some formal method (Kass and Wasserman 1996; Berger and Pericchi 2001). In general, the 
use of improper priors for model-specific parameters is not permitted in the context of model selection, 
as improper priors are determined only up to an arbitrary multiplicative constant. In inference for a 
given model, these arbitrary multiplicative constants cancel in the posterior distribution of the 
model-specific parameters. However, these constants remain in marginal likelihoods leading to 
indeterminate model probabilities and Bayes factors (Jeffreys 1961; Berger and Pericchi 2001). To avoid 
indeterminacies in posterior model probabilities, proper priors for βγ under each model are usually 
required.
}


In recent times there has been a lot of attention paid to specifying the hyperprior on
the parameter $g$ of the $g$-prior. 

It was Liang et al (2008) who showed that many of the previous specifications for $g$
were inadequate and argued that a hyperprior should be placed on $g$. \cite{Liang2008}
was able to 

 
Our main contributions are:
\begin{enumerate}
	\item We derive Monte Carlo schemes for drawing samples from the parameter posterior distributions and their corresponding Rao-Blackwellized estimators;
	
	\item We derive exact expressions for all of the parameter posteriors, except for the regression coefficient vector where we instead derive exact expressions for the first and
	second posterior moments;
	
	\item We derive convenient analytic approximations for the parameter posterior distributions and analyse conditions when these approximations are accurate;
	
	\item We show, under a particular asymptotic setting, that model averaging using the chosen prior structure is asymptotically equivalent to model averaging using the Bayesian Information Criterion (BIC); and
	
	\item We implement an {\tt R} package for efficient model averaging under this prior structure.
\end{enumerate}

\noindent Throughout the article numerical issues are discussed.

The article is organised as follows. Section \ref{sec:model} outlines and justifies our chosen model 
and prior structure. Section \ref{sec:MarginalLikelihood} derives the marginal likelihood. Section 
\ref{sec:MonteCarlo} derives our Monte Carlo approaches. Section \ref{sec:Exact} derives exact expressions
for the posterior distributions. Section \ref{sec:approximations} derives analytic approximations
and compares model averaging using our chosen prior structure with model averaging using the BIC.
In Section \ref{sec:implementation}, we discuss details of our implementation which made our approach 
computationally feasible.
In Section \ref{sec:numerical} we perform a series of numerical experiments to show the accuracy of our approach. 
Finally, in Section \ref{sec:conclusion}, we provide a conclude.









 



 



 





\newpage 
\section{The base model}
\label{sec:model}

Suppose $\vy = (y_1,\ldots,y_n)^T$ is a response vector of length $n$, $\mX$ is an $n$ by $p$ matrix 
of covariates. We consider the linear model for predicting $\vy$ with predicturs $\mX$ via
\begin{equation}
\label{eq:linearModel}
\vy | \alpha, \vbeta, \sigma^2 \sim N(\vone_n\alpha + \mX \vbeta, \sigma^2 \mI_n),
\end{equation} 


\noindent where $\alpha$ is the model intercept, $\vbeta$ is a coefficient vector of length $p$, 
$\sigma^2$ is the residual variance, and $\mI_n$ is the $n \times n$ identity matrix. 
Without loss of generality, to simplify later calculations, we will transform $\vy$ and $\mX$ 
so that $\vy$ and the columns of $\mX$ are standardized so that $\overline{y} = 0$, 
$\|\vy\|^2 = \vy^T\vy = n$, $\mX_j^T\vone = 0$,  and $\|\mX_j\|^2 = n$ where $\mX_j$ is the $j$th  column of $\mX$. 

Suppose that we wish to perform Bayesian model selection, model averaging or hypothesis 
testing. Let $\vgamma \in \{0, 1\}^p$ be a binary vector of indicators
for the inclusion of the $p$th column of $\mX$ in the model where $\mX_\vgamma$ 
denotes the design matrix formed by including only the columns of $\mX$ with 
$\gamma_j = 1$. We will adopt the following prior structure
\begin{equation}
\label{eq:priorStructure}
\begin{array}{c}
\ds p(\alpha) \propto 1,  
\qquad 
\ds p(\sigma^2) \propto (\sigma^2)^{-1} I(\sigma^2 > 0),                      
\\ [2ex]
\vbeta_\vgamma | \sigma^2, g, \vgamma \sim \N_p(\vzero, g \sigma^2 (\mX_\vgamma^T \mX_\vgamma)^{-1}),
\quad \text{ and }  \quad 
p(\vbeta_{-\vgamma}|\vgamma) = \prod_{j=1}^p \delta(\beta_j;0)^{1-\gamma_j},
\end{array}
\end{equation} 

\noindent where $\delta(x;a)$ is the Dirac delta function with location $a$. 
For the time being we will defer specification of $p(g)$ and $p(\vgamma)$.
We will now justify each element of our chosen prior structure.

The priors on $\alpha$ and $\sigma^2$ are improper Jeffreys priors and have been formally justified 
in \cite{Berger2012}. In the context Bayesian model selection, model averaging or hypothesis 
testing $\alpha$ and $\sigma^2$ appear in all models 
so that when comparing models the proportionality constants in the corresponding
Bayes factors cancel.

The prior on $\vbeta_\vgamma$ is Zellner's $g$-prior \citep[see for example,][]{Zellner1986} with prior 
hyperparameter $g$. This family of priors for Gaussian regression model where the prior covariance 
matrix of $\vbeta_\vgamma$ is taken to be a multiple of $g$ with the Fisher information matrix for $\vbeta$. 
This places the most prior mass for $\vbeta_\vgamma$ on the section of the parameter space where the data is 
least informative, and makes the marginal likelihood of the model scale-invariant. Furthermore, this 
choice of prior removes a log-determinant of $\mX_\vgamma^T\mX_\vgamma$ term from the expression for the marginal 
likelihood, which is an additional computational burden to calculate.

The prior on $\vbeta_\vgamma$ combined with the prior on $\vbeta_{-\vgamma}$
in (\ref{eq:priorStructure}) is referred to as a spike and slab prior
which was initially used in
\cite{Mitchell1988}
but was later refined in
\cite{Madigan1994},
\cite{George1993}, and
\cite{Ishwaran2005}. 
Initial  spike and slab priors considered the 
case where the Dirac delta functions in (\ref{eq:priorStructure}) 
were replaced
by a Gaussian density with small variance. The above structure implies 
that $p(\vbeta_\vgamma|\vy)$ is a point mass at $\vzero$ leading to
algebraic and computational simplifications. 

An alternative choice of prior on $\vbeta_\vgamma$ was proposed by \cite{Maruyama2011}. Let
$p_{\vgamma} = |\vgamma|$. We will now describe their prior on $\vbeta_\vgamma$ for the case where for the case
$p_{\vgamma} < n - 1$. Let $\mU\mLambda\mU^T$ be an eigenvalue decomposition of $\mX_\vgamma^T\mX_\vgamma$
where $\mU$ is an orthonormal $p_{\vgamma} \times p_{\vgamma}$ matrix, $\mLambda = \mbox{diag}(\lambda_1,\ldots,\lambda_{p_{\vgamma}})$ 
is a diagonal matrix of eigenvalues with $\lambda_1\ge\ldots,\ge \lambda_{p_{\vgamma}}>0$. Then \cite{Maruyama2011} 
propose a prior for $\vbeta_\vgamma$ of the form
\begin{equation}
\label{eq:priorBetaMG}
\vbeta_\vgamma | \sigma^2, g \sim N(\vzero, \sigma^2 (\mU\mW\mU^T)^{-1}),   
\end{equation} 

\noindent where $\mW = \mbox{diag}(w_1,\ldots,w_{p_{\vgamma}})$ with $ w_j = \lambda_j/[\nu_j(1 + g) - 1]$ for 
some prior hyperparameters $\nu_1 \ge \ldots \ge \nu_q$. \cite{Maruyama2011} suggest as a default 
choice for the $\nu_j$'s to use $\nu_j = \lambda_j/\lambda_{p_{\vgamma}}$, for $1\le j \le p_{\vgamma}$. 
This choice down-weights the prior on the rotated parameter space of $(\mU \vbeta)_j$ when the 
corresponding eigenvalue $\lambda_j$ is large and consequently prior standard errors are 
approximately the same size. Note that when $\nu_1 = \ldots = \nu_{p_{\vgamma}} = 1$ the prior 
(\ref{eq:priorBetaMG}) reduces to the prior for $\vbeta$ in (\ref{eq:priorStructure}). 

The choice between (\ref{eq:priorBetaMG}) and the prior for $\vbeta$ in (\ref{eq:priorStructure}) 
represents a trade-off over computational efficiency and desirable statistical properties. The choice
(\ref{eq:priorStructure}) avoids the computational burden of calculating an eigenvalue or a singular 
value decomposition of a $q\times q$ matrix for each model considered. It also means that we can 
exploit efficient matrix updates to traverse the entire model space in a computationally efficient 
manner allowing this to be done feasibly when $p$ is less than around 30 on a standard 2017 laptop 
(see Section \ref{sec:implementation} for details).


The marginal likelihood for the
model  (\ref{eq:linearModel}) and under prior structure
(\ref{eq:priorStructure}). 
Integrating out $\alpha$ and $\vbeta$ from 
$p(\vy,\alpha,\vbeta|\sigma^2,g)$ we find
\begin{equation}\label{eq:yGivenSigma2andG}
\begin{array}{rl}
\ds p(\vy|\sigma^2,g)
& \ds = \int \exp\left[
- \tfrac{n}{2}\log(2\pi\sigma^2) 
- \tfrac{1}{2\sigma^2}\|\vy - \vone\alpha - \mX\vbeta\|^2
- \tfrac{p}{2}\log(2\pi g\sigma^2) 
+ \tfrac{1}{2}\log|\mX^T\mX|
- \tfrac{1}{2g\sigma^2}\vbeta^T\mX^T\mX\vbeta  
\right] d\alpha d\vbeta
\\
& \ds = \int \exp\left[
- \tfrac{n}{2}\log(2\pi\sigma^2) 
- \tfrac{n}{2\sigma^2}
- \tfrac{n\alpha^2}{2\sigma^2} 
+ \sigma^{-2}\vy^T\mX\vbeta
- \tfrac{1}{2\sigma^2}(1 + g^{-1})\vbeta^T\mX^T\mX\vbeta 
- \tfrac{p}{2}\log(2\pi g\sigma^2) 
+ \tfrac{1}{2}\log|\mX^T\mX|
\right] d\alpha d\vbeta
\\
& \ds = \int \exp\left[
- \tfrac{n-1}{2}\log(2\pi\sigma^2) 
- \tfrac{1}{2}\log(n)
- \tfrac{n}{2\sigma^2}
+ \sigma^{-2}\vy^T\mX\vbeta
- \tfrac{1}{2\sigma^2}(1 + g^{-1})\vbeta^T\mX^T\mX\vbeta 
- \tfrac{p}{2}\log(2\pi g\sigma^2) 
+ \tfrac{1}{2}\log|\mX^T\mX|
\right]  d\vbeta
\\
& 
\ds = \exp\left[
- \tfrac{n-1}{2}\log(2\pi\sigma^2) 
- \tfrac{1}{2}\log(n)
- \tfrac{p}{2}\log(1 + g)
- \tfrac{n}{2 \sigma^2} \left( 1 - \tfrac{g}{1 + g} R^2 \right)  
\right],
\end{array} 
\end{equation}

\noindent 
%\joc{
	Derivation of the above expression uses the identity
	$
	\int \exp\left\{ -\tfrac{1}{2}\vx^T\mA\vx + \vb^T\vx \right\} d \vx = |2\pi\mSigma|^{1/2} \exp\left\{ \tfrac{1}{2}\vmu^T\mSigma^{-1}\vmu \right\}
	$
	where $\vmu = \mA^{-1}\vb$, and $\mSigma = \mA^{-1}$.
	It also uses the identities: $|c\mA| = c^d|\mA|$ and $|\mA^{-1}| = |\mA|^{-1}$ when $\mA \in\R^{d\times d}$.
%}
Integrating out $\alpha$, $\vbeta$, and $\sigma^2$ from $p(\vy,\alpha,\vbeta,\sigma^2|g,\vgamma)$ we
obtain
\begin{equation}\label{eq:yGivenG}
\begin{array}{rl}
\ds p(\vy|g,\vgamma)
& \ds = \int \exp\left[
- \tfrac{n-1}{2}\log(2\pi) 
- \tfrac{1}{2}\log(n)
- \tfrac{p}{2}\log(1 + g)
- \left( \tfrac{n-1}{2} + 1\right)\log(\sigma^2) 
- \left( \tfrac{n}{2} \tfrac{1 + g(1-R^2)}{1 + g} \right)\sigma^{-2} 
\right]  d\sigma^2
\\
& 
\ds = K(n)
(1 + g)^{(n - p_\vgamma - 1)/2}(1 + g \widehat{\sigma}_\vgamma^2)^{-(n-1)/2},
\end{array} 
\end{equation}

\noindent where $K(n) = [\Gamma( (n-1)/2 )]/[\sqrt{n}(n\pi)^{(n-1)/2}]$, 
$\widehat{\sigma}_\vgamma^2 = 1 - R_\vgamma^2$, 
$R_\vgamma^2 = \vy^T\mX_\vgamma^T(\mX_\vgamma^T\mX_\vgamma)^{-1}\mX_\vgamma^T\vy/n$ is the MLE of $\sigma^2$ for model
$\vgamma$ when $\overline{y}=0$ and $\|\vy\|^2 = n$
($R_\vgamma^2$
being the the usual R-squared statistic).
This is the same expression as \cite{Liang2008} Equation (5) 
after simplification. Note that
when $\vgamma = \vzero$, i.e., the null model, then $p_\vgamma = 0$, and
$R_\vgamma^2 = 0$ leading to the simplification $p(\vy|g,\vzero) = K(n)$. 

We will now discuss the specification of $g$.




\section{Dealing with $g$}
\label{sec:MarginalLikelihood}
Further criteria are discussed in  Bayarri et al. (2012).

The prior hyperparameter $g$ needs to be specified with exquisite care. 
In the following discussion let $\vgamma^*$ be the true model from which the data was generated. 
Selecting $g$ using a subjective elicitation of priors is usually criticized in the context of 
hypothesis testing. Diffuse problems, i.e., setting $g$ to a large constant, are also problematic 
due to Bartlett's paradox. 
\begin{description}
	\item[Problem 1.] Bartlett's paradox: $p(\vzero|\vy) \to 1$ as $g\to\infty$.
\end{description}

\noindent Bartlett's paradox is paradoxical in the sense that as the prior for $\vbeta_\vgamma$
is made increasingly uninformative the null model becomes increasingly preferred.

If $g$ cannot be made arbitrarily large one might consider setting $g$ to be some constant
value. The unit information prior of Kass \& Wasserman (1995), risk inflation criterion prior of 
Foster \& George (1994), and the benchmark prior of Fern\'andez et al. (2001) set $g$ to $g=n$, 
$g=p_\vgamma^2$ and $g = \max(n,p_\vgamma^2)$ respectively. 
However, these choices lead to the information paradox. 
\begin{description}
	\item[Problem 2.] Information paradox: $p(\vgamma^*|\vy) \not\to 1$ as $n\to\infty$.
\end{description}

\noindent  
\cite{Liang2008} showed that these choices
are also problematic since using any particular constant $g$ will lead to the information paradox.

A third approach might to be to select $g$ adaptively using an empirical Bayes procedure.
Selecting $g$ via an empirical Bayes procedure. 
\cite{Liang2008} considers two such procedures, a local and a global procedure.
For the local empirical Bayes for each model $\vgamma$
the value 
$$
\ds \widehat{g}_{\mbox{\scriptsize EBL}} = \argmax_{g>0} \{ p(\vy|g,\vgamma) \}
$$ 

\noindent 
is used. For the 
global empirical Baye procedure, one value of $g$ is used for all models which is 
given by
$$
\ds \widehat{g}_{\mbox{\scriptsize EBG}} = \argmax_{g>0} \left\{ \sum_{\vgamma}  p(\vy,\vgamma|g) = p(\vy|g) \right\}.
$$

\noindent 
However, \cite{Liang2008} show that setting
$g = \widehat{g}_{\mbox{\scriptsize EBL}}$ or 
$g = \widehat{g}_{\mbox{\scriptsize EBG}}$
are only partially model selection consistent in the following sense.
\begin{description}
	\item[Problem 3.] Partial model selection consistency: $p(\vgamma^*|\vy) \to 1$ as $n\to\infty$ except
	when $\vgamma^* = \vzero$.
\end{description}

Given the problems above one might consider placing a hyperprior on the parameter $g$. 
Initially, \cite{Liang2008} suggest the hyper $g$-prior where
\begin{equation}\label{eq:hyperG}
\ds p(g) = \frac{a - 2}{2}(1 + g)^{-a/2}I(g>0),
\end{equation}

\noindent for $a>2$. Combining (\ref{eq:yGivenG}) with the above prior and applying 
3.197(5) of \cite{Gradshteyn2007} leads to
\begin{equation}\label{eq:hyperGmarginal}
\ds p_{g}(\vy|\vgamma) =  \frac{K(n)(a - 2)}{p_\vgamma + a - 2} {}_2F_1\left( \frac{n-1}{2}, 1; \frac{p_\vgamma + a}{2}; R_\vgamma^2 \right),
\end{equation}

\noindent where ${}_2F_1(\cdot,\cdot;\cdot;\cdot)$ is the Gaussian hypergeometric function.
Gaussian hypergeometric function which are notoriously prone 
to overflow and numerical instability \citep{Pearson2014}. When such numerical issues arise 
\cite{Liang2008} resorted to Laplace approximation. However, recent work of Nadarajah 
(2015) showed the numerical issues associated with (\ref{eq:hyperG}) can be avoided by using properties of the 
Gaussian hypergeometric function based on the identity 
\begin{equation}\label{eq:logGuassHypergeometric}
\ds {}_2F_1(b,1;c;x) = (c-1) x^{1-c} (1-x)^{c-b-1} B_x (c-1, b-c+1),
\end{equation}

\noindent from Prudnikov, Brychkov, \& Marichev (1986 vol. 3, sec. 7.3), where $B_x(\cdot,\cdot)$
is the incomplete beta function. Note that the case where $x=0$ is not handled by 
(\ref{eq:logGuassHypergeometric}) and the value $\ds {}_2F_1(b,1;c;0) = 1$ should be used instead.
Since standard libraries exist for evaluating $B_x(\cdot,\cdot)$
on the log scale numerical overflow can be avoided.
However, while \cite{Liang2008} showed that while (\ref{eq:hyperG}) leads to a closed form for 
$p(\vy|\vgamma)$ and Nadarajah (2015) showed that the numerical problems associated with
evaluating (\ref{eq:hyperGmarginal}) can be avoided, \cite{Liang2008} also showed that
(\ref{eq:hyperG}) leads to Problem 3.

\cite{Liang2008} also considered
the Zellner-Siow prior structure \citep{Zellner1980} which is
equivalent to using $g\sim \mbox{IG}(1/2,n/2)$. However, this choice of prior does not
lead to closed form expressions for $p(\vy|\vgamma)$. Bayarri \& Garc\'ia-Donato (2005)
suggest univariate quadrature or Monte Calro integration to approximate $p(\vy|\vgamma)$. \cite{Liang2008} suggest Laplace approximation this choice
which they also show avoids Problem 3. However, whatever method is used to 
approximate $p(\vy|\vgamma)$ will lead to a computational overhead and potential
numerical issues we would like to avoid.

Finally, \cite{Liang2008} proposed the hyper-$g/n$ prior given by
\begin{equation}\label{eq:hyperGonN}
\ds p(g) = \frac{a - 2}{2n}\left( 1 + \frac{g}{n} \right)^{-a/2}I(g>0),
\end{equation}

\noindent where again $a>2$. 


$$
\begin{array}{rl}
p(\vy|\vgamma) 
& \ds 
= K(n) \frac{a - 2}{2n}  \int_0^\infty 
\left( 1 + \frac{g}{n} \right)^{-a/2}
(1 + g)^{(n-p_\vgamma-1)/2} \left[ 1 + g (1 - R_\vgamma^2) \right]^{-(n-1)/2}  dg
\\
& \ds = K(n) \frac{a - 2}{2n}  \int_0^1 
(1 - u)^{p/2 + a/2 - 2  } \left[ 1 - u \left(1  -  \tfrac{1}{n} \right) \right]^{-a/2} \left(  1 - u R^2\right)^{-(n-1)/2} du

%\\
%& \ds \approx K(n) \frac{a - 2}{2n}  \int_0^\infty 
%(1 + g)^{(n-p_\vgamma-1)/2} \left[ 1 + g (1 - R_\vgamma^2) \right]^{-(n-1)/2}  dg
%\\
%& \ds =  K(n) \frac{a - 2}{n p_\vgamma}  
%{}_2F_1\left[
%\frac{n-1}{2},1; \frac{p_\vgamma}{2} ; R_\vgamma^2
%\right]
\end{array} 
$$

\noindent Employing 
Equation 3.211 of \cite{Gradshteyn2007} (see Appendix A) leads to
\begin{equation}\label{eq:hyperGonNmarginal}
\ds p_{g/n}(\vy|\vgamma) =  \frac{K(n)(a - 2)}{n(p_\vgamma + a - 2)} F_1\left( 1, \frac{a}{2}, \frac{n-1}{2}; \frac{p_\vgamma + a}{2}; 1  -  \frac{1}{n}, R_\vgamma^2 \right).
\end{equation}

 
\noindent While \cite{Liang2008} showed that the hyper-$g/n$ prior is model
selection consistent, the expression (\ref{eq:hyperGonNmarginal}) is extremely
difficult to evaluate numerically since the second last argument of the above 
$F_1$ is asymptotically close to the radius of convergence of the $F_1$
function.

Suppose that we treat the second last argument as equal to 1. Then
$$
\begin{array}{rl}
F_1\left( 1, \frac{a}{2}, \frac{n-1}{2}; \frac{p_\vgamma + a}{2}; \frac{n - 1}{n}, R_\vgamma^2 \right) 
& \ds \approx F_1\left( 1, \frac{a}{2}, \frac{n-1}{2}; \frac{p_\vgamma + a}{2}; 1, R_\vgamma^2 \right)
\\
& \ds = \frac{\Gamma(\frac{p_\vgamma + a}{2})\Gamma(\frac{p_\vgamma}{2} - 1)}{\Gamma(\frac{p_\vgamma + a}{2} - 1)\Gamma(\frac{p_\vgamma}{2})} {}_2F_1\left( \frac{n-1}{2}, 1; \frac{p_\vgamma}{2};  R_\vgamma^2 \right)
\end{array} 
$$

\noindent which follows from Schlosser (2013) Equation 36.

\cite{Liang2008} again suggest Laplace approximation 
for this choice of prior. 

Given the closeness of the 2nd last argument above
we will approximate the marginal by
$$
\begin{array}{rl}
p(\vy|\vgamma) 

\end{array} 
$$ 

\noindent  3.197(3)
$$
\begin{array}{rl}
p(\vy|\vgamma) 
& \ds \approx K(n) \frac{a - 2}{2n}  \int_0^1 
(1 - u)^{p/2 - 2} \left(  1 - u R_\vgamma^2\right)^{-c} du
\\
& \ds = K(n) \frac{a - 2}{n(p - 2)} {}_2 F_1\left[
\frac{n-1}{2}, 1; \frac{p}{2}; R_\vgamma^2
\right]
\end{array} 
$$ 




Next we will consider the prior 
\begin{equation}\label{eq:betaPrime}
\ds p(g) = \frac{g^{b}(1 + g)^{-(a+b+2)}}{\mbox{Beta}(a+1,b+1)} I(g>0),
\end{equation}

\noindent proposed by \cite{Maruyama2011} where $a>-1$ and $b>-1$. 
This is a Pearson Type VI or beta-prime distribution. More specifically, 
$g\sim \mbox{Beta-prime}(b+1,a+1)$ using the typical parametrization of 
the beta-prime distribution (see Keeping, 1962; Johnson et al., 1995).
Then
$$
%\begin{array}{rl}
\ds p(\vy|\vgamma) 
%& \ds = \int_0^\infty                                         
%\frac{g^{b}(1 + g)^{-a-b-2}}{\mbox{Beta}(a+1,b+1)}
%K(n) (1 + g)^{(n - p - 1)/2}\left[ 1 + g(1-R^2) \right]^{-(n-1)/2}
%dg
%\\
%& \ds 
=
\frac{K(n)}{\mbox{Beta}(a+1,b+1)}
\int_0^\infty             
g^{b}(1 + g)^{(n - p_\vgamma - 1)/2 - (a + b + 2)}  (1 + g \widehat{\sigma}_\vgamma^2 )^{-(n-1)/2}  
dg.
%\end{array}
$$

\noindent If we choose $b$ such that
$a+b+2 = (n - p_\vgamma - 1)/2$, implying
$b = (n - p_\vgamma - 5)/2 - a$, then the exponent of the $(1 + g)$ term in the equation above is zero.
Using Equation 3.194 (iii) of \cite{Gradshteyn2007}
(see Appendix A) we obtain
\begin{equation}\label{eq:marginalLikelihood}
\begin{array}{rl}
\ds p(\vy) 
%& \ds =
%\frac{K(n)}{\mbox{Beta}(a+1,b+1)}
%\int_0^\infty g^{b} \left[ 1 + g(1-R^2) \right]^{-(n-1)/2}  
%dg
%\\ [2ex]
& \ds 
=  K(n)
\frac{\mbox{Beta}(p/2 + a + 1,b + 1)}{\mbox{Beta}(a+1,b+1)} (\widehat{\sigma}^2)^{-(b + 1)}
\\ [2ex]
& \ds = \widetilde{K}(n,a)

\Gamma(p/2 + a + 1)\Gamma(a + b + 2)
(\widehat{\sigma}^2)^{-(b + 1)}
\end{array}
\end{equation}

\noindent where
$\widetilde{K}(n,a) = 
K(n)/[\Gamma( (n-1)/2 )\Gamma(a + 1)]$
is a common factor for all models for the purpose of model comparison.

Note that (\ref{eq:marginalLikelihood}) proportional
to a special case of the prior structure considered by \cite{Maruyama2011}
who refer to this special case as ZE as a model selection criterion (after Zellner's $g$ prior). This choice of $b$ also ensures that $g = \BigO(n)$ so that $\tr\{\Var(\vbeta | g, \sigma^2)\} = \BigO(1)$, preventing Bartlett's paradox. 
Note that in comparison to previously discussed priors
marginal likelihood only involves gamma functions which
are well behaved from a numerical analysis perspective. 
\cite{Maruyama2011} showed the prior (\ref{eq:betaPrime}) leads to model
selection consistency.
For derivation of the above properties and further discussion see \cite{Maruyama2011}. 

Finally, the robust prior of Bayarri et al. (2012) using the default
choices of the prior hyperparameters uses 
$$
p(g) = \frac{1}{2}\left(\frac{1+n}{1 + p_\vgamma}  \right)^{1/2} (1 + g)^{-3/2}I(g > L)
$$
 
\noindent where $L = (1 + n)/(1 + p_\vgamma) - 1$. This leads to 
\begin{equation}\label{eq:yGivenG}
\ds p(\vy|\vgamma)
\ds = \frac{K(n)}{2}\left(\frac{1+n}{1 + p_\vgamma}  \right)^{1/2} 
\int_L^\infty  (1 + g)^{(n - p_\vgamma)/2 - 2}(  1 + g \widehat{\sigma}_\vgamma^2)^{-(n-1)/2} dg.
\end{equation}

\noindent Using the substitution 
$x = (1 + L)/(g - L)$ followed by Equation 3.197(5) of \cite{Gradshteyn2007}
(see Appendix A) leads to
\begin{equation}\label{eq:yGivenGammaRobust}
\ds p(\vy|\vgamma)
 \ds = K(n) \left( \frac{n + 1}{ p_\vgamma + 1} \right)^{ - p_\vgamma/2} \frac{(\widehat{\sigma}_\vgamma^2)^{-(n-1)/2}}{p_\vgamma+1}
{}_2F_1\left[ \frac{n-1}{2}, \frac{p_\vgamma+1}{2}; \frac{p_\vgamma+3}{2}  ; 
\frac{(1  - 1/\widehat{\sigma}_\vgamma^2)(p_\vgamma + 1)}{1 + n} \right],
\end{equation}

\noindent which is the same expression as Equation 26. of Bayarri et al. (2012) modulo notation
and division by $K(n)$.
This is difficult to deal with numerically for two reasons. Firstly, either of the first 
two arguments of the ${}_2F_1$ function are large relative to the third will often lead to numerical overflow problems. Secondly,
and more problematically, when $\widehat{\sigma}_\vgamma^2$ becomes small the last argument
of ${}_2F_1$ function can become less than $-1$ which falls outside the radius of convergence
of the ${}_2F_1$ function. The {\tt BayesVarSel} package deals with these problems using numerical
integration.
 
Instead suppose we begin with $x = g - L$ then
$$
\ds p(\vy|\vgamma)
\ds = \frac{K(n)}{2}\left(\frac{1+n}{1 + p_\vgamma}  \right)^{1/2} (\widehat{\sigma}_\vgamma^2)^{-(n-1)/2}
\int_0^\infty  (1 + L + h)^{(n - p_\vgamma)/2 - 2}\left[  \frac{1 + L\widehat{\sigma}_\vgamma^2}{\widehat{\sigma}_\vgamma^2} + h \right]^{-(n-1)/2} dh.
$$

\noindent Use of Equation 3.197(1) of \cite{Gradshteyn2007} leads to
\begin{equation}\label{eq:yGivenGammaRobust2}
\ds p(\vy|\vgamma)
\ds = K(n) \left( \frac{1 + n}{1 + p_\vgamma} \right)^{(n - p_\vgamma - 1)/2} \frac{\left( 1 + L\widehat{\sigma}_\vgamma^2 \right)^{-(n - 1)/2}}{1 + p_\vgamma}
{}_2F_1\left[
\frac{n-1}{2}, 1; \frac{p_\vgamma+3}{2}; \frac{1 - \widehat{\sigma}_\vgamma^2}{1 + L\widehat{\sigma}_\vgamma^2}
\right].
\end{equation}

\noindent This expression is numerically far more stable to evaluate. Due to simplifications
we have $0\le \widehat{\sigma}_\vgamma^2<1$, we also have $L>0$ so that the last argument
of the ${}_2F_1$ above is bounded in the unit interval. Furthermore, using
(\ref{eq:logGuassHypergeometric}) means that (\ref{eq:yGivenGammaRobust2}) can be evaluated
on the log scale, avoiding numerical overflow.


%$$
%\int_0^\infty x^{\nu - 1}(\beta + x)^{-\mu}(x + \gamma)^{-\varrho} dx
%= \beta^{-\mu}\gamma^{\nu - \varrho}\mbox{Beta}(\nu,\mu-\nu + \varrho)
%{}_2F_1(\mu,\nu;\mu + \varrho; 1 - \gamma/\beta)
%$$

%$$
%\beta = \frac{1 + L\widehat{\sigma}_\vgamma^2}{\widehat{\sigma}_\vgamma^2}
%$$
%$$
%\gamma = 1 + L
%$$
%$$
%\nu = 1
%$$
%$$
%\mu = \frac{n-1}{2}
%$$
%$$
%\varrho = - (n - p_\vgamma - 4)/2
%$$

\newpage 

\section{Variational Bayes}

\noindent We will consider VB with factorization
$$
q(\alpha,\vbeta,\sigma^2,g) = q(\alpha,\vbeta,\sigma^2) q(g) 
$$

\noindent Let
$$
m = \bE_q(\alpha),
\quad 
s^2 = \mbox{Var}_q(\alpha),
\quad 
\vmu = \bE_q(\vbeta), 
\quad 
\mSigma = \mbox{Cov}_q(\vbeta),
\quad 
\tau_\sigma = \bE_q(\sigma^{-2}),
\quad 
\mbox{and}
\quad 
\tau_g = \bE_q(g^{-1}).
$$

\noindent Let
$$
\begin{array}{rl}
\ds \log \underline{p}(\vy,\alpha,\vbeta,\sigma^2) 
& \ds = \bE_q\left[ \log\left( \frac{p(\vy,\alpha,\vbeta,\sigma^2)}{q(g)} \right) \right]
\\ [2ex]
& \ds = \bE_q\left[ \log p(\vy,\alpha,\vbeta,\sigma^2) \right] + E
\\
& \ds = 
- \tfrac{n}{2}\log(2\pi)
- \left( \tfrac{n}{2} + 1 \right) \log(\sigma^2)
- \tfrac{1}{2\sigma^2}\|\vy - \vone\alpha - \mX\vbeta\|^2
\\
& \ds \qquad 
- \tfrac{p}{2}\log(2\pi\sigma^2)
- \tfrac{p}{2}\bE_q[\log(g)]
+ \tfrac{1}{2}\log|\mX^T\mX|
- \tfrac{\tau_g}{2\sigma^2} \vbeta^T\mX^T\mX\vbeta 
 + E
\end{array}
$$

\noindent where $E = - \bE_q\left[ \log q(g) \right]$. 
$$
\ds \underline{p}(\alpha|\vy,\sigma^2) \sim N(0,\sigma^2/n)
$$

\noindent 
Then
$$
\begin{array}{rl}
\ds \underline{p}(\vy,\vbeta,\sigma^2) 
& \ds = \int \exp\Big[
- \tfrac{n}{2}\log(2\pi)
- \left( \tfrac{n}{2} + 1 \right) \log(\sigma^2)
- \tfrac{n\alpha^2}{2\sigma^2}
- \tfrac{1}{2\sigma^2}\|\vy - \mX\vbeta\|^2
\\
& \ds \qquad 
+ \tfrac{1}{2}\log|\mX^T\mX|
- \tfrac{\tau_g}{2\sigma^2} \vbeta^T\mX^T\mX\vbeta 
- \tfrac{p}{2}\log(2\pi\sigma^2)
- \tfrac{p}{2}\bE_q[\log(g)] + E
\Big] d\alpha 
\\
& \ds = \exp\Big[
- \tfrac{n-1}{2}\log(2\pi)
- \left( \tfrac{n-1}{2} + 1 \right) \log(\sigma^2)
- \tfrac{1}{2}\log(n)
- \tfrac{1}{2\sigma^2}\|\vy - \mX\vbeta\|^2
\\
& \ds \qquad 
+ \tfrac{1}{2}\log|\mX^T\mX|
- \tfrac{\tau_g}{2\sigma^2} \vbeta^T\mX^T\mX\vbeta 
- \tfrac{p}{2}\log(2\pi\sigma^2)
- \tfrac{p}{2}\bE_q[\log(g)] + E
\Big]   
\end{array}
$$

\noindent So that
$$
\ds \underline{p}(\vbeta|\vy,\sigma^2) \sim N\left[
(1 + \tau_g)^{-1}\widehat{\vbeta},
\sigma^2 (1 + \tau_g)^{-1} \left( \mX^T\mX \right)^{-1}
\right]
$$

\noindent Next,
$$
\begin{array}{rl}
\ds \underline{p}(\vy,\sigma^2) 
& \ds = \frac{1}{\sqrt{n}(2\pi)^{(n-1)/2}} \int \exp\Big[
- \left( \tfrac{n-1}{2} + 1 \right) \log(\sigma^2)
- \tfrac{n}{2\sigma^2} 
+ \tfrac{1}{\sigma^2}\vy^T\mX\vbeta
- \tfrac{1}{2\sigma^2}(1 + \tau_g) \vbeta^T\mX^T\mX\vbeta
\\
& \ds \qquad  
+ \tfrac{1}{2}\log|\mX^T\mX|
- \tfrac{p}{2}\log(2\pi\sigma^2)
- \tfrac{p}{2}\bE_q[\log(g)] + E
\Big] d\vbeta 
\\
& \ds = \frac{1}{\sqrt{n}(2\pi)^{(n-1)/2}} \exp\Big[
- \left( \tfrac{n-1}{2} + 1 \right) \log(\sigma^2)
- \tfrac{n}{2\sigma^2} 
\\
& \ds \qquad 
+ \tfrac{1}{2}\log|\mX^T\mX|
+ \tfrac{1}{2}\log|2\pi\widetilde{\mSigma}|
+ \tfrac{1}{2}\widetilde{\vmu}^T\widetilde{\mSigma}^{-1}\widetilde{\vmu}
- \tfrac{p}{2}\log(2\pi\sigma^2)
- \tfrac{p}{2}\bE_q[\log(g)] + E
\Big] 
\\
& \ds = \frac{1}{\sqrt{n}(2\pi)^{(n-1)/2}} \exp\Big[
- \left( \tfrac{n-1}{2} + 1 \right) \log(\sigma^2)
- \tfrac{p}{2}\log(1 + \tau_g)
- \tfrac{p}{2}\bE_q[\log(g)]
- \tfrac{n}{2\sigma^2} \left\{ 1 - (1 + \tau_g)^{-1}  R^2
\right\} + E
\Big] 
\end{array}
$$

\noindent where
$$
\begin{array}{rl}
\ds \widetilde{\mSigma} 
& \ds = \left[ (1 + \tau_g)\sigma^{-2} \mX^T\mX \right]^{-1} 
\\
& \ds = \sigma^2 (1 + \tau_g)^{-1} \left( \mX^T\mX \right)^{-1}
\\
\ds \widetilde{\vmu} 
& \ds = \mSigma\mX^T\vy\sigma^{-2} 
\\
& \ds = (1 + \tau_g)^{-1} \left( \mX^T\mX \right)^{-1}\mX^T\vy 
\\
& \ds = (1 + \tau_g)^{-1}\widehat{\vbeta}
\end{array}
$$

\noindent Then
$$
\underline{p}(\sigma^2|\vy) \sim \mbox{IG}\left[
\tfrac{n-1}{2},
\tfrac{n}{2} \left\{ 1 - (1 + \tau_g)^{-1}  R^2
\right\}
\right]
$$

\noindent 
If $\vx|z \sim N(\vmu,z\mV)$ and $z\sim \mbox{IG}(a,b)$ then
$\vx \sim t_{2a}( \vmu, (b/a)\mV)$. 
Hence,
$$
\ds \underline{p}(\alpha|\vy) \sim t_{n-1}\left[ 0, \tfrac{1}{n-1} \left\{ 1 - (1 + \tau_g)^{-1}  R^2
\right\}
\right]
$$
$$
\ds \underline{p}(\vbeta|\vy) \sim t_{n-1}\left[
(1 + \tau_g)^{-1}\widehat{\vbeta},
\tfrac{n}{n-1} (1 + \tau_g)^{-1} \left\{ 1 - (1 + \tau_g)^{-1}  R^2 \right\}  \left( \mX^T\mX \right)^{-1}
\right]
$$

\noindent Hence,
$$
\ds \tau_\sigma = \frac{n-1}{n \left\{ 1 - (1 + \tau_g)^{-1}  R^2 \right\}}
$$
$$
\vmu = (1 + \tau_g)^{-1}\widehat{\vbeta}
$$
$$
\mSigma = \tfrac{n}{n-1} (1 + \tau_g)^{-1} \left\{ 1 - (1 + \tau_g)^{-1}  R^2 \right\}  \left( \mX^T\mX \right)^{-1}
$$

\noindent The $q$-density for $g$ has the form
$$
q(g) \sim \mbox{IG}\left[
\frac{p+1}{2}, 
\frac{n}{2} + \frac{\tau_\sigma}{2}\left( \vmu^T\mX^T\mX\vmu + \mbox{tr}(\mSigma\mX^T\mX) 
\right)
\right]
$$

\noindent Now we have
$$
\vmu^T\mX^T\mX\vmu = (1 + \tau_g)^{-2} nR^2
$$
$$
\mbox{tr}(\mSigma\mX^T\mX)  = 
\tfrac{n}{n-1} (1 + \tau_g)^{-1} \left\{ 1 - (1 + \tau_g)^{-1}  R^2 \right\} p = (1 + \tau_g)^{-1} \tau_\sigma^{-1} p
$$

$$
\begin{array}{rl}
\ds \tau_g 
& \ds = \frac{p+1}{n + \tau_\sigma\left( \vmu^T\mX^T\mX\vmu + \mbox{tr}(\mSigma\mX^T\mX) \right) }
\\ [2ex]
& \ds = \frac{p+1}{n + \tau_\sigma (1 + \tau_g)^{-2} nR^2 + (1 + \tau_g)^{-1} p }
\end{array}
$$


$$
\tau_g\tau_\sigma (1 + \tau_g)^{-1} nR^2 = \tau_g(\tau_\sigma n  + 1 - n) 
$$
$$
 \tau_g \tau_\sigma (1 + \tau_g)^{-1} nR^2    = (n - 1 - \tau_g n)(1 + \tau_g) - \tau_g p
$$


$$
\tau_g\tau_\sigma n  + \tau_g   = n - 1  + n\tau_g - \tau_g - \tau_g^2 n - \tau_g p
$$
 
\section{Monte Carlo methods}
\label{sec:MonteCarlo}

We now develop  
Monte Carlo (MC) and Rao-Blackwellization
schemes for parameter posterior estimation.
In Section \ref{sec:Exact} we derive
exact expressions for most posterior quantities of interest in terms of special
functions. However, such special functions can suffer from numerical instability.
The following Monte Carlo schemes can be applied when numerical instability
issues arise in the evaluation of special functions or when
asymptotic approximations fail. 

We will now show how to obtain direct Monte Carlo samples of the posterior distribution.
Combining Equation (\ref{eq:yGivenG}), the prior for $g$ in
(\ref{eq:priorStructure}) and (\ref{eq:marginalLikelihood})
obtain
\begin{equation}\label{res:06}
p(g|\vy) = \frac{(\widehat{\sigma}^2)^{b+1} g^{b} \left[  1 + g \widehat{\sigma}^2 \right]^{-(b + d + 2)}}{\mbox{Beta}(b + 1,d+1)}I(g>0),
\end{equation}

\noindent where $c = (n-1)/2$ and $d = c - b - 2 = p/2 + a$. 
This is a generalized beta prime distribution 
or generalized beta distribution of the second kind and
is a special case of the compound gamma distribution
(see Dubey, 1970; McDonald, 1984). [{\bf Mark, can you check the previous sentence?}] The fact that $\widetilde{g} = g \widehat{\sigma}^2 \sim \mbox{Beta-prime}(b+1,d+1)$,
and
if $B\sim \mbox{Beta}(s,t)$ then $B/(1 - B)\sim \mbox{Beta-prime}(s,t)$, allow us to easily draw samples from
$g|\vy$. \\


McDonald, J.B. (1984) Some generalized functions for the size distributions of income, Econometrica 52, 647--663. \\

Dubey, Satya D. (December 1970). Compound gamma, beta and F distributions. Metrika. 16: 27--31.  \\





%\joc{ 
%	$$
%	u = \frac{g}{1+g},
%	\quad 
%	g = \frac{u}{1-u},
%	\quad 
%	1 - u = \frac{1}{(1 + g)},
%	\quad 
%	(1 - u)^{-1} = (1 + g),
%	\quad \mbox{and}\quad 
%	\frac{du}{dg} = \frac{1}{1+g} - \frac{g}{(1 +g)^2} 
%	= \frac{1}{(1+g)^2} = (1 - u)^2.
%	$$
%}

The quantity $u=g/(1+g)$ will occur frequently in the coming text and so it will be useful to derive $p(u|\vy)$.
Via a change of variables we have
\begin{equation}\label{eq:uGiveY}
p(u|\vy) = \frac{(1 -  R^2)^{b+1}}{\mbox{Beta}(b + 1,d+1)} 
u^{b}  (1 - u)^d (  1 -  uR^2 )^{-(b + d + 2)}, \quad 0<u<1.
\end{equation}

\noindent This is a special case of the
Gauss hypergeometric (GH) distribution
of Armero and Bayarri (1994).
More specifically,
$u|\vy \sim GH(b+1,d+1,b+d+2,-R^2)$
where if $x \sim GH(p,q,r,\lambda)$
then the density of $x$ is
$$
p(x) = \frac{x^{p-1}(1 - x)^{q-1}(1 + \lambda x)^{-r}}{\mbox{Beta}(p,q){}_2F_1(r,p;p+q; -\lambda)}
$$

\noindent To show (\ref{eq:uGiveY}) is a special case
of the GH distribution, we use the
Euler's identity 
${}_{2}F_{1}(a,b;c;z)=(1-z)^{c-a-b}\,{}_{2}F_{1}(c-a,c-b;c;z)$
and the fact that
${}_2F_1(0,b;c;z) = {}_2F_1(a,0;c;z) = 1$
to obtain 
${}_2F_1(b+d+2,b+1;b+d+2; R^2) = (1 -  R^2)^{-(b+1)}$.
We can sample from $u|\vy$ by sampling instead from
$g|\vy$ and then applying the transformation $u=g/(1+g)$.




 
In order to sample from $\sigma^2|\vy$ 
we first sample $g^* \sim g|\vy$ and then obtain
a  sample from $\sigma^2|\vy,g^*$.
We obtain $\sigma^2|\vy,g$
by combining (\ref{eq:yGivenSigma2andG}) with the prior for $\sigma^2$
in (\ref{eq:priorStructure}) and normalizing to obtain
\begin{equation}\label{eq:sigma2GivenYandG}
\ds \sigma^2|\vy,g 
\sim \mbox{IG}\left( \tfrac{n-1}{2},\tfrac{n}{2}\left( 1 - \tfrac{g}{1+g} R^2\right) \right).
\end{equation}

%\noindent For a given $g$ it is straightforward
%to sample from $\sigma^2|\vy,g$. 

Next, it is straightforward to show that 
\begin{equation}\label{eq:coefficientsGivenYandG}
\alpha|\vy,\sigma^2 \sim N\left( 0, \sigma^2/n \right), 
\qquad  \mbox{and} \qquad 
\vbeta|\vy,\sigma^2,g \sim \N\left(         
\tfrac{g}{1+g}\widehat{\vbeta},    
\tfrac{g}{1+g} \sigma^2 \left( \mX^T\mX \right)^{-1} 
\right),                              
\end{equation}

\noindent where $\widehat{\vbeta} = \left( \mX^T\mX \right)^{-1} \mX^T\vy$ is the least squares 
estimate for $\vbeta$.


Using the fact that $p(\vbeta|\vy,g) = \int p(\vbeta|\vy,\sigma^2,g)p(\sigma^2|\vy,g) d \sigma^2$ 
along with standard calculations leads to
$$
\alpha|\vy,g \sim t_{n-1}\left[ 0, \tfrac{n}{n-1}  \left( 1 - \tfrac{g}{1+g} R^2\right) (\mX^T\mX)^{-1} \right]
$$
\begin{equation}\label{eq:betaGivenYandU}
\ds \vbeta|\vy,g \sim t_{n-1}\left[
\tfrac{g}{1+g} \widehat{\vbeta},
 \tfrac{n}{n-1}  \tfrac{g}{1+g} \left( 1 - \tfrac{g}{1+g} R^2\right) (\mX^T\mX)^{-1}
\right].
\end{equation}

We now have all of the ingredients to generate Monte Carlo samples from the full posterior distribution of all model parameters. 
To do so we use the following steps 
\begin{enumerate}
	\item Sample $B \sim \mbox{Beta}(b+1,d+1)$.
	
	\item Calculate $\widetilde{g} = B/(1 - B)$. (Noting that $\widetilde{g} = g \widehat{\sigma}^2 \sim \mbox{Beta-prime}(b+1,d+1)$).
	
	\item Calculate $g = \widetilde{g}/\widehat{\sigma}^2$. (Noting that this value of $g$ is a sample from $g|\vy$).
	
	\item Sample $\sigma^2|\vy,g$ using  (\ref{eq:sigma2GivenYandG}).
	
	\item Sample from $\alpha|\vy,\sigma^2$ and 
	$\vbeta|\vy,\sigma^2,g$ using (\ref{eq:coefficientsGivenYandG}).
	
	%\item Sample from $\vbeta|\vy,g$ using (\ref{eq:betaGivenYandU}).
\end{enumerate}

Note we only require the summary 
statistics $\widehat{\vbeta}$, $\left( \mX^T\mX \right)^{-1}$, $n$, $p$ and $R^2$
to sample from the posterior distribution for all model parameters. Once these quantities are calculated the cost of drawing samples
from the posterior distribution is minimal. However, calculation
of credible intervals requires dealing with mixtures of distributions.

Instead of performing inferences based on MC samples
we can exploit Rao-Blackwellization leading to more
efficient estimators of the parameter posterior distributions. Let $g_1,\ldots,g_N$ be $N$ independent draws 
obtained from performing steps 1. to 3. above $N$ times. Then
the Rao-Blackwellized estimator for $p(\sigma^2|\vy)$ is
$$
\ds 
p_{\mbox{\scriptsize RB}}(\sigma^2|\vy)
= \frac{1}{N} \sum_{i=1}^{N} p(\sigma^2|\vy,g_i).
$$

\noindent Similarly let $\sigma_1^2,\ldots,\sigma_N^2$ be $N$ draws
obtained from performing steps 1. to 4. $N$ times.
Then the Rao-Blackwellized estimator for $p(\alpha|\vy)$ is
$$
\ds 
p_{\mbox{\scriptsize RB}}(\alpha|\vy)
= \frac{1}{N} \sum_{i=1}^{N} p(\alpha|\vy,\sigma_i^2).
$$

\noindent We have two potential Rao-Blackwellized
estimators for $\vbeta|\vy$. These are
$$
p(\vbeta|\vy) \approx
p_{\mbox{\scriptsize RB,1}}(\vbeta|\vy)
= \frac{1}{N} \sum_{i=1}^{N} p(\vbeta|\vy,\sigma_i^2,g_i)
\qquad \mbox{and} \qquad
p(\vbeta|\vy) \approx
p_{\mbox{\scriptsize RB,2}}(\vbeta|\vy)
= \frac{1}{N} \sum_{i=1}^{N} p(\vbeta|\vy,g_i),
$$

\noindent where $p_{\mbox{\scriptsize RB,2}}(\vbeta|\vy)$ has a smaller Monte Carlo error compared to 
$p_{\mbox{\scriptsize RB,1}}(\vbeta|\vy)$.

%Note we only require the summary 
%statistics $\widehat{\vbeta}$, $\left( \mX^T\mX \right)^{-1}$, $n$, $p$ and $R^2$
%to sample from the posterior distribution for all model parameters. Once these quantities %are calculated the cost of drawing samples
%from the posterior distribution is minimal. However, calculation
%of credible intervals requires dealing with mixtures of distributions.


\section{Exact posterior distributions in terms of special functions}
\label{sec:Exact}
 
In this section we derive the exact expressions for most Bayesian inferential quantities
of interest based on the model presented in Section 2. An exception is
the exact expression for the posterior density for $\vbeta|\vy$ for this quantity we
instead derive the first and second moments.


First, we use the fact that $p(\sigma^2|\vy) = \int_0^\infty p(\sigma^2|\vy,u) p(u|\vy)dg$ so that the density
$\sigma^2|\vy$ may be expressed as
$$
\begin{array}{rl}
\ds p(\sigma^2|\vy) 
%& \ds = 
%\left[ \frac{\left(\tfrac{n}{2}\right)^c}{\Gamma(c)} (\sigma^2)^{-(c+1)} \exp\left\{ - \frac{n}{2\sigma^2} \right\} \right] 
%\left[ \frac{(1 -  R^2)^{b+1} }{\mbox{Beta}(d + 1,b+1)} %\right] \int_0^\infty 
% g^{b} (1 + g)^{-c}  \exp\left(   \frac{g}{1+g} \frac{nR^2}{2\sigma^2} \right) dg.
%\\
& \ds = 
\left[ \frac{\left(\tfrac{n}{2}\right)^c}{\Gamma(c)} (\sigma^2)^{-(c+1)} \exp\left\{ - \frac{n}{2\sigma^2} \right\} \right] 
\left[ \frac{(\widehat{\sigma}^2)^{b+1} }{\mbox{Beta}(d + 1,b+1)} \right]
\int_0^1
u^{b} (1 - u)^d  \exp\left(  \frac{nR^2}{2\sigma^2} u \right) du.
\end{array}
$$

%\noindent \joc{
%$$
%u = \frac{g}{1+g},
%\quad 
%g = \frac{u}{1-u},
%\quad 
%1 - u = \frac{1}{(1 + g)},
%\quad 
%(1 - u)^{-1} = (1 + g),
%\quad \mbox{and}\quad 
%\frac{du}{dg} = \frac{1}{1+g} - \frac{g}{(1 +g)^2} 
%= \frac{1}{(1+g)^2} = (1 - u)^2.
%$$
%
%\noindent Using the change of variables $u = g/(1 + g)$ the integral
%above is  
%$$
%\ds \int_0^1 u^{b} (1 - u)^d  \exp\left(  %\frac{nR^2}{2\sigma^2} u \right) du
%= \mbox{Beta}(d+1,b+1) {}_1 F_1\left(b + 1; c; %\frac{nR^2}{2\sigma^2} \right)
%$$
%
%$$
%	\int_{0}^u x^{\nu - 1} (u - x)^{\mu - 1}  e^{\beta x} dx = \mbox{Beta}(\nu,\mu) {}_1 F_1(\nu;\mu+\nu;\beta u) \quad   \mbox{(assuming $\mbox{Re}(\mu)>0$ and $\mbox{Re}(\nu)>0$).}
%$$
%\noindent with $u = 1$, $\nu = b + 1$, $\mu = a + p/2 + 1 >0$, and $\beta = nR^2/(2\sigma^2)$.
%}

\noindent 
Utilising Equation 3.383(i) of \cite{Gradshteyn2007} (see Appendix A) we obtain  
$$
%\begin{array}{rl}
\ds p(\sigma^2|\vy) 
%& 
\ds = 
\frac{\left(\tfrac{n}{2}\right)^c (\widehat{\sigma}^2)^{b+1}}{\Gamma(c)} (\sigma^2)^{-(c+1)} \exp\left(  - \frac{n}{2\sigma^2} \right)
{}_1 F_1\left(b + 1; c; \frac{nR^2}{2\sigma^2} \right),
%\end{array}
$$

\noindent where 
${}_1 F_1(\alpha;\gamma;z)$ 
is the confluent hypergeometric function (also called
Kummer's function of the first kind).
To the best of our knowledge, this is a new positive valued continuous distribution.
An alternative representation using the identity ${}_1F_1(a;b;z) = e^z {}_1F_1(b-a;b;-z)$ results in
$$
%\begin{array}{rl}
\ds p(\sigma^2|\vy) 
%& 
\ds = 
\frac{\left(\tfrac{n}{2}\right)^c (\widehat{\sigma}^2)^{b+1}}{\Gamma(c)} (\sigma^2)^{-(c+1)} \exp\left(  - \frac{n \widehat{\sigma}^2}{2\sigma^2} \right)
{}_1 F_1\left( d + 1; c; -\frac{nR^2}{2\sigma^2} \right),
%\end{array}
$$

\noindent which is much more amenable to numerical evaluation. 
 
 
Next we have
$p(\alpha|\vy) = \int_0^\infty p(\alpha|\vy,\sigma^2)p(\sigma^2|\vy) d\sigma^2$.
% where $\alpha|\vy,\sigma^2 \sim N(0,\sigma^2/n)$. 
After we apply the change of variables $\sigma^2 = 1/\tau$
we obtain
$$
\begin{array}{rl}
\ds p(\alpha|\vy) 
%& \ds =
%\joc{ \frac{\left(\tfrac{n}{2}\right)^c (1 -  R^2)^{b+1}  }{\Gamma(c)\sqrt{2\pi/n}}
%\int_0^\infty 
%(\sigma^2)^{-1/2} \exp\left( -\frac{n\alpha^2}{2\sigma^2} \right) 
%(\sigma^2)^{-(c + 1)}\exp\left( -\frac{n}{2\sigma^2}  \right)  {}_1 F_1\left( b + 1; c; %\frac{nR^2}{2\sigma^2} \right)  d\sigma^2
%}
%\\ [2ex]
%& \ds =  \frac{\left(\tfrac{n}{2}\right)^c (1 -  R^2)^{b+1}  }{\Gamma(c)\sqrt{2\pi/n}} 
%\times 
%\int_0^\infty 
%(\sigma^2)^{-c -3/2} \exp\left( -\frac{n(1+\alpha^2)}{2\sigma^2} \right)   {}_1 F_1\left( b + %1; c; \frac{nR^2}{2\sigma^2} \right) d\sigma^2
%\\
& \ds = \frac{\left(\tfrac{n}{2}\right)^c (1 -  R^2)^{b+1}  }{\Gamma(c)\sqrt{2\pi/n}} 
\times 
\int_0^\infty 
\tau^{c - 1/2} \exp\left( -\frac{n(1+\alpha^2)}{2} \tau \right)   {}_1 F_1\left( b + 1; c; \frac{nR^2}{2} \tau \right) d \tau.
\end{array}
$$


%$$
%\begin{array}{rll}
%\ds \int_0^\infty e^{-st} t^{b-1} {}_1F_1(a,c,kt) dt
%& \ds = \Gamma(b)s^{-b} F(a,b,c,k s^{-1}) 
%& \mbox{(if $|s| > |k|$)}
%\\
%& \ds = \Gamma(b)(s - k)^{-b} F\left(c - a, b; c; \frac{k}{k - s} %\right)
%& \mbox{(if $|s - k| > |k|$)}
%\end{array} 
%$$

%\noindent assuming that $\mbox{Re}(b)>0$, $\mbox{Re}(s) > %\max(0,\mbox{Re}(k))$. 
%In the above case
%$$
%b \leftrightarrow c + 1/2, \quad 
%s \leftrightarrow \frac{n(1+\alpha^2)}{2} , \quad
%a \leftrightarrow b + 1, \quad 
%c \leftrightarrow c, \quad \mbox{and} \quad
%k \leftrightarrow  \frac{nR^2}{2}.
%$$


%\noindent where the second line above follows from the 
%change of variables $\sigma^2 = 1/\tau$.

%\joc{
%\noindent Equation 7.621 (4) of \cite{Gradshteyn2007} 
%(the second line corrected from EH I, 269(5))
%is

%
%\noindent The first holds since
%$$
%|s| = \frac{n(1+\alpha^2)}{2}  > |k| = \frac{nR^2}{2} \qquad %\mbox{(since $R^2 \in [0,1]$ and $\alpha^2\ge 0$)},
%$$
%
%\noindent Checking the other conditions
%$$
%b \leftrightarrow c + 1/2 = (n-1)/2 + 1/2 = n/2 > 0.
%$$
%}

\noindent 
Utilizing  Equation 7.621 (4) of \cite{Gradshteyn2007} we obtain
\begin{equation}\label{eq:alphaGivenY}
\begin{array}{rl}
\ds p(\alpha|\vy) 
& \ds = \frac{ \Gamma\left( c + \tfrac{1}{2} \right) (1 -  R^2)^{b+1}  }{ \Gamma(c)\sqrt{\pi}} 
\left( 1+\alpha^2 \right)^{-n/2} {}_2F_1\left( b+1, c + \frac{1}{2}; c; \frac{R^2}{1+\alpha^2}  \right),
\end{array}
\end{equation}

\noindent where ${}_2F_1$ is the Gaussian hypergeometric function.

The first three arguments of the ${}_2F_1$ function in (\ref{eq:alphaGivenY}) are $O(n)$. This makes 
numerical evaluation of (\ref{eq:alphaGivenY}) difficult in general. In order to handle this difficulty
we will first use the Euler transform
Euler's identity 
${}_{2}F_{1}(a,b;c;z)=(1-z)^{c-a-b}\,{}_{2}F_{1}(c-a,c-b;c;z)$ leading to
\begin{equation}\label{eq:alphaGivenY_2}
	\begin{array}{rl}
		\ds p(\alpha|\vy) 
		& \ds = \frac{ \Gamma\left( c + \tfrac{1}{2} \right)   }{ \Gamma(c)\sqrt{\pi}} 
		\frac{(1 -  R^2)^{b+1}}{(  \alpha^2 + 1 - R^2 )^{b+3/2}}
		\left( 1+\alpha^2 \right)^{ - (d+1) }   
		
		 {}_2F_1\left( d + 1,  - \frac{1}{2}; c; \frac{R^2}{1+\alpha^2}  \right),
	\end{array}
\end{equation}

\noindent which is again more numerically stable to evaluate
when $n$ is large.
 
Unfortunately, we have not been able to find a closed form expression
for $\vbeta|\vy$ in terms of known special functions. However,
it is possible, for fixed $\vbeta$ to integrate $\int p(\vbeta|\vy,u)p(u|\vy)du$
using  (\ref{eq:uGiveY})  and (\ref{eq:betaGivenYandU})
%, i.e.,
%$$
%\begin{array}{rl}
%\ds p(\vbeta|\vy)
%& \ds =  \frac{(1 -  R^2)^{b+1} \Gamma\left( \tfrac{n+p-1}{2}\right)
%	|\mX^T\mX|^{1/2} n^{(n+p-1)/2} }{\mbox{Beta}(b + 1,d+1) \Gamma\left( \tfrac{n-1}{2} \right) n^{p/2}\pi^{p/2}  } \\
%& \ds \quad \times \int_0^1
%u^{b - p/2 + (n+p-1)/2}  (1 - u)^d (  1 -  uR^2 )^{-(b + d + 2)-p/2 + (n+p-1)/2}  
%\left[ n u \left( 1 - u R^2\right) + (\vbeta - u\widehat{\vbeta})^T
%\mX^T\mX
%(\vbeta - u\widehat{\vbeta}) \right]^{-(n+p-1)/2} du
%\end{array}
%$$ 
via univariate numerical integration methods.
Instead we will concentrate on deriving the first and second moments of $\vbeta|\vy$.


In order to calculate the first and second posterior moments for $\vbeta$
we use the laws of total expectation and variance respectively, i.e.,
%$\bE(\vbeta|\vy) = \bE_{g|\vy}\left[ \vbeta|\vy,g \right]$, and
%$\mbox{Cov}(\vbeta|\vy)  = \bE_{g|\vy}\left[
%\mbox{Cov}(\vbeta|\vy,g)
%\right] + \mbox{Cov}_{g|\vy}\left[
%\bE(\vbeta|\vy,g)
%\right]$ 
%or equivalently 
$\bE(\vbeta|\vy) = \bE\left[ \bE(\vbeta|\vy,u) |\vy \right]$, and
$\mbox{Cov}(\vbeta|\vy)  = \bE\left[
\mbox{Cov}(\vbeta|\vy,g)|\vy
\right] + \mbox{Cov}\left[
\bE(\vbeta|\vy,u)|\vy
\right]$ 
to obtain
$$
\begin{array}{rll}
\ds \bE(\vbeta|\vy) 
& \ds = M_1 \widehat{\vbeta}
\\ %[1ex]
\ds \mbox{Cov}(\vbeta|\vy) 
%& \ds = \bE_{u|\vy}\left[ \frac{n - 1}{n - 3}
%\left(\frac{n}{n-1}\right) u \left( 1 - u R^2\right) %(\mX^T\mX)^{-1}
%\right] + \mbox{Cov}_{u|\vy}\left[
%u \widehat{\vbeta} \right] 
%\\ [2ex]
& \ds = \tfrac{n}{n - 3} 
\left( M_1 - M_2 R^2\right) (\mX^T\mX)^{-1}
+ (M_2 - M_1^2) \widehat{\vbeta}  \widehat{\vbeta}^T
\end{array}
$$

\noindent where
$M_1 
%= \bE\left[  \left. \frac{g}{1 + g}  \right| \vy\right] 
= \bE(u|\vy)$
and
%\qquad \mbox{and} \qquad 
$M_2 
%= \bE\left[ \left. \left(\frac{g}{1 + g}\right)^2   \right| \vy \right]
= \bE(u^2|\vy)$.


Using the normalizing constant for the GH distribution,
Euler's identity and properties of the Beta function we obtain
\begin{equation}\label{eq:M1}
\begin{array}{rl}
\ds M_1 
%& \ds \joc{ = \frac{\mbox{Beta}(b + 2,d+1)}{\mbox{Beta}(b + 1,d + 1)} (1 -  R^2)^{b+1} {}_2F_1(b+d+2,b+2;b+d+3; R^2)
%}
%\\
%& \ds \joc{= 
%	\frac{\mbox{Beta}(d+1, b+2)}{\mbox{Beta}(d + 1, b + 1)} 
%	(1 - R^2)^{b+1}
%	{}_2 F_1 (c, b + 2; c + 1; R^2)
%}
%\\ [2ex]
%& \ds  \joc{=    \frac{\mbox{Beta}(d+1, b+2)}{\mbox{Beta}(d + %1, b + 1)} 
%	(1 - R^2)^{b+1}
%	\times (1 - R^2)^{c + 1 - c - (b+2)}
%	{}_2 F_1 (c + 1 - c, c + 1 - (b + 2); c + 1; R^2)        
%}
%\\ [2ex]
%& \ds  \joc{=    
%	\frac{\Gamma(b+2)}{\Gamma(c + 1)}
%	\frac{\Gamma(c)}{\Gamma(b+1)}
%	{}_2 F_1 (d + 1, 1; c + 1; R^2)        
%}
%\\ [2ex]
& \ds  =    
\frac{b + 1}{c}
{}_2 F_1 (d + 1, 1; c + 1; R^2)    
\end{array} 
\end{equation}

%\noindent Note that the same results can be obtained using 
%\cite{Gradshteyn2007} Equation 3.197 (5).

\noindent 
and,
\begin{equation}\label{eq:M2}
\begin{array}{rl}
\ds M_2
%& \ds \joc{ = \frac{(1 - R^2)^{b+1}}{\mbox{Beta}(d + 1, b + 1)} 
%	\int_0^\infty g^{b+2} (1 + g)^{-2} [1 + g (1 - R^2)]^{-c} dg 
%}
%\\ [2ex]
%& \ds \joc{= 
%\frac{\mbox{Beta}(d + 1, b+3) }{\mbox{Beta}(d + 1, b + 1)} 
%(1 - R^2)^{b+1} {}_2 F_1 (c, b+3; c + 2; R^2)
%}
%\\
%& \ds \joc{= 
%\frac{\Gamma(b+3)}{\Gamma(c + 2)}
%\frac{\Gamma(c)}{\Gamma(b+1)}
%{}_2 F_1(d+1, 2; c+ 2; R^2)    
%}
%\\
& \ds =
\frac{(b+1)(b+2)}{c(c+1)}    {}_2 F_1(d+1, 2; c+ 2; R^2).    
\end{array} 
\end{equation}

 
%\newpage 

\section{Approximations}
\label{sec:approximations}

While the results in the previous section are exact, they are not particularly intuitive since these results depend on special functions. We will use asymptotic approximations to gain a better understanding of results from previous sections.
The asymptotic framework that we will use assumes that $R^2$ is a
fixed constant, and $p/n \to 0$ as $n$ diverges.
First, we will show how model averaging using prior structure (\ref{eq:priorStructure}) 
is asymptotically equivalent to model averaging using BIC. Second, we will show 
asymptotically that the parameter posteriors take simple parametric forms.





\subsection{Asymptotic forms of parameter posterior distributions}

We will now provide asymptotic approximations for $p(\alpha|\vy)$,
$p(\vbeta|\vy)$, and $p(\sigma^2|\vy)$. We will start with $p(\sigma^2|\vy)$.
To give some intuition behind how $p(\sigma^2|\vy)$ behaves asymptotically 
note that $b + 1 = c \left[ 1 - (p + 2a + 2)/(n-1) \right]$ so that
$b + 1 = c + O(p/n)$. Hence, if $p/n$ is small then  
$$
\begin{array}{rl}
\ds p(\sigma^2|\vy) 
%& 
%\ds \approx
%\frac{\left(\tfrac{n}{2}\right)^c (1 -  R^2)^{c}}{\Gamma(c)} %(\sigma^2)^{-(c+1)} \exp\left(  - \frac{n}{2\sigma^2} \right)
%{}_1 F_1\left(c; c; \frac{nR^2}{2\sigma^2} \right)
%\\
& \ds 
\approx \frac{\left(\tfrac{n}{2}\right)^c (1 -  R^2)^{c}}{\Gamma(c)} (\sigma^2)^{-(c+1)} \exp\left(  - \frac{n(1 - R^2)}{2\sigma^2} \right)
\end{array}
$$

\noindent which follows from the identity ${}_1 F_1(a,a,x) = e^x$, i.e.,
\begin{equation}\label{eq:sigma2givenYapprox}
\ds \sigma^2|\vy \stackrel{\mbox{\scriptsize approx.}}{\sim} \mbox{IG}\left( \tfrac{n-1}{2}, \tfrac{n}{2}\widehat{\sigma}^2 \right)
\end{equation}

\noindent where $\widehat{\sigma}^2 = 1 - R^2$ is the typical MLE for $\sigma^2$ under the 
simplifying assumptions on $\vy$ and $\mX$ stated in Section 2. 
%From the above expression we
%clearly see Bernstien von Mises Theorem come into effect.

Next, consider the series expansion for the Gaussian hypergeometric function 
given by
$$
{}_2 F_1 (a, b; c; z)        
= 1 + \frac{ab}{c} z + \frac{a(a + 1)b(b+1)}{c(c+1)} \frac{z^2}{2} + \ldots
$$

\noindent In the context of the expressions for $M_1$,
$$
M_1 = \frac{b + 1}{c}
\left[ 
1 + \frac{d+1}{c+1} R^2 + \frac{ (d+1)(d + 2) }{(c+1)(c+2)} R^4 + \ldots
\right].  
$$

\noindent 
We have $d+1 = O(1)$ and $c = O(n)$ the series expansion converges rapidly when either 
$p/n$ or $R^2$ is small. Since each term in the expansion is positive it follows
that $M_1$ is a
strictly monotonic increasing function of $R^2$. Using the fact that 
$$
{}_2F_1 (a,b;c;1)= \frac{\Gamma(c)\Gamma(c-a-b)}{\Gamma(c-a)\Gamma(c-b)}, \qquad   
$$

\noindent provided $c>a+b$. By using properties of the gamma function
 for the special case that $R^2 = 0$ and 
$R^2=1$ we have
$$
{}_2 F_1 (d + 1, 1; c + 1; 0) = 1 \quad \mbox{and} \quad  
{}_2 F_1 (d + 1, 1; c + 1; 1) = \frac{c}{b + 1}.
$$

\noindent Hence and similarly,
$$
\ds \frac{b + 1}{c} \le M_1 \le 1, \qquad \mbox{and} \qquad \frac{(b+1)(b+2)}{c(c+1)} \le M_2 \le 1.
$$

\noindent Hence,
$$
\begin{array}{rl}
0 \le \mbox{Var}(u|\vy) 
& \ds \le 1 - \frac{(b+1)^2}{c^2}
\\
& \ds = \frac{(n - 1)^2 - [(n - 1) - (p  + 2a + 4)]^2}{(n - 1)^2}
\\
& \ds 
= \frac{2(p  + 2a + 4)}{n - 1} - \frac{(p  + 2a + 4)^2}{(n - 1)^2}
\\
& \ds = O(p/n)
\end{array} 
$$

\noindent It then follows that $M_1 \to 1$ and $M_2\to 1$ either
as $n\to \infty$ or as $R^2 \to 1$. Furthermore, as $n\to \infty$ or as $R^2 \to 1$ 
we have $\mbox{Var}(u|\vy) \to 0$.

$$
p(u) = \exp(f(u))
$$
$$
p'(u) = f'(u)\exp(f(u))
$$
$$
p''(u) = [f''(u) + f'(u)^2]\exp(f(u))
$$

$$
\bE( g(u) ) \approx g(\widehat{u})  + \tfrac{1}{2} \mbox{Var}(u) g''(\widehat{u}) 
$$


$$
\begin{array}{rll}
\log p(\sigma^2|\vy,u) 
& \ds = \left( \tfrac{n-1}{2} \right) \log\left[
\tfrac{n}{2}\left( 1 - u R^2\right) \right]
- \log\Gamma\left[ \frac{n-1}{2} \right]
- \left( \frac{n-1}{2} + 1\right) \log(\sigma^2)
- \tfrac{n}{2}\left( 1 - u R^2\right)   \sigma^{-2}
\\
& \ds = \left( \tfrac{n-1}{2} \right) \log
\left( 1 - u R^2\right) 
- \tfrac{n}{2}\left( 1 - u R^2\right)   \sigma^{-2}
+ \mbox{constants in $u$}
\end{array}
$$ 
 
$$
\begin{array}{rll}
p(\sigma^2|\vy) 
& \ds = \int p(\sigma^2|\vy,u) p(u|\vy) du 
\\
& \ds = 
\bE[p(\sigma^2|\vy,u)|\vy]
\\
& \ds 
\approx p(\sigma^2|\vy,\bE(u|\vy))
+ \tfrac{1}{2}\mbox{Var}(u|\vy) \left[ \frac{\d^2 p(\sigma^2|\vy,u)}{d u^2} \right]_{u = \bE(u|\vy)}
\\
& \ds \approx
\mbox{IG}\left( \tfrac{n-1}{2},\tfrac{n}{2}\left( 1 - M_1 R^2\right) \right)
\qquad \mbox{(First order delta method)}
\\
& \ds 
\approx p(\sigma^2|\vy,u=M_1)
+ \tfrac{1}{2}(M_2 - M_1^2) \left[ 
\left\{
\frac{\d^2 \log p(\sigma^2|\vy,u)}{\d u^2}
+ \left( \frac{\d \log p(\sigma^2|\vy,u)}{\d u} \right)^2
\right\}
p(\sigma^2|\vy,u) \right]_{u = M_1}
\\
& \ds 
\propto \mbox{IG}\left( \tfrac{n-1}{2},\tfrac{n}{2}\left( 1 - M_1 R^2\right) \right)
+ \tfrac{1}{2}(M_2 - M_1^2)  
\tfrac{n^2 R^4}{2} (\sigma^2)^{-2}
p(\sigma^2|\vy,M_1) 
\\
& \ds 
= \mbox{IG}\left( \tfrac{n-1}{2},\tfrac{n}{2}\left( 1 - M_1 R^2\right) \right)
+ \tfrac{1}{2}(M_2 - M_1^2)  
\tfrac{n^2 R^4}{2} \mbox{IG}\left( \tfrac{n+1}{2},\tfrac{n}{2}\left( 1 - M_1 R^2\right) \right)
\end{array}
$$


$$
\ds \sigma^2|\vy  
\stackrel{\mbox{\scriptsize approx.}}{\sim} 
\mbox{IG}\left( \tfrac{n-1}{2},\tfrac{n}{2}\left( 1 - M_1 R^2\right) \right)
$$

%\begin{itemize}
%	\item $\ds \bE\left[ \frac{g}{g+1} \Big| \vy \right] = \bE(u|\vy) = M_1 \to 1$; and
%	
%	
%	\item $\ds \mbox{Var}\left[ \frac{g}{g+1} \Big| \vy \right] = \mbox{Var}(u|\vy) = M_2 - M_1^2 \to 0$.
%	
%\end{itemize}
%
\noindent Hence, the posterior distribution for $u$ behaves like a point mass
as $n\to \infty$ or as $R^2 \to 1$. In the context of $\sigma^2|\vy,g$ 
if we were to simply substitute $g/(1+g) = 1$ into Equation 
(\ref{eq:sigma2GivenYandG}) we would obtain Equation (\ref{eq:sigma2givenYapprox}).


If $\vx|z \sim N(\vmu,z\mV)$ and $z\sim \mbox{IG}(a,b)$ then
$\vx \sim t_{2a}( \vmu, (b/a)\mV)$. 
We have $a = (n - 1)/2$, $b = (n/2)\left( 1 - M_1 R^2\right)$, $\vmu = 0$, and $\mV = \sigma^2/n$. Hence,
$$
\alpha|\vy \stackrel{\mbox{\scriptsize approx.}}{\sim} t_{n-1}\left( 0, \frac{1 - M_1 R^2}{n-1} \right)
$$

Similarly, using the delta method on (\ref{eq:betaGivenYandU}) leads to
\begin{equation}\label{eq:betaGivenYandU}
\ds \vbeta|\vy \stackrel{\mbox{\scriptsize approx.}}{\sim} t_{n-1}\left[
M_1 \widehat{\vbeta},
\left(\tfrac{n}{n-1}\right) M_1  \left( 1 - M_1  R^2\right) (\mX^T\mX)^{-1}
\right].
\end{equation}
%\noindent Again, we see a Bernstein von Mises theorem coming into effect.

 




 
\subsection{Asymptotic equivalence of model averaging via BIC and via prior structure (\ref{eq:priorStructure})}


\noindent We will now cast all of our previous results into the context of model averaging.
In order to do so we introduce some notation.
We index the space of models by a $p$-dimensional vector of indicator variables for each variable considered. 


\noindent where the priors placed on $\alpha$ and $\sigma^2$ remain unchanged, and
the hyperparameter $b$ for the prior on $g$ is replaced with $b=(n - q - 5)/2 - a$
where $q = |\vgamma|$, and $\delta(x;a)$ is the Dirac delta function for $x$ with
location $a$. 

For the purposes of this paper we will assume a flat prior of $\vgamma$, i.e., 
$p(\vgamma) = 2^{-p}$, although extension to other priors on $\vgamma$ is
straightforward.

All previous results concerning posterior distributions for $\alpha$, $\vbeta$, $\sigma^2$ and $g$ conditioned on a particular model $\vgamma$ require small changes, i.e.,
by replacing $\widehat{\vbeta}$ with $\widehat{\vbeta}_\vgamma$ (the MLE under model
$\vgamma$), $(\mX^T\mX)^{-1}$ with $(\mX_\vgamma^T\mX_\vgamma)^{-1}$ , $p$ with $q = |\vgamma|$, and $R^2$ with $R_\vgamma^2$ (the R-squared value
under model $\vgamma$). Further, all previous expressions would be conditioned on
$\vgamma$.

The marginal likelihood is given by
$p(\vy) = \sum_{\vgamma} p(\vy|\vgamma)p(\vgamma)$
where $\sum_{\vgamma}$ denotes summation over all possible models $\vgamma$.
The posterior probability for model $\vgamma$ is given by
$$
\ds p(\vgamma|\vy) = \frac{p(\vy|\vgamma)p(\vgamma)}{\sum_{\vgamma'} p(\vy|\vgamma')p(\vgamma')}
$$

\noindent and the parameter posterior distributions are then given by
$$
p(\alpha|\vy) = \sum_{\vgamma} p(\alpha|\vy,\vgamma)p(\vgamma|\vy),
\quad
p(\vbeta|\vy) = \sum_{\vgamma} p(\vbeta|\vy,\vgamma)p(\vgamma|\vy) 
\quad \mbox{and} \quad 
p(\sigma^2|\vy) = \sum_{\vgamma} p(\sigma^2|\vy,\vgamma)p(\vgamma|\vy). 
$$

\noindent An alternative representation for $p(\vgamma|\vy)$ under a flat prior is
$$
\ds p(\vgamma|\vy) = \frac{
\exp\left[
-\tfrac{1}{2} \left\{ 
\widetilde{\mbox{BIC}}(\vgamma) - \widetilde{\mbox{BIC}}(\vgamma^*) 
\right\} \right]
}{
\sum_{\vgamma'} \exp\left[
-\tfrac{1}{2} \left\{ \widetilde{\mbox{BIC}}(\vgamma') - \widetilde{\mbox{BIC}}(\vgamma^*) \right\} \right]
}
$$

\noindent where $\widetilde{\mbox{BIC}}(\vgamma) = - 2\log p(\vy|\vgamma)$ and
$\vgamma^* \in \{0,1\}^p$ corresponds to any model in the model space.

Consider two such models indexed by $\vgamma_1$ and $\vgamma_2$ respectively.
Let $\widehat{\sigma}_k^2$, $R_k^2$, $q_k$, $\mbox{BIC}_k$ be the MLEs for $\sigma^2$, 
R-squared values, model sizes and BIC scores corresponding to $k\in\{1,2\}$.
Then treating the $R^2$ values and $q$'s constant and letting $n$ diverge we have
$$
\begin{array}{rl}
\widetilde{\mbox{BIC}}(\vgamma_2) - \widetilde{\mbox{BIC}}(\vgamma_1)
& \ds = 
- 2(b_1 + 1)\log(1 - R_1^2) 
- 2\log\left( \Gamma(d_1 + 1)\Gamma(c + q_1/2) \right)
\\
& \ds \qquad + 2(b_1 + 1)\log(1 - R_2^2) + 2\log\left( \Gamma(d_2 + 1)\Gamma(c + q_2/2) \right)
\\ [1ex]
& \ds = n\log(1 - R_2^2) - n\log(1 - R_1^2)  + 2\log\left( \frac{\Gamma(c + q_2/2
)}{\Gamma(c + q_1/2)} \right) 
+ 2\log\left( \frac{\Gamma(d_2 + 1)}{\Gamma(d_1 + 1)} \right)
 \\
& \ds \qquad + 
(q_1 - 3 + 2a)\log(1 - R_1^2) - (q_2 - 3 + 2a)\log(1 - R_2^2)

\end{array} 
$$

$$
b = \frac{n - q - 5}{2} - a
$$
$$
a + b + 2 = \frac{n - q - 1}{2} = c + q/2
$$
$$
d = q/2 + a
$$

\noindent where the second line is obtained by absorbing terms only depending on $q_1$, $q_2$, $R_1$, $R_2$ and $a$ into 
the $O(1)$ term. Using Stirling's asymptotic expansion
of $\ln\Gamma(z)$ for large
$z$, i.e.,
$\ln \Gamma (z) = z\ln(z)-z -(1/2)\ln(z) + (1/2)\log(2\pi) + O(z^{-1})$ \citep[see for example Equation 6.1.37 of][]{Abramowitz1972},
we obtain  
$$
\begin{array}{rl}
\ds \log\left( \frac{\Gamma(a + b_2 + 2)}{\Gamma(a + b_1 + 2)} \right)
& \ds = (a + b_2 + 2)\log(a + b_2 + 2) - (a + b_1 + 2)\log(a + b_1 + 2)
+ \log\left( \frac{a + b_1 + 2}{a + b_2 + 2} \right) + O(n^{-1})
\\
& \ds = (a + b_2 + 2)\log\left[ n \left( \frac{1}{2} - \frac{q_2 + 1}{2n}  \right) \right] - (a + b_1 + 2)\log\left[ n \left( \frac{1}{2} - \frac{q_1 + 1}{2n}  \right) \right] \\
& \ds \qquad 
+ \log\left( 1 +  \frac{b_1 - b_2}{a + b_2 + 2} \right) + O(n^{-1})
\\
& \ds = \tfrac{1}{2} (q_2 - q_1)\log(n) 
+ (a + b_2 + 2)\log\left( \frac{1}{2} - \frac{q_2 + 1}{2n}  \right)
\\
& \ds \qquad 
- (a + b_1 + 2)\log\left( \frac{1}{2} - \frac{q_1 + 1}{2n}  \right)
+ \log\left( 1 +  \frac{b_1 - b_2}{a + b_2 + 2} \right) + O(1).
\end{array} 
$$

\noindent For the second, third and fourth terms above
using Taylor series expansions around the $O(n^{-1})$ terms inside
the log functions it is easy to see that the 
second, third and fourth terms above are $O(1)$, $O(1)$ and $O(n^{-1})$
respectively. Hence,
$$
\begin{array}{rl}
\widetilde{\mbox{BIC}}(\vgamma_2) - \widetilde{\mbox{BIC}}(\vgamma_1)
& \ds = n\log(1 - R_2^2) - n\log(1 - R_1^2) +  (q_2 - q_1)\log(n)   + O(1)
\\
& \ds = \mbox{BIC}(\vgamma_2) - \mbox{BIC}(\vgamma_1) + O(1),
\end{array} 
$$

\noindent where $\mbox{BIC}(\vgamma) = n\log(1 - R_\vgamma^2)  + p\log(n)$ is
the BIC under model $\vgamma$.

Hence, using $\log p(\vy)$ as a model selection criterion is
first order equivalent to BIC and shares all first order optimality properties
of the BIC (see for example, Yang, 2005). Furthermore, when using a uniform
prior on the model space, i.e., $p(\vgamma) = 1/2^p$, Bayesian model averaging 
via the prior structure (\ref{eq:priorStructure}) is asymptotically equivalent
to model averaging via BIC.

 

 



 
 
\newpage 
 

 
\section{Implementation}
\label{sec:implementation}

Key to the feasibility of the model selection and averaging is an efficient implementation of the procedures. We employ two main 
strategies to achieve computational efficiency (i) efficient software implementation using
highly optimized software libraries; and (ii) efficient calculation of
$R$-squared values for all models based on using a Gray code and appropriate
matrix algebraic simplifications.
For ease of use we 
implemented an {\tt R} package called {\tt BLAH}.
This internals of {\tt BLAH} is implemented
in {\tt C++} and uses the {\tt R} packages \texttt{Rcpp} and \texttt{RcppEigen} to enhance
computational performance.  

\subsection{Gray code} 
\label{sec:GrayCode}


The Gray code was originally developed to aid in detecting errors in analog to digital conversions in
communications systems by Frank Gray in 1947. It is a sequence of binary numbers whose key feature is that
one and only one binary digit is different between binary numbers in the sequence. 
Suppose we index the space of models $\mGamma \in \{0,1\}^{2^p \times p}$ using binary numbers and let $\vgamma_i$ and $\vgamma_{i+1}$ be any two consecutive rows of $\mGamma$,
then
$$
\| \vgamma_i - \vgamma_{i+1} \| = 1
$$

\noindent for most standard vector norms.
 
Gray codes can be constructed using a sequence of ``reflect'' and ``prefix'' steps.
Let $\mGamma_1 = (0,1)^T \in \{0,1\}^{2\times 1}$ be the first Gray code matrix matrix and let $\mGamma_k$ be the $k$th Gray code matrix. Then we can obtain the $k+1$th Gray code matrix given $\mGamma_k$ via 
$$
\ds \mGamma_{k+1} = \left[\begin{array}{cc}
\vzero & \mGamma_k \\
\vone  & \mbox{reflect}(\mGamma_k)
\end{array} \right]
$$ 

\noindent where $\mbox{reflect}(\mGamma_k)$ is the matrix obtained by reversing the order of rows of $\mGamma_k$, and the $\vzero$ and $\vone$ are vectors of zeros and ones
of length $2^k$ respectively. In {\tt C++} this Gray codes can be efficiently constructed
using bit-shit operations on binary strings.

Gray codes allows the enumeration of the entire model space in an order which only adds
or removes one covariate from the previous model at a time. We can then use standard matrix
inverse results to perform rank
one updates and downdates in the calculation of the $R^2$, $(\mX^T\mX)^{-1}$ and
$\widehat{\vbeta}$ values for each model in the
model space.

\subsection{Updates and downdates} 

Both updates and downdates depend on the fact that
the inverse of a real symmetric matrix can be written as
\begin{eqnarray}
\ds \left[ \begin{array}{cc}
\mA   & \mB \\
\mB^T & \mC
\end{array} \right]^{-1}
&  = &
\ds \left[ \begin{array}{cc}
\mI & \vzero \\
-\mC^{-1}\mB^T &  \mI
\end{array} \right]
\left[ \begin{array}{cc}
\widetilde{\mA} & \vzero \\
\vzero & \mC^{-1}
\end{array} \right]
\left[ \begin{array}{cc}
\mI    & -\mB\mC^{-1}\\
\vzero & \mI
\end{array} \right] \label{eq:blockdiag1}\\
&  = &
\ds\left[
\begin{array}{cc}
\widetilde{\mA}
& - \widetilde{\mA}\mB\mC^{-1} \\
-\mC^{-1}\mB^T\widetilde{\mA}
& \mC^{-1} + \mC^{-1}\mB^T\widetilde{\mA}\mB\mC^{-1}
\end{array}\right]\label{eq:blockdiag2}
\end{eqnarray}

\noindent where $\widetilde{\mA} = \left(\mA-\mB\mC^{-1}\mB^T\right)^{-1}$
provided all inverses in (\ref{eq:blockdiag1}) and
(\ref{eq:blockdiag2}) exist. 
For both the update and downdate formula we assume that the quantities
$\mX^T\vy$, $\mX^T\mX$, $(\mX_{\vgamma_i}^T\mX_{\vgamma_i})^{-1}$, 
$\widehat{\vbeta}_{\vgamma_i}$ and $R_{\vgamma_i}^2$ values have been computed from the previous step.

We want to update the model inverse matrix, coefficient vector and $R^2$ values for the model $\vgamma_{i+1}$ where $\mX_{\vgamma_{i+1}}$ is the matrix given by $\mX_{\vgamma_i}$ with a column $\vz$ inserted into the appropriate position.
For clarity of exposition we will assume that the column $\vz$ is inserted into the lasts position, i.e., $\mX_{\vgamma_{i+1}} = [\mX_{\vgamma_{i}},\vz]$.
Then the updates for the model inverse matrix, coefficient estimates, and $R^2$ values can be obtained by following the steps bellow.
\begin{enumerate}
	\item Calculate $\widehat{\vz} = (\mX_{\vgamma_i}^T\mX_{\vgamma_i})^{-1}\mX_{\vgamma_i}^T\vz$, 
	$\kappa 
	%= 
	%1/(\vz^T(\mI - \mX_{\vgamma_i}(\mX_{\vgamma_i}^T\mX_{\vgamma_i})^{-1}\mX_{\vgamma_i}^T)\vz) 
	= 1/(n - \vz^T\widehat{\vz})$, and  $s = \vy^T(\vz - \widehat{\vz})$.
	
	\item The model inverse matrix can be updated via  
	%The update for the $(\mX_{\vgamma_{i+1}}^T\mX_{\vgamma_{i+1}})^{-1}$ using $(\mX_{\vgamma_{i}}^T\mX_{\vgamma_{i}})^{-1}$ is
	%given by the following. 
	$$
	\begin{array}{rl}
	%\left[ \begin{array}{cc}
	%\mX^T\mX & \mX^T\vz \\
	%\vz^T\mX & \vz^T\vz \\
	%\end{array} \right]^{-1}
	(\mX_{\vgamma_{i+1}}^T\mX_{\vgamma_{i+1}})^{-1}
	%& \ds = 
	%\left[ \begin{array}{cc}
	%(\mX^T\mX)^{-1} + c(\mX^T\mX)^{-1}\mX^T\vz\vz^T\mX(\mX^T\mX)^{-1}  & -(\mX^T\mX)^{-1}\mX^T\vz c \\
	%-c\vz^T\mX(\mX^T\mX)^{-1}              
	%& c
	%\end{array} \right]
	%\\
	%& \ds = 
	%\left[ \begin{array}{cc}
	%(\mX^T\mX)^{-1} + c\widehat{\vz}\widehat{\vz}^T  & - \widehat{\vz} c \\
	%-c\widehat{\vz}^T             
	%& c
	%\end{array} \right]
	
	\ds = 
	\left[ \begin{array}{cc}
	(\mX_{\vgamma_{i}}^T\mX_{\vgamma_{i}})^{-1}    & \vzero \\
	\vzero             
	& 0
	\end{array} \right] + \kappa \left[ \begin{array}{r}
	\widehat{\vz} \\
	-1 \\
	\end{array} \right] \left[ \begin{array}{r}
	\widehat{\vz} \\
	-1 \\
	\end{array} \right]^T.
	\end{array} 
	$$
	
	\item
	The coefficient estimators 
	%are given by
	%$ 
	%\ds \widehat{\vbeta}_{\vgamma_{i}} = %(\mX_{\vgamma_{i}}^T\mX_{\vgamma_{i}})^{-1}\mX_{\vgamma_{i}}^T\vy$,
	%and $\ds \widehat{\vbeta}_{\mbox{\scriptsize update}}  = (\mX_{\vgamma_{i+1}}^T\mX_{\vgamma_{i+1}})^{-1}\mX_{\vgamma_{i+1}}^T\vy$.  
	%Then using the block inverse formula we have
	can be updated via
	$$
	\begin{array}{rl}
	\ds \widehat{\vbeta}_{\vgamma_{i+1}}
	%& \ds = \left[ \begin{array}{c}
	%\widehat{\vbeta} \\
	%0 
	%\end{array} \right] + c \left[ \begin{array}{cc}
	%(\mX^T\mX)^{-1}\mX^T\vz \left\{ \vz^T\mX(\mX^T\mX)^{-1}\mX^T\vy - %\vz^T\vy\right\}  \\
	%\vz^T\vy - \vz^T\mX(\mX^T\mX)^{-1}\mX^T\vy
	%\end{array} \right]
	%\\
	%& \ds 
	= \left[ \begin{array}{c}
	\widehat{\vbeta}_{\vgamma_{i}} \\
	0 
	\end{array} \right] - \kappa s  \left[ \begin{array}{r}
	\widehat{\vz}   \\
	- 1
	\end{array} \right].
	\end{array} 
	$$
	
	\item The $R^2$ value van be update via
	% for each model (under the 
	%standardization given in Section \ref{sec:model}) are given by
	%$R_{\vgamma_{i}}^2 = \tfrac{1}{n} \vy^T\mX_{\vgamma_{i}}(\mX_{\vgamma_{i}}^T\mX_{\vgamma_{i}})^{-1}\mX_{\vgamma_{i}}^T\vy$
	%and
	%$R_{\vgamma_{i+1}}^2 = \tfrac{1}{n}\vy^T\mX_{\vgamma_{i+1}}(\mX_{\vgamma_{i+1}}^T\mX_{\vgamma_{i+1}})^{-1}\mX_{\vgamma_{i+1}}^T\vy$.
	%Then using the block inverse formula
	$$
	\begin{array}{rl}
	\ds 
	R_{\vgamma_{i+1}}^2 
	%& \ds = \tfrac{1}{n} \vy^T\mX(\mX^T\mX)^{-1}\mX^T\vy
	%+ \tfrac{c}{n}\left[ 
	%\widehat{\vy}\vz\vz^T\widehat{\vy}
	%- 2\widehat{\vy}\vz\vz^T\vy 
	%+ \vy^T\vz\vz^T\vy 
	%\right]
	= R_{\vgamma_{i}}^2
	+ \frac{\kappa s^2}{n}.
	
	\end{array}
	$$
	
	%\item
	%\noindent The model determinants are given by
	%$D = |\mX^T\mX|$
	%and
	%$D_{\mbox{\tiny update}} = |\mC^T\mC|$.
	%Using the block determinant formula we have
	%$D_{\mbox{\tiny update}} = D/c$.
\end{enumerate}

\noindent Presuming  have been precomputed
the above updates costs $O(|\vgamma_{i}|^2 + n)$ time.

Suppose want to downdate the model summary quantities for the model
$\vgamma_{i+1}$ where $\mX_{\vgamma_{i+1}}$ is the matrix given by 
$\mX_{\vgamma_i}$ with a column $\vz$ removed from the appropriate position.
Similarly as for updates for clarity of exposition we will assume that
$\vz$ will be removed from the last column of $\mX_{\vgamma_i}$, i.e., 
$\mX_{\vgamma_{i}} = [\mX_{\vgamma_{i+1}}, \vz]$. 
Then the downdates for model summary values are given by the following steps.
\begin{enumerate}
	\item 
	%The downdate for the model inverse matrix to obtain $(\mX_{\vgamma_{i+1}}^T\mX_{\vgamma_{i+1}})^{-1}$
	%from $(\mX_{\vgamma_{i}}^T\mX_{\vgamma_{i}})^{-1}$ can be found using the block-inverse formula.
	Suppose we partition the matrix
	 $(\mX_{\vgamma_{i+1}}^T\mX_{\vgamma_{i+1}})^{-1}$ so that
	$$
	\ds (\mX_{\vgamma_{i}}^T\mX_{\vgamma_{i}})^{-1} 
	= \left[ \begin{array}{cc}
	\mA   & \vb \\
	\vb^T & c \\
	\end{array} \right].
	%= 
	%\left[ \begin{array}{cc}
	%\mX^T\mX & \mX^T\vz \\
	%\vz^T\mX & \vz^T\vz \\
	%\end{array} \right]^{-1}
	$$
	
	\noindent The model inverse matrix downdate can be obtained
	via   $(\mX_{\vgamma_{i+1}}^T\mX_{\vgamma_{i+1}})^{-1} = \mA - c^{-1}\vb\vb^T$.
	
	\item Calculate
	$\widehat{\vz} = (\mX_{\vgamma_{i+1}}^T\mX_{\vgamma_{i+1}})^{-1}\mX_{\vgamma_{i+1}}^T\vz$ using step 1.,
	$\kappa = 1/(n - \vz^T\widehat{\vz})$,
	and $s = \vy^T(\vz - \widehat{\vz})$.
	
	\item 
	%Let $\widehat{\vbeta}_{\vgamma_{i+1}} = (\mX_{\vgamma_{i+1}}^T\mX)^{-1}\mX_{\vgamma_{i+1}}^T\vy$
	%and $\widehat{\vbeta}_{\vgamma_{i}} = %(\mX_{\vgamma_{i}}^T\mX_{\vgamma_{i}})^{-1}\mX_{\vgamma_{i}}^T\vy$.
	%Then
	The coefficient estimates downdate can be obtained
	via
	$$
	\widehat{\vbeta}_{\vgamma_{i+1}} = \left[ \widehat{\vbeta}_{\vgamma_{i}} \right]_{-|{\vgamma_{i}}|} + \kappa s\widehat{\vz},
	$$
	
	\noindent where $[ \widehat{\vbeta}_{\vgamma_{i}}]_{-|{\vgamma_{i}}|}$
	removes the last column from $\widehat{\vbeta}_{\vgamma_{i}}$.
	
	\item 
	%Let $R_{\vgamma_{i}}^2 = \tfrac{1}{n}\vy^T\mX_{\vgamma_{i}}(\mX_{\vgamma_{i}}^T\mX_{\vgamma_{i}})^{-1}\mX_{\vgamma_{i}}^T\vy$
	%and $R_{\vgamma_{i+1}}^2 = \tfrac{1}{n} \vy^T\mX_{\vgamma_{i+1}}(\mX_{\vgamma_{i+1}}^T\mX_{\vgamma_{i+1}})^{-1}\mX_{\vgamma_{i+1}}^T\vy$.
	The $R^2$ downdate can be obtained
	via
	$$
	R_{\vgamma_{i+1}}^2 = R_{\vgamma_{i}}^2 - \frac{\kappa s^2}{n}.
	$$
	
	
	%\item Let  $D = |\mC^T\mC|$ and $D_{\mbox{\tiny downdate}} = |\mX^T\mX|$ then
	%$D_{\mbox{\tiny downdate}} = cD$.
\end{enumerate}

\noindent Again, presuming relevant summary quantities have been precomputed
the updates for all of the above quantities costs $O(|\vgamma_{i}|^2 + n)$ time.

\subsection{Special functions}

Several special functions appear in the expressions of various inferential
quantities of interest in this paper. The main special functions used 
were (Kummer's) confluent hypergeometric function (of the first kind),
denoted ${}_1F_1$, and the Gaussian hypergeometric function, denoted
${}_2 F_1$. Great care is required to implement these functions. During the 
implementation process we tried several packages which implemented these
functions. We found that the {\tt R} package {\tt gsl} was the best performing
implementation amongst the packages we tried
[{\bf CITE:} gsl].
 

\section{Numerical results}
\label{sec:numerical}

\begin{table}
	\label{tab:numerical_results_hitters}
	\caption{Top 10 models selected for the Hitters data set ranked by posterior model probability, the difference
		in log probabilities $\log p(\vy|\vgamma)$ between the best model and this model and the difference
		between the Bayesian Information Criteria for this model and the best model}
	\begin{center}
		\begin{tabular}{r|l|l|c|r|r}
			rank & $\vgamma$ (base: 1, 2, 6, 15, 16) & $p(\vgamma|\vy)$ & $R^2$ & $\widetilde{\text{BIC}}_\text{Best} - \widetilde{\text{BIC}}$ & $\text{BIC}_\text{Best} - \text{BIC}$ \\
			\hline
			1  &  +10, +11, +13&  2.56\%&  0.53&  0.00&  0.00\\
			2  &  +11, +12, +13&  2.29\%&  0.53&  0.22&  0.23\\
			3  &  +8, +11, +12, +13&  1.60\%&  0.53&  0.94&  2.17\\
			4  &  +8, +9, +10&  1.23\%&  0.53&  1.47&  1.53\\
			5  &  +10, +11, +13, +17&  1.10\%&  0.53&  1.69&  2.95\\
			6  &  +12&  1.10\%&  0.51&  1.70&  $-$0.30\\
			7  &  +7, +11, +12, +13,&  1.07\%&  0.53&  1.75&  3.02\\
			8  &  +8, +11, +12, +13, +17&  1.03\%&  0.54&  1.83&  4.40\\
			9  &  +9, +11, +12, +13&  0.67\%&  0.53&  2.68&  3.99\\
			10 &  +8, +9, +12&  0.64\%&  0.52&  2.78&  2.89\\
		\end{tabular}
	\end{center}
\end{table}
\begin{table}
	\label{tab:numerical_results_bodyfat}
	\caption{Top 10 models selected for the Body fat data set ranked by posterior model probability, the difference
		in log probabilities $\log p(\vy|\vgamma)$ between the best model and this model and the difference
		between the Bayesian Information Criteria for this model and the best model}
	\begin{center}
		\begin{tabular}{r|l|r|c|r|r}
			rank & $\vgamma$ (base: 6, 13) & $p(\vgamma|\vy)$ & $R^2$ & $\widetilde{\text{BIC}}_\text{Best} - \widetilde{\text{BIC}}$ & $\text{BIC}_\text{Best} - \text{BIC}$ \\
			\hline
			1  &  +2&  17.64\%&  0.73&  0.00&  0.00\\
			2 &  +3&  5.85\%&  0.73&  2.21&  2.25\\
			3 &  +2, +11&  5.74\%&  0.74&  2.25&  1.98\\
			4 &  +2, +12&  3.56\%&  0.74&  3.20&  2.95\\
			5 &  +1, +3&  2.83\%&  0.74&  3.66&  3.42\\
			6 &  +1&  2.69\%&  0.73&  3.76&  3.84\\
			7 &  +2, +3&  2.07\%&  0.74&  4.28&  4.06\\
			8 &  +3, +5&  2.03\%&  0.74&  4.33&  4.11\\
			9 &  +2, +8&  1.96\%&  0.74&  4.40&  4.18\\
			10 &  +1, +2&  1.81\%&  0.74&  4.55&  4.33\\
		\end{tabular}
	\end{center}
\end{table}

\begin{table}
	\label{tab:numerical_results_wage}
	\caption{Top 10 models selected for the Wage data set ranked by posterior model probability, the difference
		in log probabilities $\log p(\vy|\vgamma)$ between the best model and this model and the difference
		between the Bayesian Information Criteria for this model and the best model}
	\begin{center}
		\begin{tabular}{r|l|r|c|r|r}
			rank & $\vgamma$ (base: 13, 16, 17) & $p(\vgamma|\vy)$ & $R^2$ & $\widetilde{\text{BIC}}_\text{Best} - \widetilde{\text{BIC}}$ & $\text{BIC}_\text{Best} - \text{BIC}$ \\
			\hline
			1 &  &  67.57\%&  0.91&  0.00&  0.00\\
			2 &  +11&  5.57\%&  0.91&  4.99&  3.70\\
			3 &  +12&  3.99\%&  0.91&  5.66&  4.37\\
			4 &  +3&  3.88\%&  0.91&  5.71&  4.43\\
			5 &  +2&  2.86\%&  0.91&  6.32&  5.04\\
			6 &  +1&  2.18\%&  0.91&  6.86&  5.58\\
			7 &  +14&  1.56\%&  0.91&  7.54&  6.25\\
			8 &  +10&  0.92\%&  0.91&  8.59&  7.31\\
			9 &  +4&  0.91\%&  0.91&  8.61&  7.33\\
			10 &  +6&  0.85\%&  0.91&  8.75&  7.47\\
		\end{tabular}
	\end{center}
\end{table}

\begin{table}
	\label{tab:numerical_results_gradrate}
	\caption{Top 10 models selected for the Graduation Rate data set ranked by posterior model probability, the
		difference in log probabilities $\log p(\vy|\vgamma)$ between the best model and this model and the difference between the Bayesian Information Criteria for this model and the best model}
	\begin{center}
		\begin{tabular}{r|l|r|c|r|r}
			rank & $\vgamma$ (base: 2, 8, 9, 10, 16, 17) & $p(\vgamma|\vy)$ & $R^2$ & $\widetilde{\text{BIC}}_\text{Best} - \widetilde{\text{BIC}}$ & $\text{BIC}_\text{Best} - \text{BIC}$ \\
			\hline
			1 &  +6, +12&  7.62\%&  0.45&  0.00&  0.00\\
			2 &  +1, +6, +12&  6.80\%&  0.46&  0.23&  1.65\\
			3 &  +6&  4.88\%&  0.45&  0.89&  $-$0.40\\
			4 &  +1, +6&  4.46\%&  0.45&  1.07&  1.08\\
			5 &  +6, +7&  2.43\%&  0.45&  2.28&  2.31\\
			6 &  +6, +7, +12&  2.37\%&  0.45&  2.33&  3.79\\
			7 &  +5, +12&  2.29\%&  0.45&  2.41&  2.43\\
			8 &  +3, +6, +12&  1.56\%&  0.45&  3.17&  4.63\\
			9 &  +5, +6, +12&  1.52\%&  0.45&  3.22&  4.69\\
			10 &  +5&  1.32\%&  0.45&  3.50&  2.24\\
		\end{tabular}
	\end{center}
\end{table}

\begin{table}
	\label{tab:numerical_results_uscrime}
	\caption{Top 10 models selected for the US Crime data set ranked by posterior model probability, the
		difference in log probabilities $\log p(\vy|\vgamma)$ between the best model and this model and the difference between the Bayesian Information Criteria for this model and the best model}
	\begin{center}
		\begin{tabular}{r|l|r|l|r|r}
			rank & $\vgamma$ (base: 1, 3, 4, 13)& $p(\vgamma|\vy)$ & $R^2$ & $\widetilde{\text{BIC}}_\text{Best} - \widetilde{\text{BIC}}$ & $\text{BIC}_\text{Best} - \text{BIC}$ \\
			\hline
			1 &  +11, +14&  2.48\%&  0.77&  0.00&  0.00\\
			2 & +14&  1.47\%&  0.74&  1.05&  1.45\\
			3 & +11, +12, +14&  0.99\%&  0.77&  1.85&  2.07\\
			4 & +10, +11, +14&  0.93\%&  0.77&  1.97&  2.22\\
			5 &  +8, +11, +14&  0.89\%&  0.77&  2.06&  2.33\\
			6 &  +11&  0.78\%&  0.73&  2.31&  2.91\\
			7 &  +12, +14&  0.71\%&  0.75&  2.51&  2.99\\
			8 &  +7, +14&  0.69\%&  0.75&  2.56&  3.04\\
			9 &  +7, +11, +14&  0.69\%&  0.77&  2.56&  2.94\\
			10 &  +7, +10, +11, +14&  0.69\%&  0.79&  2.56&  2.85\\
		\end{tabular}
	\end{center}
\end{table}


\subsection{Parameter posterior distributions}


% alpha

% beta

% sigma2

% g

% u



\section{Conclusion}
\label{sec:conclusion}

\bibliographystyle{anzsj}
\bibliography{references_mendeley}

\newpage 

\section*{Appendix A: Useful integrals}

The following integrals appear in \cite{Gradshteyn2007}.
\begin{itemize}
\item 
Equation 3.194 (iii):
$$
\int_0^\infty \frac{ x^{\mu - 1} }{(1 + \beta x)^\nu} dx = \beta^{-\mu} \mbox{Beta}(\mu,\nu - \mu) \quad   \mbox{(assuming $\mu,\nu>0$ and $\nu>\mu$).}
$$

\item 
$$
U(a,b,z) = \frac{1}{\Gamma(a)}\int_0^\infty e^{-zt} t^{a - 1} (1 + t)^{b - a - 1} dt, 
\qquad \mbox{(for $\mbox{Re}(a)>0$, $\mbox{Re}(z)>0$)}.
$$


\item 
Equation 3.383 (i):
$$
\int_{0}^u x^{\nu - 1} (u - x)^{\mu - 1}  e^{\beta x} dx = \mbox{Beta}(\nu,\mu) {}_1 F_1(\nu;\mu+\nu;\beta u) \quad   \mbox{(assuming $\mbox{Re}(\mu)>0$ and $\mbox{Re}(\nu)>0$).}
$$


\item 3.197(5) of G\&R given by
$$
\int_0^\infty x^{\lambda - 1}(1 + x)^\nu (1 + \alpha x)^\mu dx
=\mbox{Beta}(\lambda,-\mu-\nu-\lambda){}_2F_1(-\mu,\lambda;-\mu-\nu; 1 - \alpha)
$$

\noindent provided $-(\mu  + \nu) > \lambda > 0$.
 

\item Equation 7.621 (4) 
(the second line below is corrected using EH I, 269(5))
is
$$
\begin{array}{rll}
\ds \int_0^\infty e^{-st} t^{b-1} {}_1F_1(a,c,kt) dt
& \ds = \Gamma(b)s^{-b} F(a,b,c,k s^{-1}) 
& \mbox{(if $|s| > |k|$)}
\\
& \ds = \Gamma(b)(s - k)^{-b} F\left(c - a, b; c; \frac{k}{k - s} \right)
& \mbox{(if $|s - k| > |k|$)}
\end{array} 
$$

\noindent assuming that $\mbox{Re}(b)>0$, $\mbox{Re}(s) > \max(0,\mbox{Re}(k))$.

\item Equation 3.385  of \citep{Gradshteyn2007} is
\begin{equation} \label{res:04}
\int_{0}^1 x^{\nu - 1} (1 - x)^{\lambda - 1}(1 - \beta x)^{\varrho} e^{-\mu x} dx 
= \mbox{Beta}(\nu,\lambda) {}_1 F_1(\nu,\varrho,\lambda+\nu,-\mu,\beta)
\end{equation}

\noindent provided $\mbox{Re}(\lambda)>0$, $\mbox{Re}(\nu)>0$ and $|\mbox{arg}(1-\beta)|<\pi$.


 

\item Equation X.XXX (X) 
$$
\begin{array}{l}
\ds \int_0^\infty x^{\rho_n - 1}e^{-\mu x} {}_mF_n(a_1,\ldots,a_m;\rho_1,\ldots,\rho_n,\lambda x) dx
=
\Gamma(\rho_n)\mu^{-\rho_n} {}_mF_{n-1}(a_1,\ldots,a_m;\rho_1,\ldots,\rho_n,\lambda/\mu)
\end{array}
$$

\noindent Hence,
$$
\begin{array}{l}
\ds \int_0^\infty x^{\rho_1 - 1}e^{-\mu x} {}_1F_1(a_1;\rho_1,\lambda x) dx
=
\Gamma(\rho_1)\mu^{-\rho_1} {}_1F_{0}(a_1; ;\lambda/\mu)
=
\Gamma(\rho_1)\mu^{-\rho_1} (1 - \lambda/\mu)^{-a_1}
\end{array}
$$

\noindent where we have used the identity ${}_1F_{0}(a; ;z) = (1 - z)^{-a}$ for generalized hypergeometric functions.

\end{itemize}

\noindent This is a special case of the normalizing constant for the compound confluent hypergeometric
distribution of Gordy (1998) given by
$$
\mbox{CCH}(x;p,q,r,s,\nu,\theta) = \frac{ x^{p-1}(1 - \nu x)^{q-1} (\theta + (1-\theta)\nu x)^{-r}\exp(-s x)}{\mbox{Beta}(p,q)\nu^{-p}\exp(-s/\nu)\Phi_1(q,r,p+q,s/\nu,1-\theta)}, \qquad 0< x < 1/\nu, 
$$

\noindent where $p>0$, $q>0$, $s\in\bR$, $0\le \nu\le 1$ and $\theta\in(0,1)$. 



\section*{Appendix B: Markov chain Monte Carlo}

Here we will find the full conditional distributions which can be used to perform Bayesian
inference via Gibbs sampling. The full conditional distributions for $\alpha$, $\vbeta$, and
$\sigma^2$ are given by
$$
\begin{array}{ll}
\ds \alpha|\vy,\sigma^2
& \ds \sim N(0,\sigma^2/n),
\\ [1ex]
\ds \vbeta|\vy,\sigma^2,g 
& \ds \sim N\left(                  
\tfrac{g}{1+g}\widehat{\vbeta},    
\tfrac{g}{1+g} \sigma^2 \left( \mX^T\mX \right)^{-1} 
\right), \qquad \mbox{and}
\\ [1ex]
\ds \sigma^2|\vy,\valpha,\vbeta,g
& \ds \sim \mbox{IG}\left( 
\frac{n+p}{2}, \frac{\|\vy - \vone_n\alpha - \mX\vbeta\|^2}{2}
+ \frac{\vbeta^T\mX^T\mX\vbeta}{2g}
\right), 
%\propto 
%\exp\left[
%- \left( \frac{n+p}{2} + 1 \right)\log(\sigma^2)
%- \frac{\|\vy - \vone_n\alpha - \mX\vbeta\|^2}{2\sigma^2}
%- \frac{\vbeta^T\mX^T\mX\vbeta}{2g\sigma^2}
%\right]
\end{array}
$$

\noindent 
where $\widehat{\vbeta} = (\mX^T\mX)^{-1}\mX^T\vy$.
These are all standard distributions that can be easily sampled from.



The full conditional distribution for $g$ is proportional to
\begin{equation}
\label{eq:fullCondG}
\ds p(g|\vy,\vbeta,\sigma^2) \propto g^{A-1} (1 + g)^{-B}\exp( - C/g),
\end{equation}

\noindent where $A = b - p/2 + 1$, $B = a + b + 2$, and $C = \vbeta^T\mX^T\mX\vbeta/(2\sigma^2)$. 
Using the change of variables $h = 1/g$ we have
\begin{equation}
\label{eq:fullCondH}
p(h|\vy,\vbeta,\sigma^2)\propto h^{D-1}  (1 + h)^{-B}\exp( - Ch),
\end{equation}


\noindent 
where $D = B - A = a  + 1  + p/2$. If $a > - 1$ then $D>0$ since $p\ge 1$.
% Note that the full conditional distribution for $g$ (or alternatively $h$) is 
%non-standard.
We recognize (\ref{eq:fullCondH}) as 
the confluent hypergeometric U distribution of
Gordy (1998), we will call the CHU distribution for short, with
$h |\vy,\vbeta,\sigma^2 \sim \mbox{CHU}(D,B,C,1)$
where if $x\sim\mbox{UH}(p,r,s,\lambda)$ then $x$ has the density function
$$
p(x) = \frac{x^{p-1}(1 + \lambda x)^{-r}\exp(-sx)}{\Gamma(p)U(p,p+1-r,s/\lambda)}, \qquad x>0; \quad p>0, \quad r, s\in \bR  \quad \mbox{and} \quad 
s/\lambda\ge 0. 
$$

\noindent Here $U(a,b,z)$ is the 
U confluent hypergeometric function defined in Abramowitz \& Stegun (1972), denoted by $\Psi$ in \cite{Gradshteyn2007}, and is defined in Appendix A.
%The distribution is defined for $p>0$,
%$r\in\bR$, $s\in\bR$ and $s/\lambda\ge 0$. 
The full conditional distribution (\ref{eq:fullCondG}) 
is what we call the inverse
UH distribution and use the notation
$$
\ds g |\vy,\vbeta,\sigma^2 \sim \mbox{ICHU}(D,B,C,1)
$$

\noindent where if
$x\sim\mbox{ICHU}(p,r,s,\lambda)$ then it has the
density
$$
p(x) = \frac{x^{-(p+r+1)}(x + \lambda)^{-r}\exp(-s/x)}{\Gamma(p)U(p,p+1-r,s/\lambda)},\qquad x>0; \quad p>0, \quad r, s\in \bR  \quad \mbox{and} \quad 
s/\lambda\ge 0. 
$$

\noindent While we have identified the full conditional distribution for $g$ it is not a standard distribution from which can be easily sample. A Metropolis-Hastings or rejection sampling approach could be used to sample from
$g |\vy,\vbeta,\sigma^2$. 


\newpage 

\section*{Appendix C}


$$
\ds p(\vy|g)
= K(n)
(1 + g)^{(n-p-1)/2}\left[ 1 + g (1 - R^2) \right]^{-c},
$$



\subsection*{$g$-prior}

Liang et al (2008) propose the $g$-prior
$$
p(g) = \frac{(a - 2)}{2}(1 + g)^{-a/2}
$$

\noindent where $a > 2$. The marginal likelihood is given by
$$
\begin{array}{rl}
p(\vy) 
& \ds = \int_0^\infty K(n)\frac{(a - 2)}{2} 
(1 + g)^{(n-p-1 - a)/2}\left[ 1 + g (1 - R^2) \right]^{-c}  dg
\\ [2ex]
& \ds = K(n) \frac{(a - 2)}{2} \mbox{Beta}(1,(p + a)/2 - 1){}_2F_1( c, 1; (p + a)/2; R^2)
\\
%& \ds = K(n) \frac{(a - 2)}{2} \frac{\Gamma((p + a)/2 - 1)}{\Gamma((p + a)/2)} {}_2F_1( c, 1; (p + a)/2; R^2)
\\
& \ds = K(n) \frac{(a - 2)}{p + a - 2} {}_2F_1( c, 1; (p + a)/2; R^2).
\end{array} 
$$

\noindent where we have used 3.197(5) of G\&R given by
$$
\int_0^\infty x^{\lambda - 1}(1 + x)^\nu (1 + \alpha x)^\mu dx
=\mbox{Beta}(\lambda,-\mu-\nu-\lambda){}_2F_1(-\mu,\lambda;-\mu-\nu; 1 - \alpha)
$$

\noindent provided $-(\mu  + \nu) > \lambda > 0$ with
$$
\lambda \leftrightarrow 1, \quad \nu \leftrightarrow (n-p-1 + a)/2, \quad \mu \leftrightarrow -c, \quad \alpha \leftrightarrow 1- R^2.
$$

\noindent Note that when $R^2 = 0$ (which occurs under the null model),
${}_2F_1( c, 1; (p + a)/2; R^2) = 1$ and $p = 0$ resulting in
$p(\vy) = K(n)$. Hence, the null based Bayes factor is
$$
\mbox{BF}_{\gamma 0} = \frac{(a - 2)}{p + a - 2} {}_2F_1( c, 1; (p + a)/2; R^2).
$$

\newpage 
\subsection*{$g/n$-prior}

Liang et al (2008) propose the $g/n$-prior
$$
p(g) = \frac{a - 2}{2n}\left(1 + \frac{g}{n}\right)^{-a/2}
$$

\noindent Then
$$
\begin{array}{rl}
p(\vy) 
& \ds = K(n) \frac{a - 2}{2n}  \int_0^\infty 
(1 + g)^{(n-p-1)/2} \left(1 + \frac{g}{n}\right)^{-a/2} \left[ 1 + g (1 - R^2) \right]^{-c}  dg
\end{array} 
$$

%\noindent \joc{
%$$
%u = \frac{g}{1+g},
%\quad 
%g = \frac{u}{1-u},
%\quad 
%1 - u = \frac{1}{(1 + g)},
%\quad 
%(1 - u)^{-1} = (1 + g),
%\quad \mbox{and}\quad 
%\frac{du}{dg} = \frac{1}{1+g} - \frac{g}{(1 +g)^2} 
%= \frac{1}{(1+g)^2} = (1 - u)^2.
%$$

\noindent Using th change of variables $u = g/(1 + g)$ we have
$$
\begin{array}{rl}
p(\vy) 
& \ds = K(n) \frac{(a - 2)}{2n}  \int_0^1 
(1 - u)^{-(n-p-1)/2 - 2} \left(1 +  \frac{u/n}{1-u} \right)^{-a/2} \left[ 1 + \frac{u}{1-u} (1 - R^2) \right]^{-c} du
\\ [2ex]
& \ds = K(n) \frac{a - 2}{2n}  \int_0^1 
(1 - u)^{p/2 + a/2 - 2  } \left(1 - u (1  -  1/n) \right)^{-a/2} \left(  1 - u R^2\right)^{-c} du
\\ [2ex]
& \ds = \frac{K(n)}{n} \frac{a - 2}{p + a - 2}  F_1(1,a/2,c,(p + a)/2; (1  -  1/n), R^2) 
\end{array} 
$$

\noindent where we have used the result of \'Emile (1881) who showed that
$$
\int_0^1 t^{a-1} (1-t)^{c-a-1} (1-xt)^{-b_1} (1-yt)^{-b_2} dt
= \mbox{Beta}(a,c - a) F_1(a,b_1,b_2,c; x,y) 
$$

\noindent for $c>a>0$  with
$F_1$ being the Appell hypergeometric function in two variables, and
$$
a\leftrightarrow 1,
\quad 
c  \leftrightarrow (p + a)/2 
\quad 
x \leftrightarrow (1  -  1/n)
\quad 
b_1 \leftrightarrow a/2
\quad 
y \leftrightarrow R^2
\quad 
b_2 \leftrightarrow c
$$


\'Emile, Picard  (1881). Sur une extension aux fonctions de deux variables du probl\'eme de Riemann relativ aux fonctions hyperg\'eom\'etriques. Annales scientifiques de l'\'Ecole Normale Sup\'erieure. (2\'eme s\'erie) (in French). 10: 305--322.


\bigskip 
\noindent When the null is true $R^2 = 0$ and $p=0$. In this case $p(\vy) =  K(n)$. Hence,
$$
\mbox{BF}_{\gamma 0} = \frac{a - 2}{n(p + a - 2)}  F_1(1,a/2,c,(p + a)/2; (1  -  1/n), R^2) .
$$

\newpage 
\subsection*{Zellner-Siow prior}

$$
p(g) = \frac{(n/2)^{1/2}}{\Gamma(1/2)} g^{-3/2} \exp\left( -\frac{n}{2g} \right)
$$


$$
\ds p(\vy)
= K(n) \frac{(n/2)^{1/2}}{\Gamma(1/2)}
\int_0^\infty g^{-3/2}
(1 + g)^{(n-p-1)/2}\left[ 1 + g (1 - R^2) \right]^{-c}  \exp\left( -\frac{n}{2g} \right) dg
$$


$$
\ds p(\vy)
= K(n) \frac{(n/2)^{1/2}}{\Gamma(1/2)}
\int_0^1  u^{-3/2}
(1 - u)^{ (p-1)/2}\left(  1   - uR^2 \right)^{-c}  \exp\left( -\frac{n}{2u}  + \frac{n}{2} \right) du
$$

$$
p(h) = \frac{(n/2)^{1/2}}{\Gamma(1/2)} h^{-1/2} \exp\left( -\frac{n}{2}h \right)
$$


$$
\begin{array}{rl} 
\ds p(\vy)
& \ds = K(n) \frac{(n/2)^{1/2}}{\Gamma(1/2)}
\int_0^\infty h^{(p-1)/2 }
(1 + h)^{(n-p-1)/2}\left[ h +  (1 - R^2) \right]^{-c}  \exp\left( -\frac{n}{2}h \right) dg
\\
& \ds = K(n) \frac{(n/2)^{1/2}}{\Gamma(1/2)} (1 - R^2)^{-c}
\sum_{k=0}^\infty \frac{1}{k!} \left( -\frac{n}{2} \right)^k \int_0^\infty h^{k + (p - 1)/2 }
(1 + h)^{(n-p-1)/2} \left[ 1 + h/(1 - R^2) \right]^{-c}   dh
 
\end{array} 
$$

\noindent Now we use 3.197(5) of G\&R
$$
\int_0^\infty x^{\lambda-1} (1 + x)^{\nu}(1 + \alpha x)^{\mu} dx
= \mbox{Beta}(\lambda,-\mu-\nu-\lambda) {}_2 F_1(-\mu,\lambda;-\mu-\nu;1-\alpha)
$$

\noindent provided $-(\mu+\nu) > \lambda > 0$. Above we have
$$
\lambda \leftrightarrow k + p/2 + 1/2 >0
$$
$$
\nu \leftrightarrow (n-p-1)/2
$$
$$
\alpha \leftrightarrow 1/(1 - R^2)
$$
$$
\mu \leftrightarrow - c
$$

\noindent Checking the conditions
$$
p/2 >   k + p/2 + 1/2 
$$

%\noindent 
%Now we use 3.197(1) of G\&R
%$$
%\int_0^\infty x^{\nu-1} (\beta + x)^{-\mu}(x + \gamma)^{-\varrho} dx
%= \mbox{Beta}(\nu,\mu - \nu + \varrho){}_2F_1(\mu,\nu;\mu + \varrho; 1 - \gamma/\beta)
%$$

%\noindent provided $\nu>0$, $\mu > \nu - \varrho$.

%Above we have
%$$
%\nu \leftrightarrow k + (p + 1)/2 > 0
%$$

%\noindent We then either have
%$$
%\beta \leftrightarrow 1, \quad \mu \leftrightarrow - (n-p-1)/2, \quad \gamma \leftrightarrow 1 - R^2, \quad \varrho  \leftrightarrow c
%$$

%\noindent In this case
%$$
%0 > k  + 1/2  
%$$


%\noindent Or
%$$
%\gamma \leftrightarrow 1, \quad \varrho \leftrightarrow - (n-p-1)/2, \quad \beta %\leftrightarrow 1 - R^2, \quad \mu  \leftrightarrow c
%$$

%$$
%0 > k  + 1/2  
%$$




\newpage 
This leads to 
\begin{equation}\label{eq:yGivenG}
\ds p(\vy|\vgamma)
\ds = \frac{K(n)}{2}\left(\frac{1+n}{1 + p_\vgamma}  \right)^{1/2} 
\int_L^\infty  (1 + g)^{(n - p_\vgamma)/2 - 2}(  1 + g \widehat{\sigma}_\vgamma^2)^{-(n-1)/2} dg,
\end{equation}


$$
h = g - L
$$

$$
g = h + L
$$

$$
dg = dh
$$

$$
\ds p(\vy|\vgamma)
\ds = \frac{K(n)}{2}\left(\frac{1+n}{1 + p_\vgamma}  \right)^{1/2} 
\int_0^\infty  (1 + L + h)^{(n - p_\vgamma)/2 - 2}(  1 + L\widehat{\sigma}_\vgamma^2 + h\widehat{\sigma}_\vgamma^2)^{-(n-1)/2} dh,
$$

$$
x = 1/h
$$

$$
h = 1/x
$$

$$
dh = - x^{-2} dx
$$


$$
\begin{array}{rl} 
\ds p(\vy|\vgamma)
& \ds = \frac{K(n)}{2}\left(\frac{1+n}{1 + p_\vgamma}  \right)^{1/2} 
\int_0^\infty x^{-2} (1 + L + 1/x)^{(n - p_\vgamma)/2 - 2}(  1 + L\widehat{\sigma}_\vgamma^2 + \widehat{\sigma}_\vgamma^2/x)^{-(n-1)/2} dx,
\\
& \ds = \frac{K(n)}{2}\left(\frac{1+n}{1 + p_\vgamma}  \right)^{1/2} 
\int_0^\infty   x^{(p_\vgamma-1)/2} [1 + (1 + L)x]^{(n - p_\vgamma)/2 - 2}[\widehat{\sigma}_\vgamma^2 +  (1 + L\widehat{\sigma}_\vgamma^2)x]^{-(n-1)/2} dx,
\end{array}
$$


$$
z = (1 + L)x
$$

$$
x = z/(1 + L)
$$

$$
dx = dz/(1 + L)
$$


$$
\begin{array}{rl} 
\ds p(\vy|\vgamma)
& \ds = \frac{K(n)}{2}\left(\frac{1+n}{1 + p_\vgamma}  \right)^{1/2} 
\int_0^\infty 
\left( \frac{z}{1+L} \right)^{(p_\vgamma-1)/2}
\left( 1 + z\right)^{(n - p_\vgamma)/2 - 2}
\left[ \widehat{\sigma}_\vgamma^2 +  \left( \frac{1 + L\widehat{\sigma}_\vgamma^2}{1 + L} \right)z \right]^{-(n-1)/2} (1 + L)^{-1} dz
\\
& \ds = \frac{K(n)}{2}\left(\frac{1+n}{1 + p_\vgamma}  \right)^{1/2} (1 + L)^{ - (p_\vgamma+1)/2} (\widehat{\sigma}_\vgamma^2)^{-(n-1)/2}
\\ 
& \ds \quad \times
\int_0^\infty 
z^{(p_\vgamma+1)/2-1}
\left( 1 + z\right)^{(n - p_\vgamma)/2 - 2}
\left[ 1 +  \left( \frac{1 + L\widehat{\sigma}_\vgamma^2}{\widehat{\sigma}_\vgamma^2(1 + L)} \right)z \right]^{-(n-1)/2}  dz

\end{array}
$$

\noindent 
3.197(5) of G\&R given by
$$
\int_0^\infty x^{\lambda - 1}(1 + x)^\nu (1 + \alpha x)^\mu dx
=\mbox{Beta}(\lambda,-\mu-\nu-\lambda){}_2F_1(-\mu,\lambda;-\mu-\nu; 1 - \alpha)
$$

\noindent 
provided $-(\mu  + \nu) > \lambda > 0$.

$$
\lambda \leftrightarrow (p_\vgamma+1)/2
$$
$$
\nu \leftrightarrow (n - p_\vgamma)/2 - 2
$$
$$
\alpha \leftrightarrow \frac{1 + L\widehat{\sigma}_\vgamma^2}{\widehat{\sigma}_\vgamma^2(1 + L)}
$$
$$
\mu \leftrightarrow -(n-1)/2
$$

\noindent Checking
$$
(p_\vgamma+1)/2 > 0
$$

$$
(p_\vgamma + 3)/2 > (p_\vgamma+1)/2
$$

$$
- \mu - \nu = (p_\vgamma+3)/2 
$$

\noindent Hence,
$$
\begin{array}{rl} 
\ds p(\vy|\vgamma)
& \ds = \frac{K(n)}{2}\left(\frac{1+n}{1 + p_\vgamma}  \right)^{1/2} (1 + L)^{ - (p_\vgamma+1)/2} (\widehat{\sigma}_\vgamma^2)^{-(n-1)/2}
\\ 
& \ds \quad \times
\mbox{Beta}\left[ \frac{p_\vgamma+1}{2}, 1 \right]
{}_2F_1\left[ \frac{n-1}{2}, \frac{p_\vgamma+1}{2}; \frac{p_\vgamma+3}{2}  ; 1 - \frac{1 + L\widehat{\sigma}_\vgamma^2}{\widehat{\sigma}_\vgamma^2(1 + L)} \right]

\\

& \ds = \frac{K(n)}{p_\vgamma+1}\left(\frac{1+n}{1 + p_\vgamma}  \right)^{1/2} (1 + L)^{ - (p_\vgamma+1)/2} (\widehat{\sigma}_\vgamma^2)^{-(n-1)/2}
{}_2F_1\left[ \frac{n-1}{2}, \frac{p_\vgamma+1}{2}; \frac{p_\vgamma+3}{2}  ; 
\frac{1  - 1/\widehat{\sigma}_\vgamma^2}{ 1 + L} \right]


\end{array}
$$

Now we use 
$$
1 + L = \frac{n + 1}{ 1 + p_\vgamma}
$$

$$
\begin{array}{rl} 
\ds p(\vy|\vgamma)
& \ds = \left( \frac{n + 1}{ p_\vgamma + 1} \right)^{ - p_\vgamma/2} \frac{(\widehat{\sigma}_\vgamma^2)^{-(n-1)/2}}{p_\vgamma+1}
{}_2F_1\left[ \frac{n-1}{2}, \frac{p_\vgamma+1}{2}; \frac{p_\vgamma+3}{2}  ; 
\frac{(1  - 1/\widehat{\sigma}_\vgamma^2)(p_\vgamma + 1)}{1 + n} \right]


\end{array}
$$

\newpage 





















$$
\ds p(\vy|\vgamma)
\ds = \frac{K(n)}{2}\left(\frac{1+n}{1 + p_\vgamma}  \right)^{1/2} (\widehat{\sigma}_\vgamma^2)^{-(n-1)/2}
\int_0^\infty  (1 + L + h)^{(n - p_\vgamma)/2 - 2} \left(  \frac{1 + L\widehat{\sigma}_\vgamma^2}{\widehat{\sigma}_\vgamma^2} + h \right)^{-(n-1)/2} dh,
$$

Using 3.197 of G\&R which is
$$
\int_0^\infty x^{\nu - 1}(\beta + x)^{-\mu}(x + \gamma)^{-\varrho} dx
= \beta^{-\nu}
\gamma^{\nu - \varrho} 
\mbox{Beta}(\nu,\mu - \nu + \varrho)
{}_2F_1(\mu,\nu;\mu+\varrho; 1 - \gamma/\beta)
$$

\noindent provided $\nu>0$, $\mu > \nu - \varrho$.


$$
\nu \leftrightarrow 1
$$ 
$$
\mu \leftrightarrow - (n - p_\vgamma)/2 + 2
$$
$$
\varrho \leftrightarrow (n-1)/2
$$

\noindent Checking
$$
- n/2 + p_\vgamma/2 + 2 > 1 - n/2 + 1/2
$$
$$
p_\vgamma/2 + 2 > 1  + 1/2
$$
$$
p_\vgamma > -1
$$

$$
\beta \leftrightarrow 1 + L
$$

$$
\gamma \leftrightarrow \frac{1 + L\widehat{\sigma}_\vgamma^2}{\widehat{\sigma}_\vgamma^2}
$$


$$
\begin{array}{rl}
\ds p(\vy|\vgamma)
\ds 
& \ds = K(n)\left(\frac{1+n}{1 + p_\vgamma}  \right)^{1/2} 
\frac{ (\widehat{\sigma}_\vgamma^2)^{-1} }{p_\vgamma+1} 
(1 + L)^{-1}
\left( 1 + L\widehat{\sigma}_\vgamma^2 \right)^{1 - (n-1)/2} 
\\
& \ds \qquad 
{}_2F_1\left( - \frac{n - p_\vgamma + 4}{2}, 1;  \frac{ p_\vgamma+3}{2} ; 
\frac{(1 - 1/\widehat{\sigma}_\vgamma^2)(1 + p_\vgamma)}{1 + n} \right)
\end{array} 
$$


\end{document}



Note that
Theorem 2.1 of Nadarajah (2015)
does not apply here since the resulting
expression involves incomplete beta functions with negative
arguments.

\joc{ 
	$$
	{}_2 F_1(b, 1; c; x)
	= (c-1) x^{1-c}(1-x)^{c-b-1}B_x(c - 1, b - c + 1),
	$$
	
	$$
	\begin{array}{rl}
	\ds M_1
	& \ds =    
	\frac{b + 1}{c}
	%	{}_2 F_1 (d + 1, 1; c + 1; R^2)        
	c x^{-c}(1-x)^{b + 1}B_x(c, - b - 1)
	\end{array} 
	$$
	
	$$
	\begin{array}{rl} 
	\ds {}_2F_1(b, 2; c; x) 
	& \ds = 
	- \frac{(c-1)(c-2)^2}{b-1} x^{1-c} (1-x)^{c-b-1} \mbox{Beta}_x(c-2,b-c+1) \\
	& \ds \qquad - \frac{(c-1)(c-2)(c-b-1)}{b-1} x^{-c} (1-x)^{c-b-2} \mbox{Beta}_x(c-2,b-c+1) \\
	& \ds \qquad + \frac{(c-1)(c-2)}{(b-1) \mbox{Beta}(c-2,b-c+1)}x^{-1}(1-x)^{-1}
	\end{array} 
	$$
	
}







  


\joc{
	\subsection{Calculating mode of $p(g)$}
	
	\noindent To find the mode of $p(g)$ consider
	$$
	f(g) = \log p(g) = b\log(g) - (a + b + 2)\log(1 + g) + \mbox{constants in $g$}
	$$ 
	
	\noindent Then
	$$
	f'(g) = \frac{b}{g} - \frac{a + b + 2}{g + 1} = 0
	$$
	
	$$
	b - (a  + 2)g = 0
	$$
	
	\noindent So the mode is
	$$
	\mbox{[mode of $p(g)$]} = \frac{b}{a+2} = O(n)
	$$
	
	\noindent so that at least this is right.
	
	
	\subsection{Calculating of $E(g)$}
	
	$$
	\int_0^\infty g^{b + 1} (1 + g)^{-a-b-2} dg = \mbox{Beta}(b + 2,a)
	$$
	
	\noindent Then
	$$
	E(g) = \frac{\mbox{Beta}(a,b + 2)}{\mbox{Beta}(a + 1,b + 1)} 
	= \frac{\Gamma(a)\Gamma(b+2)}{\Gamma(a + b + 2)} \frac{\Gamma(a+1)\Gamma(b+1)}{\Gamma(a + b + 2)}
	= \frac{\Gamma(a)\Gamma(b+2)}{\Gamma(a+1)\Gamma(b+1)}
	= \frac{b + 1}{a}
	$$
	
	\noindent so that the mean does not exist as claimed.
	
	\subsection{Calculating of $E(g^{-1})$}
	
	$$
	\int_0^\infty g^{b - 1} (1 + g)^{-a-b-2} dg = \mbox{Beta}(b,a + 2)
	$$
	
	
	
	$$
	E(g^{-1}) = \frac{\mbox{Beta}(b,a + 2)}{\mbox{Beta}(a + 1,b + 1)} 
	= \frac{\Gamma(a+2)\Gamma(b)}{\Gamma(a + b + 2)} \frac{\Gamma(a+1)\Gamma(b+1)}{\Gamma(a + b + 2)}
	= \frac{\Gamma(a+2)\Gamma(b)}{\Gamma(a+1)\Gamma(b+1)}
	= \frac{a + 1}{b}
	$$
	
	\noindent so that
	$$
	1/E(g^{-1}) = \frac{b}{a + 1}
	$$
	
	\noindent as claimed.
}










As is happens we can perform step 2 of the algorithm in Section 2 efficiently if we reuse calculations from each step.
In each step only a single variable is added or removed from a
model $\vgamma_k$ at a time. 
Suppose that the current model to be changed is $\vgamma_k$ and that we
want to change the $j$th element of $\vgamma_k$. Let
$$
\vgamma_{jk}^{(i)} = (\gamma_{1k},\ldots,\gamma_{j-1,k},i,\gamma_{j+1,k},\ldots,\gamma_{pk})^T,
$$

\noindent be the vector $\vgamma_k$ where the $j$th element of $\vgamma_k$ is set to $i$.
For an update the current value of $\gamma_{jk}$ is 0 and want to set this value to 1. 

In this case we will employ the following shift in notation.
\begin{itemize}
	\item Let $\mX$ correspond to $\mX_{\vgamma}$; and
	\item Let $\vz$ correspond to $\mX_j$.
\end{itemize}

\noindent 
For an downdate the current value of $\gamma_{jk}$ is 1 and want to set this value to 0. 
In this case we will employ the following shift in notation.
\begin{itemize}
	\item Let $\mX$ correspond to $\mX_{\vgamma_{jk}^{(0)}}$; and
	\item Let $\vz$ correspond to $\mX_j$.
\end{itemize}

\noindent In both cases we let $\mC = [\mX,\vz]$ where $\mX$ is a $n\times d$
matrix, $\vz$ is a $n$ vector and $\mC$ is a $n\times (d+1)$ matrix. Throughout
we will assume that the quantities $\mX^T\mX$, $\mX^T\vz$, $\vz^T\vz$, $\mX^T\vy$
and $\vz^T\vy$ have been previously calculated. The update and downdate equations
are based on the following matrix block-inverse formula.



%\bigskip
%\noindent We will also use the following block determinant lemma.

%\bigskip 
%\noindent {\bf Lemma 2:} The block determinant formula is:
%$$
%\left|\begin{array}{cc}
%\mA & \mB \\
%\mC & \mD 
%\end{array}\right| = |\mA||\mD - \mC\mA^{-1}\mB|.
%$$



\bigskip 
\noindent We will use the above formula for efficient updates/downdates
of the model ``inverse'' matrices, coefficient vector, and 
$R$-squared 
%and model ``determinant'' 
values.