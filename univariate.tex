\subsection{Variational Bayes for zero-inflated count models}

\noindent Consider the model

$$
y_i = r_i x_i, 1 \leq i \leq n,
$$

\noindent where $x_i \sim \Poisson{(\lambda)}$ independent of
$r_i \stackrel{\text{ind.}}{\sim} \Bernoulli{(\rho)}, 1 \leq i \leq n$.

We employ priors $\rho \sim \Unif{(0, 1)}$ and $\lambda \sim \Gamma{(0.01, 0.01)}$. 
 
% TODO: Add graphical model

We use a factorised approximation to the full likelihood, as detailed in \citep{ormerod10}.
The use of conjugate priors in the full model yields easier mean field updates in the
variational approximation.

It can be shown via standard algebraic manipulations that the
full conditionals for $\lambda, \rho$ and $\vr$ are:

$$
\begin{array}{rl}
\lambda | \textbf{rest} &\sim \myGamma{(\alpha_\lambda + \vone^T\vx, \beta_\lambda + \vone^T\vr)}, \\ [0.5ex]
\rho | \textbf{rest} &\sim \Beta{(\alpha_\rho + \vone^T \vr, \beta_\rho + n - \vone^T\vr)} \\ [0.5ex]
\mbox{ and } \quad r_i | \textbf{rest} &\sim \Bernoulli{(\text{expit}(\eta_i))}, \quad 1 \leq i \leq n.
\end{array}
$$

% Step Two: Assume q(r_i) = Bernoulli(\rho_i), 1 \leq i \leq n for some known \rho_i. Find the
% variational Bayes updates of the q-densities q(\lambda) and q(\rho) corresponding to the
% factorisation
% q(\vr, \lambda, \rho) = q(\lambda) q(\rho) \sum_{i=1}^n q(r_i)

\noindent We assume a factorised approximation of the form

$$
q(\lambda, \rho, \vp) = q(\lambda) q(\rho) \left [ \prod_{i=1}^n q(r_i) \right ]
$$

\noindent where $q(\lambda)$ is a Gamma distribution, $q(\rho)$ is a Beta distribution 
and $q(r_i)$ are Bernoulli distributions.

\noindent This leads to the following functional forms of the optimal q-densities

$$
\begin{array}{l}
\mbox{$q^*(\lambda)$ is the $\myGamma{\alpha_{q(\lambda)}, \beta_{q(\lambda)}}$ density function,} \\ [0.5ex]
\mbox{$q^*(\rho)$ is the $\text{Beta}(\alpha_{q(\rho)}, \beta_{q(\rho)})$ density function, and} \\ [0.5ex]
\mbox{$q^*(r_i)$  is the $\text{Bernoulli}(p_{q(r_i)})$ density function, $1 \leq i \leq n$,}
\end{array}
$$

%$$
%\begin{array}{c}
%q^*(\lambda) \sim \myGamma(\alpha_{q(\lambda)}, %\beta_{q(\lambda)}),
%q^*(\rho) \sim \text{Beta}(\alpha_{q(\rho)}, \beta_{q(\rho)}),
%\quad \mbox{ and } \quad  
%q^*(r_i) \sim \text{Bernoulli}(p_{q(r_i)}), \ \ 1 \leq i \leq n,
%\end{array}
%$$

\noindent where the parameters are updated according to Algorithm \ref{algorithm1}. 

\subsection{Lower bound}
The lower bound of the univariate model can be calculated directly to be
$$
\begin{array}{rl}
\bE_q \left\{ \log{p(\vx, \vr, \lambda, \rho)} - \log{q(\vr, \lambda, \rho)} \right\} &= T_1 + T_2 \\
\end{array}
$$

\noindent where
$$
\begin{array}{rl}
T_1 & \ds =
\alpha_\lambda \log{(\beta_\lambda)} + (\alpha_\lambda - 1) [\psi(\alpha_{q(\lambda)}) - \log{(\beta_{q(\lambda)})}] - \beta_\lambda \frac{\alpha_{q(\lambda)}}{\beta_{q(\lambda)}} - \log\Gamma(\alpha_\lambda) \\
& \ds \quad -\vp^T\frac{\alpha_{q(\lambda)}}{\beta_{q(\lambda)}} + \bE_q[\vx^T \log{(\lambda \vr)}] - \log\Gamma(\vx+1)) \quad \mbox{and} 
\\ [1ex]
T_2 &= - \vp^T \log \vp - (\vone - \vp)^T \log (\vone - \vp) - \log \Beta (\alpha_\rho, \beta_\rho) + \log \Beta (\alpha_q, \beta_q)
\end{array}
$$

\noindent with 
$$
\bE [x_i \log{(\lambda r_i)}]  =
	\begin{cases}
	0 & \textbf{if } x_i = 0 \\
	\bE_q [\log \lambda] = \psi(\alpha_{q(\lambda)}) - \log{(\beta_{q(\lambda)})} & \textbf{if } x_i \ne 0. \\
	\end{cases}
$$

\begin{algorithm} 
\caption[Algorithm 1]{Iterative scheme for obtaining the parameters in the
optimal densities $q^*(\lambda)$ and $q^*(\rho)$}
\begin{algorithmic}
\REQUIRE{$\alpha_{q(\rho)} \leftarrow \alpha_\rho + \vone^T\vp, 
\alpha_{q(\lambda)} \leftarrow \alpha_\lambda + \vone^T\vx$}
\WHILE{the increase in $\log{\underline{p}}(\vx;q)$ is significant}
\STATE $\beta_{q(\rho)} \leftarrow \beta_\rho + n - \vone^T\vp$
\STATE $\eta \leftarrow -\alpha_{q(\lambda)}/\beta_{q(\lambda)} + \psi{(a_{q{(\rho)}})} - \psi{(b_{q{(\rho)}})}$
\STATE $\vp_{q(\vr_0)} \leftarrow \expit{(\eta)}$
\STATE $\beta_{q(\lambda)} \leftarrow \beta_\lambda + \vone^T\vp$
\ENDWHILE
\end{algorithmic}
\label{algorithm1}
\end{algorithm}

%By taking the expectation of each full conditional with respect to 

% This should be made the numerical experiment section.
