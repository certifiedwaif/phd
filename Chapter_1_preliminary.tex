\chapter{Preliminary material}

% Outline that John suggested
\section{Real world problems}

The advent of digital computers and the Internet have lead to an explosion in the volume of data being
collected. With technological progress marching on, this trend seems only set to continue and accelerate. But
this data is only of value if it can be analysed and understood.  This incredible increase in the volume of
data has lead to corresponding computational difficulties in processing and modelling such large amounts of
data -- so-called Big Data which is so large that it is difficult to process on one computer. This data raises
new challenges which modern statisticians must be ready to meet. This realisation has lead to an explosion of
interest in data science, incorporating ideas from both statistics and computer science, in recent years.
Machine learning problems are being tackled with algorithms which use probability models for the data --
leading to the new field of statistical learning which combines many of the best elements of statistics and
machine learning \citep{James:2014:ISL:2517747} \citep{MacKay:2002:ITI:971143}
\citep{hastie01statisticallearning} \citep{Murphy:2012:MLP:2380985}.

\section{Motivation}

Approaches to modelling data are needed which can handle large volumes of data in a computationally
efficient manner while retaining the probabilistic underpinning of classical statistics and statistical machine
learning that leads to a rigorous underlying theory for inference.

\section{Approach}

Generalised linear mixed models are an appealing way to model data, as they are flexible enough to model a
range of data types and situations. But the Bayesian versions of these models typically required 
computationally demanding MCMC, which can also be prone to convergence problems. Instead, we consider approximate Bayesian inference tecniques, which are computationally efficient and deterministic. It is
desirable to use normal priors for the reegression co-efficients of these models, as these are easily
interpreted. But for Generalised Linear Mixed Models with a non-normal response, these priors are
non-conjugate, making VB difficult to apply as the required mean field updates are intractable. We apply
Gaussian Variational Bayes -- an extension to Variational Bayes, to fit a multivariate normal distribution
to the regression co-efficients of our models.

Exact inference for model selection for linear models with normal priors is computationally feasible when the
number of  covariates is small, with $p$ below 40. But exhaustively exploring the search space is not
efficient, and often not computationally feasible for a larger number of covariates. In this situation, we
adopt a population- based technique inspired by Rockova's work on population-based EM to efficiently explore
the posterior model space.

\section{Why Bayesian?}
The difference between frequentist and Bayesian approaches begins with a difference in philosophy.
Frequentists define an event's probability as its' relative frequency after a large number of trials.
while Bayesians view probability as our reasonable expectation about an event, representing our state of knowledge about the event.

There are many practical reasons to choose Bayesian approaches to modelling data.
It is flexible to complications. Complicated models can be built by chaining together multiple levels of
simple models. These models can then be fit to data by calculating the posterior probability of the
parameters using Bayes' Rule,
\[
	p(\vtheta | \vy) = \frac{p(\vy | \vtheta) p(\vtheta)}{\int p(\vy | \vtheta) p(\vtheta) d \vtheta}
\]
providing the resulting integral can be evaluated or approximated. There are many models which are difficult
to fit under the frequentist paradigm, as the likelihood is difficult to maximise. Furthermore, as the
Bayesian paradigm treats each of the parameters in a model as uncertain, the full uncertainty associated with
all of the parameters can be estimated in the uncertainty in the posterior distribution. This approach avoids
many of the pitfalls of statistical inference encountered with the frequentist approach using significance
testing and p-values \citep{Cox2005}.


The ability to build a model a part at a time and have the uncertainty propagate makes Bayesian modelling 
particularly appropriate for mixed and hierarchical models.
Uncertainty regarding model selection is taken into account

\section{What problems in thesis}

In this section, we introduce the major problems that will be addressed in this thesis. The themes of flexible
modelling of data using Generalised Linear Mixed Models and model selection of linear models with normal
priors  will be explored.

\subsection{GLMM}

Generalised Linear Mixed Models, an extension of Generalised Linear Models to include both fixed and
random effects, are flexible to many complicated modelling situations. The standard technique for fitting
Bayesian versions of these models is to use Monte Carlo Markov Chains techniques such as Gibbs sampling.
Linear and generalised linear regression models are the standard tools used by applied statisticians to
explain the relationship between an outcome variable and one or more explanatory variables. They provide a
general method  to analyse quantified relationships between variables within a data set in an easily
interpretable way. A standard assumption is that the outcomes are independent, and that the effect of the
explanatory variables on the outcome is fixed. But if the outcomes are dependent and this assumption is not
met, then linear and generalised linear models can be extended to linear mixed models. These allow us to
incorporate dependencies amongst the  observations via the assumption of a more complicated covariance
structure, including random effects for  different subgroups or longitudinal data and other extensions such as
splines, missing data and measurement error. This additional flexibility makes their application popular in
many fields, such as public health, psychology and agriculture.

While mixed models are very useful for gaining insight into a data set, fitting them can be computationally
challenging. For all but the simplest situations, fitting these models involves computing high-dimensional
integrals which are often analytically and computationally intractable. Thus, approximations must be used.

In the frequentist paradigm, model parameters are fixed and uncertainty enters the model through random
errors, which have associated variance. The data is modelled as a combination of these fixed parameters and
random errors. In the Bayesian paradigm, the uncertainty enters the model by assuming parameters are random
variables, while the data is fixed.

\subsection{Model selection}

While new types of model are continually being developed, linear models with normal priors remain a popular
and essential modelling tool owing to the ease of fitting these models, statistical inference on the
parameters and, most importantly, the ease which which these models can be interpreted. But for a data set
with a moderate or large number of parameters, the question is immediately raised of which covariates we should
include in our model. One of the problems that we address in this thesis is \emph{model selection} on linear
models with normal priors.

\section{Generalised Linear Mixed models}
All of the models considered in this thesis belong to the family of generalised linear mixed models - a
flexible set of models which can model continuous or discrete responses and incorporate both fixed and
random effects.

\subsection{Exponential family and the canonical form of linear regresison models}

An exponential family is a set of probability distributions of the form
\[
	f_X(x|\vtheta) = h(x) g(\theta) \exp(\eta(\theta)^\top T(x)).
\]

Most commonly used probability distributions in statistics such as the normal, exponential, gamma,
chi-squared, beta, Dirichlet, Bernoulli, categorical, Poisson, Wishart and Inverse Wishart distributions can
be represented in this form. A canonical form of generalised linear regression models is
\[
	\log p(\vnu; \vy) = \vy^\top \mC \vnu - b(\mC \vnu).
\]

Different choices of the function $b:\R \to \R$ lead to different generalised linear mixed models
assuming the probability distribution for $y$ is one of the probability distributions mentioned above.

% \begin{table}
% 	\caption{Table of b functions}
% 	\label{tab:b_functions}
% 	\begin{tabular}{|l|lll|}
% 		\hline
% 		Distribution                         & Parameter & Natural parameter       & Inverse parameter mapping \\
% 		\hline
% 		Bernoulli distribution               & $p$       & $\log{\frac{p}{1 - p}}$ & $\frac{1}{1 + e^\eta}$    \\
% 		Poisson distribution                 & $\lambda$ & $\log \lambda$          & $e^\eta$                  \\
% 		Normal distribution (known variance) & $\mu$     & $\frac{\mu}{\sigma}$    & $\sigma \eta$             \\
% 		\hline
% 	\end{tabular}
% \end{table}

\subsection{Mixed Models}

We follow the convention of \citep{Zhao2006}. Generalised Linear Mixed Models from the exponential family with
Gaussian random effects take the general form
$$
\begin{array}{rl}
	\vy | \vbeta, \vu &= \exp{\{ \vy^\top (\mX \vbeta + \mZ \vu) - \vone^\top b(\mX \vbeta + \mZ \vu) + \vone^\top c(\vy) \}}, \\
	\vu | \mG &\sim \N(\vzero, \mG),
\end{array}
$$
where the fixed effects are denoted by the vector $\vbeta$ and the random effects are denoted by $\vu$. The
design matrix for the fixed effects is denoted by $\mX$ and the design matrix for the random effects are
denoted by $\mZ$. The choice of the functions $b$ and $c$ allows this general structure to be adapted to
particular situations -- a choice of $b(x) = e^x$ corresponding to the Poisson family of distributions
specifies a Poisson linear mixed model appropriate for modelling count data, while a choice of $b(x) = \log(1
+ e^x)$ corresponding to the logistic family of distributions specifies a logistic linear mixed model
appropriate to modelling binary data.

Random effects are very flexible in the variety of models they allow us to fit to our data. Through
specification of the covariance structures in the matrix $\mG$ with the appropriate data in the design matrix
$\mZ$, complicated dependencies amongst the responses $\vy$ can be specified, allowing modelling of
longitudinal data, fitting smoothing splines to the data and modelling spatial relationships between
responses. This allows us to fit hierarchical models with random intercepts and slopes, capturing levels of variation within groups within the data.
\citep{Gelman2007}
% TODO: Not happy with how this paragraph is written. I can express this idea better.

\section{Splines and smoothing}
The most general form of the univariate regression problem is
\[
	y_i = f(x_i)
\]

where $f: \R \to \R$ is unknown, and we wish to estimate it as $\hat{f}$. We treat the functional form of
$f$ as unknown, and attempt to estimate it purely from the data available to us. Fully non-parametric regression
is a difficult problem to solve, but the problem can be simplified by pre-specifying the points at which the
function may change curvature, the so-called knots.

% \subsubsection{Penalised spline}
% \subsubsection{B-splines}
\subsection{B-Splines}

% This is taken from the Wikipedia page on the subject
There are many families of basis functions which can be conveniently used for function approximation,
including orthogonal polynomials. The B-spline basis \citep{DeBoor1972} is numerically stable and efficient to
computationally evaluate. A B-Spline is a piecewise polynomial function of degree $< n$ in a variable $x$. It
is defined over a domain $t_0 \leq x \leq t_m, m=n$. The points where $x = t_j$ are known as knots or break-
points. The number of internal knots is equal to the degree of the polynomial if there are no knot
multiplicities. The knots must be in ascending order. The number of knots is the minimum for the degree of the
B-spline, which has a non-zero value in the range between the first and last knot. Each piece of the function
is a polynomial of degree $< n$ between and including adjacent knots. A B-Spline is a continuous function at
the knots. When all internal knots are distinct its derivatives are also continuous up to the derivative of
degree $n - 1$. If internal knots are coincident at a given value of x, the continuity of derivative order is
reduced by 1 for each additional knot.

For any given set of knots, the B-spline for approximating a given function is a unique linear combination
of B-spline basis functions with the basis functions defined as:
$$
\begin{array}{rl}
	B_{m, i}(x; \vkappa) &= (\kappa_{i + m  + 1} - \kappa_{i})[\kappa_i, \ldots, \kappa_{i + m + 1}](. - x)^m_+ \\
	&= \frac{x - \kappa_i}{\kappa_{i + 1} - \kappa_i} Q_{i, k-1} (x; \vkappa) + 
										\frac{\kappa_{i + k + 1} - x}{\kappa_{i + k + 1} - \kappa_{i + 1}} Q_{i, k-1} (x; \vkappa)
\end{array}
$$

\begin{align*}
	B_{i, 0}(x) & := \begin{cases}                                                                                                        
	1           & \text{if } \kappa_i \leq x < \kappa_{i+1}                                                                                         \\
	0           & \text{otherwise}                                                                                                        
	\end{cases}
	% B_{i, k}(x) & := \frac{x - \kappa_i}{\kappa_{i + 1} - \kappa_i} Q_{i, k-1} (x) + 
	% 									\frac{\kappa_{i + k + 1} - x}{\kappa_{i + k + 1} - \kappa_{i + 1}} Q_{i, k-1} (x). 
\end{align*}

\noindent where "[]" denotes the divided difference and
% B-Splines
$$
Q_{m, i}(x; \kappa) =
\begin{cases}
B_{m, i}(x; \kappa),& \kappa_{i + m} > \kappa_i \\
0, & \text{otherwise}.
\end{cases}
$$
We define the BSpline basis this way to remain correct in the case where knots are repeated in $\vkappa$. We
choose piecewise cubic splines as cubics are numerically well behaved while still capturing the curvature of
functions we wish to approximate well \citep{Press:2007:NRE:1403886}. Thus we select the knot sequence
$\vkappa$
$$
a = \kappa_1 = \kappa_2 = \kappa_3 = \kappa_4 < \kappa_5 < \ldots < \kappa_{K+5} = \kappa_{K+6} = \kappa_{K+7} = \kappa_{K+8} = b.
$$

\subsection{O'Sullivan Splines}

% $B_{ik} = B_k (x)$
% $B_x = [B_1(x), \ldots, B_K+4(x)]$
Let $B_1$, \ldots, $B_{K+4}$ be the cubic B-spline basis functions defined by the knots $\kappa_1$ to
$\kappa_{K+4}$. O'Sullivan splines are penalised splines which are penalised using the $\mOmega$ matrix.
Let $\mOmega$ be the $(K+4) \times (K+4)$ matrix where the $(k, k')-th$ element is
\[
	\mOmega_{k k'} = \int_a^b B''_k(x) B''_{k'}(x) dx.
\]
Then the O'Sullivan spline estimate of the true function $f$ at the point $x$ is
$\hat{f}_O(x; \lambda) = \mB_x \hat{\vnu}_O$, where
$\hat{\vnu}_O = (\mB^\top \mB + \lambda \mOmega)^{-1} \mB^\top \vy$.

$\mOmega$ is defined in this way to penalise oscillation, which is measured by the second derivative.
The penalty differs from P-splines in that the P-spline penalty matrix is $\mD_2^\top \mD_2$ where $\mD_2$ is
the second-order differencing matrix.

% Divided difference notation?
% Lagrange's interpolating polynomials?
% Semiparametric regression / Connection to mixed models
We follow the discussion of semiparametric regression in \cite{ruppert_wand_carroll_2003}.
Using a mixed models setup to fit spline models protects against overfitting.
Constructing a $\mZ$ matrix with the appropriate B-Spline function evaluations in each of the rows, where
each column corresponds to a knot.

  
\subsection{Degrees of freedom}

% From Section 2.5.2, Semiparametric regression

Consider the fixed effects model $\vy = \mX \vbeta$ where $\mX \in \R^{n \times p}$. If the hat matrix is
defined as $\mH = \mX (\mX^\top\mX)^{-1} \mX^\top$ then

\begin{align*}
	\tr(\mH) & = \tr[\mX(\mX^\top\mX)^{-1}\mX^\top]  \\
	         & = \tr[\mX^\top\mX (\mX^\top\mX)^{-1}] \\
	         & = \tr(\mI_p)                          \\
	         & = p                                   
\end{align*}

which is the number of parameters in the fixed effects model. Thus we use the trace of the hat matrix as
the definition of the effective degrees of freedom of the model. This definition then naturally extends to
more general smoother matrices which can be used to fit models involving splines or other kinds of smoothing.

% \subsection{Crossed effects}
% \subsection{Spatial structure}

\section{Model selection}
% Given a full data matrix $\mX \in \R^{n \times p}$, we may wish to find the subset of variables $\gamma \in \{
% 1 \ldots p \}$ such that $\mX_\gamma \vbeta_\gamma$ describe the data in $\vy$ best.

The problem of selecting a statistical model from a set of candidate models given a data set, hence referred
to as \emph{model selection}, is one of the most important problems encountered in practice by applied
statistical practitioners. It is one of the central tasks of applied statistical analysis, and there is a
correspondingly large literature on the subject.The problem of model selection for normal linear models is
particularly well studied, owing to the popularity and importance of normal linear models in applications.

The bias-variance trade-off is one of the central issues in statistical learning. The guise this issue takes
in model selection is balancing the quality of the model fit against the complexity of the model, in an
attempt to find a compromise between over-fitting and under-fitting, in the hope that the model fit will
generalise well beyond the training data we have observed to the general population and that we haven't simply
learned the noise in the training set.

There have been many approaches to model selection proposed, including criteria based approaches, variable
selection, approaches based on functions of the residual sum of squares, penalised regression such as the
lasso and $L_1$ regression, and Bayesian modelling approaches. Model selection is a difficult problem in high
--dimensional spaces in general because as the dimension of the space increases, the number of possible models
increases combinatorially. Many model selection algorithms use heuristics in an attempt to search the model
space more efficiently but still find an optimal or near-optimal model within a reasonable period of time. A
major motivation for this field of research is the need for a computationally feasible approach to performing
model selection on large scale problems where the number of covariates is large.

Many approaches to model selection have been investigated, with various model selection criteria
proposed to attempt to balance model likelihood against model complexity. One approach to model selection is
to select between entire models, using a model selection criterion. These criterion may have a fixed penalty
for model complexity, such as Akaike's Information Criterion \citep{Akaike1974}, the Risk Inflation Criterion
\citep{Foster1994}, the Schwarz criterion or Bayesian Information Criterion \citep{Schwarz1978}, the Deviance
Information Criterion \citep{Spiegelhalter2016} or the Principle of Minimum Description Length
\citep{Hansen2001}. Alternatively, the penalty may be adaptive/data--dependent as in \citep{George2000}. An
alternative to model selection criterion is to select models based on their posterior probability, such as by
selecting the median probability model as in \citep{Barbieri2004}. In a Bayesian context is to use Bayes
factors to compare the posterior likelihoods of the candidate models to see which is most probable given the
observed data. This can be done, for example, by using Bayes Factors as in \citep{Kass1993}. Rather than
selecting one candidate model, several models can be combined together using Bayesian model  averaging, as in
\citep{Hoeting1999}, \citep{Raftery1997}, \citep{Fernandez2001} or \citep{Papaspiliopoulos2016}. Or model
selection can be made implicit in the model fitting process itself, as in ridge regression \citep{Casella1980},
of which the well-known lasso is a special case \citep{Tibshirani1996}. As \citep{Breiman1996} and
\citep{Efron2013} showed, while  the standard formulation of a linear model is unbiased, the goodness of fit of
these models is numerically  unstable. Breiman showed that by introducing a penalty on the size of the
regression co- efficients such as  in ridge regression, this numerical instability can be avoided. This
reduces the variances of the co-efficient estimates, at the expense of introducing some bias --- 
which is another instance of the bias--variance trade--off.

Many approaches to model selection have been investigated, with various model selection criteria proposed to
attempt to balance model likelihood against model complexity. One approach to model selection is to select
between entire models, using a model selection criterion. These criterion may have a fixed penalty for model
complexity, such as Akaike's Information Criterion \citep{Akaike1974}, the Risk Inflation Criterion
\citep{Foster1994}, the Schwarz criterion or Bayesian Information Criterion \citep{Schwarz1978}, the Deviance
Information Criterion \citep{Spiegelhalter2016} or the Principle of Minimum Description Length
\citep{Hansen2001}. Alternatively, the penalty may be adaptive/data--dependent as in \citep{George2000}.

An alternative to model selection criterion is to select models based on their posterior probability. In a
Bayesian context, this can be accomplished by comparing the posterior likelihoods of the candidate models to
see which is most probable given the observed data. This can be done, for example, by using Bayes Factors as
in \citep{Kass1993}. Alternately, rather than selecting one candidate model, several models can be combined together using Bayesian model  averaging, as in \citep{Hoeting1999}, \citep{Raftery1997}, \citep{Fernandez2001} or \citep{Papaspiliopoulos2016}.

Model selection can also be made implicit in the model fitting process itself, as in ridge regression
\citep{Casella1980}, of which the well-known lasso is a special case \citep{Tibshirani1996}.

As shown in
papers by \citep{Breiman1996} and \citep{Efron2013} showed, while  the standard formulation of a linear model
is unbiased, the goodness of fit of these models is numerically  unstable. Breiman showed that by introducing
a penalty on the size of the regression co- efficients such as  in ridge regression or the lasso, this numerical
instability can be avoided. This reduces the variances of the co-efficient estimates, at the expense of
introducing some bias --- the bias-- variance trade--off.

A special case of model selection is variable selection, where the focus is on selecting individual
covariates, rather than entire models. This approach can either be Fully Bayesian or Empirically Bayesian as
in \citep{Cui2008}. Variable selection approaches involve a stochastic search over the variables in the model
space. This search can be driven by posterior probabilities, as in \citep{Casella2006}, or by Gibbs sampling
approaches such as in \citep{George1993}. These two approaches of model selection and variable selection can
be combined, as in \citep{Geweke1996}. Variable selection can also be accomplished by selecting the median
probability model, consisting of those models whose posterior inclusion probability is at least $1/2$, as in
\citep{Barbieri2004}.

Another approach to model selection is to focus on selecting individual covariates, rather than entire
models. This approach can either be Fully Bayesian or Empirically Bayesian as in \citep{Cui2008}. Variable
selection approaches involve a stochastic search over the variables in the model space. This search can be
driven by posterior probabilities, as in \citep{Casella2006}, or by Gibbs sampling approaches such as in
\citep{George1993}. These two approaches of model selection and variable selection can be combined, as in
\citep{Geweke1996}.

A challenge to applying this method of model selection is that exact model fitting may be computationally
infeasible for models involving even moderate numbers of observations and covariates, and popular alternatives
for fitting Bayesian models such as Monte Carlo Markov Chains (henceforth referred to as MCMC) are still
extremely computationally intensive.

Variational Bayes (see \cite{Ormerod2010}) is a computationally
efficient, deterministic method of fitting Bayesian models to data. Variational Bayes approximates the true
posterior $p(\vy, \vtheta)$ by minimising the KL divergence between the posterior and the  approximating
distribution $q(\vtheta)$.

A linear model with normal priors allows exact inference on the regression and model selection parameters in
closed form, which might appear to negate the beenfits of a variational approximation to the model. However,
the performance of our variational approximation should remain similiar if the priors are altered to cater for
complications such as robustness, while exact Bayesian inference calculations are no longer possible in closed
form in these situations.

\cite{Zellner1986} suggested a particular form of conjugate Normal-Gamma family where the Bayes factors have a
relatively simple form, incorporating a parameter $g$ to control mixing between the model fit from the data
and a prior specification of model fit. This immediately raises the question of how $g$ should be chosen, and
whether it should be fixed or have a prior specification. \cite{Liang2008} showed that fixed choices of $g$
lead to paradoxes such as Bartlett's Paradox and the Information Paradox, and so a prior specification should
be preferred. There are many ways of choosing a prior on $g$. Using a mixture of $g$-priors has the advantage
of adapting the degree of shrinkage to the prior model dependent on the data.

\section{Models and Assumptions}

We now introduce the models that will be considered in this thesis, and the assumptions which will be made
when using these models.

% Non-Bayesian
\subsection{Frequentist Approachs to Model Selection}
Many approaches to model selection have been investigated, with various model selection criteria proposed to
attempt to balance model likelihood against model complexity. One approach to model selection is to select
between entire models, using a model selection criterion. These criterion may have a fixed penalty for model
complexity, such as Akaike's Information Criterion \cite{Akaike1974}, the Risk Inflation Criterion
\cite{Foster1994}, the Schwarz criterion or Bayesian Information Criterion \cite{Schwarz1978}, the Deviance
Information Criterion \cite{Spiegelhalter2016} or the Principle of Minimum Description Length
\cite{Hansen2001}. Alternatively, the penalty may be adaptive/data--dependent as in \cite{George2000}.

Model selection can also be made implicit in the model fitting process itself, as in ridge regression
\cite{Casella1980}, of which the well-known lasso is a special case \cite{Tibshirani1996}. As shown in
papers by \cite{Breiman1996} and \cite{Efron2013} showed, while  the standard formulation of a linear model
is unbiased, the goodness of fit of these models is numerically  unstable. Breiman showed that by introducing
a penalty on the size of the regression co- efficients such as  in ridge regression, this numerical
instability can be avoided. This reduces the variances of the co-efficient estimates, at the expense of
introducing some bias -- the bias- variance trade--off.

\subsection{Information criteria}
In a frequentist context, there are many functions which can be used to judge which model is best, such as
AIC, BIC etc. These are functions $f: \gamma \to \R^+$ which allow the models under consideration to be
ranked, and the best model chosen from those available. Thus the optimal model selected by an information
criteria is $\gamma^* = \min_\gamma f(\gamma)$. These functions typically attempt to balance log likelihood
against the complexity of the model, achieving a compromise between each.

% \mgc{AIC, BIC, DIC, Mallow's $C_p$}

Information criteria are frequently used to compare amongst models. Letting $\vgamma$ denote the candidate model,
Information Criteria take the form "-2 times the log-likelihood plus a term penalising for complexity of
the model"
\[
	\text{Information Criteria} = -2 \log p(\vy | \hat{\vtheta}_\vgamma) + \text{complexity penalty}
\]

\noindent where $\log p(\vy | \hat{\vtheta_\gamma})$ is the log-likelihood of the model and the complexity
penalty is a function of the sample size $n$ and the number of parameters $p$ of the model. Information
criteria attempt to successfully compromise between goodness of fit and model complexity.

The most popular of the information criteria is the Akaike Information Criterion (AIC) \citep{Akaike1974}. AIC
calculates an estimate of the information lost when a given model is used to represent the process that
generates the data and so is an estimator of the Kullback-Leibler divergence of the true model from the fitted
model. The AIC of the model $\vgamma$ is defined as
\[
	\text{AIC}(\vgamma) = -2 \log p(\vy | \hat{\vtheta}_\vgamma) + 2 p_\vgamma
\]

\noindent where $|\vgamma|$ is the number of parameters in the model $\vgamma$. The model with the lowest AIC
is selected as the `best`.

Of a similiar form as the AIC, but derived via a more Bayesian framework is the Bayesian Information Criterion
or BIC. The BIC approximates the posterior probability of the candidate model $\vgamma$. The BIC is defined as
\[
	\text{BIC}(\vgamma) = -2 \log p(\vy | \hat{\vtheta}_\vgamma) + p_\vgamma \log(n).
\]

\noindent This is a more severe penalty for model complexity than in the Akaike's Information Criteria when
$n$ is greater than $8$. BIC can be shown to be approximately equivalent to model selection using Bayes
Factors in certain contexts, as shown by \cite{Kass1993}.

\subsection{Penalised regression}
Penalised regression methods trade introducing some bias in the estimator for reducing the variance and thus
fitting a more parsimonious model. The major advantages are that a model with fewer covariates will be
correspondingly easier to interpret, and that the variance of the estimator will be less. In penalised
regression, the regression co-efficients are subjected to a penalty or constraint.
$$
\widehat{\vbeta}_{\text{penalised}} = \argmin_\vbeta \|\vy - \mX \vbeta\|_2^2 + \text{penalty}(\vbeta)
$$

From a Bayesian perspective, the penalty can be considered as a prior distribution on the regression 
co-efficients where smaller values of $\vbeta$ are given more weight than larger ones. Here the penalised
estimate of the regression co-efficients is the mode of their posterior distribution.
\subsubsection{Lasso regression}
Lasso regression is a penalised regression method developed by \citep{Tibshirani1996}, which was directly
inspired by ridge regression.
It follows from Minkowski's inequality that this function is convex, and thus can be the optimisation problem
is convex, and can be solved using standard methods from convex optimisation \citep{Boyd2010}.

$$
\widehat{\vbeta}_{\text{lasso}} = \argmin_\vbeta \|\vy - \mX \vbeta\|_2^2 \text{ subject to } \|\vbeta\|_1 \leq \lambda
$$

The constraint on the $l_1$ norm has the effect of shrinking the co-efficients, and setting some of them to
zero. This forces the models fit by lasso regression to be sparse, providing model selection as part of the
model-fitting process.

A disadvantage of lasso regression are that the constraint on the regression co-efficients depends on the
tuning parameter which must be selected a priori or through cross-validation. But a much greater issue is that
the model selection process intrinsic to lasso regression does not take into account the uncertainty of the
model selection process itself, as Bayesian model selection methods do.

\subsubsection{Ridge regression}
Ridge regression is another penalised regression method, which predates lasso regression
\cite{Hoerl1970}. The penalty on the regression co-efficients shrinks the estimated co-efficients towards
zero, ensuring sparsity of the fitted model.
$$
\widehat{\vbeta}_{\text{lasso}} = \argmin_\vbeta \|\vy - \mX \vbeta\|_2^2 \text{ subject to } \|\vbeta\|_2 \leq \lambda
$$

\subsection{Bayesian Approaches to Model Selection}
An alternative to model selection criterion is to select models based on their posterior probability. In a
Bayesian context, this can be accomplished by comparing the posterior likelihoods of the candidate models to
see which is most probable given the observed data. This can be done, for example, by using Bayes Factors as
in \cite{Kass1993}. Rather than selecting one candidate model, several models can be combined together using
Bayesian model averaging, as in \cite{Hoeting1999}, \cite{Raftery1997}, \cite{Fernandez2001} or
\cite{Papaspiliopoulos2016}.

The problem of model selection can also be considered within a Bayesian framework. Within this framework,
candidate models can be selected by comparing their posterior likelihoods or Bayes Factors to see which is
most probable given the observed data, as in \cite{Kass1993}. Bayesian model selection approaches can
incorporate prior information, such as the prior probability of selecting each candidate model $\vgamma$.

A special case of model selection is variable selection, where the focus is on selecting individual
covariates, rather than entire models. Variable selection approaches search over the
variables in the model space for the best covariates to include in the candidate model. Due to the large
number of possible combinations of covariates -- typically $2^p$ where $p$ is the number of covariates, such
searches are often stochastic. This approach can either be Fully Bayesian or Empirically Bayesian as in
\cite{Cui2008}.  This search can be driven by posterior probabilities, as in \cite{Casella2006}, or by Gibbs
sampling approaches such as in \cite{George1993}. These two approaches of model selection and variable
selection can be combined, as in \cite{Geweke1996}. Variable selection can also be accomplished by selecting
the median probability model, consisting of those models whose posterior inclusion probability is at least
$1/2$, as in \cite{Barbieri2004}.

% \subsubsection{Linear regression with normal prior and $g$ hyperprior}

% Zellner's g-prior

% \[
% 	\vbeta_\vgamma | \sigma^2, \mathcal{M}_\vgamma \sim \N(\vzero, g \sigma^2 (\mX^\top \mX)^{-1})
% \]

% \noindent which scales the Fisher information $\sigma^2 (\mX^\top \mX)^{-1}$ by $g$, was first introduced in
% \cite{Zellner1986}. It is widely used for variable selection in linear models with normal priors owing to its'
% computational efficiency in evaluating marginal likelihoods and model selection and conceptually simple
% interpretation.

% This model specification performs shrinkage on the regression co-efficients. As stated in \cite{Hastie2015},
% it is best to `bet on sparsity': ``Use a procedure that does well in sparse  problems, since no procedure does
% well in dense problems.''. With different choices of hyper- prior on $g$, this shrinkage can be made to behave
% like Bayesian versions of the lasso, ridge regression or generalisations of these \citep{Hahn2015}.  Thus this
% model specification has the advantage of performing well on model selection problems where the true model is
% sparse in the sense that the number of true non-zero covariates $p$ is less than the number of samples $n$.

% \begin{itemize}
% % \item Sparsity
% % \item Ease of interpretation from fewer regression co-efficients
% % \item Computational convenience
% \item Convex optimisation problem
% % \item Does not take model selection uncertainty into account
% \item Bias-Variance trade-off -- trade some bias in the estimator for a decrease in variance
% \item Deal with collinearity
% \end{itemize}


\section{Approximate inference}
When the prior and model chosen for a Bayesian model is conjugate, the posterior distribution is available in
closed form and can be easily calculated.
When the prior is non-conjugate, the integral to calculate the posterior distribution is typically intractable
and so numerical methods must be used to calculate it approximately.
The gold standard for Bayesian inference is to use Monte-Carlo Markov Chain methods such as Metropolis-Hastings
or Gibbs sampling. But these methods are computationally intensive, to the point where they are simply
impractical in Big Data situations where $n$ or $p$ are large. Moreover, they can be prone to convergence 
problems.

Thus there is a need for approximate Bayesian inference methods which are less computationally intensive while
being almost as accurate.

 
\subsection{Variational Bayes}
\label{sec:vb}

We now introduce Variational Bayes, the popular approximate inference method for Bayesian models. It is used
to accelerate Bayesian model fitting by tens or hundreds of times, with only minor loss in accuracy for some
models. This method plays a central role in this thesis, particularly in the second and fourth chapters.

\subsubsection{Definition}

Much of Bayesian inference is based on the posterior distribution of a model's parameters given observed data 
defined by $p(\vtheta|\vy) = p(\vy|\vtheta)p(\vtheta)/p(\vy)$ where $\vy$ is a vector of observed data,
$\vtheta$ are the model parameters $p(\vy|\vtheta)$ is the model distribution, $p(\vtheta)$ is a prior 
distribution on $\vtheta$ and $p(\vy)=\int p(\vy|\vtheta)p(\vtheta)d\vtheta$. Here the integral is performed
over the domain of $\vtheta$. If a subset of $\vtheta$ are discrete random variables then the integral over
these parameters is replaced with a combinatorial sum over all possible values of these discrete random 
variables.

The calculation of the true posterior distribution is often either computationally intractable or no closed form
exists for the posterior distribution and so an approximation is required. Variational approximation is a
class of methods for transforming the problem of approximating a distribution $p(x)$ by another more
convenient distribution $q(x)$. Variational approximation can be viewed as minimising the Kullback- Leibler
divergence between the true posterior $p(\vtheta|\vy)$ and an approximating distribution $q(\vtheta)$,
sometimes called a $q$-density. Suppose that a class of candidate approximating distributions
$q(\vtheta)$ is parameterised by $\vxi$ and write $q(\vtheta)\equiv q(\vtheta;\vxi)$. We attempt to find an 
optimal approximating distribution $q^*(\vtheta)$ such that
\[
	\ds q^*(\vtheta) = \argmin_{\vxi \in \Xi} \text{KL} \{ {q(\vtheta;\vxi) || p(\vtheta|\vy)} \}.
\]

The density function of a random vector $\vu$ is denoted by $p(\vu)$.  The conditional density function of a
random vector $\vu$ given $\vv$ is denoted by $p(\vu|\vv)$. Consider a generic Bayesian model with parameter
vector $\vtheta \in \Theta$. Throughout this section we assume that $\vy$ and $\vtheta$ are continuous random
vectors. The KL divergence between the probability distributions $p$ and $q$ is defined as
\[
	\KL(q || p) \equiv \int q(\vtheta) \log \left \{ \frac{q(\vtheta)}{p(\vtheta | \vy)} \right \} d \vtheta.
\]

Consider a model $p(\vtheta|\vy)$ which is of interest to us but computationally difficult or intractable to 
fit. We may be able to gain much of the same insight from a given data set by fitting an accurate approximation 
of the model, allowing us to summarise the data and perform statistical inference.

Let us denote the approximating model $q(\theta)$. Then we seek a $q(\theta)$ which approximates
$p(\vtheta|\vy)$ as closely as possible in some sense. 
We may seek a best approximating distribution $q^*(\vtheta)$ to a given posterior distribution
$p(\vtheta|\vy)$ from a parameteric family of approximating distributions $Q$ by finding
\[
	q^*(\vtheta) = \argmin_{q \in Q} \KL \{ {q(\vtheta) || p(\vtheta|\vy)} \}.
\]

If $\vtheta$ is partitioned into $M$ partitions $\vtheta_1$, $\vtheta_2$, \ldots, $\vtheta_M$ then a 
simple form of approximation to adopt is the factored approximation of the form
\[
	q(\vtheta) = \Pi_{i=1}^M q(\vtheta_i)
\]

where each of the density $q(\vtheta_i)$ is a member of a parametric family of density functions. This form of
approximation is computationally convenient, but assumes that the partitions of $\vtheta$ are completely
independent of one another.

The optimal mean field update for each of the parameters $\vtheta_i$ can be shown to be
\[
	q^*(\vtheta_i) \propto \exp \{ \E [\log p(\vy; \vtheta)] \}.
\]

For details of the proof, and a more thorough introduction to the topic of variational approximations, see
\cite{Ormerod2010}.

\noindent It can easily be shown that
\[
	\ds \log p(\vy) = \int q(\vtheta;\vxi)\frac{p(\vy|\vtheta)p(\vtheta)}{q(\vtheta;\vxi)} d\vtheta + \text{KL}(q(\vtheta;\vxi)||p(\vtheta|\vy)).
\]

\noindent As the Kullback-Leibler divergence is strictly positive, the first term on the right hand side
is a lower bound on the marginal log-likelihood which we will define by
$$
\ds \log \underline{p}(\vy;\vxi) \equiv \int q(\vtheta;\vxi)\frac{p(\vy|\vtheta)p(\vtheta)}{q(\vtheta;\vxi)} d\vtheta
$$

\noindent and maximizing $\log \underline{p}(\vy;\vxi)$ with respect to $\vxi$ is eqivalent to minimizing
$\text{KL}(q(\vtheta;\vxi)||p(\vtheta|\vy))$. $\log \underline{p}(\vy;\vxi)$ is referred to as the
variational lower bound.

\noindent When the optimal distributions for each $q_i^*(\theta_i)$ are calculated, they yield a set of
equations, sometimes called the consistency conditions, which need to be  satisfied simultaneously. These
yield a series of mean field updates for the parameters of each approximating distribution. By executing the
mean field update equations in turn for each parameter in the model, the variational lower bound for the model
$\underline{p}(\vtheta; \vy)$ is iteratively increased. It can be shown that by calculating $q_i^*(\theta_i)$ for 
a particular $i$ with the remaining $q_j^*(\theta_j)$, $j\ne i$ fixed, results in a monotonic increase in the
variational lower bound, and thus a monotonic decrease in the Kullback-Leibler divergence between
$p(\vtheta|\vy)$ and $q(\vtheta)$.

\noindent The variational lower bound is maximised iteratively. On each iteration, the value of each parameter
in the model is calculated as the expectation of the full likelihood relative to the other parameters in the
model, which is referred to as the mean field update:
$$q_i^*(\theta_i) \propto \exp{\{ \bE_{-q(\theta_i)} \log p(\vy, \vtheta) \}}$$

\noindent This is done for each parameter in the model in turn until the variational lower bound's increase is
negligible and convergence is achieved.

Note that a combination of (A) and (B) can be used which has recently been formalized by \cite{Rohde2015}.

\noindent Variational Bayes aims to approximate a true, possibly intractable probability distribution $p(x)$
by a simpler, more tractable distribution $q(x)$ of known form.

This approach works well for classes of models where all of the parameters are conjugate. For more
general classes of models, mean field updates are not analytically tractable and general gradient-based
optimisation methods must be used, as for the Gaussian Variational Approximation (see \citep{Ormerod2012}) used
in this paper. These methods are generally difficult to apply in practice, as the problems can involve the
optimisation of many parameters over high-dimensional, constrained spaces whose constraints cannot be simply
expressed.


% General Design Bayesian Generalized Linear Mixed Model, as in \citep{zhao06}. This allows us to incorporate
% within-subject correlation, and smoothing splines (as in \citep{Wand2008}) in our models.

% Idea: We can use an approximation of the from q(\beta, \u, \Sigma) q(\rho) \Product q(r_i)
% and use GVA on q(\beta, \u, \Sigma) and mean field updates on \rho and r_i

% \subsection{Semiparametric Mean Field Variational Bayes}


% \subsubsection{Definitions}


% Two strategies for selecting a suitable class of approximating distributions $q$ such that
% an optimal distribution $q^*(\vtheta)$ can be found are
% (A) specifying the parametric form of $q$; or 
% (B) choosing $q$ to be of the factored form $q(\vtheta) = \prod_{i=1}^M q(\theta_i)$.
% For the second alternative,
% it can be shown (see Ormerod \& Wand, 2010, for example) that the optimal form of the
% approximating distributions $q_i$ for each parameter are of the form
% \begin{equation}\label{eq:consistency}
% 	q_i^*(\theta_i) \propto \exp{\{ \bE_{-q(\theta_i)} \log p(\vy, \vtheta) \}},  \quad 1\le i\le M.
% \end{equation}

%This approach works well for classes of models where all of the parameters are conjugate. For more general
%classes of models, mean field updates are not analytically tractable and general gradient-based optimisation
%methods must be used, as for the Gaussian Variational Approximation (see \citep{ormerod09}) used in this paper.
%These methods are generally difficult to apply in practice, as the problems can involve the optimisation of
%many parameters over high-dimensional, constrained spaces whose constraints cannot be simply expressed.

\subsection{Gaussian Variational Approximation}

In cases where there is a strong dependence between partitions of $\vtheta$,  such as between the parameters
$\vmu$ and $\mSigma$ in a hierarchical Gaussian model, a factored approximation may not approximate the true
distribution accurately. In this case, an alternate form of approximation may be used with the parameters
considered together to take their dependence into account. One such form of approximation is the Gaussian
Variational Approximation \cite{Ormerod2012}, which assumes that the distribution of the parameters being
approximated is multivariate Gaussian. The covariance matrix of the Gaussian allows the approximation to
capture the dependence amongst the elements of $\vtheta$, which increases the accuracy of the variational
approximation relative to the factored approximation.

\subsection{Laplace's Method of approximation}
\label{sec:laplace_approximation}

Laplace's method of approximation is used to approximate integrals of a unimodal function with negative second
derivative at the mode, indicating that the function is decreasing rapidly away from this point. The essential
idea is that if the function is decreasing rapidly away from the mode, the bulk of the area under the function
will be within a neighbourhood of the mode. Thus the integral of the function can be well approximated by an
integral over the neighbourhood of the mode. How large that neighbourhood needs to be is estimated using
how fast the function is changing at the mode, which is estimated by $|f''(x_0)|$.

Consider an exponential integral of the form
\[
	\int_a^b e^{M f(x)} dx
\]

\noindent where $f(x)$ is twice differentiable and $f''(x_0) < 0$, $M \in \R$ and $a, b \in \R \cup \{-\infty,
\infty\}$. Let $f(x)$ have a unique global maximum at $x_0$. Then, Taylor expanding about $x_0$, we have

\[
	f(x) = f(x_0) + f'(x_0) (x - x_0) + \frac{1}{2} f''(x_0) (x - x_0)^2 + R
\]

\noindent where $R = \BigO((x - x_0)^3)$. As $f$ has a global maximum at $x_0$, the first derivative of $f$ is
zero at $x_0$. Thus, the function $f(x)$ may be approximated by

\[
	f(x) \approx f(x_0) - \frac{1}{2} |f''(x_0)| (x - x_0)^2
\]

\noindent for $x$ sufficiently close to $x_0$, as the second derivative is negative at $x_0$. This ensures the
the approximation of the integral

\[
	\int_a^b e^{M f(x)} dx \approx e^{M f(x_0)} \int_a^b e^{-M |f''(x_0)|(x - x_0)^2} dx
\]

\noindent is accurate. This is a Gaussian integral, and thus we find that

\[
	\int_a^b e^{M f(x)} dx \approx \sqrt{\frac{2 \pi}{M |f''(x_0)|}} e^{M f(x_0)}
\]

 % See the Relative error section of the Wikipedia page, for instance.
\noindent Thus we have approximated our integral by a closed form expression. The error in the approximation
is $O(1/M)$. The approximation can be made more accurate by using a Taylor expansion beyond second order.

\subsubsection{Extending to multiple dimensions}
This approach to approximating integrals extends naturally to multiple dimensions.
Consider the second order Taylor expansion of $\log f(\vtheta)$ around the mode $\vtheta_m$
$$
\log f(\vtheta) \approx f(\vtheta_m) + (\vtheta - \vtheta_m) \nabla \log f(\vtheta) + \frac{1}{2} (\vtheta - \vtheta_m)^\top \mH_{\log f}(\vtheta_m) (\vtheta - \vtheta_m) + \sO(\|\vtheta - \vtheta_m\|^3).
$$

\noindent Assuming that $\vtheta$ is a stationary point of $\log f$, then $\nabla f(\vtheta) = \vzero$ and so
$$
\log f(\vtheta) \approx f(\vtheta_m) + \frac{1}{2} (\vtheta - \vtheta_m)^\top \mH_{\log f}(\vtheta_m) (\vtheta - \vtheta_m) + \sO(\|\vtheta - \vtheta_m\|^3)
$$

\noindent at such a point. The quadratic form in $\vtheta$ in the approximate expression for the log
likelihood above leads to a Gaussian approximation for the likelihood $\N(\vtheta_m, \mH_{\log
f}(\vtheta_m)^{-1})$. The approximation is crude but can be quite accurate if the likelihood is symmetric and
unimodal.

\subsection{Other methods: Expectation Propagation}
Expectation Propagation is an approximate Bayesian inference method, first proposed in \citep{Minka2001}.
It relies on minimising the reverse KL divergence $\text{KL}(p || q)$ between the true and approximaing
disitrbutions $p$ and $q$. A factorised form of the distribution
\[
	q(\vtheta) = \Pi_{i=1}^n q(\vtheta_i)
\]
is assumed. In general, fully minimising the KL divergence between $p$ and $q$ is intractable, so Expectation
Propagation approximates this by minimising the KL divergence of each of the factors individually.
It does this by cycling through each of the factors matching the sufficient statistics of each, incorporating
the information already in the other factors
\[
	\bE_{-q_i} [\text{KL}(p || q)]
\]
The factors are cycled through several times until convergence is achieved.

While promising, Expectation Propagation is a relatively new technique.
Unlike with Variatioal Bayes, there is no guarantee of convergence, and there is still much work to be done
before it is as mature as other approximation methods like Variational Bayes and Laplace approximation.

\section{Our Contributions}
% \begin{itemize}
% 	\item Gaussian Variational Approximation to Zero-Inflated Models. Parameterisation of the covariance structure.

In this section, we briefly outline the major contributions in this thesis.

In our second chapter, we present a Gaussian Variational Approximation to a Zero-Inflated Poisson mixed model
which can flexibly incorporate both fixed and random effects. This allows it to fit complicated models
to the data incorporating random intercepts and slopes and additive models using O'Sullivan-penalised splines.
The model is fit by optimising the conditional likelihood of the Gaussian component of the model given the
parameters governing zero-inflation and the covariance matrix $\mSigma$.
We present a new parameterisation for the covariance matrix of the Gaussian based on the Cholesky
factorisation of the precision matrix, and detail computation and numerical advantages of this
factorisation, owing to its sparsity when the form of the covariance matrix of the Gaussian is known due to
knowledge of the random effects in the model.

% \item Exact inference for some regression parameters for regression model with Maruyama and George prior on $g$.

A popular choice of Bayesian model selection is to use regression models with $g$-priors. For the Maruyama
and George Beta Prime prior \citep{Maruyama2011} we were able to derive closed form expressions for the
posterior distributions of most of the parameters of the model in terms of the hypergeometric function.

% \item Exact moments for $\vbeta$ for the regression model with the Maruyama and George prior.

% We were unable to derive a closed expression for the non-intercept regression parameters $\vbeta$, so instead
% we focused on the first and second moments. We were able to obtain closed form expressions for these,
% again in terms of the hypergeometric function. We were also able to obtain an approximation for the
% regression co-efficients which is multivariate normal using the Laplace approximation method, showing
% that the first and second moments of $\vbeta$ characterise the posterior distribution well.

% \item Posterior distribution of $g$ for different g-priors - Liang's hyper-$g$ prior, Liang's hyper-$g/n$ prior,
% Bayarri's robust $g$ prior and the Maruyama and George Beta-Prime prior.

An important consideration in model selection is being able to compare models against one another.
Calculation of the Bayes Factors for comparing models requires being able to compute the posterior
distribution of $g$. In our third chapter, we derive closed form expressions in terms of special functions for the posterior
distributions of $g$ for a number of choices of $g$ prior from the literature:
Liang's hyper-$g$ prior,
Liang's hyper-$g/n$ prior \citep{Liang2008},
Bayarri's robust $g$ prior \citep{Bayarri2012} and the 
Maruyama and George Beta-Prime \citep{Maruyama2011} prior.	

% \item CVA

For moderate to large $p$, the entire model space can be too large to perform model selection by
exhaustively searching the model space. Instead, approximate methods can be used to search the parts of the
model space for which the posterior model likelihood is the highest. In our fourth chapter, propose a population-based algorithm,
which works by adding or removing a covariate at a time to each of the fitted models in the population.
We implement this algorithm for a number of model selection priors from the literature:
the Liang's hyper-$g$ prior,
the Liang's hyper-$g/n$ prior \citep{Liang2008},
Bayarri's robust $g$ prior \citep{Bayarri2012} and the 
Maruyama and George Beta-Prime \citep{Maruyama2011} prior.	

We are able to implement this algorithm efficiently by using rank-one updates and downdates and the closed
forms of the posteriors for the model selection priors that we consider. The population-based approach allows
us to estimate the uncertainty in the model selection process.

% \end{itemize}
