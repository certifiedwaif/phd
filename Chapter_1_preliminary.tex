%! TEX root = thesis.tex


% Abstract
\chapter*{Abstract}

Bayesian models offer great flexibility, but can be computationally demanding
to fit. The gold standard for fitting Bayesian models, when posterior
distributions are not available analytically, are Monte Carlo Markov Chain
methods. However, these can be slow and prone to convergence problems.
Approximate methods of fitting Bayesian models allow these models to be fit
using deterministic algorithms in substantially less time.  Variational Bayes
(VB) is a method for approximating the posterior distributions of the model
parameters sometimes with only a slight loss of accuracy.  In this thesis, we
consider two important problems --  variable selection for linear models, and
zero inflated mixed models. 

The first problem we address is variable selection, a task of central
importance in modern statistics. Here, Bayesian model selection has the
advantage of incorporating the uncertainty of the model selection process
itself which propagates to the estimates of the model  parameters. Linear
regression models with Gaussian priors are ubiquitous in applied statistics due
to their ease of fitting and interpretation. We use the popular $g$-prior
\cite{Zellner1986} for model selection of linear models with normal priors
where $g$ is a prior hyperparameter. This raises the question of how best to
choose $g$. \cite{Liang2008} show that a fixed choice of $g$ leads to problems,
such as Bartlett's Paradox and the Information Paradox. These paradoxes, and
other problems, can be avoided by putting a prior on $g$. Using several popular
priors on $g$, we derive exact expressions for the model selection Bayes
Factors in terms of special functions depending only on  the sample size,
number of covariates and correlation of the model being considered. We show
that these expressions are accurate, fast to evaluate, and numerically stable.
An R package \texttt{blma} for doing Bayesian linear model averaging using
these exact expressions has been  released on GitHub.

For data sets with a small number of covariates, it is computationally feasible
to perform exact model averaging. As the number of covariates increases the
model space becomes too large to explore exhaustively.  Recently,
\cite{Rockova2017} introduced Particle EM (PEM), a population-based method for
efficiently exploring a subset of the model space with high posterior
probability. The population-based method allows the method to seek multiple
local modes, and captures greater total posterior mass from the model space
than choosing a single model would. We extend this method using Particle
Variational Approximation and the exact posterior marginal likelihood
expressions to derive a computationally efficient algorithm for model selection
on data sets with a large number of covariates. We demonstrate the algorithm's
performance on a number of data sets for different combinations of $g$-prior,
model selection prior and population size. We also compare our method to the
existing methods such as lasso, SCAD, and MCP penalized regression methods, and
PEM in terms of model selection performance,  and show that our method
outperforms these. We also show that total posterior mass increases and mean
marginal variable error decreases, as the number of models in the population
increases.
% Draw attention to speed 8s on 20 cores
%For $n = 600$, $p = 7200$ sized problem. 
Our algorithm performs very well relative to previous algorithms in the
literature, completing in 8 seconds on a model selection problem with a sample
size of 600 and 7200 covariates.

The second problem we address is zero-inflated models have many applications in
areas such as manufacturing and public health, but pose numerical issues when
fitting them to data. We apply a variational approximation to zero-inflated
Poisson mixed models with Gaussian distributed random effects using a
combination of VB and the Gaussian Variational Approximation (GVA). We
demonstrate that this approximation is accurate and fast on a number of
simulated and benchmark data sets. We also incorporate a novel parameterisation
of the covariance of the GVA using the Cholesky factor of the precision matrix,
similar to \cite{Tan2018}, and discuss the computational advantages of this
parameterisation due to the sparsity of the precision matrix for mixed models
and resolve associated numerical difficulties.


\chapter{Introduction}

% Outline that John suggested
\section{Motivation}

The advent of digital computers and the internet have lead to an explosion in
the volume of data being collected. With technological progress marching on,
this trend seems only set to continue and accelerate. In the future, as
technology continues to advance more data will be able to be stored and
processed, and so this trend of increasing volumes of data is set to continue
\citep{Gandomi2015}. But this data is only of value if it can be analysed and
understood.

This incredible increase in the volume of data has introduced new computational
difficulties in processing and modelling such large amounts of data, so-called
\emph{Big Data}, which is so large that it is difficult to process on one
computer. This data raises new challenges which modern statisticians must be
ready to meet. Approaches to modelling data are needed which can handle large
volumes of data in a computationally efficient manner while retaining the
probabilistic underpinning of classical statistics and statistical machine
learning, providing a rigorous underlying theory for inference. This realisation
has created an explosion of interest in \emph{Data Science}, incorporating ideas from
both statistics and computer science in recent years. Machine learning problems
are being tackled with algorithms which use probability models for the data --
motivating the development of the new field of statistical learning which
combines many of the best elements of statistics and machine learning
\citep{James:2014:ISL:2517747, MacKay:2002:ITI:971143,
hastie01statisticallearning, Murphy:2012:MLP:2380985}.

\section{Choosing an inferential paradigm}

How one proceeds given the above needs can be addressed through an inferential
paradigm. The most common of these are  the frequentist and Bayesian
statistical paradigms. The difference between frequentist and Bayesian
approaches begins with a difference in philosophy. Frequentists define an
event's probability as its' relative frequency after a large number of trials.
While Bayesians view probability as our reasonable expectation about an event,
representing our state of knowledge about the event.

There are many practical reasons to choose Bayesian approaches to modelling
data. It is flexible in modelling statistical complications, such as missing or
hierarchical data, and complicated models can be built by chaining together
multiple levels of simple models. These models can then be fit to data by
calculating the posterior probability of the parameters using Bayes' Theorem,
\begin{equation}\label{eq:bayes_theorem}
	p(\vtheta | \vy) = \frac{p(\vy | \vtheta) p(\vtheta)}{\int p(\vy | \vtheta) p(\vtheta) d \vtheta}
\end{equation}

\noindent where $\vy$ is a vector of observed data, $\vtheta$ are the model
parameters, $p(\vy|\vtheta)$ is the likelihood function, $p(\vtheta)$ is a
prior distribution on $\vtheta$, and $p(\vy)=\int
p(\vy|\vtheta)p(\vtheta)d\vtheta$.  Here the integral is performed over the
range of $\vtheta$. If a subset of $\vtheta$ are discrete random variables
then the integral over these parameters is replaced with a combinatorial sum
over all possible values of these discrete random variables.

There are many models which are difficult to fit under the frequentist
paradigm, as the likelihood can be difficult to maximise for complex models.
Furthermore, as the Bayesian paradigm treats each of the parameters in a model
as uncertain, the full uncertainty associated with all of the parameters can
be estimated via the uncertainty in the posterior distribution. This approach
avoids many of the pitfalls of statistical inference encountered with the
frequentist approach using significance testing and p-values \citep{Cox2005}.

The ability to build a model one component at a time and have the uncertainty
propagate through the model makes Bayesian modelling  particularly appropriate
for mixed effects and hierarchical models. In particular, uncertainty
regarding model selection is taken into account in the context of model
selection. Thus for the two classes of problems we consider in this thesis the
Bayesian approach is more suitable.

\subsection{Bayes Factors}

In the Bayesian inferential paradigm, two competing hypotheses can be compared
using Bayes Factors. The Bayes Factor is the ratio of the marginal likelihoods
under the assumption of each of the models being compared

\begin{equation*}
	\text{BF}(M_1, M_2)
	= \frac{P(\vy | M_1)}{P(\vy | M_2)}
	= \frac{\int P(\vtheta_1|M_1) P(\vy | \vtheta_1, M_1) d \vtheta_1}{\int P(\vtheta_2|M_2) P(\vy | \vtheta_2, M_2) d \vtheta_2}
	= \frac{P(M_1|\vy) P(M_2)}{P(M_2|\vy) P(M_1)}
\end{equation*}

\noindent where $\vy$ is the data observed, $M_1$ and $M_2$ are the models
being compared and $\vtheta_1$ and $\vtheta_2$ are the model parameter vectors
associated with each model respectively.

\section{Research problems}

In this section, we introduce the major problems that will be addressed in this
thesis. The themes of flexible modelling of data using Generalised Linear Mixed
Models and model selection of linear models with normal priors  will be
explored.

\subsection{Exponential family and the canonical form of linear regression
models}

The concept of the exponential family of probability distributions was first
introduced by \cite{Koopman1935} and \cite{pitman_1936}. The canonical form of a
regression model from the exponential family is
\begin{equation}\label{eq:exponential_family}
	p(\vy | \vtheta) = h(\vy) \exp \{ \vtheta^\top T(\vy) - b(\vtheta) \}
\end{equation}
for a parameter vector $\vtheta \in \mathbf{\Theta}$, and observed data $\vy$.
The sufficient statistic $T$ and $h$ are functions of the observed data, while
the cumulant function $b(\vtheta)$ is a function of the parameter $\vtheta$. The
cumulant function is the logarithm of the normalisation constant.

Many commonly used probability distributions of practical interest, such as the
Gaussian, Bernoulli, Poisson, Exponential and Gamma probability distributions,
can be expressed as an exponential family by making an appropriate choice of
$h$, $T$ and $b$ functions. The exponential family of distributions have several
appealing statistical and computational properties which derive from the
convexity of the parameter space $\Theta$ for which the exponential family
distribution is defined, and the convexity of the cumulant function
\citep{Jordan2010}. The mean of an exponential family distribution can be
obtained by calculating the first derivative of the cumulant function and then
evaluating at zero, while the variance can be obtained by calculating the second
derivatives of the cumulant function and evaluating at zero.

The exponential family of distributions allow us to extend linear models to more
general situations where the response variable is not normally distributed but
may be categorical, discrete or continuous and the relationship between the
response and the explanatory variables need not be of simple linear form.  By
choosing the parameterisation $\vtheta = \mX \vbeta$ where $\mX$ is the matrix
of observed covariates in $\R^{n \times p}$ and $\vbeta$ are regression
parameters in $\R^p$, for $n$ the sample size and $p$ the number of covariates,
a canonical form of generalised linear regression models may be written as
\begin{equation}\label{eq:glm}
	\log p(\vy | \vtheta) = \vy^\top \mX \vbeta - \vone^\top b(\mX \vbeta) + \vone^\top c(\vy)
\end{equation}
where $c(\vtheta)$ is the log of $h(\vy)$ from 
(\ref{eq:exponential_family}). A choice of $b(x) = e^x$ corresponding to the
Poisson family of distributions specifies a Poisson linear model appropriate for
modelling count data, while a choice of $b(x) = \log(1 + e^x)$ corresponding to
the logistic family of distributions specifies a logistic linear model
appropriate to modelling binary data.

\subsection{Generalised Linear Mixed Models}

Generalised Linear Mixed Models, an extension of Generalised Linear Models to
include both fixed and random effects, are applicable to many complicated
modelling situations.

Linear and generalised linear regression models are the standard tools used by
applied statisticians to explain the relationship between an outcome variable
and one or more explanatory variables. They provide a general method  to analyse
quantified relationships between variables within a data set in an easily
interpretable way. A standard assumption is that the outcomes are independent,
and that the effect of the explanatory variables on the outcome is fixed. But if
the outcomes are dependent and this assumption is not met, then linear and
generalised linear models can be extended to linear mixed models. These allow us
to incorporate dependencies amongst the  observations via the assumption of a
more complicated covariance structure, including random effects for  different
subgroups or longitudinal data and other extensions such as splines. This
additional flexibility makes their application popular in many fields, such as
public health, psychology and agriculture \citep{Kleinman2004, Lo2015, Kachman2000}.

In the frequentist paradigm, model parameters are fixed and uncertainty enters
the model through random errors, which have an associated covariance. The data
is modelled as a combination of these fixed parameters and random errors. In
the Bayesian paradigm, the uncertainty in the parameters and the data is
accounted for by the likelihood function.

\subsubsection{A Canonical Form for Generalised Linear Mixed Models}

The generalised form for linear models in  (\ref{eq:glm}) can easily be
extended to include random effects.  Following the conventions for Generalised
Design of \cite{Zhao2006}, we adopt the canonical form for Generalised Linear
Mixed Models exponential family with Gaussian random effects take the general
form
$$
\begin{array}{rl}\label{eq:glmm}
	p(\vy | \vbeta, \vu) &= \exp{\{ \vy^\top (\mX \vbeta + \mZ \vu) - \vone^\top b(\mX \vbeta + \mZ \vu) + \vone^\top c(\vy) \}}, \\
	\vu | \mG &\sim \N(\vzero, \mG),
\end{array}
$$

\noindent where the fixed effects are denoted by the vector $\vbeta$, the
random effects are denoted by $\vu$ and $\mG$ is the covariance matrix of
random effects. The covariance structure in $\mG$ is usually chosen to capture
the dependencies of interest between the observations in the application, such
as the dependency between repeated observations on an individual within a
longitudinal study, the dependency between observations within a cluster in a
hierarchical model or the spatial dependency between observations that are
close to one another in a spatial model.
%
The design matrix for the fixed effects is denoted by $\mX$ and the design
matrix for the random effects are denoted by $\mZ$.

Random effects are very flexible in the variety of models they allow us to fit
to our data. Through specification of the covariance structures in the matrix
$\mG$ with the appropriate data in the design matrix $\mZ$, complicated
dependencies amongst the responses $\vy$ can be specified, allowing modelling of
longitudinal data, fitting smoothing splines to the data and modelling spatial
relationships between responses. This allows us to fit hierarchical models with
random intercepts and slopes, capturing levels of variation within groups within
the data \citep{Gelman2007}.
% TODO: Not happy with how this paragraph is written. I can express this idea
% better.

% FIXME: Is this the best place for this?
While mixed models are very useful for gaining insight into a data set, fitting
them can be computationally challenging. For all but the simplest situations,
fitting these models involves computing high-dimensional integrals which are
often analytically and computationally intractable. The standard technique for
fitting Bayesian versions of these models is to use Monte Carlo Markov Chains
techniques. Thus, an approximation must be used in order to fit these models
within a reasonable time frame. Our approach to this problem is outlined in
Chapter 4.

\section{Splines and smoothing}

While linear models are statistically convenient to work with and easy to
interpret once fitted, the relationship between the response and explanatory
variables may not always be linear in practice. Thus a generalisation of linear
models to nonlinear situations is needed that still retains the beneficial
statistical and interpretive properties of linear models as much as possible.
The most general form of the univariate regression problem is
$ y_i = f(x_i) $
where $f: \R \to \R$ is unknown, and we wish to estimate it.  Fully
nonparametric regression is a difficult problem to solve, but the problem can
be simplified by prespecifying the points at which the function may change
curvature, which we refer to as \emph{knots}.

\newpage

% \subsubsection{Penalised spline}
% \subsubsection{B-splines}
\subsection{B-Splines}

% This is taken from the Wikipedia page on the subject. Yet somehow, I've managed to avoid including anything
% that's interesting or useful about B-Splines.

There are many families of basis functions which can be conveniently used for
function approximation, including orthogonal polynomials. The B-spline basis
\citep{DeBoor1972} is numerically stable and efficient to computationally
evaluate. A B-Spline is a piecewise polynomial function of degree $< n$ in a
variable $x$. It is defined over a domain $\kappa_0 \leq x \leq \kappa_m, m=n$.
The points where $x = \kappa_j$ are known as knots or break points. The number
of internal knots is equal to the degree of the polynomial if there are no knot
multiplicities. The knots must be in ascending order. The number of knots is
the minimum for the degree of the B-spline, which has a non-zero value in the
range between the first and last knot. Each piece of the function is a
polynomial of degree less than $n$ between and including adjacent knots. A
B-Spline is continuous at its' knots. When all internal knots are distinct its
derivatives are also continuous up to the derivative of degree $n - 1$. If
internal knots coincide at a given value of $x$, the continuity of derivative
order is reduced by 1 for each additional knot.

For any given set of knots, the B-spline for approximating a given function is
a unique linear combination of basis functions recursively defined as
$$
\begin{array}{rl}
	B_{i, 0}(x) & := \begin{cases}                                                                                                        
	1           & \text{if } \kappa_i \leq x < \kappa_{i+1};  \quad \text{and}                                                                                         \\
	0           & \text{otherwise,}                                                                                                        
	\end{cases}
	% B_{i, k}(x) & := \frac{x - \kappa_i}{\kappa_{i + 1} - \kappa_i} Q_{i, k-1} (x) + 
	% 									\frac{\kappa_{i + k + 1} - x}{\kappa_{i + k + 1} - \kappa_{i + 1}} Q_{i, k-1} (x). 
\end{array}
$$

\noindent for $i = 1, \ldots, K + 2M -1$ and
$$
\begin{array}{rl}
	B_{i, k}(x; \vkappa) & \ds = \frac{x - \kappa_i}{\kappa_{i + k} - \kappa_i} Q_{i, k-1} (x; \vkappa) + 
										\frac{\kappa_{i + k + 1} - x}{\kappa_{i + k + 1} - \kappa_{i + 1}} Q_{i, k-1} (x; \vkappa)
\end{array}
$$

\noindent for $i = 1, \ldots, K + 2 M - m$ with
% B-Splines
$$
Q_{m, i}(x; \kappa) =
\begin{cases}
B_{m, i}(x; \kappa),& \kappa_{i + m} > \kappa_i; \quad \mbox{and} \\
0, & \text{otherwise}.
\end{cases}
$$

We define the B-Spline basis in this way so that the definition remains correct
in the case where knots are repeated in $\vkappa$. We choose piecewise cubic
splines as cubics are numerically well behaved while still capturing the
curvature of functions we wish to approximate well
\citep{Press:2007:NRE:1403886}. Thus we select the knot sequence $\vkappa$ to be
$$
a = \kappa_1 = \kappa_2 = \kappa_3 = \kappa_4 < \kappa_5 < \ldots < \kappa_{K+5} = \kappa_{K+6} = \kappa_{K+7} = \kappa_{K+8} = b.
$$

There are many ways of choosing knots for applied statistical problems. A
typical approach is to choose the internal knots using the sample quantiles of
the data set being examined.
A common choice is to select 
$\min(n_U/4, 35)$ internal knots
where $n_U$ is the unique number of $x_i$'s.

\subsection{O'Sullivan Splines}
% $B_{ik} = B_k (x)$
% $B_x = [B_1(x), \ldots, B_K+4(x)]$
% Divided difference notation?
% Lagrange's interpolating polynomials?
% Semiparametric regression / Connection to mixed models

In this section, we follow the discussion of semiparametric regression in
\cite{ruppert_wand_carroll_2003}.  Using a mixed models setup to fit spline
models protects against overfitting, we construct a $\mZ$ matrix with the
appropriate B-Spline function evaluations in each of row of the matrix, where
each column in the matrix corresponds to one of the knots we have selected.

O'Sullivan introduced a class of penalised splines based on the B-spline basis
functions in \cite{OSullivan1986} which are a direct generalisation of
smoothing splines. Let $B_1$, \ldots, $B_{K+4}$ be the cubic B-spline basis
functions defined by the knots $\kappa_1$ to $\kappa_{K+4}$. O'Sullivan splines
are splines which are penalised using the penalty matrix $\mOmega$. Let
$\mOmega$ be the $(K+4) \times (K+4)$ matrix where the $(k, k')-th$ element is
\[ \mOmega_{k k'} = \int_a^b B''_k(x) B''_{k'}(x) dx. \] Then the O'Sullivan
spline estimate of the true function $f$ at the point $x$ is
\begin{equation*}
\widehat{f}_O(x; \lambda) = \mB_x \widehat{\vnu}_O,
\end{equation*}
where $\widehat{\vnu}_O = (\mB^\top \mB + \lambda \mOmega)^{-1} \mB^\top \vy$,
as shown in \cite{ruppert_wand_carroll_2003}.

The matrix $\mOmega$ is defined in this way to penalise oscillation, which is
measured by the second derivative. This penalty differs from the penalty for
``penalised B-Splines'' or P-splines in that the P-spline penalty matrix is
$\mD_2^\top \mD_2$ where $\mD_2$ is the second-order differencing matrix.

\section{Variable selection}

It is often the case in applied statistics that many covariates are available,
but it is unknown a priori which covariates explain the response variable of
interest. An automatic method of exploring which model among many possible
candidate models incorporating these covariates explains the response variable
best would relieve the burden of having to fit and compare the performance of
many such models manually.

The problem of selecting a statistical model from a set of candidate models
given a data set, hence referred to as \emph{model selection}, is one of the
most important problems encountered in practice by applied statisticians. It is
one of the central tasks of science, and there is a correspondingly large
literature on the subject -- \cite{Claeskens:1251912, NengjunYi2013,
Johnstone2009} together give a comprehensive overview.

The problem of model selection for normal linear models is particularly well
studied, owing to the popularity and importance of normal linear models in
applications. While new types of model are continually being developed, linear
models with normal priors remain a popular and essential modelling tool owing to
the ease of fitting these models, statistical inference on the parameters and,
most importantly, the ease which these models can be interpreted. But for a data
set with a moderate or large number of parameters, the question is immediately
raised of which covariates we should include in our model. One of the problems
that we address in this thesis is \emph{variable selection} on linear models
with normal priors.

The bias-variance trade-off is one of the central issues in statistical learning
\citep{Murphy:2012:MLP:2380985, Bishop:2006:PRM:1162264,
hastie01statisticallearning}. The guise this issue takes in model selection is
balancing the quality of the model fit against the complexity of the model, in
an attempt to find a compromise between over-fitting and under-fitting, in the
hope that the model fit will generalise well beyond the training data we have
observed to the general population and that we haven't simply learned the noise
in the training set.

There have been many approaches to model selection proposed, including criteria
based approaches, approaches based on functions of the
residual sum of squares, penalised regression such as the lasso and $L_1$
regression, and Bayesian modelling approaches. Model selection is a difficult
problem in high-dimensional spaces in general because as the dimension of the
space increases, the number of possible models increases combinatorially
\citep{Schelldorfer2010}. Many model selection algorithms use heuristics in an
attempt to search the model space more efficiently but still find an optimal or
near-optimal model within a reasonable period of time. A major motivation for
this field of research is the need for a computationally feasible approach to
performing model selection on large scale problems where the number of
covariates is large.


% Non-Bayesian
\subsection{Frequentist approaches to model selection}
\subsubsection{Information Criteria}
% Need to define gamma first before referring to it. Is this the right place for this?

Let $\vgamma$ be a $p$-dimensional vector of indicators, where a $1$ in the
$j$th position indicates that the $j$th covariate is included in the model,
while a $0$ indicates it is excluded. Thus $\vgamma$ defines a model with
covariates drawn from a $p$ column data matrix $\mX$.

In a frequentist context, there are many functions which can be used to judge
which model is best, such as Akaike's Information Criteria (AIC) and the Bayesian 
Information Criteria (BIC). These are functions $f\colon \vgamma \to \R^+$ which allow the
models under consideration to be ranked, and the best model chosen from those
available. Thus the optimal model selected by an information criteria is
$\vgamma^* = \min_\vgamma f(\vgamma)$. These functions typically attempt to
balance log-likelihood against the complexity of the model, achieving a
compromise between each.

% \mgc{AIC, BIC, DIC, Mallow's $C_p$}

Information Criteria are frequently used to compare models. Letting $\vgamma$
denote the candidate model, Information Criteria take the form ``negative twice
times the log-likelihood plus a term penalising for complexity of the mode''
$$
	\text{Information Criteria} = -2 \log p(\vy | \widehat{\vtheta}_\vgamma) + \text{complexity penalty},
$$

\noindent where $\widehat{\vtheta}_{\gamma}$ is the maximum likelihood estimate of
the model parameters $\vtheta$ for the model $\vgamma$ and $\log p(\vy |
\widehat{\vtheta_\gamma})$ is the log-likelihood of that model with that
parameter estimate and the complexity penalty is a function of the sample size
$n$ and the number of parameters $p$ of the model. Information criteria attempt
to successfully compromise between goodness of fit and model complexity.

The most popular of the Information Criteria is the 
AIC \citep{Akaike1974}. AIC calculates an estimate of the information lost
when a given model is used to represent the process that generates the data and
so is an estimator of the Kullback-Leibler divergence of the true model from the
fitted model. The AIC of the model $\vgamma$ is defined as
$$
	\text{AIC}(\vgamma) = -2 \log p(\vy | \widehat{\vtheta}_\vgamma) + 2 p_\vgamma,
$$

\noindent where $p_\vgamma$ is the number of parameters in the model $\vgamma$.
The model with the lowest AIC is selected as the `best'.

Of a similar form as the AIC, but derived via a more Bayesian framework is the
BIC\@. The BIC approximates the posterior
probability of the candidate model $\vgamma$. The BIC is defined as
$$
	\text{BIC}(\vgamma) = -2 \log p(\vy | \widehat{\vtheta}_\vgamma) + p_\vgamma \log(n).
$$

\noindent This is a more severe penalty for model complexity than in the
Akaike's Information Criteria when $n$ is greater than $8$. BIC can be shown to
be approximately equivalent to model selection using Bayes Factors in certain
contexts \citep{Kass1993}.

Alternatively, the process of model selection can be made implicit in the model fitting
process itself, ridge regression \citep{Casella1980}, of which the well-known
lasso is a special case \citep{Tibshirani1996}. As \cite{Breiman1996} and
\cite{Efron2013} showed, while  the standard formulation of a linear model is
unbiased, the goodness of fit of these models is numerically  unstable. Breiman
showed that by introducing a penalty on the size of the regression coefficients
such as  in ridge regression, this numerical instability can be avoided. This
reduces the variances of the coefficient estimates, at the expense of
introducing some bias -- which is another instance of the bias-variance
trade-off.

\subsubsection{Penalised regression}

Penalised regression methods trade introducing some bias in the estimator for
reducing the variance and thus fitting a more parsimonious model. The major
advantages are that a model with fewer covariates will be correspondingly
easier to interpret, and that the variance of the regression co-efficient
estimator will be less. In penalised regression, the regression coefficients
are subjected to a penalty or constraint. This is typically expressed as the
minimisation of the sum of a goodness of fit function such as squared Euclidean
distance and a penalty function $$ \widehat{\vbeta}_{\text{penalised}} =
\argmin_\vbeta \|\vy - \mX \vbeta\|_2^2 + \text{penalty}(\vbeta).  $$

From a Bayesian perspective, the penalty can be considered as a prior
distribution on the regression coefficients where smaller values of $\vbeta$ are
given more weight than larger ones. Here the penalised estimate of the
regression coefficients is the mode of their posterior distribution.

\subsubsection{Ridge regression}

Ridge regression is a penalised regression method, introduced in
\cite{Hoerl1970}. The penalty on the regression coefficients is the Euclidean
norm of the regression coefficients. This penalty shrinks the estimated
coefficients towards zero. The ridge regression coefficients can thus be
estimated by solving the constrained optimisation problem

$$
\widehat{\vbeta}_{\text{ridge}} = \argmin_\vbeta \|\vy - \mX \vbeta\|_2^2 \quad \text{ subject to } \quad \|\vbeta\|_2 \leq \lambda
$$

\noindent 
where $\lambda$ is a pre-specified free parameter specifying the amount of
regularisation. This constrained optimisation problem can be transformed by the
method of Lagrange multipliers into the sum of the residual sum of squares and
the product of the Lagrange multiplier and the constraint, which acts as a
penalty on the Euclidean norm of the regression coefficients.

% Both functions are quadratic in \vbeta

\subsubsection{Lasso regression}

Lasso regression is a penalised regression method developed in
\cite{Tibshirani1996}, which was directly inspired by ridge regression.  The
penalty is the $l_1$ norm of the coefficient vector.  The lasso regression
coefficients can be estimated by solving the constrained optimisation problem
$$
\widehat{\vbeta}_{\text{lasso}} = \argmin_\vbeta \|\vy - \mX \vbeta\|_2^2 \text{ subject to } \|\vbeta\|_1 \leq \lambda,
$$

\noindent 
where $\lambda$ is a pre-specified free parameter specifying the amount of
regulation. Similarly to the constrained optimisation problem for ridge
regression, the constrained optimisation problem can be transformed by the
method of Lagrange multipliers into the sum of the residual sum of squares and
the product of the Lagrange multiplier and the constraint, which acts as a
penalty on the $l_1$ norm of the regression coefficients. It follows from
Minkowski's inequality that the function above is convex, and thus the
optimisation problem is convex, and can be solved using standard methods from
convex optimisation \citep{Boyd2010}.  The constraint on the $l_1$ norm has the
effect of shrinking the coefficients, and setting some of them to zero. This
forces the models fit by lasso regression to be sparse, providing model
selection as part of the model-fitting process.

A disadvantage of lasso regression is that the constraint on the regression
coefficients depends on the free tuning parameter which must be selected a
priori or through cross-validation. But a much greater issue is that the model
selection process intrinsic to lasso regression does not take into account the
uncertainty of the model selection process itself, particularly the selection of
$\lambda$, as Bayesian model selection methods do.

\subsection{Bayesian approaches to Model Selection}

Parallel to the frequentist approaches, model selection can be performed using
a Bayesian approach. This can be done, for example, by using Bayes Factors  to
compare the posterior likelihoods of the candidate models to see which is most
probable given the observed data\citep{Kass1993}. Rather than selecting one
candidate model, several models can be combined together using Bayesian model
averaging \citep{Hoeting1999, Raftery1997, Fernandez2001,
Papaspiliopoulos2016}.

\subsubsection{Variable selection}

A special case of model selection is variable selection, where the focus is on
selecting individual covariates, rather than entire models. Variable selection
approaches search over the variables in the model space for the best covariates
to include in the candidate model. Due to the large number of possible
combinations of covariates -- typically $2^p$ where $p$ is the number of
covariates, such searches are often stochastic. This approach can either be
Fully Bayesian or Empirically Bayesian \citep{Cui2008}.  This search can be
driven by posterior probabilities  \citep{Casella2006}, or by Gibbs sampling
approaches such in \cite{George1993}. These two approaches of model selection and
variable selection can be combined  \citep{Geweke1996}. Variable selection can
also be accomplished by selecting the median probability model, consisting of
those models whose posterior inclusion probability is at least $1/2$
\citep{Barbieri2004}.

A challenge to applying this method of model selection is that exact model
fitting may be computationally infeasible for models involving even moderate
numbers of observations and covariates, and popular alternatives for fitting
Bayesian models such as Monte Carlo Markov Chains (MCMC) are still extremely computationally intensive.

% \subsubsection{Linear regression with normal prior and $g$ hyperprior}

% Zellner's g-prior

% \[
% 	\vbeta_\vgamma | \sigma^2, \mathcal{M}_\vgamma \sim \N(\vzero, g \sigma^2 (\mX^\top \mX)^{-1})
% \]

% \noindent which scales the Fisher information $\sigma^2 (\mX^\top \mX)^{-1}$ by $g$, was first introduced in
% \cite{Zellner1986}. It is widely used for variable selection in linear models with normal priors owing to its'
% computational efficiency in evaluating marginal likelihoods and model selection and conceptually simple
% interpretation.

% This model specification performs shrinkage on the regression co-efficients. As stated in \cite{Hastie2015},
% it is best to `bet on sparsity': ``Use a procedure that does well in sparse  problems, since no procedure does
% well in dense problems.''. With different choices of hyper- prior on $g$, this shrinkage can be made to behave
% like Bayesian versions of the lasso, ridge regression or generalisations of these \citep{Hahn2015}.  Thus this
% model specification has the advantage of performing well on model selection problems where the true model is
% sparse in the sense that the number of true non-zero covariates $p$ is less than the number of samples $n$.

% \begin{itemize}
% % \item Sparsity
% % \item Ease of interpretation from fewer regression co-efficients
% % \item Computational convenience
% \item Convex optimisation problem
% % \item Does not take model selection uncertainty into account
% \item Bias-Variance trade-off -- trade some bias in the estimator for a decrease in variance
% \item Deal with collinearity
% \end{itemize}


\section{Approximate Bayesian inference}

When the prior and model chosen for a Bayesian model is conjugate, the
posterior distribution is available in closed form and can be easily
calculated. When the prior is non-conjugate, the integral in Equation
~\ref{eq:bayes_theorem} to calculate the posterior distribution  is typically
intractable and so numerical methods must be used to calculate it
approximately.  The gold standard for Bayesian inference is to use MCMC methods
such as Metropolis-Hastings or Gibbs sampling.  But these methods are
computationally intensive, to the point where they are simply impractical in
Big Data situations where $n$ or $p$ are large. Moreover, they can be prone to
convergence problems.  Thus there is a need for approximate Bayesian inference
methods which are less computationally intensive while being almost as
accurate.

\subsection{Variational Bayes}
\label{sec:vb}

We now introduce Variational Bayes (VB), the popular approximate inference method for
Bayesian models. It is used to accelerate Bayesian model fitting by tens or
hundreds of times, with sometimes only minor loss in accuracy for some models.
This method plays a central role in this thesis, particularly in the third and
fourth chapters.

% John says this sounds repetitive.
% As described previously, Bayesian models offer many advantages in flexibility and ease of interpretation. But
% such models may be computationally difficult or intractable to fit.  The calculation of the true posterior
% distribution is often either computationally intractable or no closed form exists for the posterior
% distribution and so an approximation is required.

As described previously, Bayesian models may be computationally difficult or
intractable to fit. The calculation of the true posterior distribution for the
model is often either computationally intractable or no closed form exists for
the posterior distribution. We may be able to gain much of the same insight from
a given data set by fitting an accurate approximation  of the model, allowing us
to summarise the data and perform statistical inference. Variational approximation aims
to approximate a true, possibly intractable probability distribution $p(x)$ by a
simpler, more tractable distribution $q(x)$ of known form.

Variational approximation often takes the form minimising the Kullback-Leibler
divergence between the true posterior $p(\vtheta|\vy)$ and an approximating
distribution $q(\vtheta)$, sometimes called a $q$-density. For an introduction,
see \cite{Ormerod2010}.

% John says this is repetitive
% The density function of a random vector $\vu$ is denoted by $p(\vu)$.  The conditional density function of a
% random vector $\vu$ given $\vv$ is denoted by $p(\vu|\vv)$. Consider a generic Bayesian model with parameter
% vector $\vtheta \in \Theta$. Throughout this section we assume that $\vy$ and $\vtheta$ are continuous random
% vectors.

The KL divergence between the probability distributions $p$ and $q$ is defined
as
$$
	\KL(q || p) \equiv \int q(\vtheta) \log \left [ \frac{q(\vtheta)}{p(\vtheta | \vy)} \right ] d \vtheta.
$$

Suppose that a class of candidate approximating distributions $q(\vtheta)$ is
parameterised by a vector variational parameters $\vxi$ and write
$q(\vtheta)\equiv q(\vtheta;\vxi)$. We attempt to find an  optimal approximating
distribution $q^*(\vtheta)$ such that
$$
	\ds q^*(\vtheta) = \argmin_{\vxi \in \Xi} \,  \text{KL} \{ {q(\vtheta;\vxi) || p(\vtheta|\vy)} \}.
$$

\noindent If $\vtheta$ is partitioned into $M$ partitions $\vtheta_1$,
$\vtheta_2$, \ldots, $\vtheta_M$ then a simple form of approximation to adopt is
the factored approximation of the form
$$
	q(\vtheta) = \prod_{i=1}^M q(\vtheta_i)
$$

\noindent where each of the density $q(\vtheta_i)$ is a member of a parametric
family of density functions.  This form of approximation is computationally
convenient, but assumes that the partitions of $\vtheta$ are completely
independent of one another.

The optimal mean field update for each of the parameters $\vtheta_i$ can be
shown to be
$$
	q^*(\vtheta_i) \propto \exp\left[ \E_q \left\{ \log p(\vy; \vtheta)\right\} \right].
$$

\noindent For details of the proof, and a more thorough introduction to the
topic of variational approximations, see \cite{Ormerod2010}. It can easily be
shown that
$$
	\ds \log p(\vy) = \int q(\vtheta;\vxi) \log \left[ \frac{p(\vy|\vtheta)p(\vtheta)}{q(\vtheta;\vxi)} \right] d\vtheta + \text{KL}(q(\vtheta;\vxi)||p(\vtheta|\vy)).
$$

\noindent As the Kullback-Leibler divergence is strictly positive, the first
term on the right hand side is a lower bound on the marginal log-likelihood
which we will define by
$$
	\ds \log \underline{p}(\vy;\vxi) \equiv \int q(\vtheta;\vxi)  \log \left[ \frac{p(\vy|\vtheta)p(\vtheta)}{q(\vtheta;\vxi)} \right] d\vtheta
$$

\noindent and maximizing $\log \underline{p}(\vy;\vxi)$ with respect to $\vxi$
is equivalent to minimizing $\text{KL}(q(\vtheta;\vxi)||p(\vtheta|\vy))$.

\noindent  The term $\log
\underline{p}(\vy;\vxi)$ is referred to as the variational lower bound.

When the optimal distributions for each $q_i^*(\vtheta_i)$ are calculated, they
yield a set of equations, sometimes called the consistency conditions, which
need to be  satisfied simultaneously. These yield a series of mean field updates
for the parameters of each approximating distribution. By executing the mean
field update equations in turn for each parameter in the model, the variational
lower bound for the model $\underline{p}(\vtheta; \vy)$ is iteratively
increased. It can be shown that by calculating $q_i^*(\vtheta_i)$ for  a
particular $i$ with the remaining $q_j^*(\vtheta_j)$, $j\ne i$ fixed, results in
a monotonic increase in the variational lower bound, and thus a monotonic
decrease in the Kullback-Leibler divergence between $p(\vtheta|\vy)$ and
$q(\vtheta)$.

The variational lower bound is maximised iteratively. On each iteration, the
value of each parameter in the model is calculated as the expectation of the
full likelihood relative to the other parameters in the model, which is
referred to as the mean field update. This is done for each parameter in the
model in turn until the variational lower bound's increase is negligible and
convergence is achieved. Note that this approach can be extended to a wide
range of models such as semiparametric models as has been formalized by
\cite{Rohde2015}.

This approach works well for classes of models where all of the parameters are
conjugate. For more general classes of models, the mean field updates are not
analytically tractable and general gradient-based optimisation methods must be
used, such as for the Gaussian Variational Approximation \citep{Ormerod2012}.
These methods are generally difficult to apply in practice, as the problems can
involve the optimisation of many parameters over high-dimensional, constrained
spaces whose constraints cannot be simply expressed.

% TODO - merge this into the rest of the text
Recently, several stochastic Variational Bayes approaches to approximation
problems of this type have emerged.  \cite{Gershman2012} used a uniform
weighted mixture of isotropic Gaussians to approximate complex posterior
distributions. The variational lower bound is approximated with first and
second-order Taylor series expansions, and then optimised with L-BFGS.  In
\cite{Kingma2013}, the expectations in the expression for the variational lower
bound are approximated using Monte Carlo integration. The variational lower
bound is reparameterised in terms of an auxiliary noise variable such as a
standard normal, to reduce the variance of the Monte Carlo estimate.
\cite{Tan2018} takes an approach closest to the one we will adopt, using a
Gaussian Variational Approximation.  By parameterising the covariance matrix of
the Gaussian using Cholesky factors of the precision matrix, the covariance
matrix is guaranteed to be sparse due to the conditional independence between
fixed and random effects of the mixed model. The variational lower bound can be
rewritten so that it does not depend on the variational parameters.  By making
a transformation in terms of a noise variable to standardise the variational
parameters, efficient gradient estimators can be derived, then estimated using
subsampling of the data set. Sampling from the fixed normal distribution on
each iteration rather than a multivariate normal depending on the variational
parameters in the current iteration reduces the variance of the estimator.
Subsampling of the data set and sampling from the noise variable make the
fitting algorithm doubly stochastic.

Other approximate Bayesian inference techniques exist in the literature, such
as Laplace approximation \citep{Tierney1986},   integrated nested Laplace
approximation \citep{Rue2009}, and Expectation Propagation \citep{Minka2013}.
These have been applied to the problem of fitting count models
\citep{Barber2016, KimWand2017}.  But Expectation Propagation requires very
difficult algebra to complete the derivations required for the updates, and can
exhibit convergence problems. Laplace approximation relies on a Gaussian
approximation to the log of the posterior found by Taylor expanding around the
mode, which performs poorly when the true posterior is not symmetric, as is the
case for Poisson regression models.
% General Design Bayesian Generalized Linear Mixed Model, as in \citep{zhao06}.
% This allows us to incorporate within-subject correlation, and smoothing
% splines (as in \citep{Wand2008}) in our models.

% Idea: We can use an approximation of the from q(\beta, \u, \Sigma) q(\rho)
% \Product q(r_i) and use GVA on q(\beta, \u, \Sigma) and mean field updates on
% \rho and r_i

% \subsection{Semiparametric Mean Field Variational Bayes}


% \subsubsection{Definitions}

% Two strategies for selecting a suitable class of approximating distributions $q$ such that
% an optimal distribution $q^*(\vtheta)$ can be found are
% (A) specifying the parametric form of $q$; or 
% (B) choosing $q$ to be of the factored form $q(\vtheta) = \prod_{i=1}^M q(\theta_i)$.
% For the second alternative,
% it can be shown (see Ormerod \& Wand, 2010, for example) that the optimal form of the
% approximating distributions $q_i$ for each parameter are of the form
% \begin{equation}\label{eq:consistency}
% 	q_i^*(\theta_i) \propto \exp{\{ \bE_{-q(\theta_i)} \log p(\vy, \vtheta) \}},  \quad 1\le i\le M.
% \end{equation}

%This approach works well for classes of models where all of the parameters are conjugate. For more general
%classes of models, mean field updates are not analytically tractable and general gradient-based optimisation
%methods must be used, as for the Gaussian Variational Approximation (see \citep{ormerod09}) used in this paper.
%These methods are generally difficult to apply in practice, as the problems can involve the optimisation of
%many parameters over high-dimensional, constrained spaces whose constraints cannot be simply expressed.

\subsection{Gaussian Variational Approximation}

In cases where there is a strong dependence between partitions of $\vtheta$,
such as between the parameters $\vmu$ and $\mSigma$ in a hierarchical Gaussian
model, a factored approximation may not approximate the true distribution
accurately. In this case, an alternate form of approximation may be used with
the parameters considered together to take their dependence into account. One
such form of approximation is the Gaussian Variational Approximation
\citep{Ormerod2012}, which assumes that the distribution of the parameters being
approximated is multivariate Gaussian. The covariance matrix of the Gaussian
allows the approximation to capture the dependence amongst the elements of
$\vtheta$, which increases the accuracy of the variational approximation
relative to the factored approximation. This will be the approach used in
Chapter 4.

\subsection{Laplace Method of approximation}
\label{sec:laplace_approximation}

Laplace's method of approximation, as described in \cite{butler_2007} or
\cite{MacKay:2002:ITI:971143}, is used to approximate integrals of a unimodal
function $f$ with negative second derivative at the mode, indicating that the
function is decreasing rapidly away from this point. The essential idea is that
if the function is decreasing rapidly away from the mode, the bulk of the area
under the function will be within a neighbourhood of the mode. Thus, the integral
of the function can be well approximated by an integral over the neighbourhood
of the mode. How large that neighbourhood needs to be is estimated using how
fast the function is changing at the mode $x_m$, which is estimated by
$|f''(x_m)|$.

Consider an exponential integral of the form
$$
	\int_a^b e^{M f(x)} dx
$$

\noindent where $f(x)$ is twice differentiable and $f''(x_m) < 0$, $M \in \R$
and $a, b \in \R \cup \{-\infty, \infty\}$. Let $f(x)$ have a unique mode at
$x_m$. Then, Taylor expanding about $x_m$, we have
$$
	f(x) = f(x_m) + f'(x_m) (x - x_m) + \frac{1}{2} f''(x_m) (x - x_m)^2 + \BigO\left((x - x_m)^3\right).
$$

\noindent As $f$ has a global maximum at $x_m$, the first derivative of $f$ is
zero at $x_m$. Thus, the function $f(x)$ may be approximated by
$$
	f(x) \approx f(x_m) - \frac{1}{2} |f''(x_m)| (x - x_m)^2
$$

\noindent for $x$ sufficiently close to $x_m$, as the second derivative is
negative at $x_m$. This ensures the the approximation of the integral
$$
	\int_a^b e^{M f(x)} dx \approx e^{M f(x_m)} \int_a^b e^{-M |f''(x_m)|(x - x_m)^2} dx
$$

\noindent is accurate. The integral on the right-hand side of the equality is a
Gaussian integral, and thus we find that
$$
	\int_a^b e^{M f(x)} dx \approx \sqrt{\frac{2 \pi}{M |f''(x_m)|}} e^{M f(x_m)}.
$$

 % See the Relative error section of the Wikipedia page, for instance.
\noindent Thus, we have approximated our integral by a closed form expression.
The error in the approximation is $\BigO(1/M)$. The approximation can be made
more accurate by using a Taylor expansion beyond second order.

\subsubsection{Extending to multiple dimensions}

This approach to approximating integrals extends naturally to multiple
dimensions. Consider the second order Taylor expansion of $\log f(\vtheta): \R^p
\to \R$ around the mode $\vtheta_m \in \R^p$ given by
$$
\begin{array}{rl}
\ds \log f(\vtheta) 
& \ds \approx f(\vtheta_m) + (\vtheta - \vtheta_m)^\top \nabla \log f(\vtheta_m) + \tfrac{1}{2} (\vtheta - \vtheta_m)^\top \mH_{\log f}(\vtheta_m) (\vtheta - \vtheta_m)
\\
& \ds  \qquad 
+ \BigO(\|\vtheta - \vtheta_m\|^3).
\end{array} 
$$

\noindent 
where $\nabla \log f(\vtheta_m)$ is the gradient of the log-likelihood at
$\vtheta_m$ and $\mH_{\log f}(\vtheta_m)$ is the Hessian matrix of the
log-likelihood at $\vtheta_m$. Assuming that $\vtheta_m$ is a stationary point
of $\log f$, then $\nabla f(\vtheta_m) = \vzero$ and so
$$
\log f(\vtheta) \approx f(\vtheta_m) + \frac{1}{2} (\vtheta - \vtheta_m)^\top \mH_{\log f}(\vtheta_m) (\vtheta - \vtheta_m) + \BigO(\|\vtheta - \vtheta_m\|^3)
$$

\noindent at such a point. The quadratic form in $\vtheta$ in the approximate
expression for the log likelihood above leads to a Gaussian approximation for
the likelihood 
$$
\N(\vtheta_m, -\mH_{\log f}(\vtheta_m)^{-1}).
$$ 

\noindent 
The approximation
is crude but can be quite accurate if the likelihood is symmetric and unimodal, which is often the case when the sample size is large.

\subsection{Other methods: Expectation propagation}

Expectation Propagation is an approximate Bayesian inference method, first
proposed in \cite{Minka2001}.  It relies on minimising the reverse KL
divergence $\text{KL}(p || q)$ between the true and approximating distributions
$p$ and $q$. A factorised form of the distribution
\[
	q(\vtheta) = \prod_{i=1}^n q(\vtheta_i)
\]

\noindent 
is assumed. In general, fully minimising the KL divergence between $p$ and $q$
is intractable, so Expectation Propagation approximates this by minimising the
KL divergence of each of the factors individually.  It does this by cycling
through each of the factors matching the sufficient statistics of each,
incorporating the information already in the other factors.
%\[
%	\bE_{-q_i} [\text{KL}(p || q)]
%\]
The factors are cycled through several times until convergence is achieved.

While promising, unlike
with Variational Bayes, there is no guarantee of convergence, and there is still
much work to be done before it is as mature as other approximation methods like
Variational Bayes and Laplace approximation.

A linear model with normal priors allows exact inference on the regression and
model selection parameters in closed form, which might appear to negate the
benefits of a variational approximation to the model. However, the performance
of our variational approximation should remain similar if the priors are
altered to cater for complications such as robustness, while exact Bayesian
inference calculations are no longer possible in closed form in these
situations.

% TODO: How to introduce this? Is this really the best place to put it?
%\cite{Zellner1986} suggested a particular form of conjugate Normal-Gamma family
%where the Bayes factors have a relatively simple form, incorporating a parameter
%$g$ to control mixing between the model fit from the %data and a prior
%specification of model fit. This immediately raises the question of how $g$
%should be chosen, and whether it should be fixed or %have a prior specification.
%\cite{Liang2008} showed that fixed choices of $g$ lead to paradoxes such as
%Bartlett's Paradox and the Information Paradox, and so a prior specification
%should be preferred. There are many ways of choosing a prior on $g$. Using a
%mixture of $g$-priors has the advantage of adapting the degree of shrinkage to
%the prior model dependent on the data.

\section{Our contributions}
% \begin{itemize}
% 	\item Gaussian Variational Approximation to Zero-Inflated Models. Parameterisation of the covariance structure.

\noindent 
In this section, we briefly outline the major contributions in this thesis.

\begin{itemize}



% \item Exact inference for some regression parameters for regression model with
% Maruyama and George prior on $g$.

\item A popular choice of Bayesian model selection is to use regression models
with $g$-priors. For the Beta Prime prior \citep{Maruyama2011} we were able to
derive closed form expressions for the posterior distributions of most of the
parameters of the model in terms of the hypergeometric function.

% \item Exact moments for $\vbeta$ for the regression model with the Maruyama
% and George prior.

% We were unable to derive a closed expression for the non-intercept regression
% parameters $\vbeta$, so instead we focused on the first and second moments. We
% were able to obtain closed form expressions for these, again in terms of the
% hypergeometric function. We were also able to obtain an approximation for the
% regression co-efficients which is multivariate normal using the Laplace
% approximation method, showing that the first and second moments of $\vbeta$
% characterise the posterior distribution well.

% \item Posterior distribution of $g$ for different g-priors - Liang's hyper-$g$
% prior, Liang's hyper-$g/n$ prior, Bayarri's robust $g$ prior and the Maruyama
% and George Beta-Prime prior.

\item An important consideration in model selection is being able to compare
models against one another.  Calculation of the Bayes Factors for comparing
models requires being able to compute the posterior distribution of $g$. In our
second chapter, we derive closed form expressions in terms of special functions
for the posterior distributions of $g$ for a number of choices of $g$ prior from
the literature: Liang's hyper-$g$ prior, Liang's hyper-$g/n$ prior
\citep{Liang2008}, Bayarri's robust $g$ prior \citep{Bayarri2012} and the
Beta-Prime \citep{Maruyama2011} prior.

% \item CVA

\item Exact inference for model selection for linear models with normal priors
    is computationally feasible when the number of  covariates is small, with
    $p$ below 40. But exhaustively exploring the search space is not efficient,
    and often not computationally feasible for a larger number of covariates.
    To deal with this situation, in our fourth chapter, we adopt a population-
    based technique inspired by Ro\v{c}kov\'{a}'s work on population-based EM
    \citep{Rockova2017} to efficiently explore the posterior model space.
    Instead, approximate methods can be used to search the parts of the model
    space for which the posterior model likelihood is the highest. In our
    third chapter, we propose a population-based algorithm, which works by
    adding or removing a covariate at a time to each of the fitted models in
    the population. We implement this algorithm for a number of model selection
    priors from the literature: the Liang's hyper-$g$ prior, the Liang's
    hyper-$g/n$ prior \citep{Liang2008}, Bayarri's robust $g$ prior
    \citep{Bayarri2012} and the Maruyama and George Beta-Prime prior
    \citep{Maruyama2011}.


\item We are able to implement this algorithm efficiently by using rank-one
updates and downdates and the closed forms of the posteriors for the model
selection priors that we consider. The population-based approach allows us to
estimate the uncertainty in the model selection process.


\item Generalised Linear Mixed Models are an appealing way to model data, as
they are flexible enough to model a range of data types and situations. But the
Bayesian versions of these models typically require computationally demanding
MCMC, which can also be prone to convergence problems. Instead, we consider
approximate Bayesian inference techniques, which are computationally efficient
and deterministic.

\item It is desirable to use normal priors for the regression coefficients of
these models, as these are easily interpreted. But for Generalised Linear Mixed
Models with a non-normal response, these priors are non-conjugate, making VB
difficult to apply as the required mean field updates are intractable. We apply
Gaussian Variational Bayes -- an extension to Variational Bayes, to fit a
multivariate normal distribution to the regression coefficients of our models.

\item In our fourth chapter, we present a Gaussian Variational Approximation to
a zero-inflated Poisson mixed model which can flexibly incorporate both fixed
and random effects. This allows us to use our model fitting algorithm to fit
complicated models to the data incorporating random intercepts and slopes and
additive models using O'Sullivan-penalised splines.  The model is fit by
optimising the conditional likelihood of the Gaussian component of the model
given the parameters governing zero-inflation and the covariance matrix
$\mSigma$.

\item We present a new parameterisation for the covariance matrix of the
Gaussian based on the Cholesky factorisation of the precision matrix, and detail
computation and numerical advantages of this factorisation, owing to its
sparsity when the form of the covariance matrix of the Gaussian is known due to
knowledge of the random effects in the model.

\end{itemize}
