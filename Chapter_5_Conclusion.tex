\chapter{Conclusion}

\section{Zero-inflated models Discussion}
\label{sec:discussion}
		
We have described a Variational Bayes approximation to Zero-Inflated Poisson regression models which
allows such models to be fit with considerable generality. We have also devised and extensively tested a
number of alternative approaches for fitting such models, and extended one of these alternative
approaches with a new parameterisation. Using MCMC methods as the gold standard to test against, we have
assessed the accuracy and computational speed of these algoritms.
		
We applied our model fitting algorithms to a number of data sets to fit a range of models. The Cockroaches
model \ref{sec:cockroaches} had few fixed covariates, a random intercept for each apartment building and
incorporated zero-inflation. The Police stops model \ref{sec:police_stops} was a pure Poisson mixed model, no
zero-inflation and a random intercept for precincts/locality. The Biochemists model \ref{sec:biochemists} was
zero-inflated with fixed effects. The Owls model \ref{sec:owls} was zero-inflated, random intercepts for
nests. There were a large number of nests. We were able to estimate the variance component for this model very
accurately.

The use of Mean Field Variational Bayes allows estimation of Bayesian ZIP models in a fraction of the
time taken to fit the same model using even the best MCMC methods available, with only a small loss of
accuracy. This is of great utility in applications where speed matters, such as model selection or when
applied statisticians are choosing amongst many candidate models, as is typical in practice.
		
The new parameterisation of Gaussian Variational Approximation using the Cholesky factorisation of the
inverse of $\mLambda$ presented in Section \ref{sec:param} provides significant advantages.  It is well
known that the inverse of a sparse matrix need not be sparse.

Mixed models have covariance matrices with a block structure, due to the dependence structure of the
random effects. The precision parameterisation presented in this chapter is able to preserve this
sparsity within the structure of the Cholesky factors of the inverses of the covariance matrices use in
the variational lower bound by re-ordering the rows and columns of the matrices so that the random
effects blocks appear first. The Owls example presented in this chapter shows the computational
advantages of this approach when the number of groups $m$ in the model is large (m=27 in this case) --
as the covariance parameterisation takes 46 seconds to fit whereas the inverse parameterisation only
takes 3 seconds. This clearly demonstrates advantage of using sparsity to reduce the dimension of the
optimisation problem to be solved when models are being fit -- as only the non-zero values in the
covariance matrices need to be optimised over. This allows models to be fit more quickly, and with
greatly improved numerical stability and without loss of accuracy.

While all of the fitting algorithms presented in this chapter except the Laplace's approximation
algorithm were able to fit ZIP random and fixed effects models with high accuracy, and the 
Gaussian inverse parameterisation and fixed point algorithms were able to do so at high speed, they 
could be numerically unstable depending on the data the model was being fit to and their starting points.
In the case of the Gaussian inverse parameterisation algorithm, the source of the problem was tracked down
to the exponential used in the parameterisation of the diagonal of the Cholesky factor of the precision
matrix combined with the exponential that arises in the derivation of the Gaussian variational lower
bound for Poisson mixed models -- leading to frequent numeric overflows during the fitting process. This
problem, once discovered, was mitigated by replacing the exponential parameterisation of the diagonal
of the Cholesky factor with a piecewise function which is exponential beneath a threshold and quadratic
above that threshold. This was shown to greatly increase the numeric stability of the GVA inverse
parameterisation for a range of starting points.

Some of the algorithms which we experimented with were found to be very sensitive to their starting points.
While these algorithms are typically initialised with a starting point as close as possible to the final
solution, this gives some sense of the stability of each algorithm.
Stability fixes.

This article presents the essential ideas necessary for a performant implementation implementing model fitting
for ZIP regression models.%, but the performance would be even better if our algorithm was re-implemented in a
%compiled language with good numeric libraries such as C++ with Eigen.
The majority of the performance
improvements over existing approaches come from avoiding unneccessary matrix inversion, which is a
computationally expensive and numerically unstable process taking $\BigO(p^3)$ flops, and from constructing and 
calculating	with sparse matrices. The gains of these approaches, particularly from sparse matrix techniques, 
can be difficult to fully realise in R without expert knowledge of the underlying implementation and libraries.
		
Our application of these ideas to Andrew Gelman's data showed that the new parameterisation very effectively
speeds up fitting zero-inflated mixed models to real world data with a large number of groups, while still
maintaining excellent accuracy versus an MCMC approach. This demonstrates the applicability of the ideas
presented within this paper to real world data sets.
		
\section{CVA Conclusion}

We have proposed a deterministic Bayesian model selection algorithm which is computationally efficient
and simple. Like Particle EM \cite{Rockova2016}, our algorithm maintains a population of solutions and ensures
diversity of that population to explore the uncertainty of the selected model. This gives far more information
about the model selection process than simply choosing one best model. However, whereas Particle EM uses
a spike-and-slab prior for the regression co-efficients, our approach uses a g-prior, which avoids 
the Bartlett's Paradox and Information Paradox. Importantly, both approaches can be implemented using rank-one
updates and downdates and the model selection posterior probabilities are available in closed form.

While previously model selection algorithms using the Maruyama, Liang-g and robust Bayarri priors have
typically been implemented using Monte Carlo Markov Chains, our algorithm allows the advantages of these
priors while using a deterministic algorithm.

The CVA algorithm is implemented in the {\tt correlation} package in the {\tt cva} function.