%! TEX root = thesis.tex
\chapter{Conclusion and Future Directions}
\label{sec:chapter_2_conclusion}
		
% TODO: This section recounts what we've done. I'm not sure that it
% communicates what we could do in the future very well.

We have described a Variational Bayes approximation to Zero-Inflated Poisson
regression models which allows such models to be fit with considerable
generality. We have also devised and extensively tested a number of alternative
approaches for fitting such models, and extended one of these alternative
approaches with a new parameterisation. Using MCMC methods as the gold standard
to test against, we have assessed the accuracy and computational speed of these
algorithms.
		
We applied our model fitting algorithms to a number of data sets to fit a range
of models. The Cockroaches model in Section \ref{sec:cockroaches} had few fixed
covariates, a random intercept for each apartment building and incorporated
zero-inflation. The Police stops model in Section \ref{sec:police_stops} was a
pure Poisson mixed model, with no zero-inflation and a random intercept for
precincts/locality. The Biochemists model in Section \ref{sec:biochemists} was
zero-inflated with fixed effects. The Owls model in Section \ref{sec:owls} was
zero-inflated,  with a random intercepts for each nest. There were a large
number of nests $(m=27)$. We were able to estimate the variance component for
this model very accurately.

The use of Mean Field Variational Bayes allows estimation of Bayesian ZIP
models in a fraction of the time taken to fit the same model using even the
best MCMC methods available, with only a small loss of accuracy. This is of
great utility in applications where speed matters, such as when applied
statisticians are comparing and choosing amongst many candidate models, as is
typical in practice.
		
The new parameterisation of Gaussian Variational Approximation using the
Cholesky factorisation of the inverse of $\mLambda$ presented in Section
\ref{sec:param} provides significant advantages when used to estimate mixed
models.

Mixed models have covariance matrices with a block structure, due to the
dependence structure of the random effects. The precision parameterisation
presented in this chapter is able to preserve this sparsity within the
structure of the Cholesky factors of the inverses of the covariance matrices
use in the variational lower bound by re-ordering the rows and columns of the
matrices so that the random effects blocks appear first. The Owls example
presented in this chapter shows the computational advantages of this approach
when the number of groups $m$ in the model is large ($m=27$ in this case) -- as
the covariance parameterisation takes 46 seconds to fit whereas the inverse
parameterisation only takes 3 seconds. This clearly demonstrates advantage of
using sparsity to reduce the dimension of the optimisation problem to be solved
when models are being fit -- as only the non-zero values in the covariance
matrices need to be optimised over. This allows models to be fit more quickly,
and with greatly improved numerical stability and without loss of accuracy.

While all of the fitting algorithms presented in this chapter except the
Laplace's approximation algorithm were able to fit ZIP random and fixed effects
models with high accuracy, and the  Gaussian inverse parameterisation and fixed
point algorithms were able to do so at high speed, they  could be numerically
unstable depending on the data the model was being fit to and their starting
points. In the case of the Gaussian inverse parameterisation algorithm, the
source of the problem was tracked down to the exponential function used in the
parameterisation of the diagonal of the Cholesky factor of the precision matrix
combined with the exponential that arises in the derivation of the Gaussian
variational lower bound for Poisson mixed models -- leading to frequent numeric
overflows during the fitting process. This problem, once discovered, was
mitigated by replacing the exponential parameterisation of the diagonal of the
Cholesky factor with a piecewise function which is exponential beneath a
threshold and quadratic above that threshold. This was shown to greatly
increase the numeric stability of the GVA inverse parameterisation for a range
of starting points.

Some of the algorithms which we experimented with were found to be very
sensitive to their initial conditions.  While these algorithms are typically
initialised with a starting point as close as possible to the final solution,
this gives some sense of the stability of each algorithm. We were able to
develop a variant of the algorithm that employs a parameterisation which is
much more numerically stable, and demonstrate this numerical stability for a
range of models.

This thesis chapter presents the essential ideas necessary for a high performance
implementation for model fitting of ZIP regression models.
%, but the performance would be even better if our algorithm was re-implemented
%in a compiled language with good numeric libraries such as C++ with Eigen.
The majority of the performance improvements over existing approaches come from
avoiding unnecessary matrix inversion, which is a computationally expensive
and numerically unstable process taking $\BigO(p^3)$ flops, and  from
constructing and calculating with sparse matrices. The gains of these
approaches, particularly from sparse  matrix techniques, can be difficult to
fully realise in R without expert knowledge of the underlying implementation
and libraries.
		
Our application of these ideas to Andrew Gelman's data showed that the new
parameterisation very effectively speeds up fitting zero-inflated mixed models
to real world data with a large number of groups, while still maintaining
excellent accuracy versus an MCMC approach. This demonstrates the applicability
of the ideas presented within this chapter to real world data sets.
\section{Zero-inflated models via GVA future directions}
		
The first directions for future research stemming from this chapter would be
generalising the approximation to other zero-inflated models which handle
overdispersion in the data without the need for a random intercept, such as the
zero-inflated negative binomial model.

Furthermore, much more exploration could be done on alternative
parameterisations of the covariance matrix in the Gaussian Variational
Approximation. The specific parameterisation of the diagonal of the Cholesky
factor as a piecewise exponential/quadratic polynomial function was chosen
largely for convenience.

The current mean field update and Gaussian Variational Approximation algorithms
use the entire sample. For large samples in the Big Data era, this may not be
computationally feasible. Other authors such as \cite{Tan2018} have used doubly
stochastic algorithms which both sub-sample the data and use noise to
approximate the integral expression for the expectation of the variational
lower bound. The sub-sampling in particular is very appealing in a Big Data
context. We wish to experiment with this class of algorithm, and compare the
performance and accuracy of this kind of doubly stochastic algorithm with the
more traditional mean field and Gaussian variational algorithms presented in
Chapter 2.

\section{Calculating Bayes factors for $g$-priors future directions}

We have reviewed the prior structures that lead to closed form expressions for
Bayes factors for linear models. We have described ways that each of these
priors, except for the hyper-$g/n$ prior can be evaluated in a numerically stable
manner and have implemented a package \texttt{blma} for performing full exact
Bayesian model averaging using this methodology. Our package is competitive
with \texttt{BAS} and \texttt{BMS} in terms of computational speed, is
numerically more stable and accurate, and offers some different priors
structures not offered in \texttt{BAS}. Our package is much faster than
\texttt{BayesVarSelect} and is also numerically more stable and accurate.

We are currently working on several extensions to this work. Firstly, we are
working on a parallel implementation of the package which will allow for exact
Bayesian inference for problems roughly the size $p\approx 30$. While a
prototype of this work has been developed which runs on Linux using
\texttt{OpenMP}, and displayed good performance as the number of cores used
increased, we found it difficult to get the parallel code to work reliably on
Macintosh and Windows computers.  As the majority of R users use these two
platforms, we feel this is an important technical issue to resolve. Our
algorithm could also be ported to GPUs.

Secondly, we are currently implementing  Markov Chain Monte Carlo (MCMC) and
population based MCMC methods for exploring the model space when $p>30$.  We
can also see several obvious paths to extend this work to Generalised Linear
Models - either using the approach described in \cite{Li2015} or by using
Laplace approximations.  Lastly, we are deriving exact expressions for
parameter posterior distributions under some of the prior structures we have
considered here.

% TODO: I still feel that you're missing something here. Look at Statistical
% Learning with Sparsity. John calls what we've already done the `building
% block' for a whole host of other approaches. To write this section, you need
% to read more of that book, understand what John's comment meant and then
% write out what you've understood.

\section{PVA future directions}

There are several planned future extensions to this work. Firstly, we would
like to generalise the PVA approach to generalised linear models as well as
linear models, to be able to perform model selection for regression models
applicable to a wider range of types of data. The computational approach would
again either use the approach described in \cite{Li2015} or by using Laplace
approximations.  

Secondly, although the algorithm already runs in parallel on multicore CPUs
using \texttt{OpenMP}, we believe even greater gains in performance could be
achieved by porting the algorithm to run on GPUs, or by using distributed
computing such as \texttt{OpenMPI}.

Thirdly, and most excitingly, we could examine modifications to the PVA
algorithm itself. The way the algorithm currently ensures diversity amongst the
particles in the population is to reward increases in entropy, weighted by the
hyperparameter $\lambda$. The current version of the algorithm hard codes
$\lambda$ to $1$, but it would be interesting to alter $\lambda$ and as in the
Population EM algorithm \cite{Tan2018} and observe the effect on model
selection performance. The algorithm also currently maintains diversity in the
population by maintaining uniqueness of every particle within the population.
It would be interesting to relax this constraint and compare the effect on
model performance.
