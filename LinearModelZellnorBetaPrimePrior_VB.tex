\documentclass{article}

\usepackage{amsmath,amssymb,amsfonts,amsthm,latexsym,color,url}

\usepackage{graphicx}
\usepackage{graphics}
\usepackage{geometry}
\geometry{verbose,a4paper,tmargin=5.5mm,bmargin=5.5mm,lmargin=5.5mm,rmargin=5.5mm}


\newtheorem{Lemma}{Lemma}
 
% special symbols and abbreviations
% Sets or statistical values
\def\sI{{\mathcal I}}                            % Current Index set
\def\sJ{{\mathcal J}}                            % Select Index set
\def\sL{{\mathcal L}}                            % Likelihood
\def\sl{{\ell}}                                  % Log-likelihood
\def\sN{{\mathcal N}}                            
\def\sS{{\mathcal S}}                            
\def\sP{{\mathcal P}}                            
\def\sQ{{\mathcal Q}}                            
\def\sB{{\mathcal B}}                            
\def\sD{{\mathcal D}}                            
\def\sT{{\mathcal T}}
\def\sE{{\mathcal E}}                            
\def\sF{{\mathcal F}}                            
\def\sC{{\mathcal C}}                            
\def\sO{{\mathcal O}}                            
\def\sH{{\mathcal H}} 
\def\sR{{\mathcal R}}                            
\def\sJ{{\mathcal J}}                            
\def\sCP{{\mathcal CP}}                            
\def\sX{{\mathcal X}}                            
\def\sA{{\mathcal A}} 
\def\sZ{{\mathcal Z}}                            
\def\sM{{\mathcal M}}                            
\def\sK{{\mathcal K}}     
\def\sG{{\mathcal G}}                         
\def\sY{{\mathcal Y}}                         
\def\sU{{\mathcal U}}  


\def\sIG{{\mathcal IG}}                            


\def\cD{{\sf D}}
\def\cH{{\sf H}}
\def\cI{{\sf I}}

% Vectors
\def\vectorfontone{\bf}
\def\vectorfonttwo{\boldsymbol}
\def\va{{\vectorfontone a}}                      %
\def\vb{{\vectorfontone b}}                      %
\def\vc{{\vectorfontone c}}                      %
\def\vd{{\vectorfontone d}}                      %
\def\ve{{\vectorfontone e}}                      %
\def\vf{{\vectorfontone f}}                      %
\def\vg{{\vectorfontone g}}                      %
\def\vh{{\vectorfontone h}}                      %
\def\vi{{\vectorfontone i}}                      %
\def\vj{{\vectorfontone j}}                      %
\def\vk{{\vectorfontone k}}                      %
\def\vl{{\vectorfontone l}}                      %
\def\vm{{\vectorfontone m}}                      % number of basis functions
\def\vn{{\vectorfontone n}}                      % number of training samples
\def\vo{{\vectorfontone o}}                      %
\def\vp{{\vectorfontone p}}                      % number of unpenalized coefficients
\def\vq{{\vectorfontone q}}                      % number of penalized coefficients
\def\vr{{\vectorfontone r}}                      %
\def\vs{{\vectorfontone s}}                      %
\def\vt{{\vectorfontone t}}                      %
\def\vu{{\vectorfontone u}}                      % Penalized coefficients
\def\vv{{\vectorfontone v}}                      %
\def\vw{{\vectorfontone w}}                      %
\def\vx{{\vectorfontone x}}                      % Covariates/Predictors
\def\vy{{\vectorfontone y}}                      % Targets/Labels
\def\vz{{\vectorfontone z}}                      %

\def\vone{{\vectorfontone 1}}
\def\vzero{{\vectorfontone 0}}

\def\valpha{{\vectorfonttwo \alpha}}             %
\def\vbeta{{\vectorfonttwo \beta}}               % Unpenalized coefficients
\def\vgamma{{\vectorfonttwo \gamma}}             %
\def\vdelta{{\vectorfonttwo \delta}}             %
\def\vepsilon{{\vectorfonttwo \epsilon}}         %
\def\vvarepsilon{{\vectorfonttwo \varepsilon}}   % Vector of errors
\def\vzeta{{\vectorfonttwo \zeta}}               %
\def\veta{{\vectorfonttwo \eta}}                 % Vector of natural parameters
\def\vtheta{{\vectorfonttwo \theta}}             % Vector of combined coefficients
\def\vvartheta{{\vectorfonttwo \vartheta}}       %
\def\viota{{\vectorfonttwo \iota}}               %
\def\vkappa{{\vectorfonttwo \kappa}}             %
\def\vlambda{{\vectorfonttwo \lambda}}           % Vector of smoothing parameters
\def\vmu{{\vectorfonttwo \mu}}                   % Vector of means
\def\vnu{{\vectorfonttwo \nu}}                   %
\def\vxi{{\vectorfonttwo \xi}}                   %
\def\vpi{{\vectorfonttwo \pi}}                   %
\def\vvarpi{{\vectorfonttwo \varpi}}             %
\def\vrho{{\vectorfonttwo \rho}}                 %
\def\vvarrho{{\vectorfonttwo \varrho}}           %
\def\vsigma{{\vectorfonttwo \sigma}}             %
\def\vvarsigma{{\vectorfonttwo \varsigma}}       %
\def\vtau{{\vectorfonttwo \tau}}                 %
\def\vupsilon{{\vectorfonttwo \upsilon}}         %
\def\vphi{{\vectorfonttwo \phi}}                 %
\def\vvarphi{{\vectorfonttwo \varphi}}           %
\def\vchi{{\vectorfonttwo \chi}}                 %
\def\vpsi{{\vectorfonttwo \psi}}                 %
\def\vomega{{\vectorfonttwo \omega}}             %


% Matrices
%\def\matrixfontone{\sf}
%\def\matrixfonttwo{\sf}
\def\matrixfontone{\bf}
\def\matrixfonttwo{\boldsymbol}
\def\mA{{\matrixfontone A}}                      %
\def\mB{{\matrixfontone B}}                      %
\def\mC{{\matrixfontone C}}                      % Combined Design Matrix
\def\mD{{\matrixfontone D}}                      % Penalty Matrix for \vu_J
\def\mE{{\matrixfontone E}}                      %
\def\mF{{\matrixfontone F}}                      %
\def\mG{{\matrixfontone G}}                      % Penalty Matrix for \vu
\def\mH{{\matrixfontone H}}                      %
\def\mI{{\matrixfontone I}}                      % Identity Matrix
\def\mJ{{\matrixfontone J}}                      %
\def\mK{{\matrixfontone K}}                      %
\def\mL{{\matrixfontone L}}                      % Lower bound
\def\mM{{\matrixfontone M}}                      %
\def\mN{{\matrixfontone N}}                      %
\def\mO{{\matrixfontone O}}                      %
\def\mP{{\matrixfontone P}}                      %
\def\mQ{{\matrixfontone Q}}                      %
\def\mR{{\matrixfontone R}}                      %
\def\mS{{\matrixfontone S}}                      %
\def\mT{{\matrixfontone T}}                      %
\def\mU{{\matrixfontone U}}                      % Upper bound
\def\mV{{\matrixfontone V}}                      %
\def\mW{{\matrixfontone W}}                      % Variance Matrix i.e. diag(b'')
\def\mX{{\matrixfontone X}}                      % Unpenalized Design Matrix/Nullspace Matrix
\def\mY{{\matrixfontone Y}}                      %
\def\mZ{{\matrixfontone Z}}                      % Penalized Design Matrix/Kernel Space Matrix

\def\mGamma{{\matrixfonttwo \Gamma}}             %
\def\mDelta{{\matrixfonttwo \Delta}}             %
\def\mTheta{{\matrixfonttwo \Theta}}             %
\def\mLambda{{\matrixfonttwo \Lambda}}           % Penalty Matrix for \vnu
\def\mXi{{\matrixfonttwo \Xi}}                   %
\def\mPi{{\matrixfonttwo \Pi}}                   %
\def\mSigma{{\matrixfonttwo \Sigma}}             %
\def\mUpsilon{{\matrixfonttwo \Upsilon}}         %
\def\mPhi{{\matrixfonttwo \Phi}}                 %
\def\mOmega{{\matrixfonttwo \Omega}}             %
\def\mPsi{{\matrixfonttwo \Psi}}                 %

\def\mone{{\matrixfontone 1}}
\def\mzero{{\matrixfontone 0}}

% Fields or Statistical
\def\bE{{\mathbb E}}                             % Expectation
\def\bP{{\mathbb P}}                             % Probability
\def\bR{{\mathbb R}}                             % Reals
\def\bI{{\mathbb I}}                             % Reals
\def\bV{{\mathbb V}}                             % Reals
\def\bN{{\mathbb N}}

\def\d{\partial}
\def\ds{\displaystyle}

\begin{document}
	

\section{Variational Bayes for Linear Model with mixture of $g$-priors}

Consider the linear model
$$
\ds \vy|\vbeta,\sigma^2 \sim N_n(\mX\vbeta,\sigma^2\mI) 
$$

\noindent with priors
\begin{equation}\label{eq:model}
\begin{array}{rl}
\ds \vbeta|\sigma^2,g 
& \ds \sim N_p(\vzero,g\sigma^2(\mX^T\mX)^{-1}) 
\\ [1ex]
\ds p(\sigma^2) 
& \ds = (\sigma^2)^{-1} \bI(\sigma^2>0) 
\\ [1ex]
\ds p(g) 
& \ds = \frac{g^{b}(1 + g)^{-a-b-2}}{\mbox{Beta}(a+1,b+1)}  \bI(g>0)
\end{array}
\end{equation}

\noindent where $a= - 3/4$, $b = (n - p)/2 - a - 2$, 
$\vy^T\vone = 0$
and $\|\vy\|^2 = n$. Note that $b > 0$ provided $n > p + 5$. 

\medskip 
\noindent Writing the densities in full
$$
\begin{array}{rl}
\ds \log p(\vy|\vbeta,\sigma^2)
& \ds = - \frac{n}{2}\log(2\pi\sigma^2) - \frac{1}{2\sigma^2}\left[
\|\vy\|^2  - 2\vbeta^T\mX^T\vy + \vbeta^T\mX^T\mX\vbeta \right] 
\\ [1ex]
\ds \log p(\vbeta|\sigma^2,g)
& \ds = \frac{1}{2}\log|\mX^T\mX| - \frac{p}{2}\log(2\pi\sigma^2 g) - \frac{\vbeta^T\mX^T\mX\vbeta}{2\sigma^2 g}
\\ [1ex]
\ds \log p(\sigma^2)
& \ds = -\log(\sigma^2)
\\ [1ex]
\log p(g) 
& \ds = b\log(g) - (a+b+2)\log(1 + g) - \log\mbox{Beta}(a+1,b+1)
\end{array}
$$

\medskip 
\noindent Consider the VB approximation corresponding to the factorization
$q(\vtheta) = q(\vbeta)q(\sigma^2)q(g)$
where $\vtheta = (\vbeta,\sigma^2,g)$.

\medskip 
\noindent
{\bf Result 1:} If $q(\vx) \propto \exp( -\vx^T\mA\vx/2 + \vx^T\vbeta)$ then $q(\vx)$ is a $N(\mA^{-1}\vb,\mA^{-1})$ density.

\medskip 
\noindent
{\bf Result 2:} If $q(\vx)$ 
is a $N(\vmu,\mSigma)$ density then
$$
\begin{array}{rl}
\bE_q(\vx^T\mA\vx) 
& \ds = \bE_q(\mbox{tr}(\mA\vx\vx^T))
\\
& \ds = \mbox{tr}(\mA\bE_q(\vx\vx^T))
\\
& \ds = \mbox{tr}(\mA(\vmu\vmu^T + \mSigma))
\\
& \ds = \mbox{tr}(\mA\vmu\vmu^T) + \mbox{tr}(\mA\mSigma)
\\
& \ds = \mbox{tr}(\vmu^T\mA\vmu) + \mbox{tr}(\mA\mSigma)
\\
& \ds = \vmu^T\mA\vmu + \mbox{tr}(\mA\mSigma).
\end{array}
$$


\subsection{$q(\vbeta)$}

\noindent The $q$-density for $\vbeta$ is given by
$$
\begin{array}{rl}
\ds q(\vbeta)
& \ds \propto \exp\left[ \bE_{-q(\vbeta)} \left\{ \log p(\vy,\vtheta) \right\} \right]
\\ [2ex]
& \ds \propto \exp\left[ \bE_{-q(\vbeta)} \left\{  \log p(\vy|\vbeta,\sigma^2) 
+ \log p(\vbeta|\sigma^2,g) 
\right\} \right]
\\ [2ex]
& \ds \propto \exp\left[ \bE_{-q(\vbeta)} \left\{ - \frac{1}{2\sigma^2}\left( 
2\vbeta^T\mX^T\vy + \vbeta^T\mX^T\mX\vbeta \right)  - \frac{\vbeta^T\mX^T\mX\vbeta}{2\sigma^2 g} \right\} \right]
\\ [2ex]
& \ds =\exp\left[
-\frac{1}{2}\vbeta^T\left[ \tau_\sigma(1 + \tau_g)\mX^T\mX \right]\vbeta
+ \tau_\sigma\vbeta^T\mX^T\vy
\right] 
\end{array}
$$

\noindent so that $q(\vbeta)$ is a $N(\vbeta;\vmu,\mSigma)$ density where from Result 1 we have
$$
\mSigma = \tau_\sigma^{-1}(1 + \tau_g)^{-1}(\mX^T\mX)^{-1}
\qquad \mbox{and} \qquad 
\vmu 
= (1 + \tau_g)^{-1}(\mX^T\mX)^{-1}\mX^T\vy 
= (1 + \tau_g)^{-1}\widehat{\vbeta}_{\mbox{\tiny LS}}
$$

\noindent where $\widehat{\vbeta}_{\mbox{\tiny LS}} = (\mX^T\mX)^{-1}\mX^T\vy$,
$\tau_\sigma = \bE_q(\sigma^{-2})$ and
$\tau_g = \bE_q(g^{-1})$.

\subsection{$q(\sigma^2)$}

\noindent The $q$-density for $\sigma^2$ is given by
$$
\begin{array}{rl}
q(\sigma^2) 
& \ds \propto \exp\left[ \bE_{-q(\sigma^2)} \left\{ \log p(\vy,\vtheta) \right\} \right]
\\ [2ex]
& \ds \propto \exp\left[ \bE_{-q(\vbeta)} \left\{  \log p(\vy|\vbeta,\sigma^2) 
+ \log p(\vbeta|\sigma^2,g)
+ \log p(\sigma^2) \right\} \right]
\\ [2ex]
& \ds \propto
\exp\Big[
- \frac{n}{2}\log(\sigma^2) 
- \frac{1}{2\sigma^2}\bE_q\left(
\|\vy\|^2  - 2\vbeta^T\mX^T\vy + \vbeta^T\mX^T\mX\vbeta \right)
- \frac{p}{2}\log(\sigma^2) 
- \bE_q\Big( \tfrac{\vbeta^T\mX^T\mX\vbeta}{2\sigma^2 g}\Big)
- \log(\sigma^2) 
\Big]
\\ [2ex]
& \ds =
\exp\left[
- \left( \tfrac{n+p}{2} + 1 \right)\log(\sigma^2)
- \sigma^{-2}\tfrac{1}{2}\left\{
\|\vy\|^2 
- 2\vmu^T\mX^T\vy 
+ (1 + \tau_g)\vmu^T\mX^T\mX\vmu  
+ (1 + \tau_g)\mbox{tr}(\mX^T\mX\mSigma)
\right\}
\right]
\end{array}
$$

\noindent where we have used Result 2 twice to go from 
the second last line to the last line.
Hence, $q(\sigma^2)$ is a $\mbox{IG}(r,s)$ density where
$$
\ds r = \tfrac{n+p}{2}
\qquad \mbox{and} \qquad 
s  = \tfrac{1}{2}\left[ \|\vy\|^2 
- 2\vmu^T\mX^T\vy 
+ (1 + \tau_g)\vmu^T\mX^T\mX\vmu  
+ (1 + \tau_g)\mbox{tr}(\mX^T\mX\mSigma)
\right].
$$

\noindent Note that $\tau_\sigma = r/s$.

%\noindent
%Later we will show that
%$$
%s^* 
%= \frac{ n^2 \left\{ 1
%	- (1 + \tau_g)^{-1} R^2\right\}}{2(n - p)}
%$$

\subsection{$q(g)$}


\noindent The $q$-density for $g$ is given by
$$
\begin{array}{rl}
q(g)
& \ds \propto \exp\left[ \bE_{-q(g)} \left\{ \log p(\vy,\vtheta) \right\} \right]
\\ [2ex]
& \ds \propto \exp\left[ \bE_{-q(g)} \left\{ 
\log p(\vbeta|\sigma^2,g) 
+ \log p(g) 
\right\} \right]
\\ [2ex]
& \ds \propto \exp\left[
- \frac{p}{2}\log(g) 
- \bE_q\left\{ \frac{\vbeta^T\mX^T\mX\vbeta}{2\sigma^2 g} \right\}
+ b\log(g) - (a+b+2)\log(1 + g)
\right]
\\ [3ex]
& \ds = \exp\left[
\left( b - \tfrac{p}{2}\right) \log(g) 
- (a+b+2)\log(1 + g)
- g^{-1}\tfrac{\tau_\sigma}{2}\left\{\vmu^T\mX^T\mX\vmu + \mbox{tr}(\mX^T\mX\mSigma) \right\} 
\right]

\\ [2ex]
& \ds = Z^{-1} g^{n/2 - p - a - 2}(1 + g)^{-(n - p)/2}
\exp(
- c/g
)
\end{array}
$$

\noindent since $b - p/2 = n/2 - p - a - 2$ and $-a-b-2 = - (n-p)/2$
where
$$
\ds c = \tfrac{\tau_\sigma}{2}\left\{ \vmu^T\mX^T\mX\vmu + \mbox{tr}(\mX^T\mX\mSigma)\right\},
\qquad 
Z = \int_0^\infty g^{n/2 - p - a - 2}(1 + g)^{-(n-p)/2}e^{-c/g} dg,
$$

\noindent  and
$$
\tau_g = \frac{\ds \int_0^\infty g^{n/2 - p - a - 3}(1 + g)^{-(n-p)/2}e^{-c/g} dg}{
	\ds \int_0^\infty g^{n/2 - p - a - 2}(1 + g)^{-(n-p)/2}e^{-c/g} dg}.
$$


\subsection{ELBO}

An expression of the $\log \underline{p}(\vy;q)$
is given by
$$
\begin{array}{rl}
\log \underline{p}(\vy;q)
& \ds = \bE_q\left[ \log\left\{
\frac{p(\vy,\vtheta)}{q(\vtheta)}
\right\} 
\right]
\\ [2ex]
& \ds = \bE_q\Big[ - \frac{n}{2}\log(2\pi\sigma^2) - \frac{1}{2\sigma^2}\left[
\|\vy\|^2  - 2\vbeta^T\mX^T\vy + \vbeta^T\mX^T\mX\vbeta \right] 
\\ [1ex]
& \ds  \qquad  + \frac{1}{2}\log|\mX^T\mX| - \frac{p}{2}\log(2\pi\sigma^2 g) - \frac{\vbeta^T\mX^T\mX\vbeta}{2\sigma^2 g} - \log(\sigma^2)
\\ [1ex]
& \ds  \qquad  + b\log(g) - (a+b+2)\log(1 + g) - \log\mbox{Beta}(a+1,b+1)  
\\ [1ex]
& \ds  \qquad 
- r\log(s) + \log\Gamma(r) + \left( r + 1\right)\log(\sigma^2) + \frac{s}{\sigma^2}
\\ [1ex]
& \ds  \qquad - (b - p/2)\log(g) + (a+b+2)\log(1 + g) + \tfrac{c}{g} + \log(Z) \Big]
\\ [1ex]
& \ds  \qquad  + \tfrac{1}{2}\log|2e\pi\mSigma|


\\ [2ex]
& \ds =  
\frac{p}{2} 
- \frac{n}{2}\log(2\pi)
+ \frac{1}{2}\log|\mX^T\mX| 
+  \bE_q\left\{ \frac{s}{\sigma^2} - \frac{1}{2\sigma^2}\left[
\|\vy\|^2  - 2\vbeta^T\mX^T\vy + \vbeta^T\mX^T\mX\vbeta \right] 
 - \frac{\vbeta^T\mX^T\mX\vbeta}{2 g\sigma^2}  \right\} 
\\ [1ex]
& \ds  \qquad  - \log\mbox{Beta}(a+1,b+1)  - \left(\tfrac{n+p}{2} \right)\log(s) + \log\Gamma\left(\tfrac{n+p}{2} \right)
+ c\tau_g + \log(Z)  + \tfrac{1}{2}\log|\mSigma| 
\end{array}
$$ 

\noindent where
$$
\bE_q \log q(\vbeta) = -\tfrac{1}{2}\log|2e\pi\mSigma|.
$$


\noindent After the update for $q(\sigma^2)$ this simplifies to
$$
\begin{array}{rl}
\log \underline{p}(\vy;q)
& \ds =  
\tfrac{p}{2} 
- \tfrac{n}{2}\log(2\pi)
+ \tfrac{1}{2}\log|\mX^T\mX| 
- \log\mbox{Beta}(a+1,b+1)  
- \left(\tfrac{n+p}{2} \right)\log(s) 
+ \log\Gamma\left(\tfrac{n+p}{2} \right)
+ c\tau_g + \log(Z)  + \tfrac{1}{2}\log|\mSigma| 
\end{array}
$$ 
 


\subsection{Naive updates}\label{sec:Naive}

Inputs: $\vy$, $\mX$, $n$ and $p$. Initialize $\tau_\sigma>0$ and $\tau_g>0$.

\medskip 
\noindent 
Set $a = -3/4$, $b = (n-p)/2 - a - 2$ and $r = (n+p)/2$.

\medskip 
\noindent 
Calculate $\widehat{\vbeta}_{\mbox{\tiny LS}} = (\mX^T\mX)^{-1}\mX^T\vy$.

\medskip 
\noindent Loop:
\begin{itemize}
	\item $\mSigma = \tau_\sigma^{-1}(1 + \tau_g)^{-1}(\mX^T\mX)^{-1}$
	
	\item $\vmu = (1 + \tau_g)^{-1}\widehat{\vbeta}_{\mbox{\tiny LS}}$
	
	\item $s  = \tfrac{1}{2}\left[ \|\vy\|^2 
	- 2\vmu^T\mX^T\vy 
	+ (1 + \tau_g)\vmu^T\mX^T\mX\vmu  
	+ (1 + \tau_g)\mbox{tr}(\mX^T\mX\mSigma)
	\right]$
	
	\item $\tau_\sigma = r/s$
	
	\item $c 
	= \tfrac{\tau_\sigma}{2}\left\{ \vmu^T\mX^T\mX\vmu + \mbox{tr}(\mX^T\mX\mSigma)\right\}$
	
	\item $\ds \tau_g = \frac{\ds \int_0^\infty g^{n/2 - p - a - 3}(1 + g)^{-(n-p)/2}
		\exp( 
		- c/g) dg
	}{\ds \int_0^\infty g^{n/2 - p - a - 2}(1 + g)^{-(n-p)/2}
	\exp( 
	- c/g) dg}$
\end{itemize}



\newpage 

\section{Simplifications}

\noindent At convergence
the equations inside the loop in Section 
(\ref{sec:Naive}) hold simultaneously 
for the optimal $q$-densities.

\medskip 
\noindent {\bf Simplification 1:} 
$$
\vy^T\mX(\mX^T\mX)^{-1}\mX^T\vy = \|\vy\|^2 R^2 = nR^2
$$

\noindent 
which follows from using the identities for $R^2$ and the fact that $\|\vy\|^2 = \vy^T\vy = n$.  


\medskip 
\noindent {\bf Simplification 2:} 
When $\vmu = (1 + \tau_g)^{-1}\widehat{\vbeta}_{\mbox{\tiny LS}}$
and $\mSigma = \tau_\sigma^{-1}(1 + \tau_g)^{-1}(\mX^T\mX)^{-1}$ we have
$$
\begin{array}{rl}
\ds \vmu^T\mX^T\mX\vmu  
& \ds = (1 + \tau_g)^{-2}\widehat{\vbeta}_{\mbox{\tiny LS}}^T\mX^T\mX \widehat{\vbeta}_{\mbox{\tiny LS}} \\ [2ex]  
& \ds = (1 + \tau_g)^{-2}\vy^T\mX(\mX^T\mX)^{-1}\mX^T\mX (\mX^T\mX)^{-1}\mX^T\vy \\ [2ex]  
& \ds = (1 + \tau_g)^{-2} n R^2
\end{array}
$$

\noindent and
$\mbox{tr}(\mX^T\mX\mSigma) = \tau_\sigma^{-1}(1 + \tau_g)^{-1}p$.

\medskip 
\noindent {\bf Simplification 3:} When $\vmu = (1 + \tau_g)^{-1}\widehat{\vbeta}_{\mbox{\tiny LS}}$
and $\mSigma = \tau_\sigma^{-1}(1 + \tau_g)^{-1}(\mX^T\mX)^{-1}$
and using simplifications 1 and 2 we have
$$
\begin{array}{rl}
s 
& \ds 
= \tfrac{1}{2}\left[ \|\vy\|^2 
- 2\vmu^T\mX^T\vy 
+ (1 + \tau_g)\vmu^T\mX^T\mX\vmu  
+ (1 + \tau_g)\mbox{tr}(\mX^T\mX\mSigma)
\right]
\\ [2ex]
& \ds = \tfrac{1}{2}\left\{ \|\vy\|^2 
- 2(1 + \tau_g)^{-1} \vy^T\mX(\mX^T\mX)^{-1}\mX^T\vy 
+ (1 + \tau_g)^{-1}\vy^T\mX(\mX^T\mX)^{-1}\mX^T\vy 
+ p\tau_\sigma^{-1} \right\}
\\ [2ex]
& \ds = \tfrac{1}{2}\left\{ n
- (1 + \tau_g)^{-1} nR^2
+ p\tau_\sigma^{-1}\right\}.
\end{array}
$$

\noindent Now note that $\tau_\sigma = \bE_q(\sigma^{-2}) = r/s$. Hence,
$$
\tau_\sigma = \frac{n+p}{ n 
	- (1 + \tau_g)^{-1} nR^2
	+ p\tau_\sigma^{-1} }.
$$

\noindent Solving for $\tau_\sigma$ we have
$$
\tau_\sigma 
= \frac{1}{1 - (1 + \tau_g)^{-1} R^2}.
$$

\noindent and
$$
s =  r/\tau_\sigma = \tfrac{1}{2}(n + p )( 1 - (1 + \tau_g)^{-1} R^2).
$$


\medskip 
\noindent {\bf Simplification 4:} When $\vmu = (1 + \tau_g)^{-1}\widehat{\vbeta}_{\mbox{\tiny LS}}$
and $\mSigma = \tau_\sigma^{-1}(1 + \tau_g)^{-1}(\mX^T\mX)^{-1}$
and using simplifications 1-3 we have
$$
\begin{array}{rl}
\ds c 
& \ds = \frac{\tau_\sigma}{2}\left\{ \vmu^T\mX^T\mX\vmu + \mbox{tr}(\mX^T\mX\mSigma)\right\}
\\ [2ex]
& \ds = \frac{\tau_\sigma}{2}\left\{ (1 + \tau_g)^{-2} n R^2 + \tau_\sigma^{-1}(1 + \tau_g)^{-1}p \right\}
\\ [2ex]
& \ds = \frac{(1 + \tau_g)^{-2}nR^2}{2\left\{ 1
	- (1 + \tau_g)^{-1} R^2\right\} } + (1 + \tau_g)^{-1}\frac{p}{2} 
\\ [2ex]
& \ds = \frac{nR^2}{2( 1 + \tau_g)( 1 + \tau_g
	-   R^2)} + \frac{p}{2(1 + \tau_g)} 
\\ [2ex]
& \ds = \frac{nR^2}{2( 1 + \tau_g)( 1 + \tau_g
	-   R^2)} + \frac{p( 1 + \tau_g
	-   R^2)}{2(1 + \tau_g)( 1 + \tau_g
	-   R^2)}
\\ [2ex]
& \ds = \frac{(n-p)R^2}{2( 1 + \tau_g)( 1 + \tau_g
	-   R^2)} + \frac{p}{2( 1 + \tau_g
	-   R^2)} 
\end{array}
$$

\subsection{Simplified updates}

Inputs: $\vy$, $\mX$, $n$ and $p$. Initialize $\tau_g$.

\medskip 
\noindent 
Set $a=-3/4$, $b = (n-p)/2 - a - 2$ and $r = (n+p)/2$.

\begin{itemize}
\item Calculate the model summary
	statistics 
	$$\widehat{\vbeta}_{\mbox{\tiny LS}} = (\mX^T\mX)^{-1}\mX^T\vy \qquad \mbox{and} \qquad  R^2 = \vy^T\mX(\mX^T\mX)^{-1}\mX^T\vy/n.
	$$
	
\item Solve
$$\ds \tau_g = \frac{\ds \int_0^\infty g^{n/2 - p - a - 3}(1 + g)^{-(n-p)/2}
		\exp( 
		- c/g) dg
	}{\ds \int_0^\infty g^{n/2 - p - a - 2}(1 + g)^{-(n-p)/2}
	\exp( 
	- c/g) dg}
\quad \mbox{where} \quad 
\ds c = \frac{nR^2}{2( 1 + \tau_g)( 1 + \tau_g
	-   R^2)} + \frac{p}{2(1 + \tau_g)}
$$

\item $
\ds \tau_\sigma 
= \frac{1}{ 1
	- (1 + \tau_g)^{-1} R^2}
$

\item
$\mSigma = \tau_\sigma^{-1}(1 + \tau_g)^{-1}(\mX^T\mX)^{-1}$

\item $\vmu = (1 + \tau_g)^{-1}\widehat{\vbeta}_{\mbox{\tiny LS}}$
\end{itemize}
 

\subsection{EBLO}

Consider the above simplified algorithm.
An expression of the $\log p(\vy;q)$
is given by
$$
\begin{array}{rl}
\log \underline{p}(\vy;q)
& \ds =  
\tfrac{p}{2} 
- \tfrac{n}{2}\log(2\pi)
- \log\mbox{Beta}(a+1,b+1)  
- \tfrac{n}{2}\log(s) 
+ \log\Gamma\left(\tfrac{n+p}{2} \right)
+ c\tau_g + \log(Z)  
- \tfrac{p}{2}\log\left(\tfrac{n+p}{2} \right)
- \tfrac{p}{2}\log(1 + \tau_g)

\\ [2ex]
& \ds =  
\tfrac{p}{2} 
- \tfrac{n}{2}\log(2\pi)
- \log\mbox{Beta}(a+1,b+1)  
+ \tfrac{n}{2}\log(\tau_\sigma)
- \tfrac{n}{2}\log(r)
\\ [2ex]
& \ds \qquad 
+ \log\Gamma\left(\tfrac{n+p}{2} \right)
+ c\tau_g + \log(Z)  
- \tfrac{p}{2}\log\left(\tfrac{n+p}{2} \right)
- \tfrac{p}{2}\log(1 + \tau_g)


\\ [2ex]
& \ds =  
\tfrac{p}{2} 
- \tfrac{n}{2}\log(2\pi)
- \log\mbox{Beta}(a+1,b+1)  
- \tfrac{n}{2}\log(1 + \tau_g -  R^2)
\\ [2ex]
& \ds \qquad 
- \left(\tfrac{n+p}{2}\right)\log\left(\tfrac{n+p}{2} \right)
+ \log\Gamma\left(\tfrac{n+p}{2} \right)
+ c\tau_g + \log(Z)  
+ \left( \tfrac{n-p}{2} \right) \log(1 + \tau_g)



\end{array}
$$ 

\noindent Might try sometime to show this is approximately BIC.


\begin{figure}[ht]
	\centering
	\includegraphics[width=0.9\textwidth]{LowerBound.pdf}
	%\begin{minipage}[t]{0.8\textwidth}
	\caption{VB lower bound versus exact lower bound}
	\label{fig:06}
	%\end{minipage}
\end{figure}


\section{Approximations for $Z$ and $\tau_g$}

It will be useful to explore some alternatives for solving for $\tau_g$.
Some include
\begin{itemize}
\item Exact expressions.

\item Plug in $\bE(g^{-1}|\vy)$.
	
\item Laplace approximation.

\item Fully exponential Laplace approximation.



\item One dimensional quadrature.
\end{itemize}

\noindent Let 
$$
\ds I_1 =  \int_0^\infty g^{n/2 - p - a - 2}(1 + g)^{-(n-p)/2}
\exp( - c/g) dg.
$$

\noindent and 
$$
I_2 = \ds \int_0^\infty g^{n/2 - p - a - 3}(1 + g)^{-(n-p)/2}\exp( - c/g) dg
$$


\subsection{Exact expression for $Z$ and $\tau_g$ -- when $n > 2p + 1/2$}

\noindent From G\&R 3.471(7) we have
\begin{equation}\label{eq:WhittakerIntegral}
\int_0^\infty x^{\nu - 1} (\gamma + x)^{\mu - 1} e^{-\beta/x} dx 
= \beta^{(\nu-1)/2}
\gamma^{(\nu-1)/2}
\Gamma(1 - \mu - \nu)
e^{\beta/(2\gamma)}W_{(\nu-1)/2 + \mu,-\nu/2}(\beta/\gamma)
\end{equation}

\noindent provided $|\mbox{arg}(\gamma)|<\pi$, $\mbox{Re}(1 - \mu)>\mbox{Re}(\nu)>0$
where $W_{\lambda,\mu}(z)$ is a Whittaker function.

\medskip 
\noindent 
For our current case
$$
\nu_1 = \frac{n}{2} - p - a - 1, 
\quad 
1 - \mu = \frac{(n-p)}{2},
\quad 
\gamma = 1,
\quad 
\mbox{and}
\quad 
\beta = c.
$$

\noindent To apply (\ref{eq:WhittakerIntegral}) we need $n > 2p + 1/2$
and  the condition $(1 - \mu)>\nu_1$ always holds.
Under this condition
$$
I_1 = c^{n/2 - p - a - 2}
\Gamma(p/2  + a + 1)
e^{c/2}
W_{(\nu_1-1)/2 + \mu,-\nu_1/2}(c)
$$

\medskip 
\noindent 
When calculating $I_2$ we have
$$
\nu_2 = \frac{n}{2} - p - a - 2, 
$$

\noindent and identical values for $\mu$, $\gamma$ and $\beta$.
The condition $\nu_2>0$ implies
$n >  2p + 5/2$ while the condition $(1 - \mu)>\nu_2$ always holds.
Hence,
$$
I_2 = c^{n/2 - p - a - 3}
\Gamma(p/2  + a + 2)
e^{c/2}
W_{(\nu_2-1)/2 + \mu_2,-\nu_2/2}(c)
$$

\noindent and
$$
\ds \tau_g
= \frac{\ds c^{(n/2 - p - a - 3)/2}
	\Gamma(p/2  + a + 2)
	e^{c/2}
	W_{(\nu_2-1)/2 + \mu,-\nu_2/2}(c)}{
	\ds c^{(n/2 - p - a - 2)/2}
	\Gamma(p/2  + a + 1)
	e^{c/2}
	W_{(\nu_1-1)/2 + \mu,-\nu_1/2}(c)}
=  c^{-1/2} (p/2  + a + 1)\frac{\ds 
	W_{(\nu_2-1)/2 + \mu,-\nu_2/2}(c)}{
	W_{(\nu_1-1)/2 + \mu,-\nu_1/2}(c)}.
$$

\subsection{Alternative integral representations}

\noindent It will be convenient to define
$$
\ds J_1(A,B,C) = \int_0^\infty g^A(1 + g)^{B}e^{-C/g} dg
$$

\noindent then
$$
Z = J_1(n/2 - p - a - 2,-(n-p)/2,c)
$$

\noindent and
$$
\tau_g = \frac{\ds \int_0^\infty g^{n/2 - p - a - 1}(1 + g)^{-(n-p)/2}e^{-c/g} dg}{
	\ds \int_0^\infty g^{n/2 - p - a - 2}(1 + g)^{-(n-p)/2}e^{-c/g} dg} 
= \frac{I_1(n/2 - p - a - 3,-(n-p)/2,c)}{I_1(n/2 - p - a - 2,-(n-p)/2,c)}
$$

\noindent Let $h = 1/g$ then
$$
dg = -1/h^2 dh
$$

$$
\begin{array}{rl}
\ds q(h) 
    & \ds = Z^{-1}  (1/h)^{n/2 - p - a - 2}(1 + 1/h)^{-(n-p)/2}e^{-ch} h^{-2}
    \\
	& \ds = Z^{-1}  h^{-n/2 + p + a + 2}\left( \frac{1 +h}{h}\right)^{-(n-p)/2}e^{-ch} h^{-2}
	\\
	& \ds = Z^{-1}  h^{-n/2 + p + a + 2 + (n-p)/2 - 2}(1 + h)^{-(n-p)/2}e^{-ch} 
	\\
	& \ds = Z^{-1}  h^{p/2 + a}(1 + h)^{-(n-p)/2}e^{-ch} 
\end{array}
$$

\noindent Let 
$$
J_2(A,B,C) = \int_0^\infty h^A(1 + h)^{B}e^{-Ch} dh 
$$

\noindent then
$$
Z = J_2(p/2 + a,-(n-p)/2,c)
$$

\noindent and
$$
\tau_g = \bE_{q(g)}(g^{-1}) = \bE_{q(h)}(h) = \frac{\ds \int_0^\infty h^{p/2 + a +1}(1 + h)^{-(n-p)/2}e^{-ch}  dh}{\ds 
	\int_0^\infty h^{p/2 + a}(1 + h)^{-(n-p)/2}e^{-ch}  dh} 
= \frac{J_2(p/2 + a+1,-(n-p)/2,c)}{J_2(p/2 + a,-(n-p)/2,c)}.
$$

\subsection{Using the trapezoid rule}

This approach uses the iterations

\begin{enumerate}
	\item[(A)] $\ds c \leftarrow \frac{nR^2}{2( 1 + \tau_g)( 1 + \tau_g
		-   R^2)} + \frac{p}{2(1 + \tau_g)}$
	
	\item[(B)] $\tau_g \leftarrow \frac{\ds \int_0^\infty g^{n/2 - p - a - 1}(1 + g)^{-(n-p)/2}e^{-c/g} dg}{
		\ds \int_0^\infty g^{n/2 - p - a - 2}(1 + g)^{-(n-p)/2}e^{-c/g} dg} $.
\end{enumerate}

\noindent where the above integrals are approximated using the 
composite trapezoid rule.

\begin{figure}[ht]
	\centering
	\includegraphics[width=0.9\textwidth]{Trapint.pdf}
	%\begin{minipage}[t]{0.8\textwidth}
	\caption{VB vs Exact shrinkage where VB uses an iterative trapezoidal rule approach}
	\label{fig:01}
	%\end{minipage}
\end{figure}

\subsection{Plug-in approximation for $\tau_g$ using exact inference}


\noindent Using full Bayesian inference it can be shown that
$$
\ds \bE(g^{-1}|\vy) = (1 -  R^2)\left[ 1 + \frac{p/2 + a + 1}{b}   \right] \\
$$

\noindent Note that when $n < p + 5$ this value can produce undesirable results.
This approximation then uses 
$$
\tau_g \approx (1 -  R^2)\left[ 1 + \frac{p/2 + a + 1}{b}   \right].
$$.

\begin{figure}[ht]
	\centering
	\includegraphics[width=0.9\textwidth]{Plugin.pdf}
	%\begin{minipage}[t]{0.8\textwidth}
	\caption{VB vs Exact shrinkage where VB uses plug-in value from exact inference}
	\label{fig:02}
	%\end{minipage}
\end{figure}



 

\newpage

\subsection{Fast iterative approximation using fully exponential Laplace
	approximation}


The fully exponential Laplace approximation is used to approximate 
ratios of integrals and usually achieves a higher order rate of convergence
compared to the Laplace approximation.
Suppose that
$$
\ds \tau_g = \frac{\ds \int h f(h) dh}{\ds \int f(h) dh}
$$

\noindent where $f(h) = h^{p/2 + a}(1+h)^{-(n-p)/2}\exp(-ch)$.
Then the fully exponential Laplace approximation uses
$$
\ds \tau_g \approx  \frac{\ds \sqrt{\sigma_1^2}\widehat{h}_1 f(\widehat{h}_1) }{\ds \sqrt{\sigma_0^2} f(\widehat{h}_0) } = \mbox{FEL}(p/2+a,-(n-p)/2,c)
$$

\noindent where
$$
\widehat{h}_1 = \mbox{argmax}_h \left\{\log(h) + \log f(h) \right\},
\qquad 
\widehat{h}_0 = \mbox{argmax}_h \left\{ \log f(h) \right\},
$$
$$
\sigma_1^2 = \left[\left\{ - \frac{d^2 \log f(h)}{d h^2}\right\}_{h = \widehat{h}_1} \right]^{-1}
\qquad \mbox{and} \qquad 
\sigma_0^2 = \left[\left\{ - \frac{d^2 \log f(h)}{d h^2}\right\}_{h = \widehat{h}_0} \right]^{-1}.
$$

\noindent Then iterative method which uses 
the fully exponential Laplace approximation iterates over the following
two steps (for an initialized value of $\tau_g$)
\begin{enumerate}
\item[(A)] $\ds c \leftarrow \frac{nR^2}{2( 1 + \tau_g)( 1 + \tau_g
	-   R^2)} + \frac{p}{2(1 + \tau_g)}$

\item[(B)] $\tau_g \leftarrow \mbox{FEL}(p/2+a,-(n-p)/2,c)$.
\end{enumerate}


\begin{figure}[ht]
	\centering
	\includegraphics[width=0.9\textwidth]{FEL.pdf}
	%\begin{minipage}[t]{0.8\textwidth}
	\caption{VB vs Exact shrinkage where VB uses iterative fully exponential Laplace approximation}
	\label{fig:03}
	%\end{minipage}
\end{figure}



\newpage 




\subsection{Plug-in approximation for $\tau_g$ based on $q(h)$ using Laplace's method}

From above
$$
q(h) = Z^{-1} h^{p/2 + a}(1+h)^{-(n-p)/2}\exp(-ch) \bI(h>0).
$$

\noindent The Laplace approximation uses
$$
\tau_g =\bE_{q(g)}(g^{-1}) = \bE_{q(h)}(h) \approx \mbox{mode}\left\{ q(h) \right\}.
$$

\noindent Note that 
$\mbox{mode}\left\{ q(h) \right\}$ is the maximizer of
$$
\log q(h) = \frac{A}{2}\log(h) - \frac{n-p}{2}\log(1 + h) - ch - \log(Z).
$$

\noindent where $A = p + 2a$.
From first order optimality conditions
$$
\frac{A}{2h} - \frac{n-p}{2(1+h)} - c = 0
$$

\noindent Equating $\tau_g$ to the mode to the maximizer of $q(h)$
and substituting $c$ as a function of $\tau_g$
we have
\begin{equation}\label{eq:solveTau}
\begin{array}{l}
\ds 0 = \frac{A}{2\tau_g} - \frac{n-p}{2(1+\tau_g)} - \left[
\frac{nR^2}{2( 1 + \tau_g)( 1 + \tau_g
	-   R^2)} + \frac{p}{2(1 + \tau_g)} 
\right] 
\\
[2ex] 
\ds \qquad \implies
0 = \frac{A}{\tau_g} - \frac{n-p}{1+\tau_g} - \left[
\frac{nR^2}{( 1 + \tau_g)( 1 + \tau_g
	-   R^2)} + \frac{p}{1 + \tau_g}  \right] 
\\
[2ex] 
\ds \qquad \implies
0 = \frac{A}{\tau_g} - \frac{n}{1+\tau_g} - 
\frac{nR^2}{( 1 + \tau_g)( 1 + \tau_g
	-   R^2)}  


\\
[2ex] 
\ds \qquad \implies
0 = \frac{A}{\tau_g} - \left[ \frac{n( 1 + \tau_g
	-   R^2) + nR^2}{(1+\tau_g)( 1 + \tau_g
	-   R^2)}    \right]

\\
[2ex] 
\ds \qquad \implies
0 = \frac{A}{\tau_g} -  \frac{n(1 + \tau_g)
	}{(1+\tau_g)( 1 + \tau_g
	-   R^2)}    

\\
[2ex] 
\ds \qquad \implies
0 = \frac{A}{\tau_g} -  \frac{n
}{ 1 + \tau_g -   R^2}   

\\
[2ex] 
\ds \qquad \implies
0 = A( 1 + \tau_g -   R^2) - n\tau_g 

\\
[2ex] 
\ds \qquad \implies
0 = (A - n)\tau_g  + A(1  -   R^2) 

\\
[2ex] 
\ds \qquad \implies
\tau_g = \frac{A(1  -   R^2)}{n - A} 

\end{array}
\end{equation}
 
\begin{figure}[ht]
	\centering
	\includegraphics[width=0.9\textwidth]{PluginLaplace.pdf}
	%\begin{minipage}[t]{0.8\textwidth}
	\caption{VB vs Exact shrinkage where VB uses a plug-in Laplace approximation}
	\label{fig:03}
	%\end{minipage}
\end{figure}

\newpage 

\section{Comparisons to be made}

For fixed $R^2$, $n$ and $p$.
\begin{itemize}
\item Shrinkage (coefficient posterior mean).
\item Coefficient posterior variance.
\item $p(\sigma^2|\vy)$.
\item $p(g|\vy)$
\item $p(\vy)$
\item Variable inclusion probabilities (on real data)
\item Model ranking (on real data)
\end{itemize}



\begin{figure}[ht]
	\centering
	\includegraphics[width=0.5\textwidth]{g.pdf}
	%\begin{minipage}[t]{0.8\textwidth}
	\caption{Accuracies for $p(g|\vy)$ vs $q(g)$}
	\label{fig:03}
	%\end{minipage}
\end{figure}

\begin{figure}[ht]
	\centering
	\includegraphics[width=0.5\textwidth]{sigma2.pdf}
	%\begin{minipage}[t]{0.8\textwidth}
	\caption{Accuracies for $p(\sigma^2|\vy)$ vs $q(\sigma^2)$}
	\label{fig:03}
	%\end{minipage}
\end{figure}



$$
q(h) = Z^{-1} h^{p/2 + a}(1+h)^{-(n-p)/2}\exp(-ch) \bI(h>0).
$$

$$
I(A,B,C) = \int_0^\infty h^{A}(1+h)^{-B}\exp(-Ch) dh.
$$

$$
(uv)' = v'u + u'v
$$

$$
\int v'u dh = uv - int u'v  dh
$$

$$
u = h^{A}\exp(-Ch) 
$$
$$
u' = A h^{A - 1} \exp(-Ch) - Ch^{A}\exp(-Ch) 
$$

$$
v' = (1 + h)^{-B}
$$

$$
v = -\frac{1}{B-1} (1 + h)^{-B+1}
$$

$$
\begin{array}{rl}
\ds I(A,B,C) 
& \ds = 
\left[ -\frac{1}{B-1} (1 + h)^{-B+1} h^{A}\exp(-Ch)  \right]_0^\infty 
+ \frac{1}{B-1}[A I(A-1,B-1,C) - C I(A,B-1,C)]
\\
& \ds = \frac{A}{B-1}I(A-1,B-1,C) - \frac{C}{B-1} I(A,B-1,C) 
\end{array}
$$

$$
\bE_q(h) = \frac{I(A+1,B,C)}{I(A,B,C)} 
$$


\end{document}
 