% Literature_review.tex
\documentclass{amsart}[12pt]

\addtolength{\oddsidemargin}{-.75in}%
\addtolength{\evensidemargin}{-.75in}%
\addtolength{\textwidth}{1.5in}%
\addtolength{\textheight}{1.3in}%
\addtolength{\topmargin}{-.8in}%
\addtolength{\marginparpush}{-.75in}%
% \setlength\parindent{0pt}
% \setlength{\bibsep}{0pt plus 0.3ex}

\usepackage[authoryear]{natbib}
\usepackage{graphicx}
\usepackage{algorithm,algorithmic}
\usepackage{cancel}
\usepackage{amsthm}
\usepackage{mathtools}
\usepackage{algorithm,algorithmic}
\usepackage[inner=2.5cm,outer=1.5cm,bottom=2cm]{geometry}
\usepackage{setspace}
\onehalfspacing
\usepackage{microtype}

\newtheorem{theorem}{Theorem}[section]

\title{Literature review}
\author{Mark Greenaway, John T. Ormerod}

% include.tex
\newcommand{\Bernoulli}[1]{\text{Bernoulli} \left( #1 \right)}
\newcommand{\mydigamma}[1]{\psi \left( #1 \right)}
%\newcommand{\diag}[1]{\text{diag}\left( #1 \right)}
\newcommand{\tr}[1]{\text{tr}\left( #1 \right)}
\newcommand{\Poisson}[1]{\text{Poisson} \left( #1 \right)}
\def \half {\frac{1}{2}}
\def \R {\mathbb{R}}
\def \vbeta {\vec{\beta}}
\def \vy {\vec{y}}
\def \vmu {\vec{\mu}}
\def \vmuqbeta {\vmu_{q(\vbeta)}}
\def \vmubeta {\vmu_{\vbeta}}
\def \Sigmaqbeta {\Sigma_{q(\vbeta)}}
\def \Sigmabeta {\Sigma_{\vbeta}}
\def \va {\vec{a}}
\def \vtheta {\vec{\theta}}
\def \mX {\vec{X}}

\def\ds{{\displaystyle}}

\def\diag{{\mbox{diag}}}


\usepackage{latexsym,amssymb,amsmath,amsfonts}
%\usepackage{tabularx}
\usepackage{theorem}
\usepackage{verbatim,array,multicol,palatino}
\usepackage{graphicx}
\usepackage{graphics}
\usepackage{fancyhdr}
\usepackage{algorithm,algorithmic}
\usepackage{url}
%\usepackage[all]{xy}



\def\approxdist{\stackrel{{\tiny \mbox{approx.}}}{\sim}}
\def\smhalf{\textstyle{\frac{1}{2}}}
\def\vxnew{\vx_{\mbox{{\tiny new}}}}
\def\bib{\vskip12pt\par\noindent\hangindent=1 true cm\hangafter=1}
\def\jump{\vskip3mm\noindent}
\def\etal{{\em et al.}}
\def\etahat{{\widehat\eta}}
\def\thick#1{\hbox{\rlap{$#1$}\kern0.25pt\rlap{$#1$}\kern0.25pt$#1$}}
\def\smbbeta{{\thick{\scriptstyle{\beta}}}}
\def\smbtheta{{\thick{\scriptstyle{\theta}}}}
\def\smbu{{\thick{\scriptstyle{\rm u}}}}
\def\smbzero{{\thick{\scriptstyle{0}}}}
\def\boxit#1{\begin{center}\fbox{#1}\end{center}}
\def\lboxit#1{\vbox{\hrule\hbox{\vrule\kern6pt
      \vbox{\kern6pt#1\kern6pt}\kern6pt\vrule}\hrule}}
\def\thickboxit#1{\vbox{{\hrule height 1mm}\hbox{{\vrule width 1mm}\kern6pt
          \vbox{\kern6pt#1\kern6pt}\kern6pt{\vrule width 1mm}}
               {\hrule height 1mm}}}


%\sloppy
%\usepackage{geometry}
%\geometry{verbose,a4paper,tmargin=20mm,bmargin=20mm,lmargin=40mm,rmargin=20mm}


%%%%%%%%%%%%%%%%%%%%%%%%%%%%%%%%%%%%%%%%%%%%%%%%%%%%%%%%%%%%%%%%%%%%%%%%%%%%%%%%
%
% Some convenience definitions
%
% \bf      -> vector
% \sf      -> matrix
% \mathcal -> sets or statistical
% \mathbb  -> fields or statistical
%
%%%%%%%%%%%%%%%%%%%%%%%%%%%%%%%%%%%%%%%%%%%%%%%%%%%%%%%%%%%%%%%%%%%%%%%%%%%%%%%%

% Sets or statistical values
\def\sI{{\mathcal I}}                            % Current Index set
\def\sJ{{\mathcal J}}                            % Select Index set
\def\sL{{\mathcal L}}                            % Likelihood
\def\sl{{\ell}}                                  % Log-likelihood
\def\sN{{\mathcal N}}                            
\def\sS{{\mathcal S}}                            
\def\sP{{\mathcal P}}                            
\def\sQ{{\mathcal Q}}                            
\def\sB{{\mathcal B}}                            
\def\sD{{\mathcal D}}                            
\def\sT{{\mathcal T}}
\def\sE{{\mathcal E}}                            
\def\sF{{\mathcal F}}                            
\def\sC{{\mathcal C}}                            
\def\sO{{\mathcal O}}                            
\def\sH{{\mathcal H}} 
\def\sR{{\mathcal R}}                            
\def\sJ{{\mathcal J}}                            
\def\sCP{{\mathcal CP}}                            
\def\sX{{\mathcal X}}                            
\def\sA{{\mathcal A}} 
\def\sZ{{\mathcal Z}}                            
\def\sM{{\mathcal M}}                            
\def\sK{{\mathcal K}}     
\def\sG{{\mathcal G}}                         
\def\sY{{\mathcal Y}}                         
\def\sU{{\mathcal U}}  


\def\sIG{{\mathcal IG}}                            


\def\cD{{\sf D}}
\def\cH{{\sf H}}
\def\cI{{\sf I}}

% Vectors
\def\vectorfontone{\bf}
\def\vectorfonttwo{\boldsymbol}
\def\va{{\vectorfontone a}}                      %
\def\vb{{\vectorfontone b}}                      %
\def\vc{{\vectorfontone c}}                      %
\def\vd{{\vectorfontone d}}                      %
\def\ve{{\vectorfontone e}}                      %
\def\vf{{\vectorfontone f}}                      %
\def\vg{{\vectorfontone g}}                      %
\def\vh{{\vectorfontone h}}                      %
\def\vi{{\vectorfontone i}}                      %
\def\vj{{\vectorfontone j}}                      %
\def\vk{{\vectorfontone k}}                      %
\def\vl{{\vectorfontone l}}                      %
\def\vm{{\vectorfontone m}}                      % number of basis functions
\def\vn{{\vectorfontone n}}                      % number of training samples
\def\vo{{\vectorfontone o}}                      %
\def\vp{{\vectorfontone p}}                      % number of unpenalized coefficients
\def\vq{{\vectorfontone q}}                      % number of penalized coefficients
\def\vr{{\vectorfontone r}}                      %
\def\vs{{\vectorfontone s}}                      %
\def\vt{{\vectorfontone t}}                      %
\def\vu{{\vectorfontone u}}                      % Penalized coefficients
\def\vv{{\vectorfontone v}}                      %
\def\vw{{\vectorfontone w}}                      %
\def\vx{{\vectorfontone x}}                      % Covariates/Predictors
\def\vy{{\vectorfontone y}}                      % Targets/Labels
\def\vz{{\vectorfontone z}}                      %

\def\vone{{\vectorfontone 1}}
\def\vzero{{\vectorfontone 0}}

\def\valpha{{\vectorfonttwo \alpha}}             %
\def\vbeta{{\vectorfonttwo \beta}}               % Unpenalized coefficients
\def\vgamma{{\vectorfonttwo \gamma}}             %
\def\vdelta{{\vectorfonttwo \delta}}             %
\def\vepsilon{{\vectorfonttwo \epsilon}}         %
\def\vvarepsilon{{\vectorfonttwo \varepsilon}}   % Vector of errors
\def\vzeta{{\vectorfonttwo \zeta}}               %
\def\veta{{\vectorfonttwo \eta}}                 % Vector of natural parameters
\def\vtheta{{\vectorfonttwo \theta}}             % Vector of combined coefficients
\def\vvartheta{{\vectorfonttwo \vartheta}}       %
\def\viota{{\vectorfonttwo \iota}}               %
\def\vkappa{{\vectorfonttwo \kappa}}             %
\def\vlambda{{\vectorfonttwo \lambda}}           % Vector of smoothing parameters
\def\vmu{{\vectorfonttwo \mu}}                   % Vector of means
\def\vnu{{\vectorfonttwo \nu}}                   %
\def\vxi{{\vectorfonttwo \xi}}                   %
\def\vpi{{\vectorfonttwo \pi}}                   %
\def\vvarpi{{\vectorfonttwo \varpi}}             %
\def\vrho{{\vectorfonttwo \rho}}                 %
\def\vvarrho{{\vectorfonttwo \varrho}}           %
\def\vsigma{{\vectorfonttwo \sigma}}             %
\def\vvarsigma{{\vectorfonttwo \varsigma}}       %
\def\vtau{{\vectorfonttwo \tau}}                 %
\def\vupsilon{{\vectorfonttwo \upsilon}}         %
\def\vphi{{\vectorfonttwo \phi}}                 %
\def\vvarphi{{\vectorfonttwo \varphi}}           %
\def\vchi{{\vectorfonttwo \chi}}                 %
\def\vpsi{{\vectorfonttwo \psi}}                 %
\def\vomega{{\vectorfonttwo \omega}}             %


% Matrices
%\def\matrixfontone{\sf}
%\def\matrixfonttwo{\sf}
\def\matrixfontone{\bf}
\def\matrixfonttwo{\boldsymbol}
\def\mA{{\matrixfontone A}}                      %
\def\mB{{\matrixfontone B}}                      %
\def\mC{{\matrixfontone C}}                      % Combined Design Matrix
\def\mD{{\matrixfontone D}}                      % Penalty Matrix for \vu_J
\def\mE{{\matrixfontone E}}                      %
\def\mF{{\matrixfontone F}}                      %
\def\mG{{\matrixfontone G}}                      % Penalty Matrix for \vu
\def\mH{{\matrixfontone H}}                      %
\def\mI{{\matrixfontone I}}                      % Identity Matrix
\def\mJ{{\matrixfontone J}}                      %
\def\mK{{\matrixfontone K}}                      %
\def\mL{{\matrixfontone L}}                      % Lower bound
\def\mM{{\matrixfontone M}}                      %
\def\mN{{\matrixfontone N}}                      %
\def\mO{{\matrixfontone O}}                      %
\def\mP{{\matrixfontone P}}                      %
\def\mQ{{\matrixfontone Q}}                      %
\def\mR{{\matrixfontone R}}                      %
\def\mS{{\matrixfontone S}}                      %
\def\mT{{\matrixfontone T}}                      %
\def\mU{{\matrixfontone U}}                      % Upper bound
\def\mV{{\matrixfontone V}}                      %
\def\mW{{\matrixfontone W}}                      % Variance Matrix i.e. diag(b'')
\def\mX{{\matrixfontone X}}                      % Unpenalized Design Matrix/Nullspace Matrix
\def\mY{{\matrixfontone Y}}                      %
\def\mZ{{\matrixfontone Z}}                      % Penalized Design Matrix/Kernel Space Matrix

\def\mGamma{{\matrixfonttwo \Gamma}}             %
\def\mDelta{{\matrixfonttwo \Delta}}             %
\def\mTheta{{\matrixfonttwo \Theta}}             %
\def\mLambda{{\matrixfonttwo \Lambda}}           % Penalty Matrix for \vnu
\def\mXi{{\matrixfonttwo \Xi}}                   %
\def\mPi{{\matrixfonttwo \Pi}}                   %
\def\mSigma{{\matrixfonttwo \Sigma}}             %
\def\mUpsilon{{\matrixfonttwo \Upsilon}}         %
\def\mPhi{{\matrixfonttwo \Phi}}                 %
\def\mOmega{{\matrixfonttwo \Omega}}             %
\def\mPsi{{\matrixfonttwo \Psi}}                 %

\def\mone{{\matrixfontone 1}}
\def\mzero{{\matrixfontone 0}}

% Fields or Statistical
\def\bE{{\mathbb E}}                             % Expectation
\def\bP{{\mathbb P}}                             % Probability
\def\bR{{\mathbb R}}                             % Reals
\def\bI{{\mathbb I}}                             % Reals
\def\bV{{\mathbb V}}                             % Reals

\def\vX{{\vectorfontone X}}                      % Targets/Labels
\def\vY{{\vectorfontone Y}}                      % Targets/Labels
\def\vZ{{\vectorfontone Z}}                      %

% Other
\def\etal{{\em et al.}}
\def\ds{\displaystyle}
\def\d{\partial}
\def\diag{\text{diag}}
%\def\span{\text{span}}
\def\blockdiag{\text{blockdiag}}
\def\tr{\text{tr}}
\def\RSS{\text{RSS}}
\def\df{\text{df}}
\def\GCV{\text{GCV}}
\def\AIC{\text{AIC}}
\def\MLC{\text{MLC}}
\def\mAIC{\text{mAIC}}
\def\cAIC{\text{cAIC}}
\def\rank{\text{rank}}
\def\MASE{\text{MASE}}
\def\SMSE{\text{SASE}}
\def\sign{\text{sign}}
\def\card{\text{card}}
\def\notexp{\text{notexp}}
\def\ASE{\text{ASE}}
\def\ML{\text{ML}}
\def\nullity{\text{nullity}}

\def\logexpit{\text{logexpit}}
\def\logit{\mbox{logit}}
\def\dg{\mbox{dg}}

\def\Bern{\mbox{Bernoulli}}
\def\sBernoulli{\mbox{Bernoulli}}
\def\sGamma{\mbox{Gamma}}
\def\sInvN{\mbox{Inv}\sN}
\def\sNegBin{\sN\sB}

\def\dGamma{\mbox{Gamma}}
\def\dInvGam{\mbox{Inv}\Gamma}

\def\Cov{\mbox{Cov}}
\def\Mgf{\mbox{Mgf}}

\def\mis{{mis}} 
\def\obs{{obs}}

\def\argmax{\operatornamewithlimits{\text{argmax}}}
\def\argmin{\operatornamewithlimits{\text{argmin}}}
\def\argsup{\operatornamewithlimits{\text{argsup}}}
\def\arginf{\operatornamewithlimits{\text{arginf}}}


\def\minimize{\operatornamewithlimits{\text{minimize}}}
\def\maximize{\operatornamewithlimits{\text{maximize}}}
\def\suchthat{\text{such that}}


\def\relstack#1#2{\mathop{#1}\limits_{#2}}
\def\sfrac#1#2{{\textstyle{\frac{#1}{#2}}}}


\def\comment#1{
\vspace{0.5cm}
\noindent \begin{tabular}{|p{14cm}|}  
\hline #1 \\ 
\hline 
\end{tabular}
\vspace{0.5cm}
}


\def\mytext#1{\begin{tabular}{p{13cm}}#1\end{tabular}}
\def\mytextB#1{\begin{tabular}{p{7.5cm}}#1\end{tabular}}
\def\mytextC#1{\begin{tabular}{p{12cm}}#1\end{tabular}}

\def\jump{\vskip3mm\noindent}

\def\KL{\text{KL}}
\def\N{\text{N}}
\def\Var{\text{Var}}

\def \E {\mathbb{E}}
\def \BigO {\text{O}}
\def \IG {\text{IG}}
\def \Beta {\text{Beta}}



\begin{document}

\maketitle

\section{Chapter 1 -- Zero-Inflated Models}

Ormerod and Wand 2010

Explaining Variational Approximations

Highly cited paper explaining variational approximations using a number of examples.
Defines basic terms, and gives proofs of lower bound and optimal mean field update.

Starts by proving that the variational lower bound  bounds the true marginal log probability $\log p(\vy)$
from below. Then shows that assuming a product form for the approximating distribution $q(\vtheta)$, the
optimal approximating distribution for each $\vtheta_i$ has the form

\[
	q_i^*(\vtheta_i) \equiv \exp{\{\E_{-\vtheta_i}[\log p(\vy, \vtheta)]\}}
\]

Points out that only nodes in the Markov blanket of $\vtheta_i$ need to be considered, which helps reduce the
amount of work needed to calculate $q^*(\vtheta_i)$.

Gaussian Variational Approximations

Gives the derivations that allow the variational lower bound of a mixed effects model to be found. Approximate
mean and covariance of mixed models using multivariate normal with mean $\vmu$ and covariance matrix
$\mSigma$, using convex optimsation algorithms to optimise the variational lower bound directly.

B-Splines

Approximate functions with piecewise cubic polynomials.

The O'Sullivan splines paper shows that the integrals involved in calculating the penalty matrix using Simpson's Rule are exact, when the polynomials being integrated are cubic.

Zero-inflated and Bayesian models

Challis and Barber 2013

Gaussian Kullback-Leibler Approximate Inference

Constrained parameterisations of Gaussian covariance that make G-KL methods fast and scalable. But I'm fairly
sure they don't have our exact parameterisation. Discusses complexity. Numerical results.

This paper is worth reading closely. Our parameterisation is a mixture of the (c) Chevron and (d) Subspace
structures.

Challis and Barber 2011

Concave Gaussian Variational Approximations for Inference in Large-Scale Bayesian Linear Models

Looks at relationships between two popular approaches to forming bounds in approximate Bayesian inference
(local variational methods and minimal Kullback-Leibler divergence methods, the latter of which we use).

Also looks at the computational complexity of calculating the lower bound for various forms of
parameterisation of the covariance matrix.

Eklund and Karlsson 2007

Computational Efficiency in Bayesian Model and Variable Selection

Talks about MCMC schemes for model averaging by adding and removing single covariates or swapping covariates,
as well as more sophisticated schemes.

Also talks about the computational complexity of solving the least squares problem: how people very rarely
invert $\mX^\top \mX$, and instead work with a matrix factorisation of the form $A B$ that is easy to solve,
either the $QR$ factorisation which is considered to have good numerical properties or the Cholesky
factorisation when the matrix involved is a large $p$ by $p$ matrix.

\section{Chapter 3 -- Variational Bayes for Linear Model Selection Using Mixtures of g-Priors}

Zellner 1980

'Although Zellner and Siow 1980 did not explicitly use a $g$-prior formulation with a prior on $g$,
their recommendation of a multivariate Cauchy form for $p(\beta|\sigma^2)$ implicitly corresponds to using a
$g$-prior with an inverse Gamma prior'

Proposed Normal-Gamma conjugate model which includes $g$, with $g$ controlling the degree of shrinkage
from the fitted mean towards the prior mean.

George and Foster 2000

Proposed selecting the model maximising the posterior probability of model $\gamma$ based on empirical
Bayes estimates of $g$ and the standard unbiased estimate of $\sigma^2$

Cui and George 2008

Proposed marginning out $g$ with respect to a prior

Liang 2008

Builds on Zellner 1980, by proving that any fixed choice of $g$ leads to various kinds of paradoxes e.g.
Information Paradox, Bartlett's Paradox.
Proposed marginning out $g$ and $\sigma^2$ with respect to priors.

Full of good information, and serves as a great introduction to fitting normal linear models with a g-prior.
It discusses:
\begin{itemize}
\item $g$--priors, the simplified form of the Bayes Factor $BF[\mathcal{M}_\gamma : \mathcal{M}_b]$ and the definition of $BF[\mathcal{M}_\gamma : \mathcal{M}_{\gamma'}]$.
\item the paradoxes that arises from fixed choices of $g$.
\item the benefits of using a mixture of g--priors, analytic expressions are available for the posterior
distribution of $g$, $p(g|\vy)$ etc. etc. I should do a careful literature review of this paper.
\end{itemize}

Murayama and George 2011

Fully Bayes Factors with a Generalised $g$-Prior
Normal linear model, model selection based on a fully Bayes formulation with a generalisation of Zellner's
$g$-prior which allows for $p > n$

This is a more theoretical paper, which proposes a selection criteria based on a fully Bayesian formulation
with a generalisation of Zellner's $g$--prior, which allows for $p > n$. A special case of the prior
formulation is seen to yield tractable closed forms for marginal densities and Bayes factors which reveal new
model evaluation characteristics of potential interest.

Uses a Beta-Prime prior
\[
p(g) = \frac{g^b (1 + g)^{-a-b-2}}{\Beta(a + 1, b + 1)} \I_{(0, \infty)}(g)
\]

Makes special choices of $a = -3/4$ and $b = (n - q - 5)/2 - a$.

Model selection consistency is shown, which is an important property for model selection criteria to have. Do
we have this? I think we do, or can show it.

They then do some simulations to show that their approach works.

Bayorri, Berger, Forte and Garci\'{a} 2012

Criteria for Bayesian Model Choice with Application to Variable Selection

Has many good references for model selection in a Bayesian framework. I should probably go through all of them
to make sure I've made a thorough survey of the field:

Jeffreys (1961)
Zellner and Siox (1980, 1984)
Laud and Ibrahim (1995) Criterion-Based Methods for Bayesian Model Assessment
Kass and Wasserman (1995)
Berger and Pericchi (1996)
Moreno, Bertolino and Racugno (1998) Can't find
De Santis and Spezzaferri (1999)
Perez and Berger (2002)
Bayorri and Garci\'{a}--Donato (2008)
Liang et al. (2008)
Cui and Goerge (2008)
Maruyama and George (2008)
Maruyama and Strawderman (2010)

Laud and Ibrahim (1995) offers another criterion for Bayesian model assessment. The normal linear model case
gives a closed form. Some simulation studies.

Kass and Wasserman (1995) A Reference Bayesian Test for Nested Hypothesis and its Relationship to the
Schwarz Criterion. log of the Bayes Factor is approximately the Schwarz criterion $O_P(n^{-1/2})$. The
Schwarz Criterion is another name for the BIC.

Berger and Pericchi (1996) The Intrinsic Bayes Factor for Model Selection and Prediction
Yet another model selection criterion, with some claimed advantages. Advocates automatic methods of model 
selection, as "analysis of nonnested and/or multiplemodels or hypotheses is very difficult in a frequentist
framework".

De Santis and Spezzaferri (1999) Automatic and robust methods for model comparison using fractional Bayes
Factors, an alternative to Bayes Factors. Points out that Bayes Factors are very sensitive to prior
distributions. Problem in the presence of weak prior information, does not work at all in the case of improper
priors (isn't defined).

Bayorri and Garci\'{a}--Donato (2008) Divergence Based Priors for Bayesian Hypothesis Testing.

Cui and George (2008) Empirical Bayes versus fully Bayes variable selection.

Maruyama and Strawderman (2010). A New Class of Generalised Bayes Minimax Ridge Regression Estimators.

\end{document}