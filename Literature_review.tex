% Literature_review.tex
\documentclass{amsart}[12pt]

\addtolength{\oddsidemargin}{-.75in}%
\addtolength{\evensidemargin}{-.75in}%
\addtolength{\textwidth}{1.5in}%
\addtolength{\textheight}{1.3in}%
\addtolength{\topmargin}{-.8in}%
\addtolength{\marginparpush}{-.75in}%
% \setlength\parindent{0pt}
% \setlength{\bibsep}{0pt plus 0.3ex}

\usepackage[authoryear]{natbib}
\usepackage{graphicx}
\usepackage{algorithm,algorithmic}
\usepackage{cancel}
\usepackage{amsthm}
\usepackage{mathtools}
\usepackage{algorithm,algorithmic}
\usepackage[inner=2.5cm,outer=1.5cm,bottom=2cm]{geometry}
\usepackage{setspace}
\onehalfspacing
\usepackage{microtype}

\newtheorem{theorem}{Theorem}[section]

\title{Literature review}
\author{Mark Greenaway, John T. Ormerod}

% include.tex
\newcommand{\Bernoulli}[1]{\text{Bernoulli} \left( #1 \right)}
\newcommand{\mydigamma}[1]{\psi \left( #1 \right)}
%\newcommand{\diag}[1]{\text{diag}\left( #1 \right)}
\newcommand{\tr}[1]{\text{tr}\left( #1 \right)}
\newcommand{\Poisson}[1]{\text{Poisson} \left( #1 \right)}
\def \half {\frac{1}{2}}
\def \R {\mathbb{R}}
\def \vbeta {\vec{\beta}}
\def \vy {\vec{y}}
\def \vmu {\vec{\mu}}
\def \vmuqbeta {\vmu_{q(\vbeta)}}
\def \vmubeta {\vmu_{\vbeta}}
\def \Sigmaqbeta {\Sigma_{q(\vbeta)}}
\def \Sigmabeta {\Sigma_{\vbeta}}
\def \va {\vec{a}}
\def \vtheta {\vec{\theta}}
\def \mX {\vec{X}}

\def\ds{{\displaystyle}}

\def\diag{{\mbox{diag}}}


\usepackage{latexsym,amssymb,amsmath,amsfonts}
%\usepackage{tabularx}
\usepackage{theorem}
\usepackage{verbatim,array,multicol,palatino}
\usepackage{graphicx}
\usepackage{graphics}
\usepackage{fancyhdr}
\usepackage{algorithm,algorithmic}
\usepackage{url}
%\usepackage[all]{xy}



\def\approxdist{\stackrel{{\tiny \mbox{approx.}}}{\sim}}
\def\smhalf{\textstyle{\frac{1}{2}}}
\def\vxnew{\vx_{\mbox{{\tiny new}}}}
\def\bib{\vskip12pt\par\noindent\hangindent=1 true cm\hangafter=1}
\def\jump{\vskip3mm\noindent}
\def\etal{{\em et al.}}
\def\etahat{{\widehat\eta}}
\def\thick#1{\hbox{\rlap{$#1$}\kern0.25pt\rlap{$#1$}\kern0.25pt$#1$}}
\def\smbbeta{{\thick{\scriptstyle{\beta}}}}
\def\smbtheta{{\thick{\scriptstyle{\theta}}}}
\def\smbu{{\thick{\scriptstyle{\rm u}}}}
\def\smbzero{{\thick{\scriptstyle{0}}}}
\def\boxit#1{\begin{center}\fbox{#1}\end{center}}
\def\lboxit#1{\vbox{\hrule\hbox{\vrule\kern6pt
      \vbox{\kern6pt#1\kern6pt}\kern6pt\vrule}\hrule}}
\def\thickboxit#1{\vbox{{\hrule height 1mm}\hbox{{\vrule width 1mm}\kern6pt
          \vbox{\kern6pt#1\kern6pt}\kern6pt{\vrule width 1mm}}
               {\hrule height 1mm}}}


%\sloppy
%\usepackage{geometry}
%\geometry{verbose,a4paper,tmargin=20mm,bmargin=20mm,lmargin=40mm,rmargin=20mm}


%%%%%%%%%%%%%%%%%%%%%%%%%%%%%%%%%%%%%%%%%%%%%%%%%%%%%%%%%%%%%%%%%%%%%%%%%%%%%%%%
%
% Some convenience definitions
%
% \bf      -> vector
% \sf      -> matrix
% \mathcal -> sets or statistical
% \mathbb  -> fields or statistical
%
%%%%%%%%%%%%%%%%%%%%%%%%%%%%%%%%%%%%%%%%%%%%%%%%%%%%%%%%%%%%%%%%%%%%%%%%%%%%%%%%

% Sets or statistical values
\def\sI{{\mathcal I}}                            % Current Index set
\def\sJ{{\mathcal J}}                            % Select Index set
\def\sL{{\mathcal L}}                            % Likelihood
\def\sl{{\ell}}                                  % Log-likelihood
\def\sN{{\mathcal N}}                            
\def\sS{{\mathcal S}}                            
\def\sP{{\mathcal P}}                            
\def\sQ{{\mathcal Q}}                            
\def\sB{{\mathcal B}}                            
\def\sD{{\mathcal D}}                            
\def\sT{{\mathcal T}}
\def\sE{{\mathcal E}}                            
\def\sF{{\mathcal F}}                            
\def\sC{{\mathcal C}}                            
\def\sO{{\mathcal O}}                            
\def\sH{{\mathcal H}} 
\def\sR{{\mathcal R}}                            
\def\sJ{{\mathcal J}}                            
\def\sCP{{\mathcal CP}}                            
\def\sX{{\mathcal X}}                            
\def\sA{{\mathcal A}} 
\def\sZ{{\mathcal Z}}                            
\def\sM{{\mathcal M}}                            
\def\sK{{\mathcal K}}     
\def\sG{{\mathcal G}}                         
\def\sY{{\mathcal Y}}                         
\def\sU{{\mathcal U}}  


\def\sIG{{\mathcal IG}}                            


\def\cD{{\sf D}}
\def\cH{{\sf H}}
\def\cI{{\sf I}}

% Vectors
\def\vectorfontone{\bf}
\def\vectorfonttwo{\boldsymbol}
\def\va{{\vectorfontone a}}                      %
\def\vb{{\vectorfontone b}}                      %
\def\vc{{\vectorfontone c}}                      %
\def\vd{{\vectorfontone d}}                      %
\def\ve{{\vectorfontone e}}                      %
\def\vf{{\vectorfontone f}}                      %
\def\vg{{\vectorfontone g}}                      %
\def\vh{{\vectorfontone h}}                      %
\def\vi{{\vectorfontone i}}                      %
\def\vj{{\vectorfontone j}}                      %
\def\vk{{\vectorfontone k}}                      %
\def\vl{{\vectorfontone l}}                      %
\def\vm{{\vectorfontone m}}                      % number of basis functions
\def\vn{{\vectorfontone n}}                      % number of training samples
\def\vo{{\vectorfontone o}}                      %
\def\vp{{\vectorfontone p}}                      % number of unpenalized coefficients
\def\vq{{\vectorfontone q}}                      % number of penalized coefficients
\def\vr{{\vectorfontone r}}                      %
\def\vs{{\vectorfontone s}}                      %
\def\vt{{\vectorfontone t}}                      %
\def\vu{{\vectorfontone u}}                      % Penalized coefficients
\def\vv{{\vectorfontone v}}                      %
\def\vw{{\vectorfontone w}}                      %
\def\vx{{\vectorfontone x}}                      % Covariates/Predictors
\def\vy{{\vectorfontone y}}                      % Targets/Labels
\def\vz{{\vectorfontone z}}                      %

\def\vone{{\vectorfontone 1}}
\def\vzero{{\vectorfontone 0}}

\def\valpha{{\vectorfonttwo \alpha}}             %
\def\vbeta{{\vectorfonttwo \beta}}               % Unpenalized coefficients
\def\vgamma{{\vectorfonttwo \gamma}}             %
\def\vdelta{{\vectorfonttwo \delta}}             %
\def\vepsilon{{\vectorfonttwo \epsilon}}         %
\def\vvarepsilon{{\vectorfonttwo \varepsilon}}   % Vector of errors
\def\vzeta{{\vectorfonttwo \zeta}}               %
\def\veta{{\vectorfonttwo \eta}}                 % Vector of natural parameters
\def\vtheta{{\vectorfonttwo \theta}}             % Vector of combined coefficients
\def\vvartheta{{\vectorfonttwo \vartheta}}       %
\def\viota{{\vectorfonttwo \iota}}               %
\def\vkappa{{\vectorfonttwo \kappa}}             %
\def\vlambda{{\vectorfonttwo \lambda}}           % Vector of smoothing parameters
\def\vmu{{\vectorfonttwo \mu}}                   % Vector of means
\def\vnu{{\vectorfonttwo \nu}}                   %
\def\vxi{{\vectorfonttwo \xi}}                   %
\def\vpi{{\vectorfonttwo \pi}}                   %
\def\vvarpi{{\vectorfonttwo \varpi}}             %
\def\vrho{{\vectorfonttwo \rho}}                 %
\def\vvarrho{{\vectorfonttwo \varrho}}           %
\def\vsigma{{\vectorfonttwo \sigma}}             %
\def\vvarsigma{{\vectorfonttwo \varsigma}}       %
\def\vtau{{\vectorfonttwo \tau}}                 %
\def\vupsilon{{\vectorfonttwo \upsilon}}         %
\def\vphi{{\vectorfonttwo \phi}}                 %
\def\vvarphi{{\vectorfonttwo \varphi}}           %
\def\vchi{{\vectorfonttwo \chi}}                 %
\def\vpsi{{\vectorfonttwo \psi}}                 %
\def\vomega{{\vectorfonttwo \omega}}             %


% Matrices
%\def\matrixfontone{\sf}
%\def\matrixfonttwo{\sf}
\def\matrixfontone{\bf}
\def\matrixfonttwo{\boldsymbol}
\def\mA{{\matrixfontone A}}                      %
\def\mB{{\matrixfontone B}}                      %
\def\mC{{\matrixfontone C}}                      % Combined Design Matrix
\def\mD{{\matrixfontone D}}                      % Penalty Matrix for \vu_J
\def\mE{{\matrixfontone E}}                      %
\def\mF{{\matrixfontone F}}                      %
\def\mG{{\matrixfontone G}}                      % Penalty Matrix for \vu
\def\mH{{\matrixfontone H}}                      %
\def\mI{{\matrixfontone I}}                      % Identity Matrix
\def\mJ{{\matrixfontone J}}                      %
\def\mK{{\matrixfontone K}}                      %
\def\mL{{\matrixfontone L}}                      % Lower bound
\def\mM{{\matrixfontone M}}                      %
\def\mN{{\matrixfontone N}}                      %
\def\mO{{\matrixfontone O}}                      %
\def\mP{{\matrixfontone P}}                      %
\def\mQ{{\matrixfontone Q}}                      %
\def\mR{{\matrixfontone R}}                      %
\def\mS{{\matrixfontone S}}                      %
\def\mT{{\matrixfontone T}}                      %
\def\mU{{\matrixfontone U}}                      % Upper bound
\def\mV{{\matrixfontone V}}                      %
\def\mW{{\matrixfontone W}}                      % Variance Matrix i.e. diag(b'')
\def\mX{{\matrixfontone X}}                      % Unpenalized Design Matrix/Nullspace Matrix
\def\mY{{\matrixfontone Y}}                      %
\def\mZ{{\matrixfontone Z}}                      % Penalized Design Matrix/Kernel Space Matrix

\def\mGamma{{\matrixfonttwo \Gamma}}             %
\def\mDelta{{\matrixfonttwo \Delta}}             %
\def\mTheta{{\matrixfonttwo \Theta}}             %
\def\mLambda{{\matrixfonttwo \Lambda}}           % Penalty Matrix for \vnu
\def\mXi{{\matrixfonttwo \Xi}}                   %
\def\mPi{{\matrixfonttwo \Pi}}                   %
\def\mSigma{{\matrixfonttwo \Sigma}}             %
\def\mUpsilon{{\matrixfonttwo \Upsilon}}         %
\def\mPhi{{\matrixfonttwo \Phi}}                 %
\def\mOmega{{\matrixfonttwo \Omega}}             %
\def\mPsi{{\matrixfonttwo \Psi}}                 %

\def\mone{{\matrixfontone 1}}
\def\mzero{{\matrixfontone 0}}

% Fields or Statistical
\def\bE{{\mathbb E}}                             % Expectation
\def\bP{{\mathbb P}}                             % Probability
\def\bR{{\mathbb R}}                             % Reals
\def\bI{{\mathbb I}}                             % Reals
\def\bV{{\mathbb V}}                             % Reals

\def\vX{{\vectorfontone X}}                      % Targets/Labels
\def\vY{{\vectorfontone Y}}                      % Targets/Labels
\def\vZ{{\vectorfontone Z}}                      %

% Other
\def\etal{{\em et al.}}
\def\ds{\displaystyle}
\def\d{\partial}
\def\diag{\text{diag}}
%\def\span{\text{span}}
\def\blockdiag{\text{blockdiag}}
\def\tr{\text{tr}}
\def\RSS{\text{RSS}}
\def\df{\text{df}}
\def\GCV{\text{GCV}}
\def\AIC{\text{AIC}}
\def\MLC{\text{MLC}}
\def\mAIC{\text{mAIC}}
\def\cAIC{\text{cAIC}}
\def\rank{\text{rank}}
\def\MASE{\text{MASE}}
\def\SMSE{\text{SASE}}
\def\sign{\text{sign}}
\def\card{\text{card}}
\def\notexp{\text{notexp}}
\def\ASE{\text{ASE}}
\def\ML{\text{ML}}
\def\nullity{\text{nullity}}

\def\logexpit{\text{logexpit}}
\def\logit{\mbox{logit}}
\def\dg{\mbox{dg}}

\def\Bern{\mbox{Bernoulli}}
\def\sBernoulli{\mbox{Bernoulli}}
\def\sGamma{\mbox{Gamma}}
\def\sInvN{\mbox{Inv}\sN}
\def\sNegBin{\sN\sB}

\def\dGamma{\mbox{Gamma}}
\def\dInvGam{\mbox{Inv}\Gamma}

\def\Cov{\mbox{Cov}}
\def\Mgf{\mbox{Mgf}}

\def\mis{{mis}} 
\def\obs{{obs}}

\def\argmax{\operatornamewithlimits{\text{argmax}}}
\def\argmin{\operatornamewithlimits{\text{argmin}}}
\def\argsup{\operatornamewithlimits{\text{argsup}}}
\def\arginf{\operatornamewithlimits{\text{arginf}}}


\def\minimize{\operatornamewithlimits{\text{minimize}}}
\def\maximize{\operatornamewithlimits{\text{maximize}}}
\def\suchthat{\text{such that}}


\def\relstack#1#2{\mathop{#1}\limits_{#2}}
\def\sfrac#1#2{{\textstyle{\frac{#1}{#2}}}}


\def\comment#1{
\vspace{0.5cm}
\noindent \begin{tabular}{|p{14cm}|}  
\hline #1 \\ 
\hline 
\end{tabular}
\vspace{0.5cm}
}


\def\mytext#1{\begin{tabular}{p{13cm}}#1\end{tabular}}
\def\mytextB#1{\begin{tabular}{p{7.5cm}}#1\end{tabular}}
\def\mytextC#1{\begin{tabular}{p{12cm}}#1\end{tabular}}

\def\jump{\vskip3mm\noindent}

\def\KL{\text{KL}}
\def\N{\text{N}}
\def\Var{\text{Var}}

\def \E {\mathbb{E}}
\def \BigO {\text{O}}
\def \IG {\text{IG}}
\def \Beta {\text{Beta}}



\begin{document}

\maketitle

\section{Chapter 1 -- Zero-Inflated Models}

Ormerod and Wand 2010

Explaining Variational Approximations

Highly cited paper explaining variational approximations using a number of examples.
Defines basic terms, and gives proofs of lower bound and optimal mean field update.

Starts by proving that the variational lower bound  bounds the true marginal log probability $\log p(\vy)$
from below. Then shows that assuming a product form for the approximating distribution $q(\vtheta)$, the
optimal approximating distribution for each $\vtheta_i$ has the form

\[
	q_i^*(\vtheta_i) \equiv \exp{\{\E_{-\vtheta_i}[\log p(\vy, \vtheta)]\}}
\]

Points out that only nodes in the Markov blanket of $\vtheta_i$ need to be considered, which helps reduce the
amount of work needed to calculate $q^*(\vtheta_i)$.

Gaussian Variational Approximations

Gives the derivations that allow the variational lower bound of a mixed effects model to be found. Approximate
mean and covariance of mixed models using multivariate normal with mean $\vmu$ and covariance matrix
$\mSigma$, using convex optimsation algorithms to optimise the variational lower bound directly.

B-Splines

Approximate functions with piecewise cubic polynomials.

The O'Sullivan splines paper shows that the integrals involved in calculating the penalty matrix using Simpson's Rule are exact, when the polynomials being integrated are cubic.

Zero-inflated and Bayesian models

Challis and Barber 2013

Gaussian Kullback-Leibler Approximate Inference

Constrained parameterisations of Gaussian covariance that make G-KL methods fast and scalable. But I'm fairly
sure they don't have our exact parameterisation. Discusses complexity. Numerical results.

This paper is worth reading closely. Our parameterisation is a mixture of the (c) Chevron and (d) Subspace
structures.

Challis and Barber 2011

Concave Gaussian Variational Approximations for Inference in Large-Scale Bayesian Linear Models

Looks at relationships between two popular approaches to forming bounds in approximate Bayesian inference
(local variational methods and minimal Kullback-Leibler divergence methods, the latter of which we use).

Also looks at the computational complexity of calculating the lower bound for various forms of
parameterisation of the covariance matrix.

Eklund and Karlsson 2007

Computational Efficiency in Bayesian Model and Variable Selection

Talks about MCMC schemes for model averaging by adding and removing single covariates or swapping covariates,
as well as more sophisticated schemes.

Also talks about the computational complexity of solving the least squares problem: how people very rarely
invert $\mX^\top \mX$, and instead work with a matrix factorisation of the form $A B$ that is easy to solve,
either the $QR$ factorisation which is considered to have good numerical properties or the Cholesky
factorisation when the matrix involved is a large $p$ by $p$ matrix.

\section{Chapter 3 -- Variational Bayes for Linear Model Selection Using Mixtures of g-Priors}

Zellner 1980

'Although Zellner and Siow 1980 did not explicitly use a $g$-prior formulation with a prior on $g$,
their recommendation of a multivariate Cauchy form for $p(\beta|\sigma^2)$ implicitly corresponds to using a
$g$-prior with an inverse Gamma prior'

Proposed Normal-Gamma conjugate model which includes $g$, with $g$ controlling the degree of shrinkage
from the fitted mean towards the prior mean.

George and Foster 2000

Proposed selecting the model maximising the posterior probability of model $\gamma$ based on empirical
Bayes estimates of $g$ and the standard unbiased estimate of $\sigma^2$

Cui and George 2008

Proposed marginning out $g$ with respect to a prior

Liang 2008

Builds on Zellner 1980, by proving that any fixed choice of $g$ leads to various kinds of paradoxes e.g.
Information Paradox, Bartlett's Paradox.
Proposed marginning out $g$ and $\sigma^2$ with respect to priors.

Full of good information, and serves as a great introduction to fitting normal linear models with a g-prior.
It discusses:
\begin{itemize}
\item $g$--priors, the simplified form of the Bayes Factor $BF[\mathcal{M}_\gamma : \mathcal{M}_b]$ and the definition of $BF[\mathcal{M}_\gamma : \mathcal{M}_{\gamma'}]$.
\item the paradoxes that arises from fixed choices of $g$.
\item the benefits of using a mixture of g--priors, analytic expressions are available for the posterior
distribution of $g$, $p(g|\vy)$ etc. etc. I should do a careful literature review of this paper.
\end{itemize}

Murayama and George 2011

Fully Bayes Factors with a Generalised $g$-Prior
Normal linear model, model selection based on a fully Bayes formulation with a generalisation of Zellner's
$g$-prior which allows for $p > n$

This is a more theoretical paper, which proposes a selection criteria based on a fully Bayesian formulation
with a generalisation of Zellner's $g$--prior, which allows for $p > n$. A special case of the prior
formulation is seen to yield tractable closed forms for marginal densities and Bayes factors which reveal new
model evaluation characteristics of potential interest.

Uses a Beta-Prime prior
\[
p(g) = \frac{g^b (1 + g)^{-a-b-2}}{\Beta(a + 1, b + 1)} \I_{(0, \infty)}(g)
\]

Makes special choices of $a = -3/4$ and $b = (n - q - 5)/2 - a$.

Model selection consistency is shown, which is an important property for model selection criteria to have. Do
we have this? I think we do, or can show it.

They then do some simulations to show that their approach works.

Bayorri, Berger, Forte and Garci\'{a} 2012

Criteria for Bayesian Model Choice with Application to Variable Selection

Has many good references for model selection in a Bayesian framework. I should probably go through all of them
to make sure I've made a thorough survey of the field:

Jeffreys (1961)
Zellner and Siox (1980, 1984)
Laud and Ibrahim (1995) Criterion-Based Methods for Bayesian Model Assessment
Kass and Wasserman (1995)
Berger and Pericchi (1996)
Moreno, Bertolino and Racugno (1998) Can't find
De Santis and Spezzaferri (1999)
Perez and Berger (2002)
Bayorri and Garci\'{a}--Donato (2008)
Liang et al. (2008)
Cui and Goerge (2008)
Maruyama and George (2008)
Maruyama and Strawderman (2010)

Laud and Ibrahim (1995) offers another criterion for Bayesian model assessment. The normal linear model case
gives a closed form. Some simulation studies.

Kass and Wasserman (1995) A Reference Bayesian Test for Nested Hypothesis and its Relationship to the
Schwarz Criterion. log of the Bayes Factor is approximately the Schwarz criterion $O_P(n^{-1/2})$. The
Schwarz Criterion is another name for the BIC.

Berger and Pericchi (1996) The Intrinsic Bayes Factor for Model Selection and Prediction
Yet another model selection criterion, with some claimed advantages. Advocates automatic methods of model 
selection, as "analysis of nonnested and/or multiple models or hypotheses is very difficult in a frequentist
framework".

De Santis and Spezzaferri (1999) Automatic and robust methods for model comparison using fractional Bayes
Factors, an alternative to Bayes Factors. Points out that Bayes Factors are very sensitive to prior
distributions. Problem in the presence of weak prior information, does not work at all in the case of improper
priors (isn't defined).

Bayorri and Garci\'{a}--Donato (2008) Divergence Based Priors for Bayesian Hypothesis Testing.

Cui and George (2008) Empirical Bayes versus fully Bayes variable selection.

Maruyama and Strawderman (2010). A New Class of Generalised Bayes Minimax Ridge Regression Estimators.

Zellner, Jeffreys-Bayes Posterior Odds Ratio and the Akaike Information Criterion for Discriminating Between
Models

\subsection{References from Liang et al 2008}
Barbieri and Berge 2004 Optimal Predictive Model Selection
Choose a model for future prediction and it is natural to measure the accuracy of a future prediction by
squared error loss. Under the Bayesian approach, it is commonly perceived that the optimal predictive model is
the model with highest probability, but this is not necessarily the case. In this paper we show that for
selection among the normal linear models, the optimal predictive model is often the median probability model,
which is defined as the model consisting of those variagbles which have overall posterior probabliity greater
than or equal to $\frac{1}{2}$ of being in a model. The median probability model often differs from the highest
probability model.

Bartlett ``A Comment on D.v. Lindley's Statistical Paradox'', Biometrika
Contradiction between hypothesis test results and posterior probability of the null hypothesis. Correcting 
Lindley.

Berger and Pericchi, 2001, ``Objective Bayesian Methods for Model Selection: Introduction and Comparison''
The basics of the Bayesian approach to model selection are first presented, as well as the motivations for the
Bayesian approach. We then review four methods of developing default Bayesian procedures that have undergone
considerable recent development, the Conventional Prior approach, the Bayes Information Criterion, the
Intrinsic Bayes Factor, and the Fractional Bayes Factor. As part of the review, these methods are illustrated
on examples involving the normal linear model. The later part of the chapter focuses on comparison of the
four approaches, and includes an extensive discussion of criteria for judging model selection procedures.

Berger, Pericchi and Varshavsky, ``Bayes Factors and Marginal Distributions in Invariant Situation''
In Bayesian analysis with a ``minimal'' data set and common noninformative priors, the (formal) marginal
density of the data is surprisingly often independent of the error distribution. This results in great
simplifications in certain model selection metholodologies; for instance, the Intrinsic Bayes Factor for
models with this property reduces simply to Bayes factor with respect to the noninformative priors. The basic
result holds for comparison of models which are invariant with respect to the same group structure. Indeed the
condition reduces to a condition on the distributions of the common maximal invariant. In these situations,
the marginal density of a ``minimal'' data set is typically available in closed form, regardless of the error
distribution. This provides very useful expressions for computation of Intrinsic Bayes Factors in more general
settings. The conditions for the results to hold are explored in some detail for nonnormal linear models and
various transformations thereof.

Breiman and Friedman, ``Estimating Optimal Transformations for Multiple Regression and Correlation''
In regression analysis the response variable $\vy$ and the predictor variables $\mX_1, \ldots, \mX_p$ are often
replaced by functions $\theta(\vy)$ and $\phi(X_1)$, \ldots, $\phi(X_p)$. We discuss a procedure for estimating
those functions $\theta^*$ and $\phi_1^*$, \ldots, $\phi_p^*$ that minimise
$e^2 = \frac{\E\{ [\theta(\vy) - \sum_{j=1}^p \phi_j (X_j)]^2 \}}{\text{var}[\theta(\vy)]}$
given on a sample $\{ (\vy_k, \vx_{k1}, \ldots, \vx_{kp}), 1 \leq k \leq N \}$ and making minimal assumptions
concerning the data distribution or the form of the solution functions. For the bivariate case, $p=1$,
$\theta^*$ and $\rho^*=\rho(\theta^*, \phi^*) = \text{max}_{\theta, \phi} \rho[\theta(\vy), \psi(\mX)]$, where
$\rho$ is the product moment correlation coefficient and $\rho^*$ is the maximum correlation between $X$ and
$Y$. Our procedure thus also provides a method for estimating the maximum correlation between two variables.

Butler and Wood (2002), ``Laplace Approximations for Hypergeometric Functions with Matrix Argument''
In this paper we present Laplace approximations for two functions of matrix argument: the Type I confluent
hypergeometric function and the Gauss hypergeometric function. Both of these functions play an important role
in distribution theory in multivariate analysis, but from a practical point of view they have proved
challenging, and they have acquired a reputation for being difficult to approximate. Appealing features of the
approximations we present are: (i) they are fully explicit (and simple to evaluate in practice) and
(ii) typically, they have excellent numerical accuracy. \ldots Relative error properties of these
approximations are also studied, and it is noted that the approximations have uniformly bounded relative
errors in important cases.

Casella and Moreno, (2006) ``Objective Bayes Variable Selection''
A novel fully automatic Bayesian procedure for variable selection in normal regression models is proposed. The
procedure uses the posterior probabilities of the models to drive a stochastic search. The posterior
probabilities are computed using intrinsic priors, which can be considered default priors for model selection
problems; that is, they are derived from the model structure and are free from tuning parameters. Thus they
can be seen as objective priors for model selection. The stochastic search is based on a Metropolis-Hastings
algorithm with a stationary distribution proportional to the model posterior probabilities. The procedure is
illustrated on both simulated and real examples.

Clyde and George, 2000, Flexible Empirical Bayes Information for Wavelets
Wavelet shrinkage estimation is an increasingly popular method for signal denoising and compression. Although
Bayes estimators can provide excellent mean-squared error (MSE) properties, the selection of an effective
prior is a difficult task. To address this problem, we propose empirical Bayes (EB) prior selection methods for
various error distributions including the normal and heavier-tailed Student's t-distributions. Under such EB
prior distributions, we obtain threshold shrinkage estimators based on model selection, and multiple-shrinkage
estimators based on model averaging. These EB estimators are seen to be computationally competitive with standard
classical thesholding methods, and to be robust to outliers in both the data and wavelet domains. Simulated and
real examples are used to illustrate the flexibility and improved performance of these methods in a wide variety
of settings.

Cui and George, 2007, Empirical Bayes versus Fully Bayes Variable Selection
For the problem of variable selection for the normal linear model, fixed penalty selection criteria such as
AIC, $C_p$, BIC and RIC correspond to the posterior modes of a hierarchical Bayes model for various fixed
hyperparameter settings. Adaptive selection criteria obtained by empirical Bayes estimation of the
hyperparameters have been shown by George and Foster [2000, Calibration and Empirical Bayes variable selection,
Biometrika] to improve on these fixed selection criteria. In this paper, we study the potential of alternative
fully Bayes methods which instead margin out the hyperparameters with respect to prior distributions. Several
structured prior formulations are considered for which fully Bayes selection and estimation methods are
obtained. Analytical and simulation comparisons with empirical Bayes counterparts are studied.

Fernandez, Ley and Steel, 2000, Benchmark priors for Bayesian model averaging
In contrast to a posterior analysis given a particular sampling model posterior model probabilities in the
context of model uncertainty are typically rather sensitive to the specification of the prior. In particular,
`diffuse' priors on model-specific parameters can lead to quite unexpected consequences. Here we focus on the
practically relevant situation where we need to entertain a (large) number of sampling models and we have (or wish to use) little or no subjective prior information. We aim at providing an `automatic' or `benchmark'
prior structure that can be used in such cases. We focus on the normal regression model with uncertainty in
the choice of regressors. We propose a partly non-informative prior structure related to a natural conjugate
$g$-prior specification, where the amount of subjective information requested from the user is limited to the
choice of a single scalar hyperparameter $g_{0j}$. The consequences of different choices for $g_{0j}$ are
examined. We investigate theoretical properties such as consistency of the implied Bayesian procedure. Links
with classical information criteria are provided. More importantly, we examine the finite sample implications
of several choices of $g_{0j}$ in a simulation study. \ldots In addition to posterior criteria, we shall also
compare the predictive performance of different priors. A classic example concerning the economics of crime
will also be provided and contrasted with results in the literature. The main findings of the paper will lead
us to propose a `benchmark' prior specification in a linear regression context with model uncertainty.

Foster and George, 1994, The Risk Inflation Criterion for Multiple Regression
A new criterion is proposed for the evaluation of variable selection procedures in multiple regression. This
criterion, which we call the risk inflation, is based on an adjustment to the risk. Essentially, the risk
inflation is the maximum increase in risk due to selecting rather than knowing the ``correct'' predictiors.
A new variable selection procedure is obtained which, in the case of orthogonal predictors, substantially
improves on AIC, C_p and BIC and is close to optimal. In contrast to AIC, C_P and BIC which use dimensionality
penalties of 2, 2 and log n respectively, this new procedure uses a penalty 2 log p, where p is the number of
available predictors. For the case of nonorthogonal predictors, bounds for the optimal penalty are obtained.

George, E (1999) Discussion of ``Model Averaging and Model Search Strategies''
2000, ``The Variable Selection Problem''
The problem of variable selection is one of the most pervasive model selection problems in statistical
applications. Often referred to as the problem of subset selection, it arises when one wants to model the
relationship between a variable of interest and a subset of potential explanatory variables or predictors but
there is uncertainty about which subset to use. This vignette reviews some of the key developments that have led 
to the wide variety of approaches to the problem.

George and Foster (2000) Calibration and Empirical Bayes Variable Selection
For the problem of variable selection for the normal linear model, selection criteria such as AIC, C_p, BIC
and RIC have fixed dimension penalties. Such criteria are shown to correspond to selection of maximum
posterior models under implicit hyperparameter choices for a particular hierarchical Bayes formulation. Based
on this calibration, we propose empirical BAyes selection criteria that use hyperparameter estimates instead
of fixed choices. For obtaining these estimates, both marginal and conditional maximum likelihood methods are
considered. As opposed to traditional fixed penalty criteria, these empirical Bayes criteria have dimensionality
penalties that depend on the data. Their performance is seen to approximate adaptively the perform of the best
fixed-penalty criterion across a variety of orthogonal and non-orthogonal set-ups, incuding wavelet regression.
Empirical Bayes shrinkage estimators of the selected coefficients are also proposed.

George and McCulloch (1993) ``Variable Selection via Gibbs Sampling''
A crucial problem in building a multiple regression model is the selection of predictors to include. The main
thrust of this article is to propose and develop a procedure that uses probabilistic considerations for selecting
promising subsets. This procedure entails embedding the regression setup in a hierarchical normal mixture model 
where latent variables are used to identify subset choices. In this framework the promising subsets of predictors
can be identified as those with higher posterior probability. The computational burden is then alleviated by
using a Gibbs sampler to indirectly sample from this multinomial posterior distribution on the set of possible
subset choices. Those subsets with higher probability -- the promising ones -- can then be identified by their
more frequent appearance in the Gibbs sample.

Geweke, J. (1996), ``Variable Selection and Model Comparison in Regression''
In the specification of linear regression models it is common to indicate a list of candidate variables from
which a subset enters the model with nonzero coefficients. This paper interprets this specification as a mixed
continuous-discrete prior distribution for coefficient values. It then utilises a Gibbs sampler to construct
posterior moments. It is shown how this method can incorporate sign-constraints and provide posterior 
probabilities for all possible subsets of regressors. The methods are illustrated using some standard data sets.

Hansen and Yu, 2001, ``Model Selection and the Principle of Minimum Description Length''
This article reviews the principle of minimum description length (MDL) for problems of model selection. By
viewing statistical modeling as a means of generating \emph{descriptions} of observed data, the MDL frameowkr
discriminates between competing models based on the \emph{complexity} of each description. This approach began
with Kolmogorov's theory of algorithmic complexity, matured in the literature on information theory, and has
recently received renewed attention within the statistics community. Here we review both the practical and the
theoretical aspects of MDL as a tool for model selection, emphasising the rich connections between information
theory and statistics. At the boundary between these two disciplines we find many interesting interpretations
of popular frequentist and Bayesian procedures. As we show, MDL provides an objective umbrella under which rather disparate approaches to statistical modelling can coexist and be compared. We illustrate the MDL principle by
considering problems in linear regression, nonparametric curve estimation, cluster analysis and time series
analysis. Because model selection in linear regression is an extremely common problem that arises in many
applications, we present details derivations of several MDL criteria in this context and discuss their properties
through a number of examples. Our emphasis is on the practical application of MDL, and hence we make extensive
use of real data sets. In writing this review, we tried to make a descriptive philosophy of MDL natural to a
statistics audience by examining classical roblems in model selection. In the engineering literature, however,
MDL is being applied to ever more exotic modelling situations. As a principle for statistical modelling in 
general, one strength of MDL is that it can be intuitively extended to provide useful tools for new problems.

Hoeting, Madigan, Raftery, Valinksy (1999) ``Bayesian Model Averaging: A Tutorial''
Standard statistical practice ignores model uncertainty. Data analysts typically select a model from some class
of models and then process as if the select model had generated the data. This approach ignores the uncertainty
in model selection, leading to overconfident inferences and decisions that are more risky than one thinks they
are. Bayesian model averaging (BMA) provides a coherent mechanism for accounting for this model uncertainty.
Several methods for implementing BMA have recently emerged. We discuss these methods and present a number of
examples. In those examples, BMA provides improved out-of-sample predictive performance. We also provide a
catalogue of currently available BMA software.

Jeffreys (1961) Theory of Probability

Johnstone and Silverman (2005) ``Empirical Bayes Selection of Wavelet Thresholds''
Not very relevant

Kass, Raftery (1995) Bayes Factors
Bayes factors are very general and do not required alternative models to be nested
The Schwarz criterion (or BIC) gives a rough approximation to the log of the Bayes factor, which is easy
to use and does not require evaluation of prior distributions.
Bayes factors can be converted into weights to be attached to various models \ldots takes account of model
uncertainty

Leamer, E. E. (1978a) ``Regression Selection Strategies and Revealed Priors''
The computation and selection of constrained regressions may be motivated by prior information and, if so,
a regression selection strategy reveals the implicit prior. The selection strategies of principal component
regression, stepwise regression, and imposing equality constraints are connected with prior densities which are
uniform on spheres, hyperbolas, and cones respectively. Omitting variables in a predetermined order reveals
lexicographic priors.

Miller, A. J. (2001) Subsets Selection in Regression
Fitting and choosing models that are linear in their parameters and to understanding and correcting the bias
introduced by selecting a model that fits only slightly better than the others.

Mitchell and Beauchamp (1988) ``Bayesian Variable Selection in Linear Regression''
Complicated, not fully Bayesian

Pauler, Wakefield and Kass (1999) ``Bayes Factors and Approximations for Variance Component Models''
Tests of variance components

Raftery, Madigan and Hoeting (1997) ``Bayesian Model Averaging for Linear Regression Models''
Conditioning on a single selected model ignores model uncertainty and thus leads to the underestimation of
uncertainty when making inferences. A Bayesian solution to this problem involves averaging over all possible
models, but this is often not practical. First, we describe an ad hoc procedure, ``Occam's Window'', which
indicates a small set of models over which a model average can be computed. Second, we describe a Markov chain
method

Zellner, A (1971) ``An Introduction to Bayesian Inference in Econometrics''

(1986) ``On Assessing Prior Distributions and Bayesian Regression Analysis With g-prior Distributions''

Zellner, A. and Min, C. (1997) ``Bayesian Analysis, Model Selection and Prediction''

Zellner, A. and Siow, A. (1980) ``Posterior Odds Ratios for Selected Regression Hypotheses''

\subsection{References from Murayama and George}
Akaike, H. (1974) ``A new look at the statistical model identification''
The history of the development of statistical hypothesis testing in time series is reviewed briefly and it is
pointed out that the hypothesis testing procedure is not adequately defined as the procedure for statistical
identification. The classical maximum likelihood estimation procedure is reviewed and a new estimate minimum
information theoretical criterion (AIC) estimate (MAICE) which is designed for the purpose of statistical
information is introduced. When there are several completing models the MAICE is deinfed by the model and the
maximum likelihood estimates of the parameters which give the minimum of AIC defined by

\[
	\text{AIC} = -2 ( \text{log likelihood} ) + 2 (\text{number of independently adjusted parameters within the model})
\]

Casella, G. (1980) ``Minimax ridge regression estimation''
The technique of ridge regression, first proposed by Hoerl and Kennard, has become a poopular tool for data
analysts faced with a high degree of multicollinearity in their data. By using a ridge estimator, one hopes to
both stabilise one's estimates (lower the condition number of the matrix) and improve upon the squared error
loss of the least squares estimator.

Recently, much attention has been focused on the latter objective. Building on the work of Stein and others,
Strawderman and Thisted have developed classes of ridge regression estimators which dominate the usual estimator
of risk, and hence are minimax. The unwieldy form of the risk function, however, has led these authors to
minimise conditions which are stronger than needed.

In this paper, using an entirely new method of proof, we derive the conditions that are necessary and sufficient
for minimaxity of a large class of ridge regression estimators. The conditions derived here are very similiar
to those derived for minimaxity of some Stein-style estimators.

We also show, however, that if one forces a ridge regression estimator to satisfy the minimax conditions, it is
quite likely that the other goal of Hoerl and Kennard (stability of estimates) cannot be realised.

Casella, G. (1985) ``Condition numbers and minimax regression estimators''
Ridge regression was originally formulated with two goals in mind: improvement in mean squared error and numerical
stability of the coefficients estimates. Conditions are given under which a minimax ridge regression estimator
can also improve numerical stability, a quantity that can be measured with the condition number of the matrix
to be inverted. The consequences of trading numerical stability for minimaxity are also discussed.

Geluk, J. L. and De Haan, L. (1987) ``Regular variatioan, extensions and Tauberian theorems''

Hurwich, C. M. and Tsai, C.-L. (1989) ``Regression and time series model selection in small samples''
A bias correction to the Akaike information criterion, AIC, is derived for regression and autoregressive time
series models. The connection is of particular use when the sample size is small, or when the number of fitted
parameters is a moderate to large fraction of the sample size. The corrected method, called AICC, is 
asymptotically efficient if the true model is infinite dimensional. Furthermore, when the true model is of
finite dimension, AICC is found to provide better model order choices than any other asymptotically efficient
method. Applications to nonstationary autoregressive and mixed autoregressive moving average time series models
are also discussed.

Knight, K. and F, W. (2000) ``Assymptotics for lasso-type estimators''
We consider the asymptotics behaviour of regression estimators that minimise the residual sum of squares plus
a penalty proportional to $\sum_j \|\beta_j\|^\lambda$ for some $\lambda > 0$. These estimators include
the lasso as a special case when $\lambda = 1$. Under appropriate conditions, we show that the limiting 
distributions can have positive probability mass at $0$ when the true valueof the parameter is $0$. We also 
consider asymptotics for ``nearly singular'' designs.

\end{document}