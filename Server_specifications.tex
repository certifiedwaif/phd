\documentclass{amsart}
\author{Mark Greenaway}
\title{Server specifications}
\begin{document}
\section{Budget}
\$15k to \$25k.

\section{CPU}
We can save some money here - four cores should be fine. There
are many other machines within the department which have fast
CPUs. If we need more processing power, we can use those.

\section{Memory}
A minimum of 128 GB of RAM, of the fastest type that we can buy.
256 GB of RAM would be better, but it's not essential.

\section{Hard disks}
At least $12 \times 3$ terabyte hard disks, for a total of 36 terabytes of storage.
There is a trade-off between rotational speed and storage capacity. From my
research, I think the best current trade-off for us is 10,000 RPM hard disks.

This is the most important thing, due to the large volumes of data that we
routinely process. We should focus the majority of our budget on
getting the fastest, best hard disks that we can. The entire point of
purchasing this server is to increase our storage capacity, and the
speed with which we can process the data on that storage.

\section{RAID}
We have at least two options here:

\begin{enumerate}
\item A hardware RAID controller that allows us to do striping so that
we can get good I/O performance.
\item Software RAID, which the Linux kernel supports.
\end{enumerate}

\section{Network adapter}
1 gigabit Ethernet is standard now.

10 gigabit Ethernet would be better, but at present we would have no
other 10 gigabit Ethernet devices (switches, other servers etc.) to
connect it to. This should probably be a future upgrade.

\section{Actually, scratch all of that \ldots}
I met with Ellis, Kashuala and Dario. They liked the idea of Directly Attached
Storage. Cheaper, simpler and it solves current our problem.

\end{document}