\documentclass{amsart}
\author{Mark Greenaway}
\title{Bioinformatics research group new server specification}
\begin{document}
\maketitle

\section{Budget}
\$15k to \$25k.

\section{CPU}
At least 16 or 32 cores in a server seems standard now. Intel processors. The supplier
can probably help us optimise how to get the best price to performance ratio, but we
shouldn't go below 2.4 GHz per core.

\section{Memory}
At least 128 GB of RAM. 256 GB would be better. 1333 MHz or 1600 MHz server RAM.

\section{Hard disks}
At least $12 \times 3$ terabyte hard disks, for a total of 36 terabytes of storage.
There is a trade-off between rotational speed and storage capacity. We can probably talk to 
the supplier about which mix of disks gives us the best performance/price ratio. But we 
should be getting at least this much storage.

As long as we can meet our storage requirements, the next thing to optimise for is the
fastest disks that we can get. This is probably the most important thing in the
specification, as our code currently spends a lot of time waiting for I/O.

\section{Disk controller}
We should also make sure that whatever disk controller is part of the server, it is able
to keep up with all of the drives that we buy.

\section{RAID}
We need either hardware RAID, or to set RAID up in software. The Linux kernel supports
RAID in software.  We should use RAID-0 (striping data across all disks). A single hard disk 
doesn't have a very  high transfer rate relative to the amounts of data that we have to 
process, but using RAID-0 or something similiar we can be using all of our hard disks in 
parallel. This will speed I/O up considerably.

\section{Network interface}
1 gigabit Ethernet is now the entry level standard. 10 gigabit Ethernet would be good to
future proof us, but we don't currently have any other 10 gigabit Ethernet switches or 
machines to talk to, so this would be wasted at present.

\end{document}