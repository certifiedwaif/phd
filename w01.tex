\documentclass[11pt,a4paper]{article}
\usepackage{amsbsy, graphics, amsmath, amssymb, color, mdwlist}


\parskip=5pt
\renewcommand{\baselinestretch}{1}
\parindent=0pt
\textwidth=6.25in
\oddsidemargin=0pt
\evensidemargin=0pt
\textheight=10in
\topmargin=-0.75in
\baselineskip=11pt


% coloring and subsectioning
\definecolor{mycol}{rgb}{0.1, 0.1,0.4}
\definecolor{mycolb}{rgb}{0.7, 0.2,0.4}
\definecolor{mycolr}{rgb}{0.25, 0.25,0.25}
\definecolor{mycolc}{rgb}{0.2, 0.4,0.65} % {1,1,1} to out-comment 
\definecolor{mycold}{rgb}{0.3, 0.5,0.7} % {1,1,1} to out-comment 
\newcommand{\structure}[1]{{\color{mycolb}#1}}
\newcommand{\rcode}[1]{{\tt\color{mycolr}#1}}
\newcommand{\phead}[1]{{\bf{\color{mycol}#1}}}


\begin{document}



\begin{center}
\rule{\textwidth}{1pt}

\vspace{10pt}
{\Large STAT 3012 Applied Linear Models -- Semester 1, 2016\\ \bigskip
 {\bf Tutorial \& Lab Sheet Week 01 }}

\rule{\textwidth}{1pt}
\end{center}


\phead{Tutorial Problems for Week 01}

There is no tutorial in week 01.

\bigskip
\phead{Computer Problems for Week 01}

Problems 4 and 5 should be done during the lab. The following is some basic information to login, download datasets and to create reports using Sweave. Anyone who prefers to use R markdown, knitr are any other similar tool is encouraged to do so. Problems 1 to 3 are for self-study to refresh assumed knowledge and should be done before the computer lab.

\vspace{-2mm}\begin{description}
\item[Login] Use your unikey and unikey password.
\item[Datasets] Store datasets in a folder of your choice (e.g. create a folder {data} in your home directory $\sim/$). Load datasets as a data frame in R by
\begin{verbatim}
DAT = read.table(file="~/data/filename.txt",header=TRUE)
\end{verbatim}
or point directly to the file using 
\begin{verbatim}
DAT = read.table(url("http://www.PATH/filename.txt"),header=TRUE)
\end{verbatim}
(This only works for files that are not behind a firewall).
\item[Work with an R Script] Before starting to write reports, work with R Scripts (end in {.R}), through {File : New : R Script}. You can send single or multiple lines to the R Console.
\item[Create OpenOffice reports] Copy and paste code chunk and output. For graphs export the figures or copy and paste it from the Clipboard.
\item[Create pdf reports using knit as in STAT2x11 and STAT2x12] In case you learned how to create pdf reports using knitr you can of course continue using this over Sweave.

The only significant difference between knitr and Sweave for the students are the code for graphics chunks. Sweave requires 
\begin{verbatim}
<<fig=TRUE>>= ....@
\end{verbatim}
and the graphic chunks for knitr are the same as non-graphics chunks  ie. 
\begin{verbatim}
 <<>>= .... @
\end{verbatim}

Attention: In this course there will be only one command that is not going
to work when using Sweave, which is the R command \texttt{setwd}.

\item[Create pdf reports using Sweave] Create a {.Rnw} file through {File : New : R Sweave} and store it in a folder of your choice. These files have the following basic structure:

\begin{verbatim}

Heading

Text (using LaTeX) followed by code chunks (using R). 

<<>>=
z=rnorm(10)
summary(z)
@

more text followed by graph chunks, one each for each figure:

\begin{figure}
\begin{center}
<<fig=TRUE>>=
hist(z)
@
\caption{Your caption, e.g. Histogram of $z$}
\label{figure xyz}
\end{center}
\end{figure}

This allows you to automatically refer to Figure \ref{figure xyz}
\end{verbatim}

To create a `nice' pdf click on {Compile PDF}

Attention: In this course there will be only one command that is not going
to work when using Sweave, which is the R command \texttt{setwd}.

\end{description}

\begin{enumerate}
% New Question
\item (Assumed knowledge) Use R to find the following probabilities
\begin{enumerate}
\item $P(Z > - 0.785), \ Z \sim N(0,1)$, with \texttt{pnorm(q, lower.tail = FALSE)}.
\item $P(t_2 \geq -1.26)$, with \texttt{pt(q,df,lower.tail=FALSE)}.
\item $P(\chi_4^2 < 4.7)$, with \texttt{pchisq(q,df)}. 
\item $P(|t_{9}| > 1.85)$, with \texttt{pf(q**2,df1,df2,lower.tail=FALSE)} after thinking about how the $t$ and the $F$ distribution relate to each other.
\end{enumerate}

% New Question
\bigskip
\item (Assumed knowledge) Use {qnorm}, {qt}, {qchisq}, and {qf} to find \ $c$ \ in the following
\begin{enumerate}
\item $P(t_4 \geq c) = .995$ with \texttt{qt},
\item $P(|Z| \leq c) = 1/11$ with both, \texttt{qnorm} and \texttt{qchisq},
\item $P(F_{3,12} \leq c) = .90$ with \texttt{qf}.
\end{enumerate}

% New Question
\bigskip
\item (Assumed knowledge) A machine produces metal pieces that are cylindrical in shape.  A sample of 8 pieces
is taken and the diameters are $$ 1.01, \ 0.97, \ 0.39 \ 1.03, \ 1.04, \ 0.99, \ 0.98, \ 0.99. $$
\begin{enumerate}
\item  Construct a box plot representation of this data set. (With \texttt{x=c(1.01,..., 0.99)} and 
\texttt{boxplot()}).
\item   Estimate the average diameter, $\mu$, produced by
the machine. Estimate the standard error ($=s/\sqrt{n}$) of your estimate? (\texttt{mean(x)}, $s=$ \texttt{sd()}, and $n=$ \texttt{length(x)}). 
\item  Assuming that the diameter can be modelled by a normal distribution,
calculate a 98\% confidence interval for $\mu$. (With \texttt{t.test(x,mu=1,conf.level=0.98)}, gives you the solution for (d) as well).
\item  Would you reject the hypothesis \ $H_0 : \mu = 1.00$ \ at significance level
\ $\alpha = .02$ \ on the basis of these data?
%\item Is there an outlier or error in the data? If yes, what can you do about it? Would the statistical conclusion change?
\end{enumerate}

% New Question
\item Analyse lengths of time of passages of play data from ten international
rugby matches involving the `All Blacks'. This as all other course data is available from\\ \texttt{http://www.maths.usyd.edu.au/u/mueller/stat3012/loc/datasets/a-index.html}. This exercise helps you to digest part of Lecture 2 and to revisit assumed knowledge on R and graphical displays.

\begin{enumerate}
\item Download the data and read it into R, storing them as a data frame \texttt{rugby}. A possible solution is to save the file in a \texttt{data} folder, followed by running
{\color{mycolr}
\begin{verbatim}
rugby = read.table(file="~/data/rugby.txt",header=TRUE)
\end{verbatim}}


\item Look at the data frame by simply typing its name, \texttt{rugby}. You 
should see that the data frame has two columns. Scroll up to see that these columns are headed \texttt{Game} and \texttt{Time}
respectively. (These headings were read in from the text file, \texttt{rugby.txt};
R was alerted to the presence of these headings by the \texttt{header=T} syntax in
the \texttt{read.table} command.) The variable \texttt{Game} identifies the match 
(labelled A, B, $\ldots$, K) and the variable \texttt{Time} contains the times of 
passages of play, in seconds.

\item In reports it is often preferable to only show the first couple of lines in a data frame. Try the following: \texttt{rugby[1:3,]}, \texttt{head(rugby)} or \texttt{head(rugby,2)}.

\item Type in \texttt{Game}. You should get an error. Attach the data frame with 
\begin{verbatim}
attach(rugby)
\end{verbatim}  
then try typing \texttt{Game}. You should get some output now that the data frame
is attached (so that R can `see' the variables inside).

\item The variable type for \texttt{Game} is categorical (or factor as synonym). You get frequencies for each category by 
\texttt{table(Game)}.
Which game had the most separate passages of play? Which had the least?

\item We can display the data using a bar plot. Type
\texttt{barplot(table(Game))}

There are various ways of beautifying this plot. 
You can add axis names, 
change colours and so on. (The flexibility and quality of graphical display
in R is one of the attractions of the package.) 

Try for example
\begin{verbatim}
barplot(table(Game),col="cyan",xlab="Game",ylab="Frequency")
\end{verbatim} 

You can use the help function to learn more -- try
{help(barplot)}.

\item The passage of play time is a continuous numerical variable. Try
displaying it using a histogram: \texttt{hist(Time)}.
Guess how you might add colour to the plot (try \texttt{help(hist)} if necessary.) Is 
the distribution of \texttt{Time} normal? If not, have you seen any other data sets with similarly shaped histograms?

\item Finally, we can look at the times broken down by individual match. Type
\begin{verbatim}
Time[Game=="A"]
\end{verbatim}  
That gives you just the passage times for match A. Try producing histograms
of the passage times for both game A and game H.

\item After the lab and as a preparation to handing in lab reports, produce a brief report using either OpenOffice or Sweave. Look at the files \texttt{w01s.R} and \texttt{w01s.Rnw} (once they are released) in case you have difficulties.
\end{enumerate}

% New Question
\bigskip
\item Produce a \texttt{list()} called \texttt{STAB}, consisting of two matrices reproducing Table 1 and Table 2 of 
\begin{center}
\texttt{http://www.maths.usyd.edu.au/u/mueller/stat3012/loc/tables.pdf}. \end{center}
Relabel row and column names to give the value for $z$ and $p$. This exercise helps you to digest part of Lecture 2 and to revisit assumed knowledge on the generation of statistical tables.

\end{enumerate}



\rule{\textwidth}{1pt}
\hfill last adjustments: \today \; by SM

\end{document}


