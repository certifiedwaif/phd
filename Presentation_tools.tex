\documentclass{beamer}

\usetheme{Warsaw}
\usepackage{graphicx}
\usepackage{ulem}
% include.tex
\newcommand{\Bernoulli}[1]{\text{Bernoulli} \left( #1 \right)}
\newcommand{\mydigamma}[1]{\psi \left( #1 \right)}
%\newcommand{\diag}[1]{\text{diag}\left( #1 \right)}
\newcommand{\tr}[1]{\text{tr}\left( #1 \right)}
\newcommand{\Poisson}[1]{\text{Poisson} \left( #1 \right)}
\def \half {\frac{1}{2}}
\def \R {\mathbb{R}}
\def \vbeta {\vec{\beta}}
\def \vy {\vec{y}}
\def \vmu {\vec{\mu}}
\def \vmuqbeta {\vmu_{q(\vbeta)}}
\def \vmubeta {\vmu_{\vbeta}}
\def \Sigmaqbeta {\Sigma_{q(\vbeta)}}
\def \Sigmabeta {\Sigma_{\vbeta}}
\def \va {\vec{a}}
\def \vtheta {\vec{\theta}}
\def \mX {\vec{X}}

\def\ds{{\displaystyle}}

\def\diag{{\mbox{diag}}}


\usepackage{latexsym,amssymb,amsmath,amsfonts}
%\usepackage{tabularx}
\usepackage{theorem}
\usepackage{verbatim,array,multicol,palatino}
\usepackage{graphicx}
\usepackage{graphics}
\usepackage{fancyhdr}
\usepackage{algorithm,algorithmic}
\usepackage{url}
%\usepackage[all]{xy}



\def\approxdist{\stackrel{{\tiny \mbox{approx.}}}{\sim}}
\def\smhalf{\textstyle{\frac{1}{2}}}
\def\vxnew{\vx_{\mbox{{\tiny new}}}}
\def\bib{\vskip12pt\par\noindent\hangindent=1 true cm\hangafter=1}
\def\jump{\vskip3mm\noindent}
\def\etal{{\em et al.}}
\def\etahat{{\widehat\eta}}
\def\thick#1{\hbox{\rlap{$#1$}\kern0.25pt\rlap{$#1$}\kern0.25pt$#1$}}
\def\smbbeta{{\thick{\scriptstyle{\beta}}}}
\def\smbtheta{{\thick{\scriptstyle{\theta}}}}
\def\smbu{{\thick{\scriptstyle{\rm u}}}}
\def\smbzero{{\thick{\scriptstyle{0}}}}
\def\boxit#1{\begin{center}\fbox{#1}\end{center}}
\def\lboxit#1{\vbox{\hrule\hbox{\vrule\kern6pt
      \vbox{\kern6pt#1\kern6pt}\kern6pt\vrule}\hrule}}
\def\thickboxit#1{\vbox{{\hrule height 1mm}\hbox{{\vrule width 1mm}\kern6pt
          \vbox{\kern6pt#1\kern6pt}\kern6pt{\vrule width 1mm}}
               {\hrule height 1mm}}}


%\sloppy
%\usepackage{geometry}
%\geometry{verbose,a4paper,tmargin=20mm,bmargin=20mm,lmargin=40mm,rmargin=20mm}


%%%%%%%%%%%%%%%%%%%%%%%%%%%%%%%%%%%%%%%%%%%%%%%%%%%%%%%%%%%%%%%%%%%%%%%%%%%%%%%%
%
% Some convenience definitions
%
% \bf      -> vector
% \sf      -> matrix
% \mathcal -> sets or statistical
% \mathbb  -> fields or statistical
%
%%%%%%%%%%%%%%%%%%%%%%%%%%%%%%%%%%%%%%%%%%%%%%%%%%%%%%%%%%%%%%%%%%%%%%%%%%%%%%%%

% Sets or statistical values
\def\sI{{\mathcal I}}                            % Current Index set
\def\sJ{{\mathcal J}}                            % Select Index set
\def\sL{{\mathcal L}}                            % Likelihood
\def\sl{{\ell}}                                  % Log-likelihood
\def\sN{{\mathcal N}}                            
\def\sS{{\mathcal S}}                            
\def\sP{{\mathcal P}}                            
\def\sQ{{\mathcal Q}}                            
\def\sB{{\mathcal B}}                            
\def\sD{{\mathcal D}}                            
\def\sT{{\mathcal T}}
\def\sE{{\mathcal E}}                            
\def\sF{{\mathcal F}}                            
\def\sC{{\mathcal C}}                            
\def\sO{{\mathcal O}}                            
\def\sH{{\mathcal H}} 
\def\sR{{\mathcal R}}                            
\def\sJ{{\mathcal J}}                            
\def\sCP{{\mathcal CP}}                            
\def\sX{{\mathcal X}}                            
\def\sA{{\mathcal A}} 
\def\sZ{{\mathcal Z}}                            
\def\sM{{\mathcal M}}                            
\def\sK{{\mathcal K}}     
\def\sG{{\mathcal G}}                         
\def\sY{{\mathcal Y}}                         
\def\sU{{\mathcal U}}  


\def\sIG{{\mathcal IG}}                            


\def\cD{{\sf D}}
\def\cH{{\sf H}}
\def\cI{{\sf I}}

% Vectors
\def\vectorfontone{\bf}
\def\vectorfonttwo{\boldsymbol}
\def\va{{\vectorfontone a}}                      %
\def\vb{{\vectorfontone b}}                      %
\def\vc{{\vectorfontone c}}                      %
\def\vd{{\vectorfontone d}}                      %
\def\ve{{\vectorfontone e}}                      %
\def\vf{{\vectorfontone f}}                      %
\def\vg{{\vectorfontone g}}                      %
\def\vh{{\vectorfontone h}}                      %
\def\vi{{\vectorfontone i}}                      %
\def\vj{{\vectorfontone j}}                      %
\def\vk{{\vectorfontone k}}                      %
\def\vl{{\vectorfontone l}}                      %
\def\vm{{\vectorfontone m}}                      % number of basis functions
\def\vn{{\vectorfontone n}}                      % number of training samples
\def\vo{{\vectorfontone o}}                      %
\def\vp{{\vectorfontone p}}                      % number of unpenalized coefficients
\def\vq{{\vectorfontone q}}                      % number of penalized coefficients
\def\vr{{\vectorfontone r}}                      %
\def\vs{{\vectorfontone s}}                      %
\def\vt{{\vectorfontone t}}                      %
\def\vu{{\vectorfontone u}}                      % Penalized coefficients
\def\vv{{\vectorfontone v}}                      %
\def\vw{{\vectorfontone w}}                      %
\def\vx{{\vectorfontone x}}                      % Covariates/Predictors
\def\vy{{\vectorfontone y}}                      % Targets/Labels
\def\vz{{\vectorfontone z}}                      %

\def\vone{{\vectorfontone 1}}
\def\vzero{{\vectorfontone 0}}

\def\valpha{{\vectorfonttwo \alpha}}             %
\def\vbeta{{\vectorfonttwo \beta}}               % Unpenalized coefficients
\def\vgamma{{\vectorfonttwo \gamma}}             %
\def\vdelta{{\vectorfonttwo \delta}}             %
\def\vepsilon{{\vectorfonttwo \epsilon}}         %
\def\vvarepsilon{{\vectorfonttwo \varepsilon}}   % Vector of errors
\def\vzeta{{\vectorfonttwo \zeta}}               %
\def\veta{{\vectorfonttwo \eta}}                 % Vector of natural parameters
\def\vtheta{{\vectorfonttwo \theta}}             % Vector of combined coefficients
\def\vvartheta{{\vectorfonttwo \vartheta}}       %
\def\viota{{\vectorfonttwo \iota}}               %
\def\vkappa{{\vectorfonttwo \kappa}}             %
\def\vlambda{{\vectorfonttwo \lambda}}           % Vector of smoothing parameters
\def\vmu{{\vectorfonttwo \mu}}                   % Vector of means
\def\vnu{{\vectorfonttwo \nu}}                   %
\def\vxi{{\vectorfonttwo \xi}}                   %
\def\vpi{{\vectorfonttwo \pi}}                   %
\def\vvarpi{{\vectorfonttwo \varpi}}             %
\def\vrho{{\vectorfonttwo \rho}}                 %
\def\vvarrho{{\vectorfonttwo \varrho}}           %
\def\vsigma{{\vectorfonttwo \sigma}}             %
\def\vvarsigma{{\vectorfonttwo \varsigma}}       %
\def\vtau{{\vectorfonttwo \tau}}                 %
\def\vupsilon{{\vectorfonttwo \upsilon}}         %
\def\vphi{{\vectorfonttwo \phi}}                 %
\def\vvarphi{{\vectorfonttwo \varphi}}           %
\def\vchi{{\vectorfonttwo \chi}}                 %
\def\vpsi{{\vectorfonttwo \psi}}                 %
\def\vomega{{\vectorfonttwo \omega}}             %


% Matrices
%\def\matrixfontone{\sf}
%\def\matrixfonttwo{\sf}
\def\matrixfontone{\bf}
\def\matrixfonttwo{\boldsymbol}
\def\mA{{\matrixfontone A}}                      %
\def\mB{{\matrixfontone B}}                      %
\def\mC{{\matrixfontone C}}                      % Combined Design Matrix
\def\mD{{\matrixfontone D}}                      % Penalty Matrix for \vu_J
\def\mE{{\matrixfontone E}}                      %
\def\mF{{\matrixfontone F}}                      %
\def\mG{{\matrixfontone G}}                      % Penalty Matrix for \vu
\def\mH{{\matrixfontone H}}                      %
\def\mI{{\matrixfontone I}}                      % Identity Matrix
\def\mJ{{\matrixfontone J}}                      %
\def\mK{{\matrixfontone K}}                      %
\def\mL{{\matrixfontone L}}                      % Lower bound
\def\mM{{\matrixfontone M}}                      %
\def\mN{{\matrixfontone N}}                      %
\def\mO{{\matrixfontone O}}                      %
\def\mP{{\matrixfontone P}}                      %
\def\mQ{{\matrixfontone Q}}                      %
\def\mR{{\matrixfontone R}}                      %
\def\mS{{\matrixfontone S}}                      %
\def\mT{{\matrixfontone T}}                      %
\def\mU{{\matrixfontone U}}                      % Upper bound
\def\mV{{\matrixfontone V}}                      %
\def\mW{{\matrixfontone W}}                      % Variance Matrix i.e. diag(b'')
\def\mX{{\matrixfontone X}}                      % Unpenalized Design Matrix/Nullspace Matrix
\def\mY{{\matrixfontone Y}}                      %
\def\mZ{{\matrixfontone Z}}                      % Penalized Design Matrix/Kernel Space Matrix

\def\mGamma{{\matrixfonttwo \Gamma}}             %
\def\mDelta{{\matrixfonttwo \Delta}}             %
\def\mTheta{{\matrixfonttwo \Theta}}             %
\def\mLambda{{\matrixfonttwo \Lambda}}           % Penalty Matrix for \vnu
\def\mXi{{\matrixfonttwo \Xi}}                   %
\def\mPi{{\matrixfonttwo \Pi}}                   %
\def\mSigma{{\matrixfonttwo \Sigma}}             %
\def\mUpsilon{{\matrixfonttwo \Upsilon}}         %
\def\mPhi{{\matrixfonttwo \Phi}}                 %
\def\mOmega{{\matrixfonttwo \Omega}}             %
\def\mPsi{{\matrixfonttwo \Psi}}                 %

\def\mone{{\matrixfontone 1}}
\def\mzero{{\matrixfontone 0}}

% Fields or Statistical
\def\bE{{\mathbb E}}                             % Expectation
\def\bP{{\mathbb P}}                             % Probability
\def\bR{{\mathbb R}}                             % Reals
\def\bI{{\mathbb I}}                             % Reals
\def\bV{{\mathbb V}}                             % Reals

\def\vX{{\vectorfontone X}}                      % Targets/Labels
\def\vY{{\vectorfontone Y}}                      % Targets/Labels
\def\vZ{{\vectorfontone Z}}                      %

% Other
\def\etal{{\em et al.}}
\def\ds{\displaystyle}
\def\d{\partial}
\def\diag{\text{diag}}
%\def\span{\text{span}}
\def\blockdiag{\text{blockdiag}}
\def\tr{\text{tr}}
\def\RSS{\text{RSS}}
\def\df{\text{df}}
\def\GCV{\text{GCV}}
\def\AIC{\text{AIC}}
\def\MLC{\text{MLC}}
\def\mAIC{\text{mAIC}}
\def\cAIC{\text{cAIC}}
\def\rank{\text{rank}}
\def\MASE{\text{MASE}}
\def\SMSE{\text{SASE}}
\def\sign{\text{sign}}
\def\card{\text{card}}
\def\notexp{\text{notexp}}
\def\ASE{\text{ASE}}
\def\ML{\text{ML}}
\def\nullity{\text{nullity}}

\def\logexpit{\text{logexpit}}
\def\logit{\mbox{logit}}
\def\dg{\mbox{dg}}

\def\Bern{\mbox{Bernoulli}}
\def\sBernoulli{\mbox{Bernoulli}}
\def\sGamma{\mbox{Gamma}}
\def\sInvN{\mbox{Inv}\sN}
\def\sNegBin{\sN\sB}

\def\dGamma{\mbox{Gamma}}
\def\dInvGam{\mbox{Inv}\Gamma}

\def\Cov{\mbox{Cov}}
\def\Mgf{\mbox{Mgf}}

\def\mis{{mis}} 
\def\obs{{obs}}

\def\argmax{\operatornamewithlimits{\text{argmax}}}
\def\argmin{\operatornamewithlimits{\text{argmin}}}
\def\argsup{\operatornamewithlimits{\text{argsup}}}
\def\arginf{\operatornamewithlimits{\text{arginf}}}


\def\minimize{\operatornamewithlimits{\text{minimize}}}
\def\maximize{\operatornamewithlimits{\text{maximize}}}
\def\suchthat{\text{such that}}


\def\relstack#1#2{\mathop{#1}\limits_{#2}}
\def\sfrac#1#2{{\textstyle{\frac{#1}{#2}}}}


\def\comment#1{
\vspace{0.5cm}
\noindent \begin{tabular}{|p{14cm}|}  
\hline #1 \\ 
\hline 
\end{tabular}
\vspace{0.5cm}
}


\def\mytext#1{\begin{tabular}{p{13cm}}#1\end{tabular}}
\def\mytextB#1{\begin{tabular}{p{7.5cm}}#1\end{tabular}}
\def\mytextC#1{\begin{tabular}{p{12cm}}#1\end{tabular}}

\def\jump{\vskip3mm\noindent}

\def\KL{\text{KL}}
\def\N{\text{N}}
\def\Var{\text{Var}}

\def \E {\mathbb{E}}
\def \BigO {\text{O}}
\def \IG {\text{IG}}
\def \Beta {\text{Beta}}



\usefonttheme{serif}

\title{Recent developments in the R ecosystem}
\author{Mark Greenaway\\PhD candidate\\markg@maths.usyd.edu.au}

\mode<presentation>
{ \usetheme{boxes} }

\begin{document}
\begin{frame}
\titlepage
\end{frame}

\begin{frame}
\frametitle{About me}
\begin{itemize}
\item I did computer science first
\item Then I got bored and switched to maths/statistics
\item Now I'm somewhere in the intersection. I've spent a little bit of time working
			as a data scientist, whatever that means
\item Still have some ties to the tech industry
\item Computer scientists and statisticians generally care about different things
\item It's been interesting seeing these worlds converge, for better or for worse
\item I've tried to focus on things that are generally useful, and keep the techno-babble to a
			minimum
\end{itemize}
\end{frame}

\begin{frame}
\frametitle{It's an R world \ldots}
\begin{itemize}
\item There's been an explosion of interest in data science in the last few years, which has 
			lead to a lot of people from the tech. industry coming into the community
\item The data science world has basically settled on two main languages: R and Python
\item R is also an important programming language in Bioinformatics
\item Most statisticians would rather not learn another language
\item The good news is you don't have to: R is recognised as a first-class citizen in the data
			science world, and everything is expected to integrate with it
\item R's approach has been so successful that other languages have begun to copy it e.g. 					data frames, which is a clear signal that we're doing something right
\item Other languages are a \emph{long} way behind us
\end{itemize}
\end{frame}

\begin{frame}
\frametitle{Why R won}
\begin{itemize}
\item Designed for interactive data analysis
\item De facto standard in statistical research
\item Designed by statisticians, for statisticians. Different mindset to
			general programming
\item Good data manipulation/cleaning, data frames, vector oriented
\item Good graphing: base graphics, lattice, ggplot
\item Formula syntax
\item We understand the importance of missing data: NA/NULL
\item Virtually every modelling approach is implemented in R. New ones are relatively easy
			to add, building on existing work
\item Can build on the existing work from the R community, and several thousand
			CRAN packages
\item Other languages are still catching up, although Python is rapidly gaining
			ground. They're essentially re-inventing R. Meanwhile, we're forging ahead \ldots
\end{itemize}
\end{frame}

% \begin{frame}
% \frametitle{Text editors/Integrated Development Environments}
% \begin{itemize}
% \item Most of us use RStudio
% \item Lots of other options depending on your needs: Emacs, Vim, Sublime
% \item All have syntax highlighting, ability to run R commands, navigate R functions, code
% 			folding etc.
% \end{itemize}
% \end{frame}

\begin{frame}
\frametitle{Hadley Wickham/RStudio's contributions \ldots}
\begin{itemize}
\item ggplot2 - Grammar of Graphics, now in maintenance mode (!)
\item ggvis - Interactive Grammar of Graphics for R - the next version of ggplot2
\item readr - Read flat files (CSV, TSV, FWF) into R up to 10x faster than base
\item tidyr - Data cleaning with spread and gather functions
\item dplyr - select/filter/mutate/summarise/arrange/group by; plyr for data frames
\item magrittr - ``Ceci n'est pas un pipe'' Piping. Ideas from functional programming.
			This library is becoming extremely popular.
\item stringr - String handling
\item lubridate - Date and time
\item testthat - Regression testing framework
\item Advanced R - the first few chapters are useful. The later chapters are perhaps a little
			esoteric for most people
\item R Packages - Learn to create R packages quickly
\end{itemize}
\end{frame}

\begin{frame}
\frametitle{Version control/GitHub}
\begin{itemize}
\item "Dropbox for code"
\item Several popular version control systems: Git/GitHub, Subversion, Mercurial
\item GitHub most popular in the wider tech. community, and within the R community
\item Keep track of changes
\item Collaborate with others
\item Option of keeping repositories private if you don't want your code to be public yet
\item All popular R environments and text editors integrate with version control, so using it
			is not too much harder than saving files once it's set up
\item Many packages are now released on GitHub first, before being released on CRAN
\item Example: https://github.com/hadley/ggplot2
\end{itemize}
\end{frame}

\begin{frame}
\frametitle{Visualisation/Web applications}
\begin{itemize}
\item A lot of visualisation libraries are available in Javascript
\item htmlwidgets - Integrates R with Javascript. The following packages are already
			available:
			\begin{itemize}
			\item Leaflet - Geo-spatial mapping
			\item dygraphs - Time series charting
			\item MetricsGraphics - Scatterplots and line charts with D3
			\item networkD3 - Graph data visualisation with D3
			\item d3heatmap - Interactive heatmaps with D3
			\item DataTables - Tabular data display
			\item threejs - 3D scatterplots and globes
			\item DiagrammeR - Diagrams and flowcharts
			\end{itemize}
\item shiny - Build a web application in R in five minutes
\end{itemize}
\end{frame}

\begin{frame}
\frametitle{The cloud \ldots}
\begin{itemize}
\item The cloud is someone else's computer
\item Cloud providers in Sydney: Amazon Web Services, Anchor, Rackspace etc.
\item Choice of Windows, Linux or whatever you like. If you choose Windows, Microsoft's 
			licencing fees will be included in your monthly bill
\item Bioconductor is available as an AMI (Amazon Machine Image)
\item You can rent a machine as powerful as verona on demand for ~$\$2.31$ an hour, if you 
			just need the CPUs
\item It'll cost you $\$6.96$ if you want 244 gig. of RAM and 24 2 gig. hard disks as well
\item Assuming verona cost ~$\$30,000$, that's enough to rent the machine above for ~6 months
\item Can also rent machines with a GPU, or several GPUs
\item It still makes sense to buy machines. But it's also nice to be able to rent them for
			short periods.
\item It's good to have options.
\end{itemize}
\end{frame}

\begin{frame}
\frametitle{Hadoop}
\begin{itemize}
\item Started the Big Data/data science revolution in tech.
\item Based on Google's Map/Reduce - perform computation on multiple nodes, and then combine
			the results.
\item Imagine rewriting your entire program in terms of lapply() and Reduce().
\item Designed to run on clusters
\item Distributed file system - HDFS
\item Batch-oriented
\item Let's rewrite everything \emph{again}
\item Debugging is a nightmare if something goes wrong
\item Not necessarily well-suited to iterative algorithms
\end{itemize}
\end{frame}

\begin{frame}
\frametitle{Spark}
\begin{itemize}
\item Successor to Hadoop in some sense
\item R support integrated into the main release: SparkR, MapR
\item Resilient Distributed Datasets (RDDs), SparkR data frame
\item Can be used interactively. Try to make distributed computing more
			transparent to the user
\item Designed to run iterative algorithms
\item Run programs up to 100x faster than Hadoop in memory, or 10x faster on disk
\item Let's rewrite everything \emph{again}
\item I'm still wondering what happens if something goes wrong
\end{itemize}
\end{frame}

\begin{frame}
\frametitle{MLlib - distributed machine learning framework for Spark}
\begin{itemize}
\item summary statistics, correlations, stratified sampling, hypothesis testing, random data generation
\item classification and regression: SVMs, logistic regression, linear regression, decision trees, naive Bayes
\item collaborative filtering: alternating least squares (ALS)
\item clustering: k-means, Latent Dirichlet Allocation (LDA)
\item dimensionality reduction: singular value decomposition (SVD), principal component analysis (PCA)
\item feature extraction and transformation
\item optimization primitives: stochastic gradient descent, limited-memory BFGS (L-BFGS)
\end{itemize}
Think about what's not here. Limited compared to what's in R already, but probably workable.
Machine learning focus.
\end{frame}

\begin{frame}
\frametitle{Keeping up with all this stuff \ldots}
\begin{itemize}
\item The pace of change in this area is \emph{relentless}
\item Twitter - Hadley Wickham and many other prominent R developers are on Twitter. If you 
			have a problem with something you can often ask them directly
\item Blogs - http://www.r-bloggers.com/ is a good starting point. Manage with an RSS reader 				like Feedly
\item If you're spending too much time on social media, you could try Leechblock.
			I've found this helpful.
\end{itemize}
\end{frame}

\end{document}