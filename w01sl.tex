\documentclass[a4paper]{article}\usepackage[]{graphicx}\usepackage[]{color}
%% maxwidth is the original width if it is less than linewidth
%% otherwise use linewidth (to make sure the graphics do not exceed the margin)
\makeatletter
\def\maxwidth{ %
  \ifdim\Gin@nat@width>\linewidth
    \linewidth
  \else
    \Gin@nat@width
  \fi
}
\makeatother

\definecolor{fgcolor}{rgb}{0.345, 0.345, 0.345}
\newcommand{\hlnum}[1]{\textcolor[rgb]{0.686,0.059,0.569}{#1}}%
\newcommand{\hlstr}[1]{\textcolor[rgb]{0.192,0.494,0.8}{#1}}%
\newcommand{\hlcom}[1]{\textcolor[rgb]{0.678,0.584,0.686}{\textit{#1}}}%
\newcommand{\hlopt}[1]{\textcolor[rgb]{0,0,0}{#1}}%
\newcommand{\hlstd}[1]{\textcolor[rgb]{0.345,0.345,0.345}{#1}}%
\newcommand{\hlkwa}[1]{\textcolor[rgb]{0.161,0.373,0.58}{\textbf{#1}}}%
\newcommand{\hlkwb}[1]{\textcolor[rgb]{0.69,0.353,0.396}{#1}}%
\newcommand{\hlkwc}[1]{\textcolor[rgb]{0.333,0.667,0.333}{#1}}%
\newcommand{\hlkwd}[1]{\textcolor[rgb]{0.737,0.353,0.396}{\textbf{#1}}}%

\usepackage{framed}
\makeatletter
\newenvironment{kframe}{%
 \def\at@end@of@kframe{}%
 \ifinner\ifhmode%
  \def\at@end@of@kframe{\end{minipage}}%
  \begin{minipage}{\columnwidth}%
 \fi\fi%
 \def\FrameCommand##1{\hskip\@totalleftmargin \hskip-\fboxsep
 \colorbox{shadecolor}{##1}\hskip-\fboxsep
     % There is no \\@totalrightmargin, so:
     \hskip-\linewidth \hskip-\@totalleftmargin \hskip\columnwidth}%
 \MakeFramed {\advance\hsize-\width
   \@totalleftmargin\z@ \linewidth\hsize
   \@setminipage}}%
 {\par\unskip\endMakeFramed%
 \at@end@of@kframe}
\makeatother

\definecolor{shadecolor}{rgb}{.97, .97, .97}
\definecolor{messagecolor}{rgb}{0, 0, 0}
\definecolor{warningcolor}{rgb}{1, 0, 1}
\definecolor{errorcolor}{rgb}{1, 0, 0}
\newenvironment{knitrout}{}{} % an empty environment to be redefined in TeX

\usepackage{alltt}

% you can copy and paste these page settings from here to the next % to make the pdf look a little nicer
\usepackage{amsmath}
\usepackage{amscd}
\usepackage[tableposition=top]{caption}
\usepackage{ifthen}
\usepackage[utf8]{inputenc}

\parskip=5pt
\parindent=0pt
\textwidth=6.25in
\oddsidemargin=0pt
\evensidemargin=0pt
\textheight=10in
\topmargin=-0.75in
\baselineskip=11pt
% end of page and other style settings
\IfFileExists{upquote.sty}{\usepackage{upquote}}{}
\begin{document}
\SweaveOpts{concordance=TRUE}

% the following four lines force that lines break after 80 characters in the R-output
\SweaveOpts{keep.source=FALSE}


\title{Sweave Report - Week 1}
          \author{Samuel M\"uller (000000)}
          \maketitle

\medskip\noindent
Copy and paste from the script file and from w01.pdf (note that the command \texttt{setwd()} does not work within a \texttt{.Rnw} document)

\begin{knitrout}
\definecolor{shadecolor}{rgb}{0.969, 0.969, 0.969}\color{fgcolor}\begin{kframe}
\begin{alltt}
\hlstd{rugby} \hlkwb{=} \hlkwd{read.table}\hlstd{(}\hlkwc{file}\hlstd{=}\hlstr{"/Users/mueller/Dropbox/usyd/teaching/S3012/data/rugby.txt"}\hlstd{,}\hlkwc{header}\hlstd{=}\hlnum{TRUE}\hlstd{)}
\end{alltt}


{\ttfamily\noindent\color{warningcolor}{\#\# Warning in file(file, "{}rt"{}): cannot open file '/Users/mueller/Dropbox/usyd/teaching/S3012/data/rugby.txt': No such file or directory}}

{\ttfamily\noindent\bfseries\color{errorcolor}{\#\# Error in file(file, "{}rt"{}): cannot open the connection}}\end{kframe}
\end{knitrout}

Only ask for the first couple of rows of large data frames \texttt{rugby} and this is also done by the command \texttt{head}:

\begin{knitrout}
\definecolor{shadecolor}{rgb}{0.969, 0.969, 0.969}\color{fgcolor}\begin{kframe}
\begin{alltt}
\hlstd{rugby[}\hlnum{1}\hlopt{:}\hlnum{3}\hlstd{,]}
\end{alltt}


{\ttfamily\noindent\bfseries\color{errorcolor}{\#\# Error in eval(expr, envir, enclos): object 'rugby' not found}}\begin{alltt}
\hlkwd{head}\hlstd{(rugby)}
\end{alltt}


{\ttfamily\noindent\bfseries\color{errorcolor}{\#\# Error in head(rugby): object 'rugby' not found}}\begin{alltt}
\hlkwd{head}\hlstd{(rugby,}\hlnum{2}\hlstd{)}
\end{alltt}


{\ttfamily\noindent\bfseries\color{errorcolor}{\#\# Error in head(rugby, 2): object 'rugby' not found}}\end{kframe}
\end{knitrout}

\begin{knitrout}
\definecolor{shadecolor}{rgb}{0.969, 0.969, 0.969}\color{fgcolor}\begin{kframe}
\begin{alltt}
\hlkwd{names}\hlstd{(rugby)}
\end{alltt}


{\ttfamily\noindent\bfseries\color{errorcolor}{\#\# Error in eval(expr, envir, enclos): object 'rugby' not found}}\begin{alltt}
\hlkwd{attach}\hlstd{(rugby)}
\end{alltt}


{\ttfamily\noindent\bfseries\color{errorcolor}{\#\# Error in attach(rugby): object 'rugby' not found}}\begin{alltt}
\hlkwd{table}\hlstd{(Game)}
\end{alltt}


{\ttfamily\noindent\bfseries\color{errorcolor}{\#\# Error in table(Game): object 'Game' not found}}\end{kframe}
\end{knitrout}

For each figure start a new code chunk as follows:

\begin{knitrout}
\definecolor{shadecolor}{rgb}{0.969, 0.969, 0.969}\color{fgcolor}\begin{kframe}
\begin{alltt}
\hlkwd{barplot}\hlstd{(}\hlkwd{table}\hlstd{(Game))}
\end{alltt}


{\ttfamily\noindent\bfseries\color{errorcolor}{\#\# Error in table(Game): object 'Game' not found}}\end{kframe}
\end{knitrout}

You can show such a figure within the LaTeX figure environment:

\begin{figure}
\begin{center}
\begin{knitrout}
\definecolor{shadecolor}{rgb}{0.969, 0.969, 0.969}\color{fgcolor}\begin{kframe}
\begin{alltt}
\hlkwd{barplot}\hlstd{(}\hlkwd{table}\hlstd{(Game))}
\end{alltt}


{\ttfamily\noindent\bfseries\color{errorcolor}{\#\# Error in table(Game): object 'Game' not found}}\end{kframe}
\end{knitrout}
\caption{Barplot} \label{Barplot}
\end{center}
\end{figure}

This allows to automatically refer to Figure \ref{Barplot}, without color, or to Figure \ref{Barplot 2} below, which has color and proper labels.

\begin{figure}
\begin{center}
\begin{knitrout}
\definecolor{shadecolor}{rgb}{0.969, 0.969, 0.969}\color{fgcolor}\begin{kframe}
\begin{alltt}
\hlkwd{barplot}\hlstd{(}\hlkwd{table}\hlstd{(Game),}\hlkwc{col}\hlstd{=}\hlstr{"cyan"}\hlstd{,}\hlkwc{xlab}\hlstd{=}\hlstr{"Game"}\hlstd{,}\hlkwc{ylab}\hlstd{=}\hlstr{"Frequency"}\hlstd{)}
\end{alltt}


{\ttfamily\noindent\bfseries\color{errorcolor}{\#\# Error in table(Game): object 'Game' not found}}\end{kframe}
\end{knitrout}
\caption{Barplot} \label{Barplot 2}
\end{center}
\end{figure}

\begin{knitrout}
\definecolor{shadecolor}{rgb}{0.969, 0.969, 0.969}\color{fgcolor}\begin{kframe}
\begin{alltt}
\hlkwd{hist}\hlstd{(Time)}
\end{alltt}


{\ttfamily\noindent\bfseries\color{errorcolor}{\#\# Error in hist(Time): object 'Time' not found}}\end{kframe}
\end{knitrout}

As you can see, if you put everything into a LaTeX figure environment, then code and plot are shown close to each other, otherwise it can be rather far apart.


\end{document}
