\documentclass{beamer}
\mode<presentaiton>
\title{Floating point and all that ...}
\author{Mark Greenaway}
\begin{document}
\begin{frame}
\frametitle{Double precision floating point numbers}
\begin{itemize}
\item Representation: 64 bits
\item 1 sign bit
\item 53 mantissa bits
\item 11 exponent bits

\item Can represent between $10^{-308}$ and $10^{308}$, accuracy of about 16 decimal places.
\end{itemize}
\end{frame}

\begin{frame}
Consider machine numbers of the form $x_1 \ldots x_{L_B} e$ with mantissa length L, base B, digits $x_1, \ldots, x_L$ and exponent $e \in \mathbb{Z}$. 
By definition, such a number is assigned the value
\begin{align*}
\sum_{i=1:L} x_i B^{e-i} &= x_1 B^{e-1} + x_2 B^{e-2} + \ldots + x_L B^{e-L} \\
&= B^{e-L}(x_1 B^{L-1} + x_2 B^{L-2} + \ldots + x_L)
\end{align*}

i.e. the number is essentially the value of a polynomial at the point B.

\end{frame}

\begin{frame}
\frametitle{What happens when we add a large number to a small number?}
Let's say we have 5 digits in our mantissa, and work in base 10 for clarity.
\begin{align*}
1.2345 = 12345 \times 10^{-4}
\end{align*}
Let's try adding 12345 to 1.2345.
\begin{align*}
1234&5. +\\
&1.2345\\
=1234&6.2345 \\
\approx 1234&6
\end{align*}
due to rounding.
\end{frame}

\begin{frame}
\frametitle{Implications}
\begin{itemize}
\item We must approximate the reals with a finite representation
\item All arithmetic is approximate
\item Associativity is lost $(a + b) + c \ne a + (b + c)$
\item Example: Calculating means $\bar{x} = \frac{1}{n} \sum_{i=1}^n x_i \ne \frac{1}{n} \sum_i=1^n x_{(i)}$
\item R implements the second of these, as it's more accurate
\item Evaluation of expressions requires understanding of round-off, truncation etc.
\item How should we reason about this? Interval arithmetic
\end{itemize}
\end{frame}

\begin{frame}
\frametitle{Problems with solving linear systems of equations}
How would you solve the following?
\begin{align*}
A = \begin{pmatrix}
1&10^{20}&10^{10}&1\\
10^{20}&10^{20}&1&10^{40}\\
10^{10}&1&10^{40}&10^{50}\\
1&10^{40}&10^{50}&1
\end{pmatrix}
\end{align*}
How should we choose a pivot? These problems are of considerable practical interest e.g.
in solving least squares problems:
\begin{align*}
\hat{\beta} = (X^TX)^{-1}X^T Y
\end{align*}
\end{frame}

\end{document}