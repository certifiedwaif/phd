\chapter{Normal linear models}

\section*{Abstract}

% What is done in general

We develop mean field and structured variational Bayes approximations for Bayesian model selection on linear
models using Zellner's g prior. Our mean field updates only depend on a single variational parameter $\tau_g$
and other values which are fixed for each model considered. An algorithm is developed which allows these
models to be fit. Applications to a range of data sets are presented, showing  empirically that
our method performs well on real-world data.

\section{Introduction}

% Structured Variational Bayes for model selection, Wand and Ormerod Variational Bayes for Elaborate 
% Distributions (\citep{Wand2011})

% Application

% VB theory

% Our main contribution
In this paper, we develop Variational Bayes approximations to model selection of linear models using Zellner's
g prior as in \citep{Liang2008}. We show that in this situation, our variational approximation is almost exact
-- that is, the variational approximation of the Bayesian linear model gives almost perfect estimates.

% By searching of the model space as one covariate changes between each sub-model and  using rank-1 updates on
% $(\mX^\top \mX)^{-1}$, we are able to exhaustively search the model space in $\BigO(2^p np^2)$ rather than
% $\BigO(2^p np^3)$.

% This article is organised as follows. In Section \ref{sec:model_selection}, we review previous Bayesian
% approaches to model selection. 
In Section \ref{sec:methodology} we develop our approach. In Section
\ref{sec:num_exp} we perform a series of numerical experiments to show the accuracy of our approach. Finally,
in Section \ref{sec:conclusion}, we provide a Conclusion and Discussion.

\section{Methodology}
\label{sec:methodology}

\subsection{Notation}

% Definitions

Let $n > 0$ be the number of observations and $p > 0$ be the number of covariates. Let $\vy \in \R^n$ be the
vector of responses, $\vtheta \in \R^p$ be the vector of parameters and $\mX \in \R^{n \times p}$ be the
matrix of covariates. Let $p(\vtheta)$ be the prior distribution of $\vtheta$, $p(\vy, \vtheta)$ be the full
probability distribution of $\vy$, $p(\vtheta | \vy)$ the posterior distribution and $q(\vtheta)$ be the
approximating probability distribution. 

\subsection{Model}
\label{sec:model}

Zellner constructed the $g$-prior family of priors for a Gaussian regression model using a particular form of
conjugate Normal-Gamma model, where the prior covariance matrix of $\vbeta$ is taken to be a multiple of the
Fisher information  matrix by the parameter $g$ \citep{Zellner1986}. This could be thought of as regularisation
in the principle components of the estimated covariance matrix. This places the most prior mass for $\vbeta$
on the section of the parameter space where the data is least informative.

We consider a normal linear model on $\vy$ with conjugate normal prior on $\vbeta$, and covariance $g \sigma^2
(\mX^\top \mX)^{-1}$ where the prior on $g$ is Zellner's g-prior on the covariance matrices, 
We consider a normal linear model with a $g$-prior on the regression co-efficients $\vbeta$.
This introduces a hyperparameter $g$ which controls the shrinkage of the regressions co-efficients.
The prior on $g$ can be carefully chosen so that the other parameters in the model can be integrated out
analytically, yielding tractable marginal and posterior distributions for $g$, as shown by \citep{Liang2008}.
Several priors on $g$ have been considered in the literature, including the hyper-$g$ and hyper-$g/n$ priors
\cite{Liang2008}, Bayarri's robust prior \cite{Bayarri2012} and the Beta-Prime prior introduced by
\citep{Maruyama2011}. We choose the popular Beta-Prime prior, as it is numerically well-behaved. 
We choose $a$  and $b$ to be $-3/4$ and $(n - p)/2 - a - 2$ respectively, following \citep{Maruyama2011}.
Our model is thus
\begin{align*}
	\vy | \vbeta, \sigma^2 \sim \N_n(\mX \vbeta, \sigma^2 \mI) 
\end{align*}

with priors

\begin{align*}
	\vbeta | \sigma^2, g & \sim \N_p(\vzero, g \sigma^2 (\mX^T \mX)^{-1}),                     \\
	p(\sigma^2)          & = (\sigma^2)^{-1} \I(\sigma^2 > 0), \text{ and }                    \\
	p(g)                 & = \frac{g^b (1 + g)^{-(a + b + 2)}}{\Beta(a + 1, b + 1)} \I(g > 0). 
\end{align*}


% \mgc{expand on this? See STAT260 combined lecture notes, page 55}

This leads to the posterior distribution
\[
	\vbeta | \vy \sim \N_p\left(\frac{g}{1+g} \vbetahatls, g \sigma^2 (\mX^\top \mX)^{-1} \right).
\]
which is a mixture between the null model and the least squares fit, with the degree of mixing controlled by controlled by $g/(1 + g)$.

\subsection{Model Selection}

Let $\vgamma \in \{0, 1\}^p$ be the vector of indicators of inclusion of the $p$th column of $\mX$ in the
model $\vgamma$. Then $\mX_\vgamma$ is the covariate matrix formed by including only the columns of $\mX$
indicated by $\vgamma$.

We index the space of models by a $p$-dimensional vector of indicator variables for each variable considered 
for inclusion, $\vgamma$. For each model $\mathcal{M}_\vgamma$, the response vector $\vy$ is modelled by
\begin{equation*}
	\mathcal{M}_\vgamma: \vmu_\vgamma = \vone_n \alpha + \mX_\vgamma \vbeta_\vgamma.
\end{equation*}

\section{Fully Bayesian inference}

In this section, we derive the expression required for fully Bayesian inference for the model presented in
Section \ref{sec:model}. Throughout the rest of this article, we will assume that assume without loss of
generality that $\vy^\top\vone = 0$ and that $\|\vy\|^2 = n$.

\subsection{Derivation of the marginal likelihood}

%\medskip 
%\noindent {\bf Result 4:}
%$\begin{array}{rl}
%\ds p(\vy|\sigma^2,g)  = \exp\left\{
%-\tfrac{n}{2}\log(2\pi\sigma^2) 
%- \tfrac{p}{2}\log(1+g)
%- \sigma^{-2} \tfrac{n}{2}\left( 
%1 -
% \tfrac{g}{1+g} R^2  \right)
%\right\}
%\end{array}$
 
\noindent First we derive the conditional likelihood of $\vy$ given $\sigma^2$ and $g$ via
$$
\begin{array}{rl}
	\ds p(\vy|\sigma^2,g) 
	  & \ds = \int \exp\left\{ 
	-\tfrac{n}{2}\log(2\pi\sigma^2) - \tfrac{1}{2\sigma^2}\|\vy - \mX\vbeta\|^2
	-\tfrac{p}{2}\log(2\pi g\sigma^2) + \tfrac{1}{2}\log|\mX^T\mX| - \tfrac{1}{2g\sigma^2}\vbeta^T\mX^T\mX\vbeta
	\right\} d\vbeta
	\\ [2ex]
	  & \ds = \exp\left\{      
	-\tfrac{n}{2}\log(2\pi\sigma^2) - \tfrac{1}{2\sigma^2}\|\vy\|^2 -\tfrac{p}{2}\log(2\pi g\sigma^2) + \tfrac{1}{2}\log|\mX^T\mX| \right\}
	\\
	  & \ds \quad \times \int 
	\exp\left\{ - \tfrac{1}{2}\vbeta^T \left[ \sigma^{-2}(1+g^{-1})\mX^T\mX \right]\vbeta + \sigma^{-2}\vbeta^T\mX^T\vy
	\right\} d\vbeta \\ [2ex]
	  & \ds = \exp\left\{      
	-\tfrac{n}{2}\log(2\pi\sigma^2) - \tfrac{1}{2\sigma^2}\|\vy\|^2 -\tfrac{p}{2}\log(2\pi g\sigma^2) + \tfrac{1}{2}\log|\mX^T\mX| \right\}
	\\
	  & \ds \quad \times      
	|2\pi \left[ \sigma^{-2}(1+g^{-1})\mX^T\mX \right]^{-1}|^{1/2}\exp\left\{  \tfrac{1}{2\sigma^2}  \vy^T\mX\left[ (1+g^{-1})\mX^T\mX \right]^{-1}\mX^T\vy
	\right\}
	\\ [2ex]
	  & \ds = \exp\left\{      
	-\tfrac{n}{2}\log(2\pi\sigma^2) 
	- \tfrac{1}{2\sigma^2}\|\vy\|^2 
	-\tfrac{p}{2}\log(2\pi g\sigma^2) 
	+ \tfrac{1}{2}\log|\mX^T\mX| \right\}
	\\
	  & \ds \quad \times      
	\exp\left\{ 
	\tfrac{p}{2}\log(2\pi) 
	+ \tfrac{p}{2}\log(\sigma^2)
	- \tfrac{p}{2}\log(1+g^{-1})
	- \tfrac{1}{2}\log|\mX^T\mX| 
	+ \tfrac{1}{2\sigma^2}  \vy^T\mX\left[ (1+g^{-1})\mX^T\mX \right]^{-1}\mX^T\vy
	\right\}
	\\ [2ex]
	  & \ds = \exp\left\{      
	-\tfrac{n}{2}\log(2\pi\sigma^2) 
	- \tfrac{p}{2}\log(1+g)
	- \tfrac{n}{2\sigma^2} 
	+ \tfrac{g}{2\sigma^2(1+g)} n R^2
	\right\}
\end{array}
$$

\noindent where the third and last lines follow from
Equation \ref{res:01} and Equation \ref{res:02} respectively.
We also make use of the fact that $\|\vy\|^2 = n$ and the property of determinants that $|c\mA| = c^d|\mA|$ and $|\mA^{-1}| = |\mA|^{-1}$ when $\mA \in\bR^{d\times d}$.

%\medskip 
%\noindent {\bf Result 4:}
%$\ds p(\vy|g) = \frac{\Gamma(n/2)}{\pi^{n/2} \|\vy\|^n} (1 + g)^{(n-p)/2} \left[  1 + g(1 -  R^2) \right] ^{-n/2}$.

Next, we obtain the likelihood of $\vy$ conditional on $g$ by integrating out $\sigma^2$ via
$$
\begin{array}{rl}
	\ds p(\vy|g) 
	  & \ds = \exp\left\{
	- \tfrac{n}{2}\log(2\pi) - \tfrac{p}{2}\log(1+g) 
	- \tfrac{1}{2\sigma^2}n
	\right\}
	\\ [1ex]
	  & \ds \quad \times                                                                                 
	\int_0^\infty (\sigma^2)^{-(n/2 + 1)}
	\exp\left\{
	- \sigma^{-2} \left[ \tfrac{n}{2} - \tfrac{g}{2(1+g)} nR^2 \right] 
	\right\} d\sigma^2
	\\ [2ex]
	  & \ds = \exp\left\{                                                                                 
	- \tfrac{n}{2}\log(2\pi) - \tfrac{p}{2}\log(1+g)
	\right\}
	\times 
	\frac{\Gamma(n/2)}{\ds \left[ \tfrac{1}{2}\|\vy\|^2 - \tfrac{g}{2(1+g)}nR^2 \right] ^{n/2}}
	\\ [2ex]
	%    & \ds = \frac{\Gamma(n/2)}{(n\pi)^{n/2}} (1 + g)^{-p/2} \left[  1 - \frac{g}{(1+g)}  R^2 \right] ^{-n/2}
	%    \\ [2ex]
	  & \ds = \frac{\Gamma(n/2)}{(n\pi)^{n/2}} (1 + g)^{(n-p)/2} \left[  1 + g(1 -  R^2) \right] ^{-n/2}. 
\end{array}
$$

\noindent Finally, we derive the marginal likelihood of $\vy$ by integrating out $g$. Using $b= (n-p)/2 - 2 - a$ we have
$$
\begin{array}{rl}
	\ds p(\vy) 
	  & \ds = \int_0^\infty                                         
	\frac{g^{b}(1 + g)^{-a-b-2}}{\mbox{Beta}(a+1,b+1)}
	\frac{\Gamma(n/2)}{(n\pi)^{n/2}} (1 + g)^{(n-p)/2} \left[  1 + g(1 -  R^2) \right]^{-n/2}
	dg
	\\ [2ex]
	  & \ds = \frac{\Gamma(n/2)}{(n\pi)^{n/2} \mbox{Beta}(a+1,b+1)} 
	\int_0^\infty g^{b} \left[  1 + g(1 -  R^2) \right]^{-n/2}
	\\ [2ex]
	  & \ds = \frac{\Gamma(n/2) }{(n\pi)^{n/2}}                     
	\frac{\mbox{Beta}( p/2 + a + 1,b+1)}{\mbox{Beta}(a+1,b+1)}
	(1 -  R^2)^{-(b+1)}.
	\\ [2ex]
	  & \ds                                                         
	= \frac{\Gamma( p/2 + a + 1)}{(n\pi)^{n/2}} 
	\frac{\Gamma((n-p)/2)}{\Gamma(a+1)} (1 -  R^2)^{-((n-p)/2 - a - 1)} \\
\end{array}
$$

\subsection{Posterior distributions of the parameters}
We now derive the expressions for the posterior distributions of each of the parameters. Using Equation
\ref{res:03} and Equation \ref{res:04} we have
\begin{equation}
\begin{array}{rl}\label{res:06}
	p(g|\vy) & = \frac{(1 -  R^2)^{b+1} g^{b} \left[  1 + g(1 -  R^2) \right]^{-n/2}}{\mbox{Beta}(p/2 + a + 1,b+1)} \text{ and } \\
	\ds \frac{\mbox{Beta}(p/2 + a + 1,b)}{\mbox{Beta}(p/2 + a + 1,b+1)}
	  & \ds = \frac{\Gamma(p/2 + a + 1)\Gamma(b)}{\Gamma(p/2 + a + 1 + b)} 
	\frac{\Gamma(p/2 + a + 1 + b + 1)}{\Gamma(p/2 + a + 1)\Gamma(b+1)}
	\\
	  & \ds = \frac{\Gamma(b)}{\Gamma(p/2 + a + 1 + b)}                    
	\frac{\Gamma(p/2 + a + 1 + b + 1)}{\Gamma(b+1)}
	\\
	  & \ds = \frac{p/2 + a + 1 + b}{b}                                    
	\\
	  & \ds = 1 + \frac{p/2 + a + 1}{b}.                                   
\end{array}
\end{equation}
\noindent Using the change of variables $h=g/(1+g)$ we have
\begin{equation}
	p(h|\vy) = \frac{(1 -  R^2)^{b+1}}{\mbox{Beta}(p/2 + a + 1,b+1)} h^{b}(1 - h)^{2-b+n/2}  (1  - h R^2)^{-n/2}.
\end{equation}
 

\noindent The full conditional for $\vbeta$ is given by
\begin{equation}
	\begin{array}{rl}
		\vbeta|\vy,\sigma^2,g \sim \N\left[                  
		\tfrac{g}{1+g}\widehat{\vbeta}_{\mbox{\tiny LS}},    
		\tfrac{g}{1+g} \sigma^2 \left( \mX^T\mX \right)^{-1} 
		\right]                                              
	\end{array} 
\end{equation}
\noindent where $\widehat{\vbeta}_{\mbox{\tiny LS}} = \left( \mX^T\mX \right)^{-1}\mX^T\vy$.

The full conditional for $\sigma^2$ obtained by integrating out $\vbeta$ is
\begin{equation}
	\sigma^2|\vy,g \sim \mbox{IG}\left[\tfrac{n}{2},\tfrac{n}{2}\left( 
		1 -
	\tfrac{g}{1+g} R^2\right) \right].
\end{equation}

The density for $p(\sigma^2|\vy)$ is obtained by evaluating the integral
$$
\begin{array}{rl}
	\ds p(\sigma^2|\vy) 
	  & \ds = \int_0^\infty p(\sigma^2|\vy,g) p(g|\vy) dg                                                                                                                                  
	    
	\\ [2ex]
	    
	  & \ds = \int_0^\infty                                                                                                                                                                
	\left[ \frac{\left[\tfrac{n}{2}\left( 
	1 -
	\tfrac{g}{1+g} R^2\right) \right]^{n/2}}{\Gamma(n/2)} (\sigma^2)^{-(n/2 + 1)} \exp\left\{ - \sigma^{-2} \tfrac{n}{2}\left( 
	1 -
	\tfrac{g}{1+g} R^2\right) \right\} \right]
	\\ [2ex]
	  & \ds \quad \times \left[                                                                                                                                                           
	\frac{(1 -  R^2)^{b+1} g^{b} \left[  1 + g(1 -  R^2) \right]^{-n/2}}{\mbox{Beta}(p/2 + a + 1,b+1)}
	\right] dg
	    
	\\ [2ex]
	    
	    
	    
	  & \ds = \frac{(1 -  R^2)^{b+1} \left( \frac{n}{2}\right)^{n/2}(\sigma^2)^{-(n/2 + 1)}\exp\left\{ -  \tfrac{n}{2\sigma^2} \right\}}{\Gamma(n/2)\mbox{Beta}(p/2 + a + 1,b+1)}          
	\\ [2ex]
	  & \ds \quad \times \int_0^\infty                                                                                                                                                     
	\left( 
	1 -
	\tfrac{g}{1+g} R^2\right)^{n/2} g^{b} \left[  1 + g(1 -  R^2) \right]^{-n/2}
	\exp\left\{ 
	\frac{g}{1+g} \frac{nR^2}{2\sigma^2} \right\} dg
	
	\\ [2ex]
	
	  & \ds = \frac{(1 -  R^2)^{b+1} \left( \frac{n}{2}\right)^{n/2}(\sigma^2)^{-(n/2 + 1)}\exp\left\{ -  \tfrac{n}{2\sigma^2} \right\}}{\Gamma(n/2)\mbox{Beta}(p/2 + a + 1,b+1)}          
	\\ [2ex]
	  & \ds \quad \times \int_0^1                                                                                                                                                          
	\left( 
	1 -
	h R^2\right)^{n/2} \left( \frac{h}{1 - h}\right)^{b} \left[  1 + \frac{h}{1 - h}(1 -  R^2) \right]^{-n/2}
	\exp\left\{ 
	h \frac{nR^2}{2\sigma^2} \right\}\frac{1}{(1-h)^2} dh
	    
	\\ [2ex]
		
	  & \ds = \frac{ (1 -  R^2)^{b+1}\left( \frac{n}{2}\right)^{n/2}(\sigma^2)^{-(n/2 + 1)}\exp\left\{ -  \tfrac{n}{2\sigma^2} \right\}}{\Gamma(n/2)\mbox{Beta}(p/2 + a + 1,b+1)} \int_0^1 
	h^{b}(1-h)^{a+p/2} 
	\exp\left\{ 
	h \frac{nR^2}{2\sigma^2} \right\}  dh    
\end{array}
$$

% \mgc{How is this used?}
% $$
% -b-2+n/2 = p/2 + a  
% $$

\noindent using the substitution $h = g/(1 + g)$, to transform the integral to be over the interval $[0, 1]$
and into a form where we are able to use
% $g = \frac{h}{1-h}$,
% $1 - h = \frac{1}{(1 + g)}$,
% and
% $\frac{dh}{dg} = \frac{1}{1+g} - \frac{g}{(1 +g)^2} = \frac{1}{(1+g)^2} = (1 - h)^2$.
Equation 3.383(i) of \citep{Gradshteyn1988},
$$
\int_{0}^u x^{\nu - 1} (u - x)^{\mu - 1}  e^{\beta x} dx = \mbox{Beta}(\nu,\mu) {}_1 F_1(\nu;\mu+\nu;\beta u)
$$
to complete the resulting integral.

\noindent provided $\mbox{Re}(\mu)>0$, $\mbox{Re}(\nu)>0$.

\noindent We have
$u = 1$, $\nu = b + 1 = \frac{n-p}{2} - a - 1 >0$, $\mu = a + p/2 + 1 >0$, and 
$\beta = \frac{nR^2}{2\sigma^2}$, so $\nu + \mu = \frac{n}{2}$. Hence the above integral evaluates to the
expression
$$
\begin{array}{rl}
	\ds p(\sigma^2|\vy) 
	  & \ds = \frac{(1 -  R^2)^{b+1} ( n/2)^{n/2}}{\Gamma(n/2)} 
	(\sigma^2)^{-(n/2 + 1)}\exp\left\{ -  \frac{n}{2} \sigma^{-2} \right\}  {}_1 F_1\left(
	b + 1; \frac{n}{2}; \frac{nR^2}{2} \sigma^{-2} \right)
\end{array}
$$

\noindent where ${}_1 F_1(\alpha;\gamma;z) \equiv \Phi_1(\alpha;\gamma;z) = M(\alpha;\gamma;z)$ 
is the
degenerate hypergeometric function or
confluent hypergeometric function. To the best of our knowledge, this is a new distribution.

\noindent Parameterising the distribution on $\sigma^2$ in terms of the precision $\tau= 1/\sigma^2$ we obtain
the expression
$$
\begin{array}{rl}
	\ds p(\tau|\vy) 
	  & \ds = \frac{(1 -  R^2)^{b+1} ( n/2)^{n/2}}{\Gamma(n/2)} 
	\tau^{n/2 - 1}\exp\left\{ -  \frac{n}{2} \tau \right\} {}_1 F_1\left(
	b + 1; \frac{n}{2}; \frac{nR^2}{2} \tau \right).
\end{array}
$$

\noindent
This expression is integrable, and we are able to evaluate the integral
using the result from \citep{Gradshteyn1988}. \mgc{Which result, Mark?}
$$
\int_0^\infty \tau^{n/2 - 1}\exp\left\{ -  \tfrac{n}{2} \tau \right\}{}_1 F_1\left(
b + 1; \frac{n}{2}; \frac{nR^2}{2} \tau \right) d \tau = 
\frac{\Gamma(n/2)}{(n/2)^{n/2}\left( 1 - R^2\right)^{b+1}}
$$
\noindent with $\rho_1 = n/2$, $\mu = n/2$, $a_1 = b+1$, $\lambda = nR^2/2$. Hence, $p(\sigma|\vy)$ and $p(\tau|\vy)$ are indeed probability density functions.


\subsection{Posterior moments of the $\vbeta$ and $\sigma^2$}
We now calculate the posterior expectation of the precision, $\bE(\sigma^{-2}|\vy)$
$$
\begin{array}{rl}
	\ds \int_0^\infty \sigma^{-2} p(\sigma^2|\vy) d\sigma^2 
	  & \ds = \int_0^\infty \int_0^\infty \sigma^{-2} p(\sigma^2|\vy,g) d\sigma^2 p(g|\vy) dg 
	\\ [2ex]
	  & \ds =  \int_0^\infty \frac{1}{\left(                                                  
	1 -
	\tfrac{g}{1+g} R^2\right)} \frac{(1 -  R^2)^{b+1} g^{b} \left[  1 + g(1 -  R^2) \right]^{-n/2}}{\mbox{Beta}(p/2 + a + 1,b+1)} dg
	\\ [2ex]
	  & \ds =  \frac{(1 -  R^2)^{b+1}}{\mbox{Beta}(p/2 + a + 1,b+1)}                          
	\int_0^\infty \left( 
	1 -
	\tfrac{g}{1+g} R^2\right)^{-1} g^{b} \left[  1 + g(1 -  R^2) \right]^{-n/2}dg
	\\ [2ex]
	  & \ds =  \frac{(1 -  R^2)^{b+1}}{\mbox{Beta}(p/2 + a + 1,b+1)}                          
	\int_0^1 \left( 
	1 -
	h R^2\right)^{-1} \left( \frac{h}{1 - h}\right)^{b} \left[  1 + \frac{h}{1 - h}(1 -  R^2) \right]^{-n/2} \frac{1}{(1 - h)^2} dh
	
	
	\\ [2ex]
	    
	  & \ds =  \frac{(1 -  R^2)^{b+1}}{\mbox{Beta}(p/2 + a + 1,b+1)}                          
	\int_0^1  h^b (1 - h)^{n/2 - b - 2} ( 1 -  hR^2)^{-(n/2+1)}  dh
	    
	    
	\\ [2ex]
	  & \ds =  (1 -  R^2)^{b+1}                                                               
	{}_1 F_2(n/2+1,b+1,n/2;R^2)  .
\end{array}
$$

% \mgc{Why are these here?}
% $A = n/2 + 1$

% $B - 1 = b$

% $B = b + 1$

% $C = n/2$

\noindent Using the Law of Total Expectation,
\begin{align*}
	\Var(\vbeta | \vy) & = \E_g[\Var(\vbeta | \vy)] + \Var_g[\E(\vbeta|\vy, g)]                                                                                                                                                 \\
	                   & = \E_g\left[\frac{g}{1 + g} \E[\sigma^2|\vy, g] (\mX^\top \mX)^{-1} \Bigm| \vy \right] + \Var_g\left[\frac{g}{1 + g} \vbetahatls \vbetahatls^\top \Bigm| \vy \right]                                   \\
	                   & = \frac{n}{n - 2} \E_g\left[\frac{g}{1 + g} \left(1 - \frac{g}{1 + g} R^2 \right) \Bigm| \vy \right] (\mX^\top \mX)^{-1} + \Var_g\left[\frac{g}{1 + g} \Bigm| \vy \right] \vbetahatls \vbetahatls^\top \\
	                   & = \left( \frac{n}{n - 2} \right) (M_1 - M_2 R^2) (\mX^\top \mX)^{-1} + (M_2 - M_1^2) \vbetahatls \vbetahatls^\top                                                                                      
\end{align*}

\noindent where $G_1 = \E\left[\frac{g}{1 + g} \Bigm| \vy \right]$ and $G_2 = \E\left[ \left(\frac{g}{1 + g} \right)^2 \Bigm| \vy \right]$. We will calculate these expectations below.

To calculate $G_1$, we evaluate the integral

\begin{align*}
	\E\left[\frac{g}{1 + g} \Bigm| \vy \right] & = \int_0^\infty \frac{g}{1 + g} p(g | \vy) dg                                                                      \\
	                                           & = \frac{(1 - R^2)^{b+1}}{\Beta(p/2 + a + 1, b + 1)} \int_0^\infty g^{b+1} (1 + g)^{-1} [1 + g (1 - R^2)]^{-n/2} dg \\
	                                           & = \frac{\Beta(b + 2, n/2 + b + 3)}{\Beta(p /2 + a + 1, b + 1)} (1 - R^2)^{b+1} {}_2 F_1(n/2, b + 2; n/2 + 1; R^2)  \\
	                                           & = \frac{\Beta(p/2 + a + 1, b + 2)}{\Beta(p /2 + a + 1, b + 1)} {}_2 F_1(p/2 + a + 1, 1; n/2 + 1; R^2)              
\end{align*}

\noindent using 3.197 Equation 5 from \citep{Gradshteyn1988},

\[
	\int_0^\infty x^{\lambda-1} (1+x)^\nu (1 + \alpha x)^{\mu} dx = \Beta(\lambda, -\mu-\nu-\lambda) {}_2 F_1 (\mu, \lambda; -\mu-\nu; 1 - \alpha)
\]

\noindent when $[|\arg \alpha| < \pi], -\Re(\mu + \mu) > \Re \lambda > 0$ to obtain a closed form for the integral, and using Euler's identity
\[
	{}_2 F_1(a, b; c; d) = (1 - d)^{c - a -  b} {}_2 F_1(c - a, c - b; c; d)
\]
\noindent to simplify the expression further.

Similiarly, we compute $G_2$ by evaluating

\[
	\E\left[ \left(\frac{g}{1 + g} \right)^2 \Bigm| \vy \right] = \frac{\Beta(p/2 + a + 1, b + 2)}{\Beta(p /2 + a + 1, b + 1)} (1 - R^2)^{b+2} {}_2 F_1(p/2 + a + 1, 1; n/2 + 1; R^2).
\]

\noindent Thus we obtain expressions for $M_1$ and $M_2$ in terms of the hypergeometric function.
\begin{align*}
	M_1 & = \frac{\Beta(p/2 + a + 1, b + 2)}{\Beta(p /2 + a + 1, b + 1)} (1 - R^2)^{b+1} {}_2 F_1(n/2, b + 1; n/2 + 1; R^2), \text{ and } \\
	M_2 & = \frac{\Beta(p/2 + a + 1, b + 3)}{\Beta(p /2 + a + 1, b + 1)} (1 - R^2)^{b+1} {}_2 F_1(n/2, b + 3; n/2 + 2; R^2).              
\end{align*}

% TODO: Posterior distributions of regression parameters
% TODO: Posterior distributions of g for different specifications of priors on g

\section{Numerical experiments}
\label{sec:num_exp}

Let $\kappa$ be the inflation factor such that for the examples under consideration $n = \kappa p$.

\subsection{Implementation details}
\label{sec:implementation}

We traverse the models in $\Gamma$ in Graycode order.
This ensures that as the algorithm moves from one model to another, only one covariate either enters or leaves
the model.
This allows us to use Rank-1 updates and downdates of $(\mX^\top \mX)^{-1}$ to greatly decrease the
computational cost of calculating $R_\vgamma^2$ from $\BigO(np^3)$ to $\BigO(np^2)$.
Special care was taken to minimise memory allocation and deallocation.
The \texttt{Rcpp} and \texttt{RcppEigen} libraries were used to ensure the implementation was performant.

Initially, $\vgamma = (1, 0, 0, \ldots, 0, 0)^\top$. So $(\mX_\vgamma^\top \mX_\vgamma)^{-1} =
(\vx_\vgamma^\top \vx_\vgamma)^{-1}$ which is a scalar, and so can be computed in $\BigO(n)$. Then the set of
possible models $\vgamma$ is iterated through in graycode order. This ensures that only one entry of $\vgamma$
changes as each model is visited, allowing the new $(\mX_\vgamma^\top \mX_\vgamma)^{-1}$ to be calculated
using the previous $(\mX_\vgamma^\top \mX_\vgamma)^{-1}$ and a rank-1 update. This allows us to avoid directly
computing the new $\mX_\vgamma^\top \mX_\vgamma$ and then performing a full inversion, reducing a $\BigO(n
p^3)$ operation to $\BigO(n p^2)$.

\mgc{Think about distribution of bit strings. What is the expectation of $p$? I think it's simply $p/2$}

\subsection{Results}

To assess the accuracy of our variational approximations to the exact posteriors, we examined a number of
measures. All of the measures that we examined were functions of $\kappa$, $p$ and $R^2$, and so serve to
characterise the performance of our approximation on any possible data set with a number of observations
$n$, number of covariates $p$ and correlation $R$ between the response $\vy$ and the covariate matrix $\mX$.

\subsection{Shrinkage}

Exact posterior and approximate shrinkage $\left( \frac{g}{1 + g} \right)$ were calculated for a range of
values of $p$, $n$ and $R^2$ to compare their values. As can be seen from Figure $\ref{fig:shrinkage}$, the
values of the exact posterior shrinkage and approximate shrinkage are almost the same over most of the range
of these values, with deviation only noticeable in the $p=10$ and $p=20$ cases.

\begin{figure}[p]
	\includegraphics[width=17cm, height=17cm]{code/taug/Shrinkage.pdf}
	\caption{The ratio $\frac{g}{1 + g} | \vy$ controls the degree to which the model fit tends back towards
		the prior mean. This ratio is approximated in the variational approximation by $(1 + \tau_g^{-1})^{-1}$. We
		compare these two quantities in the figure above for a range of numbers of covariates, sample sizes and $R^2$
		values. Here $n = \kappa p$.}
	\label{fig:shrinkage}
\end{figure}

Coefficient posterior variance \\

We can see from Figure \ref{fig:variance} that as $p$ and $\kappa$ increase, the approximation to the
posterior variance of \ldots becomes more and more accurate.

\begin{figure}[p]
	\includegraphics[width=17cm, height=17cm]{code/taug/Variance.pdf}
	\caption{The posterior variance of the example model $p(\sigma^2 | \vy)$ is compared against the approximation
		to the posterior variance $q(\sigma^2)$ across a range of numbers of covariates, sample sizes and $R^2$
		values. Here $n = \kappa p$.}
	\label{fig:variance}
\end{figure}

Accuracy of approximation to $p(\sigma^2 | \vy)$ \\

We assessed the accuracy of the approximation to $p(\sigma^2 | \vy)$ by $q(\sigma^2)$ by numerically evaluating
the integral
\[
	1 - \frac{1}{2} \int_0^\infty |p(\sigma^2 | \vy) - q(\sigma^2)| d \sigma^2
\]
for a range of values of $p$, $\kappa$ and $R^2$. The results are presented in Figure \ref{fig:accuracy_sigma2}.

\begin{figure}[p]
	\includegraphics[width=17cm, height=17cm]{code/taug/Accuracy_sigma2.pdf}
	\caption{The accuracy of the approximation $q(\sigma^2)$ is assessed by computing the integral   $1 -
		\frac{1}{2} \int_0^\infty |p(\sigma^2 | \vy) - q(\sigma^2)| d \sigma^2$ and graphing the result over a range
		of numbers of covariates, sample sizes and $R^2$  values. Here $n = \kappa p$.}
	\label{fig:accuracy_sigma2}
\end{figure}

Accuracy of approximation to $p(g | \vy)$ \\

\begin{figure}[p]
	\includegraphics[width=17cm, height=17cm]{code/taug/Accuracy_g.pdf}
	\caption{The accuracy of the approximation $q(g)$ is assessed by computing the integral   $1 -
		\frac{1}{2} \int_0^\infty |p(g | \vy) - q(g)| d g$ and graphing the result over a range
		of numbers of covariates, sample sizes and $R^2$  values. Here $n = \kappa p$.}
	\label{fig:accuracy_g}
\end{figure}

We assessed the accuracy of the approximation to $p(g | \vy)$ by $q(g)$ by numerically evaluating the integral
\[
	1 - \frac{1}{2} \int_0^\infty |p(g | \vy) - q(g)| d g
\]
for a range of values of $p$, $\kappa$ and $R^2$. The results are presented in Figure \ref{fig:accuracy_g}.


$\log{p(\vy)}$ versus ELBO \\

We assessed the accuracy of the variational lower bound by calculating the relative error of
$\log{\underline{p}(\vy)}$ versus $\log{p(\vy)}$. The results are presented in Figure
\ref{fig:relative_error}.

\begin{figure}[p]
	\includegraphics[width=17cm, height=17cm]{code/taug/Relative_error_log_p.pdf}
	\caption{The accuracy of the variational lower bound was assessed by computing the relative error
		$\frac{\log p(\vy) - \log \underline{p}(\vy)}{p(\sigma^2 | \vy)}$ over a range
		of numbers of covariates, sample sizes and $R^2$  values. Here $n = \kappa p$.}
	\label{fig:relative_error}
\end{figure}

Precision \\

The accuracy of the precision was assessed by calculating the relative error of the approximate precision
versus the exact posterior precision. The results are presented in Figure \ref{fig:precision}. As $p$ and
$\kappa$ increase, we can see that the approximate precision is converging towards the exact precision for all
values of $R^2$.

\begin{figure}[p]
	\includegraphics[width=17cm, height=17cm]{code/taug/Precision.pdf}
	\caption{The accuracy of the approximation to the posterior precision was assessed by plotting
		$p(\sigma^{-2} | \vy)$ against $q(\sigma^{-2})$ over a range
		of numbers of covariates, sample sizes and $R^2$  values. Here $n = \kappa p$.}
	\label{fig:precision}
\end{figure}

\subsection{Marginal covariate inclusion probabilities}

The marginal covariate inclusion probabilities from the variational approximation match those from the the
exact posterior likelihood very closely. Comparing these marginal covariate inclusion probabilities to those
produced by weighting each model considered by AIC and BIC, we see that the probabilities produced by the
variational approximation and exact posterior likelihood are more conservative.

\begin{figure}[p]
	\includegraphics[width=17cm, height=17cm]{code/taug/Log_of_Relative_error_of_Variance_of_g.pdf}
	\caption{The accuracy of the approximation to the variance of $g$ was assessed by computing the relative
		error of $q(g)$ relative to $p(g | \vy)$ over a range
		of numbers of covariates, sample sizes and $R^2$  values. Here $n = \kappa p$.}
	\label{fig:rel_error_var_g}
\end{figure}


% vw1 <- read.csv("Hitters_vw1.csv", header=FALSE)
% r <- hist(as.matrix(vw1), breaks=19, axes=FALSE, prob=TRUE, main="", xlab="")
% axis(1, r$mid, c("1", "2", "3", "4", "5", "6", "7", "8", "9", "10", "11", "12", "13", "14", "15", "16", "17", "18"))

% bodyfat
% Major League Baseball Data from the 1986 and 1987 seasons.
% An Introduction to Statistical Learning with Applications in R

% \begin{tabular}{|l|lllllllllllllllllll|}
% 	\hline
% 	$\vp$ & 0.137 & 0.130 & 0.257 & 0.982 & 0.921 & 0.173 & 0.633 & 0.562 & 0.623 & 0.480 & 0.441 & 0.499 & 0.197 & 0.926 & 0.131 & 0.174 & 0.128 & 0.901 & 0.851 \\
% 	$\vq$ & 0.137 & 0.130 & 0.257 & 0.982 & 0.922 & 0.172 & 0.635 & 0.562 & 0.625 & 0.480 & 0.440 & 0.499 & 0.197 & 0.927 & 0.130 & 0.173 & 0.127 & 0.902 & 0.853 \\
% 	\hline
% \end{tabular}

\subsubsection{Hitters}

The Hitters data set was taken from the StatLib library maintained by Carnegie Mellon University, and used in
\citep{James:2014:ISL:2517747}. The data was gathered from the the performance of players during the Major
League Baseball seasons for 1986 and 1987, and includes the following covariates:

\begin{tabular}{|ll|}
	\hline
	Covariate & Description                                                                        \\
	\hline
	AtBat     & Number of times at bat in 1986                                                     \\
	Hits      & Number of hits in 1986                                                             \\
	HmRun     & Number of home runs in 1986                                                        \\
	Runs      & Number of runs in 1986                                                             \\
	RBI       & Number of runs batted in in 1986                                                   \\
	Walks     & Number of walks in 1986                                                            \\
	Years     & Number of years in the major leagues                                               \\
	CAtBat    & Number of times at bat during his career                                           \\
	CHits     & Number of hits during his career                                                   \\
	CHmRun    & Number of home runs during his career                                              \\
	CRuns     & Number of runs during his career                                                   \\
	CRBI      & Number of runs batted in during his career                                         \\
	CWalks    & Number of walks during his career                                                  \\
	League    & A factor with levels A and N indicating player’s league at the end of 1986       \\
	Division  & A factor with levels E and W indicating player’s division at the end of 1986     \\
	PutOuts   & Number of put outs in 1986                                                         \\
	Assists   & Number of assists in 1986                                                          \\
	Errors    & Number of errors in 1986                                                           \\
	Salary    & 1987 annual salary on opening day in thousands of dollars                          \\
	NewLeague & A factor with levels A and N indicating player’s league at the beginning of 1987 \\
	\hline
\end{tabular}

The marginal covariate inclusion probabilities are presented in Figure \ref{fig:Hitters_inclusion}.

\begin{figure}[p]
	\includegraphics[scale=.4]{Hitters_variable_selection.pdf}
	\caption{Hitters marginal inclusion probablities}
	\label{fig:Hitters_inclusion}
\end{figure}

The model ranking scatterplots are presented in Figure \ref{fig:Hitters_model_ranking}.



\begin{figure}[p]
	\includegraphics[scale=.4]{code/Model_selection_scatter_plot_1.pdf}
	\caption{Hitters model ranking scatterplot}
	\label{fig:Hitters_model_ranking}
\end{figure}

\subsubsection{Body Fat}
The Body Fat data set was taken from \citep{Tarr2015}, and is on the relationship between percentage of body
fat and simple body measurements. The data set consists of 128 observations, and the covariates in the data set
are:

\begin{tabular}{|ll|}
	\hline
	Covariate & Description                                                                  \\
	\hline
	Id        & Identifier                                                                   \\
	Bodyfat   & Bodyfat percentage                                                           \\
	Age       & Age (years)                                                                  \\
	Weight    & Weight (kg)                                                                  \\
	Height    & Height (inches)                                                              \\
	Neck      & Neck circumference (cm)                                                      \\
	Chest     & Chest circumference (cm)                                                     \\
	Abdo      & Abdomen circumference (cm) "at the umbilicus and level with the iliac crest" \\
	Hip       & Hip circumference (cm)                                                       \\
	Thigh     & Thigh circumference (cm)                                                     \\
	Knee      & Knee circumference (cm)                                                      \\
	Ankle     & Ankle circumference (cm)                                                     \\
	Bic       & Extended biceps circumference (cm)                                           \\
	Fore      & Forearm circumference (cm)                                                   \\
	Wrist     & Wrist circumference (cm) "distal to the styloid processes"                   \\
	\hline
\end{tabular}

The marginal covariate inclusion probabilities are presented in Figure \ref{fig:bodyfat_inclusion}.

% \begin{tabular}{|l|lllllllllllll|}
% 	\hline
% 	$\vp$ & 0.938 & 0.136 & 0.182 & 0.072 & 0.071 & 0.108 & 0.147 & 1.000 & 0.134 & 0.141 & 0.323 & 0.619 & 0.221 \\
% 	$\vq$ & 0.939 & 0.136 & 0.182 & 0.071 & 0.070 & 0.107 & 0.146 & 1.000 & 0.134 & 0.140 & 0.323 & 0.620 & 0.221 \\
% 	\hline
% \end{tabular}

\begin{figure}[p]
	\includegraphics[scale=.4]{bodyfat_variable_selection.pdf}
	\caption{bodyfat marginal inclusion probablities}
	\label{fig:bodyfat_inclusion}
\end{figure}

The model ranking scatterplots are presented in Figure \ref{fig:bodyfat_model_ranking}.

\begin{figure}[p]
	\includegraphics[scale=.4]{code/Model_selection_scatter_plot_2.pdf}
	\caption{bodyfat model ranking scatterplot}
	\label{fig:bodyfat_model_ranking}
\end{figure}

\subsubsection{Wage}

The Wage gap data set was taken from \citep{James:2014:ISL:2517747}, on 3000 workers in the Mid-Atlantic
region. The data set includes 12 covariates:

\begin{tabular}{|ll|}
	\hline
	Covariate        & Description                                                                           \\
	\hline
	Year             & Year that the wage information was recorded                                           \\
	Age              & Age of worker                                                                         \\
	Sex              & Gender                                                                                \\
	Marital status   & Marital status -- Never Married, Married, Widowed, Divorced or Seperated              \\
	Race             & White, Black, Asian or Other                                                          \\
	Education        & Education level -- Did not graduate high school, Graduated high school, Some College, 
	College Graduate \\
	                 & or Advanced College Degree                                                            \\
	Region           & Region of the country                                                                 \\
	Job class        & Type of job                                                                           \\
	Health           & Either Good or Less, or Very Good or Better                                           \\
	Health insurance & Yes or No                                                                             \\
	Log of wage      & The log of the worker's wage                                                          \\
	Wage             & The worker's wage                                                                     \\
	\hline
\end{tabular}

The marginal covariate inclusion probabilities are presented in Figure \ref{fig:Wage_inclusion}.

% \begin{tabular}{|l|lllllllllllllllll|}
% 	\hline
% 	$\vp$ & 1 & 1 & 0.010 & 0.024 & 1 & 0.054 & 0.083 & 0.019 & 0.011 & 0.010 & 0.013 & 0.014 & 0.011 & 0.014 & 0.057 & 0.042 & 0.033 \\
% 	$\vq$ & 1 & 1 & 0.010 & 0.024 & 1 & 0.054 & 0.083 & 0.019 & 0.011 & 0.010 & 0.013 & 0.014 & 0.011 & 0.014 & 0.057 & 0.042 & 0.033 \\
% 	\hline
% \end{tabular}

\begin{figure}[p]
	\includegraphics[scale=.4]{Wage_variable_selection.pdf}
	\caption{Wage marginal inclusion probablities}
	\label{fig:Wage_inclusion}
\end{figure}

The model ranking scatterplots are presented in Figure \ref{fig:Wage_model_ranking}.

\begin{figure}[p]
	\includegraphics[scale=.4]{code/Model_selection_scatter_plot_3.pdf}
	\caption{Wage model ranking scatterplot}
	\label{fig:Wage_model_ranking}
\end{figure}


\subsubsection{Graduation Rate}

The Graduation Rate data set was taken from \citep{James:2014:ISL:2517747}. The data is drawn from the 1995
issue of US News and World Report, and is on a large number of US Colleges. The data set consists of 777
observations, and includes the following covariates:

\begin{tabular}{|ll|}
	\hline
	Covariate   & Description                                                             \\
	\hline
	Private     & A factor with levels No and Yes indicating private or public university \\
	Apps        & Number of applications received                                         \\
	Accept      & Number of applications accepted                                         \\
	Enroll      & Number of new students enrolled                                         \\
	Top10perc   & Percentage of new students from top 10\% of H.S. class                  \\
	Top25perc   & Percentage of new students from top 25\% of H.S. class                  \\
	F.Undergrad & Number of fulltime undergraduates                                       \\
	P.Undergrad & Number of parttime undergraduates                                       \\
	Outstate    & Out-of-state tuition                                                    \\
	Room.Board  & Room and board costs                                                    \\
	Books       & Estimated book costs                                                    \\
	Personal    & Estimated personal spending                                             \\
	PhD         & Percentage of faculty with Ph.D.’s                                    \\
	Terminal    & Percentage of faculty with terminal degree                              \\
	S.F.Ratio   & Student/faculty ratio                                                   \\
	perc.alumni & Percentage of alumni who donate                                         \\
	Expend      & Instructional expenditure per student                                   \\
	Grad.Rate   & Graduation rate                                                         \\
	\hline
\end{tabular}

The marginal covariate inclusion probabilities are presented in Figure \ref{fig:GradRate_inclusion}.

% \begin{tabular}{|l|lllllllllllllllll|}
% 	\hline
% 	$\vp$ & 0.913 & 1.000 & 0.090 & 0.108 & 0.110 & 0.602 & 0.127 & 0.945 & 0.999 & 0.999 & 0.201 & 0.864 & 0.262 & 0.105 & 0.146 & 0.977 & 0.437 \\
% 	$\vq$ & 0.914 & 1.000 & 0.090 & 0.108 & 0.110 & 0.602 & 0.127 & 0.945 & 0.999 & 0.999 & 0.201 & 0.864 & 0.262 & 0.105 & 0.146 & 0.977 & 0.437 \\
% 	\hline
% \end{tabular}

\begin{figure}[p]
	\includegraphics[scale=.4]{GradRate_variable_selection.pdf}
	\caption{GradRate marginal inclusion probablities}
	\label{fig:GradRate_inclusion}
\end{figure}

The model ranking scatterplots are presented in Figure \ref{fig:GradRate_model_ranking}.

\begin{figure}[p]
	\includegraphics[scale=.4]{code/Model_selection_scatter_plot_4.pdf}
	\caption{GradRate model ranking scatterplot}
	\label{fig:GradRate_model_ranking}
\end{figure}

\subsubsection{US Crime}
The US Crime data set was taken from the \texttt{MASS} package \citep{Venables2002}, and is on the effect of
punishment regimes on crime rates. The data set includes 47 states of the United States of America. The
variables have been re-scaled for convenience, and include the following covariates:

\begin{tabular}{|ll|}
	\hline
	Covariate & Description                                                     \\
	\hline
	M         & percentage of males aged 14–24.                               \\
	So        & indicator variable for a Southern state.                        \\
	Ed        & mean years of schooling.                                        \\
	Po1       & police expenditure in 1960.                                     \\
	Po2       & police expenditure in 1959.                                     \\
	LF        & labour force participation rate.                                \\
	M.F       & number of males per 1000 females.                               \\
	Pop       & state population.                                               \\
	NW        & number of non-whites per 1000 people.                           \\
	U1        & unemployment rate of urban males 14–24.                       \\
	U2        & unemployment rate of urban males 35–39.                       \\
	GDP       & gross domestic product per head.                                \\
	Ineq      & income inequality.                                              \\
	Prob      & probability of imprisonment.                                    \\
	Time      & average time served in state prisons.                           \\
	y         & rate of crimes in a particular category per head of population. \\
	\hline
\end{tabular}

The marginal covariate inclusion probabilities are presented in Figure \ref{fig:USCrime_inclusion}.

% \begin{tabular}{|l|lllllllllllllll|}
% 	\hline
% 	$\vp$ & 0.226 & 0.849 & 0.997 & 0.216 & 0.502 & 0.244 & 0.358 & 0.569 & 0.324 & 0.202 & 0.424 & 0.696 & 0.869 & 0.229 & 0.655 \\
% 	$\vq$ & 0.220 & 0.856 & 0.997 & 0.210 & 0.507 & 0.240 & 0.358 & 0.573 & 0.318 & 0.196 & 0.418 & 0.699 & 0.876 & 0.224 & 0.661 \\
% 	\hline
% \end{tabular}

\begin{figure}[p]
	\includegraphics[scale=.4]{USCrime_variable_selection.pdf}
	\caption{USCrime marginal inclusion probablities}
	\label{fig:USCrime_inclusion}
\end{figure}

The model ranking scatterplots are presented in Figure \ref{fig:USCrime_model_ranking}.

\begin{figure}[p]
	\includegraphics[scale=.4]{code/Model_selection_scatter_plot_5.pdf}
	\caption{USCrime model ranking scatterplot}
	\label{fig:USCrime_model_ranking}
\end{figure}

\section{Conclusion and Discussion}
\label{sec:conclusion}

The Variational Bayes approximation produces results which are almost identical to the exact likelihood.
The VB metholodology extends naturally to new situations, such as robust model fitting, mixed effects, missing
data, measurement error and splines. We are able to retain high accuracy with less computational overhead than
exact or MCMC.

All of the variational parameters in Algorithm \ref{alg:algorithm_two} except $\tau_g$ depend only on
$\tau_g$ and the fixed quantities $n$, $p$, $R^2$, $\mX$ and $\vy$. This allows us to optimise $\tau_g$
only, and then set the rest of the variational parameters at the end of the algorithm. The fact that only
univariate optimisation is required reduces the computation required to fit our approximation considerably,
which makes this algorithm applicable when speed and/or the ability to parallelise the algorithm are
paramount, such as model selection via structured Variational Bayes.

\appendix
\subsection{Useful Results}	

We first present the following results which will aid us in deriving the expressions required for fully Bayesian
inference over the parameters in the model.

\begin{equation}\label{res:01}
	\int \exp\left\{ -\tfrac{1}{2}\vx^T\mA\vx + \vb^T\vx \right\} d \vx = |2\pi\mSigma|^{1/2} \exp\left\{ \tfrac{1}{2}\vmu^T\mSigma^{-1}\vmu \right\}
\end{equation}
where $\vmu = \mA^{-1}\vb$ and $\mSigma = \mA^{-1}$.
 
If $\vone^T\vy=0$ the $R$-squared statistic can be expressed
\begin{equation} \label{res:02}
	R^2 = \frac{\vy^T\mX(\mX^T\mX)^{-1}\mX^T\vy}{\|\vy\|^2}
\end{equation}
where $R^2$ is the usual $R$-squared statistic associated with least squares regression.

Equation 3.194 (iii) of \citep{Gradshteyn1988} is
\begin{equation}\label{res:03}
	\int_0^\infty \frac{ x^{\mu - 1} }{(1 + \beta x)^\nu} dx = \beta^{-\mu} \mbox{Beta}(\mu,\nu - \mu) \quad \quad \mbox{(assuming $\mu,\nu>0$ and $\nu>\mu$).}
\end{equation}

Equation 3.385 of \citep{Gradshteyn1988} is
\begin{equation} \label{res:04}
	\int_{0}^1 x^{\nu - 1} (1 - x)^{\lambda - 1}(1 - \beta x)^{\varrho} e^{-\mu x} dx = \mbox{Beta}(\nu,\lambda) \Phi_1(\nu,\varrho,\lambda+\nu,-\mu,\beta)
\end{equation}

\noindent provided $\mbox{Re}(\lambda)>0$, $\mbox{Re}(\nu)>0$ and $|\mbox{arg}(1-\beta)|<\pi$.

\section{Derivation of Naive Mean Field Updates}
\label{sec:appendix}
