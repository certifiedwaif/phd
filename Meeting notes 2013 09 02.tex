\documentclass{amsart}
\begin{document}
\section{Meeting notes}
\subsection{Bipolar study}
John Ormerod talked about the bipolar study. Some data has come in, which is survey 
data, categorical and discretised. He wanted to know whether we can add value to
the study.

\subsection{Linear mixed models}
John looked over the code that I wrote to fit linear mixed models with variational
approximations. He said my crossover model with two intercepts and four random
observations was ``dinky'' and that I should fit it to a more substantial example.

He asked me to do some things:
\begin{itemize}
\item Re-organise the code so that the fitting code is in a function in a seperate
source file.
\item Use the code to fit the Orthodont model.
\end{itemize}

The first of these is easy, I just quickly edit the source files and re-organise
them.

The second may be substantially more difficult, depending on the structure of the
Orthodont data.

\subsection{Logistic model}
John took me through how the augmented variables introduced make this model
tractable.

Consider a model $y|\beta \sim \text{}(\phi(X\beta)))$, which is preferred by
Bayesians because of its augmented representation.

The y's are observed, but the $\beta$'s are not.

\[
p(y|\beta) = \Pi_{i=1}^n \Phi(x_i ^T \beta)^{y_i} (1 - \Phi(x_i ^T \beta))^{1 - y_i}
\]

Let each of the i multiplicands be denoted $p(y_i|\beta)$.
Then we can rewrite each multiplicand as
\[
	p(y_i|\beta) = \int I(a_i > 0)^y_i I(a_i < 0) ^{1-y_i} \phi(a_i - x_i^T \beta) d a_i
\]

Proof:
\begin{align*}
p(y_i = 1) &= \int_{-\infty}^{\infty} I(a_i \geq 0) \phi(a_i - x_i^T \beta) \\
&= \int_{0}^{\infty} I(a_i - x_i^T \beta) d a_i
\end{align*}

Make the substitution $z_i = a_i - x_i^T \beta$. Then the integral above becomes
\begin{align*}
\int_{-x_i^T \beta}^{\infty} \phi(z_i) d z_i
&= 1 - \int_{-\infty}^{-x_i^T \beta} \phi(z_i) d z_i \\
&= 1 - \Phi(-x_i^T \beta) \\
&= \Phi(x_i^T \beta)
\end{align*}

\end{document}