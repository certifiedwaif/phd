% %\maketitle

\chapter{Numerical aspects of calculating Bayes factors for linear models using
	mixture $g$-priors
	}



\noindent
In this chapter, we consider the numerical evaluation of Bayes factors for linear models using different mixture g-priors. In particular, we consider hyperpriors for $g$ leading to closed-form expressions for the Bayes factor including the hyper-$g$ and hyper-$g/n$ priors of \cite{Liang2008}, the beta-prime prior of \cite{Maruyama2011}, the robust prior of \cite{Bayarri2012}, and the cake prior of \cite{OrmerodEtal2017}. In particular, we describe how each of these Bayes factors, except for Bayes factor under the hyper-$g/n$ prior, can be evaluated in efficient, accurate and numerically stable manner. We also derive a closed form expression for the Bayes factor under
the hyper-$g/n$ for which we develop a convenient numerical approximation. We implement an R package for Bayesian linear model averaging, and discuss some associated computational issues. We illustrate the advantages of our implementation over several existing packages on several small datasets.


\vfill
{\footnotesize
\noindent	
	This chapter corresponds to the collaborative paper: \\
	Greenaway M.J. \& Ormerod J.T (2018).
	Numerical aspects of calculating Bayes factors for linear models using mixture $g$-priors. Submitted to the Journal of Computational and Graphical Statistics.
}

\newpage 

 
\section{Introduction}

 
There has been a large amount of research in recent years into the appropriate choice of suitable 
and meaningful priors for linear regression models in the context of Bayesian model selection and 
averaging. Specification of the prior structure of these models must be made with great care in 
order for Bayesian model selection and averaging procedures to have good theoretical properties. 
A key problem in this context occurs when the models have differing dimensions and non-common 
parameters where inferences are typically highly sensitive to the choice of priors for the 
non-common parameters due to the Jeffreys-Lindley-Bartlett paradox \citep{Lindley1957,Bartlett1957,OrmerodEtal2017}.
Furthermore, this sensitivity does not necessarily vanish as the sample size 
grows \citep{Kass1995,Berger2001}.  

Bayes factors in the context of linear model selection 
\citep{Zellner1980,
	Zellner1980b,
	Mitchell1988,
	George1993,
	Fernandez2001,
	Liang2008,
	Maruyama2011,
	Bayarri2012}
have received an 
enormous amount of attention. A landmark paper in this field is \cite{Liang2008}.
\cite{Liang2008} considers a particular prior structure for the model parameters. 
In particular they consider a Zellner's $g$-prior \citep{Zellner1980,Zellner1986} 
for the regression coefficients where $g$ is a prior hyperparameter. The parameter $g$
requires special consideration. If $g$ is set to a large constant most of the posterior
mass is placed on the null model, a phenomenon sometimes referred to as Bartlett's paradox.
Due to this problem they discuss previous approaches which set $g$ to a constant, e.g., setting $g=n$ \citep{Kass1995b},  $g=p^2$ \citep{Foster1994},
and $g=\max(n,p^2)$ \citep{Fernandez2001}. However, \cite{Liang2008} showed that 
all of these choices
lead to what they call the information paradox, where the posterior probability of the
true model does not tend to 1 as the sample size grows. Finally, \cite{Liang2008} also consider
a local and global empirical Bayes (EB) procedure for selecting $g$. In these cases \cite{Liang2008}
show that these EB procedures are model selection consistent except when the true model is the null
model (the model containing the intercept only). 

The above problems suggest that a hyperprior should
be placed on $g$. \cite{Bayarri2012} also discuss in some depth desirable properties
priors should have in the context of linear model averaging and selection. 
In this chapter we review the prior structures, specifically the hyperpriors on $g$, 
that lead to closed form expressions of Bayes factors for linear models.
These include the hyper-$g$ prior of \cite{Liang2008}, the beta-prime prior of \cite{Maruyama2011}, and the robust 
prior of \cite{Bayarri2012}, and most recently the cake prior of \cite{OrmerodEtal2017} leads to a Bayes factor which is a simple function of the Bayesian Information Criterion (BIC). We concern ourselves with the efficient, accurate and numerically stable evaluation of Bayes factors, Bayesian model averaging,
and Bayesian model selection  for linear models 
under the above choices of prior structures for the model parameters.


Our main contributions in this paper are as follows.
\begin{enumerate}
	\item To the above list of hyperpriors on $g$ leading to closed form Bayes factors
	we add the  hyper-$g/n$ prior of \cite{Liang2008} for which we
	derive a new closed form expression for the Bayes factor in terms
	of the Appell hypergeometric function.
	
	\item We derive an alternative expression for the Bayes factor when using the
	robust prior of \cite{Bayarri2012} in terms of the Gaussian hypergeometric function.
	
	\item We describe how the  Bayes factors corresponding to the hyper-$g$ prior of \cite{Liang2008} 
	and robust prior of \cite{Bayarri2012} can be calculated in an efficient, accurate and numerically 
	stable manner without the need for special software or approximation.
	
	\item We derive a reasonably accurate approximation for the Appell hypergeometric function
	which can be calculated in an efficient and numerically 
	stable manner when the number of non-zero coefficients in a particular model is strictly greater than 2.
	
	\item We make available a highly efficient and {\it numerically stable} {\tt R}
	package called
	{\tt blma} available
	for exact Bayesian linear model averaging 
	using the above prior structures which is available for download from the following web address.
	
	\begin{center}
		\url{http://github.com/certifiedwaif/blma}
	\end{center}

\end{enumerate}

\noindent 
We demonstrate the advantages of our implementation of exact Bayesian model
averaging over some existing {\tt R}
packages using several small datasets.


The chapter is organised as follows. Section \ref{sec:bma} describes Bayesian model averaging and model
selection for linear models. Section \ref{sec:model} outlines and justifies our chosen model and prior structure for the linear regression model parameters. Section 
\ref{sec:hyperpriors} derives closed form expressions for various marginal likelihoods using 
different hyperpriors for $g$ and, wherever possible, describes how these may be evaluated well numerically.
In Section \ref{sec:implementation}, we discuss details of our implementation which made our implementation 
computationally feasible.
In Section \ref{sec:numerical_g_prior} we perform a series of numerical experiments to show the advantages of our approach. 
Finally, in Section \ref{sec:conclusion} we provide a conclusion and discuss future
directions.
















\section{Bayesian linear model selection and averaging}
\label{sec:bma}



Suppose $\vy = (y_1,\ldots,y_n)^T$ is a response vector of length $n$, $\mX$ is an $n$ by $p$ matrix 
of covariates where we anticipate a linear relationship between $\vy$ and $\mX$, but do not know
which of the columns of $\mX$ are important to the prediction of $\vy$.
Bayesian model averaging seeks to improve prediction by averaging over multiple
predictions over different choices of combinations of predictors.

We consider the linear model for predicting $\vy$ with design matrix $\mX$ via
\begin{equation}
	\label{eq:linearModel}
	\vy | \alpha, \vbeta, \sigma^2 \sim N(\vone\alpha + \mX \vbeta, \sigma^2 \mI),
\end{equation} 


\noindent where $\alpha$ is the model intercept, $\vbeta$ is a coefficient vector of length $p$, 
$\sigma^2$ is the residual variance, and $\mI$ is the $n \times n$ identity matrix. 
Without loss of generality, to simplify later calculations, we will standardize $\vy$ and $\mX$ 
so that $\overline{y} = 0$, 
$\|\vy\|^2 = \vy^T\vy = n$, $\mX_j^T\vone = 0$,  and $\|\mX_j\|^2 = n$ where $\mX_j$ is the $j$th 
column of $\mX$. 


Suppose that we wish to perform Bayesian model selection, model averaging or hypothesis 
testing where we are interested in comparing how different subsets of predictors 
(which correspond to different columns of the matrix $\mX$) have on the response $\vy$. To this end, 
let $\vgamma \in \{0, 1\}^p$ be a binary vector of indicators for the inclusion of the $p$th column 
of $\mX$ in the model where $\mX_\vgamma$ denotes the design matrix formed by including only the 
$j$th column of $\mX$ when $\gamma_j = 1$, and excluding it otherwise. 

In order to keep our exposition as general as possible we will assume a prior structure of
the from 
$p(\alpha,\vbeta_{\vgamma}|\vgamma)p(\vgamma)$ but,
for
the time being, we will leave the specific form of $p(\alpha,\vbeta_{\vgamma}|\vgamma)$ and $p(\vgamma)$ unspecified. 
We adopt a prior on $\vbeta_{-\vgamma}$  
of the form
\begin{equation}
	\label{eq:spikeAndSlab}
	\ds p(\vbeta_{-\vgamma}|\vgamma) = \prod_{j=1}^p \delta(\beta_j;0)^{1-\gamma_j},
\end{equation} 

\noindent where $\delta(x;a)$ is the Dirac delta function with location $a$.  
The prior on $\vbeta_{-\vgamma}$ in (\ref{eq:spikeAndSlab}) is the spike 
in a spike and slab prior where the prior on $\vbeta_{\vgamma}$ is assumed to be flat (the slab). There are
several variants of the spike and slab prior initially used in
\cite{Mitchell1988}
and later refined in
\cite{George1993}. 
The above structure implies 
that $p(\vbeta_{-\vgamma}|\vy)$ is a point mass at $\vzero$
and leads to
algebraic and computational simplifications for components of $\vbeta$ when corresponding elements of $\vgamma$ are zero.
Thus, $\gamma_j=0$ is equivalent to excluding the corresponding predictor $\mX_j$ from the model.


Exact Bayesian model averaging revolves around the posterior
probability of a model $\vgamma$ using Bayes theorem
$$
\ds p(\vgamma|\vy) = \frac{p(\vy|\vgamma)p(\vgamma)}{\sum_{\vgamma'} p(\vy|\vgamma')p(\vgamma')} = \frac{p(\vgamma)\mbox{BF}(\vgamma)}{\sum_{\vgamma'} p(\vgamma')\mbox{BF}(\vgamma')}
\quad \mbox{where} \quad 
p(\vy|\vgamma) = \int p(\vy,\vtheta|\vgamma) \, d\vtheta,
$$

\noindent letting $\vtheta = (\alpha,\vbeta,\sigma^2)$, using $\sum_{\vgamma}$ to denote a combinatorial sum over all
$2^p$ possible values of $\vgamma$, and $\mbox{BF}(\vgamma) = p(\vy|\vgamma)/p(\vy|\vzero)$
is the null based Bayes factor for model $\vgamma$.
Note that the Bayes factor is a statistic commonly used in Bayesian hypothesis testing 
\citep{Kass1995,OrmerodEtal2017}.
Prediction is based on the 
the posterior distributions of $\alpha$ and $\vbeta$ where
$p(\vbeta|\vy) = \sum_{\vgamma} p(\vbeta|\vy,\vgamma) \cdot p(\vgamma|\vy)$
(with similar expressions for $\alpha$ and $\sigma^2$).
The 
posterior expectation of $\vgamma$ is given by
$\bE(\vgamma|\vy) = \sum_{\vgamma} \vgamma \cdot p(\vgamma|\vy)$.

If one is required to select a single model, say $\vgamma^*$, two common choices are the 
highest posterior model (HPM) which uses $\vgamma^* = \vgamma_{\mbox{\tiny HPM}} = \argmax_\vgamma \{ \, p(\vy|\vgamma) \, \}$, or the
median posterior model (MPM) where $\vgamma^*$ is obtained by
rounding each element of $\bE(\vgamma|\vy)$ to the nearest integer.
The MPM has predictive optimality properties \citep{Barbieri2004}.
If the MPM is used for model selection
the quantity $\bE(\vgamma|\vy)$ is sometimes referred to as the
posterior (variable) inclusion probability (PIP) vector.

Ignoring for the moment the problems associated with specifying
$p(\alpha,\vbeta_{-\vgamma},\vgamma)$, all of the above quantities are conceptually
straightforward. In practice the computation of the quantities $p(\vgamma|\vy)$, $p(\vbeta|\vy)$ and 
$\bE(\vgamma|\vy)$ are only feasible for small values of $p$ (say around $p=30$). For large values of $p$ we need to pursue
alternatives to exact inference.



\section{Prior specification for linear model parameters}
\label{sec:model}

We will specify the prior $p(\alpha,\vbeta,\sigma^2|\vgamma)$.
as follows
\begin{equation}
	\label{eq:priorStructure}
	\begin{array}{c}
		\ds p(\alpha) \propto 1,  
		\qquad 
		\vbeta_\vgamma | \sigma^2, g, \vgamma \sim N_p(\vzero, g \sigma^2 (\mX_\vgamma^T \mX_\vgamma)^{-1})
		\quad \text{ and }  \quad 
		\ds p(\sigma^2) \propto (\sigma^2)^{-1},                      
	\end{array}
\end{equation} 

\noindent where we have introduced a new prior hyperparameter $g$.
For the time being we will defer specification of $p(g)$ and $p(\vgamma)$.
We will now justify each element of the above prior structure.

The priors on $\alpha$ and $\sigma^2$ are improper Jeffreys priors and have been justified 
in \cite{Berger1998}. In the context of Bayesian model selection, model averaging or hypothesis 
testing $\alpha$ and $\sigma^2$ appear in all models 
so that when comparing models the proportionality constants in the corresponding
Bayes factors cancel.

The prior on $\vbeta_\vgamma$ is Zellner's $g$-prior \citep[see for example,][]{Zellner1986} with prior 
hyperparameter $g$. This family of priors for a Gaussian regression model where the prior covariance 
matrix of $\vbeta_\vgamma$ is taken to be a multiple of $g$ with the Fisher information matrix for $\vbeta$. 
This places the most prior mass for $\vbeta_\vgamma$ on the section of the parameter space where the data is 
least informative, and makes the marginal likelihood of the model scale-invariant. Furthermore, this 
choice of prior removes a log-determinant of $\mX_\vgamma^T\mX_\vgamma$ term from the expression for the marginal 
likelihood, which is an additional computational burden to calculate.
The prior on $\vbeta_\vgamma$ combined with the prior on $\vbeta_{-\vgamma}$
in (\ref{eq:priorStructure}) constitutes one variant of the spike and slab prior for $\vbeta$.

An alternative choice of prior on $\vbeta_\vgamma$ was proposed by \cite{Maruyama2011}. Let
$p_{\vgamma} = |\vgamma|$, the number of non-zero elements in $\vgamma$. We will now describe their prior on $\vbeta_\vgamma$ for the case where for the case
$p_{\vgamma} < n - 1$. Let $\mU\mLambda\mU^T$ be an eigenvalue decomposition of $\mX_\vgamma^T\mX_\vgamma$
where $\mU$ is an orthonormal $p_{\vgamma} \times p_{\vgamma}$ matrix, and $\mLambda = \mbox{diag}(\lambda_1,\ldots,\lambda_{p_{\vgamma}})$ 
is a diagonal matrix of eigenvalues with $\lambda_1\ge\ldots,\ge \lambda_{p_{\vgamma}}>0$. Then \cite{Maruyama2011} 
propose a prior for $\vbeta_\vgamma$ of the form
\begin{equation}
	\label{eq:priorBetaMG}
	\vbeta_\vgamma | \sigma^2, g \sim N(\vzero, \sigma^2 (\mU\mW\mU^\top)^{-1}),   
\end{equation} 

\noindent where $\mW = \mbox{diag}(w_1,\ldots,w_{p_{\vgamma}})$ with $ w_j = \lambda_j/[\nu_j(1 + g) - 1]$ for 
some prior hyperparameters $\nu_q < \ldots < \nu_1$. \cite{Maruyama2011} suggest as a default 
choice for the $\nu_j$'s to use $\nu_j = \lambda_j/\lambda_{p_{\vgamma}}$, for $1\le j \le p_{\vgamma}$. 
This choice down-weights the prior on the rotated parameter space of $(\mU \vbeta)_j$ when the 
corresponding eigenvalue $\lambda_j$ is large, which leads to prior standard errors that are 
approximately the same size. Note that when $\nu_1 = \ldots = \nu_{p_{\vgamma}} = 1$ the prior 
(\ref{eq:priorBetaMG}) reduces to the prior for $\vbeta$ in (\ref{eq:priorStructure}). 

The choice between (\ref{eq:priorBetaMG}) and the prior for $\vbeta$ in (\ref{eq:priorStructure}) 
represents a trade-off over computational efficiency and desirable statistical properties. We choose
(\ref{eq:priorStructure}) because it avoids the computational burden of calculating an eigenvalue or a singular 
value decomposition of a $p_{\vgamma}\times p_{\vgamma}$ matrix for every model considered,
which typically costs $O(p_{\vgamma}^3)$. 
It also means that we can 
exploit efficient matrix updates to traverse the entire model space in a computationally efficient 
manner allowing this to be done feasibly when $p$ is less than around 30 on a standard 2017 laptop 
(see Section \ref{sec:implementation} for details).


The marginal likelihood for the
model  (\ref{eq:linearModel}) and under prior structure
(\ref{eq:priorStructure}). 
%Integrating out $\alpha$ and $\vbeta$ from 
%$p(\vy,\alpha,\vbeta|\sigma^2,g)$ we find
%\begin{equation}\label{eq:yGivenSigma2andG}
%\begin{array}{rl}
%\ds p(\vy|\sigma^2,g)
%& \ds = \int \exp\left[
%- \tfrac{n}{2}\log(2\pi\sigma^2) 
%- \tfrac{1}{2\sigma^2}\|\vy - \vone\alpha - \mX\vbeta\|^2
%- \tfrac{p}{2}\log(2\pi g\sigma^2) 
%+ \tfrac{1}{2}\log|\mX^T\mX|
%- \tfrac{1}{2g\sigma^2}\vbeta^T\mX^T\mX\vbeta  
%\right] d\alpha d\vbeta
%\\
%& \ds = \int \exp\left[
%- \tfrac{n}{2}\log(2\pi\sigma^2) 
%- \tfrac{n}{2\sigma^2}
%- \tfrac{n\alpha^2}{2\sigma^2} 
%+ \sigma^{-2}\vy^T\mX\vbeta
%- \tfrac{1}{2\sigma^2}(1 + g^{-1})\vbeta^T\mX^T\mX\vbeta 
%- \tfrac{p}{2}\log(2\pi g\sigma^2) 
%+ \tfrac{1}{2}\log|\mX^T\mX|
%\right] d\alpha d\vbeta
%\\
%& \ds = \int \exp\left[
%- \tfrac{n-1}{2}\log(2\pi\sigma^2) 
%- \tfrac{1}{2}\log(n)
%- \tfrac{n}{2\sigma^2}
%+ \sigma^{-2}\vy^T\mX\vbeta
%- \tfrac{1}{2\sigma^2}(1 + g^{-1})\vbeta^T\mX^T\mX\vbeta 
%- \tfrac{p}{2}\log(2\pi g\sigma^2) 
%+ \tfrac{1}{2}\log|\mX^T\mX|
%\right]  d\vbeta
%\\
%& 
%\ds = \exp\left[
%- \tfrac{n-1}{2}\log(2\pi\sigma^2) 
%- \tfrac{1}{2}\log(n)
%- \tfrac{p}{2}\log(1 + g)
%- \tfrac{n}{2 \sigma^2} \left( 1 - \tfrac{g}{1 + g} R^2 \right)  
%\right],
%\end{array} 
%\end{equation}
%
%\noindent 
%\joc{
%	Derivation of the above expression uses the identity
%	$
%	\int \exp\left\{ -\tfrac{1}{2}\vx^T\mA\vx + \vb^T\vx \right\} d \vx = |2\pi\mSigma|^{1/2} \exp\left\{ \tfrac{1}{2}\vmu^T\mSigma^{-1}\vmu \right\}
%	$
%	where $\vmu = \mA^{-1}\vb$, and $\mSigma = \mA^{-1}$.
%	It also uses the identities: $|c\mA| = c^d|\mA|$ and $|\mA^{-1}| = |\mA|^{-1}$ when $\mA \in\R^{d\times d}$.
%}
Integrating out $\alpha$, $\vbeta$, and $\sigma^2$ from $p(\vy,\alpha,\vbeta,\sigma^2|g,\vgamma)$ we
obtain
\begin{equation}\label{eq:yGivenG}
	\begin{array}{rl}
		\ds p(\vy|g,\vgamma)
		%& \ds = \int \exp\left[
		%- \tfrac{n-1}{2}\log(2\pi) 
		%- \tfrac{1}{2}\log(n)
		%- \tfrac{p}{2}\log(1 + g)
		%- \left( \tfrac{n-1}{2} + 1\right)\log(\sigma^2) 
		%- \left( \tfrac{n}{2} \tfrac{1 + g(1-R^2)}{1 + g} \right)\sigma^{-2} 
		%\right]  d\sigma^2
		%\\
		%& 
		\ds = K(n)
		(1 + g)^{(n - p_\vgamma - 1)/2}(1 + g (1 - R_\vgamma^2))^{-(n-1)/2},
	\end{array} 
\end{equation}

\noindent where $K(n) = [\Gamma( (n-1)/2 )]/[\sqrt{n}(n\pi)^{(n-1)/2}]$, and
$R_\vgamma^2 = \vy^T\mX_\vgamma^T(\mX_\vgamma^T\mX_\vgamma)^{-1}\mX_\vgamma^T\vy/n$ is 
the the usual R-squared statistic for model $\vgamma$.
This is the same expression as \cite{Liang2008} Equation (5) 
after simplification. Note that
when $\vgamma = \vzero$, i.e., the null model, then $p_\vgamma = 0$, and
$R_\vgamma^2 = 0$ leading to the simplification $p(\vy|g,\vzero) = K(n)$
for all $g$. Hence, $p(\vy|\vzero) = K(n)$ provided the hyperprior for $g$ is a proper density. We will now discuss the specification of $g$.


\section{Hyperpriors on $g$}
\label{sec:hyperpriors}

Here we outline some of the choices of hyperpriors for $g$ used in the literature, their
properties, and where possible how to implement these in an efficient, 
accurate, and
numerically stable manner. We cover the 
the hyper-$g$ and hyper-$g/n$ priors of \cite{Liang2008}, the beta-prime prior
of \cite{Maruyama2011}, the robust prior of \cite{Bayarri2012}, and the cake
prior of \cite{OrmerodEtal2017}.
We also considered the prior structure implied by \cite{Zellner1980}, but were able to make no
meaningful progress on existing methodology for this case.

We show that many of the hyperpriors on $g$ result in Bayes factors which can be expressed
in terms of the Gaussian hypergeometric function denoted 
${}_2F_1(\,\cdot\,,\,\cdot\,;\,\cdot\,;\,\cdot\,)$ \citep[see for example Chapter 15 of ][]{Abramowitz1972}.
The Gaussian hypergeometric function is notoriously prone 
to overflow and numerical instability \citep{Pearson2017}. 
When such numerical issues arise 
\cite{Liang2008} derive a Laplace approximation to ${}_2F_1$ implemented in the {\tt R} package {\tt BAS}.
Key to achieving accuracy, efficiency and numerical stability for several different mixture $g$-priors is the following result.

 
\noindent 
{\bf Result 1:} {\it For $x\in(0,1)$, $c>1$, and $b +1 > c$ we have}
\begin{equation}\label{eq:logGuassHypergeometric2}
	\ds {}_2F_1(a+b,1;a+1;x) = \frac{a}{x(1 - x)}   \frac{\mbox{pbeta}(x,a,b)}{\mbox{dbeta}(x,a,b)},
\end{equation}

\noindent 
{\it where} $\mbox{pbeta}(x;a,b)$ {\it and} $\mbox{dbeta}(x;a,b)$ {\it are the cdf and pdf of the beta 
	distribution respectively.}

 
\noindent 
{\bf Proof:} Using identity 2.5.23 of \cite{Abramowitz1972} the cdf of the beta distribution
can be written as
$$
\mbox{pbeta}(x;a,b) = \frac{x^a}{a\mbox{Beta}(a,b)} \cdot {}_2F_1(a,1-b;a+1;x) 
$$

\noindent where 
$\mbox{Beta}(a,b)$ is the beta function.
Using the Euler transformation
${}_2 F_1(a,b;c,x) = (1 - x)^{c-a-b} {}_2 F_1(c-a,c-b;c,x)$,
and the fact that ${}_2 F_1(a,b;c,x)={}_2 F_1(b,a;c,x)$,  we obtain
$$
\mbox{pbeta}(x;a,b) = \frac{x^a(1 - x)^{b}}{a\mbox{Beta}(a,b)} \cdot {}_2F_1(a+b,1;a+1;x). 
$$

\noindent Lastly, after rearranging we obtain Result 1.
\vspace{-0.5cm}\begin{flushright}$\Box$\end{flushright}
%$$
%{}_2F_1(a+b,1;a+1;x)  = \frac{\mbox{pbeta}(x;a,b)a\mbox{Beta}(a,b)}{x^a(1 - x)^b} = \frac{a}{x(1-x)}\frac{\mbox{pbeta}(x;a,b)}{\mbox{dbeta}(x;a,b)}
%$$

\noindent Numerical overflow can be avoided
since standard libraries exist for evaluating $\mbox{pbeta}(x,a,b)$ and $\mbox{dbeta}(x,a,b)$
on the log scale. Recently, \cite{Nadarajah2015} stated an equivalent result
originally derived in \cite{PrudnikovEtal1986}. 




\subsection{The hyper-$g$ prior}

\noindent 
Initially, \cite{Liang2008} suggest the hyper $g$-prior where
\begin{equation}\label{eq:hyperG}
	\ds p_{g}(g) = \frac{a - 2}{2}(1 + g)^{-a/2},
\end{equation}

\noindent for $a>2$ and $g>0$. Combining (\ref{eq:yGivenG}) with (\ref{eq:hyperG}), we have
\begin{equation}\label{eq:hyperGmarginalIntegral}
	p_{g}(\vy|\vgamma) = K(n) \frac{a - 2}{2}  \int_0^\infty 
	\left( 1 + g \right)^{-a/2}
	(1 + g)^{(n-p_\vgamma-1)/2} \left[ 1 + g (1 - R_\vgamma^2) \right]^{-(n-1)/2}  dg.
\end{equation}

\noindent After applying 
3.197(5) of \cite{Gradshteyn2007}, i.e.,
\begin{equation}\label{eq:31975}
	\ds 
	\int_0^\infty x^{\lambda - 1}(1 + x)^\nu (1 + \alpha x)^\mu dx
	=\mbox{Beta}(\lambda,-\mu-\nu-\lambda){}_2F_1(-\mu,\lambda;-\mu-\nu; 1 - \alpha),
\end{equation}
\noindent (which holds provided $-(\mu  + \nu) > \lambda > 0$),
leads to
\begin{equation}\label{eq:hyperGmarginal}
	\ds \mbox{BF}_{g}(\vgamma) = \frac{p_{g}(\vy|\vgamma)}{p_{g}(\vy|\vzero)} =  \left( \frac{a - 2}{p_\vgamma + a - 2} \right) \cdot {}_2F_1\left( \frac{n-1}{2}, 1; \frac{p_\vgamma + a}{2}; R_\vgamma^2 \right).
\end{equation}

\noindent Using Result 1 the Bayes factor under the hyper-$g$ prior can be written as
\begin{equation}\label{eq:hyperGmarginal2}
	\ds \mbox{BF}_{g}(\vgamma) 
	=  
	\frac{a - 2}{2 R_\vgamma^2(1 - R_\vgamma^2)} 
	\frac{\mbox{pbeta}\left(R_\vgamma^2,\tfrac{p_\vgamma + a - 2}{2},\tfrac{n-p_\vgamma - a+1}{2}\right)}{
		\mbox{dbeta}\left(R_\vgamma^2,\tfrac{p_\vgamma + a - 2}{2},\tfrac{n-p_\vgamma - a+1}{2}\right)}.
\end{equation}


\noindent 
Unfortunately, \cite{Liang2008} also showed that
(\ref{eq:hyperGmarginal}) is not model selection consistent when the
true model is the null model (the model only containing the intercept) and so alternative hyperpriors for $g$ should be used.




\subsection{The hyper-$g/n$ prior}

Given the problems with the hyper-$g$ prior, \cite{Liang2008} 
proposed a modified variant of the hyper-$g$ prior which uses
\begin{equation}\label{eq:hyperGonN}
	\ds p_{g/n}(g) = \frac{a - 2}{2n}\left( 1 + \frac{g}{n} \right)^{-a/2},
\end{equation}

\noindent which they call the hyper-$g/n$ prior where again $a>2$ and $g>0$.
They show that this prior leads to model selection consistency.
Combining (\ref{eq:yGivenG}) with (\ref{eq:hyperGonN}), and using the transform
$g = u/(1 - x)$, the quantity $p(\vy|\vgamma)$ 
can be expressed as the integral
\begin{equation}\label{eq:hyperGonNmarginalIntegral}
	\begin{array}{rl}
		p_{g/n}(\vy|\vgamma) 
		%& \ds 
		%= K(n) \frac{a - 2}{2n}  \int_0^\infty 
		%\left( 1 + \frac{g}{n} \right)^{-a/2}
		%(1 + g)^{(n-p_\vgamma-1)/2} \left[ 1 + g (1 - R_\vgamma^2) \right]^{-(n-1)/2}  dg
		%\\ [2ex]
		& \ds = K(n) \frac{a - 2}{2n}  \int_0^1 
		(1 - u)^{p/2 + a/2 - 2  } \left(  1 - u \left(1  -  \tfrac{1}{n} \right) \right)^{-a/2} \left(  1 - u R^2\right)^{-(n-1)/2} du.
	\end{array} 
\end{equation}

\noindent  Employing 
Equation 3.211 of \cite{Gradshteyn2007}, i.e.,
$$
\int_0^1 x^{\lambda-1}(1 - x)^{\mu - 1}(1 - u x)^{-\delta}(1 - vx)^{-\sigma} dx = \mbox{Beta}(\mu,\lambda) F_1(\lambda,\delta,\sigma,\lambda+\mu;u,v) 
%F_1(a,b_1,b_2,c; x,y) = \frac{\Gamma(c)} {\Gamma(a)\Gamma(c-a)} 
%\int_0^1 t^{a-1} (1-t)^{c-a-1} (1-xt)^{-b_1} (1-yt)^{-b_2} \, dt,
$$

\noindent provided $\lambda>0$ and $\mu>0$ where $F_1$ is the Appell hypergeometric function in two variables
\citep{Weisstein2009} leads to
\begin{equation}\label{eq:hyperGonNmarginal}
	\ds \mbox{BF}_{g/n}(\vgamma) =  \frac{a - 2}{n(p_\vgamma + a - 2)} F_1\left( 1, \frac{a}{2}, \frac{n-1}{2}; \frac{p_\vgamma + a}{2}; 1  -  \frac{1}{n}, R_\vgamma^2 \right),
\end{equation}

\noindent which is to our knowledge a new expression for the Bayes factor under the hyper $g/n$-prior.


Unfortunately, the expression (\ref{eq:hyperGonNmarginal}) is extremely
difficult to evaluate numerically since the second last argument of the above 
$F_1$ is asymptotically close to the radius of convergence of the $F_1$
function.
\cite{Liang2008} again suggest Laplace approximation 
for this choice of prior. We now derive an alternative approximation.
Using the fact that
$$
F_1(1,b_1,b_2,c; 1,y) 
= (c - 1)
\int_0^1  (1-t)^{c-b_1-2} (1-yt)^{-b_2} \, dt
= (c - 1) \frac{\, _2F_1(1,b_2;c-b_1;y)}{c-b_1-1}
$$

\noindent and the approximation $F_1(1,b_1,b_2,c; 1-1/n,y)  \approx F_1(1,b_1,b_2,c; 1,y)$
(which should be reasonable for large $n$),
for $p_\vgamma > 2$ we obtain
\begin{equation}\label{eq:hyperGonNmarginalApprox}
	%\begin{array}{rl}
	\ds \mbox{BF}_{g/n}(\vgamma) 
	%& \ds =  \frac{a - 2}{n(p_\vgamma - 2)} 
	%\, _2F_1\left( \frac{n-1}{2}, 1;  \frac{p_\vgamma}{2}; R_\vgamma^2 \right)
	%\\
	%& \ds 
	\approx    
	\frac{a - 2}{2n R_\vgamma^2(1 - R_\vgamma^2)}   \frac{
		\mbox{pbeta}\left( R_\vgamma^2, \frac{p_\vgamma-2}{2}, \frac{n-p_\vgamma+1}{2} \right)
	}{
		\mbox{dbeta}\left( R_\vgamma^2, \frac{p_\vgamma-2}{2}, \frac{n-p_\vgamma+1}{2} \right)
	}.
	%\end{array}
\end{equation}

\noindent For the cases where $p\in \{1,2\}$ we
will use numerical quadrature. When $p=0$, we also have that $R_\vgamma^2= 0$) so $\mbox{BF}_{g/n}(\vgamma) = 1$.
Figure \ref{fig:gonnapprox} illustrates the differences between ``exact'' values of the $\mbox{BF}_{g/n}$
(obtained using numerical quadrature) as a function of $n$, $p_\vgamma$, and $R^2$. From this figure we
see that the approximation has a good relative error except for values close to 1 when the approximation
overestimates the true value of the log Bayes factor. We found numerical quadrature to be more reliable
than using (\ref{eq:hyperGonNmarginal}) evaluated using the {\tt appell} function in the package 
{\tt Appell}.


\begin{figure}
	\centering
	\includegraphics[width=0.9\linewidth]{gOnNapprox}
	\caption{On the left side panels are plotted the values of log of $\mbox{BF}_{g/n}$ (light versions of the
		colours) and their corresponding approximation (dark version of the colours) 
		as a function of $n$, $p$ over a the range $R^2\in(0,0.999)$. Right side panels display
		the exact values of log of $\mbox{BF}_{g/n}$ minus the corresponding approximations.}
	\label{fig:gonnapprox}
\end{figure}


\subsection{Robust prior}  

\noindent
Next we will consider the robust hyperprior for $g$ as proposed by \cite{Bayarri2012} designed to have several nice theoretical
properties outlined there. Using the default parameter
choices the hyperprior 
for $g$ used by \cite{Bayarri2012} corresponds to:
\begin{equation}\label{eq:robustPrior}
	p_{{rob}}(g) = \tfrac{1}{2}r^{1/2} (1 + g)^{-3/2},
\end{equation}

\noindent for $g>L$  where $L = r - 1$ and $r = (1 + n)/(1 + p_\vgamma)$. 
Combining (\ref{eq:yGivenG}) with (\ref{eq:robustPrior})
leads to an expression for $p(\vy|\vgamma)$ of the form
\begin{equation}\label{eq:marginalLikelihoodRobust}
	\ds p_{rob}(\vy|\vgamma)
	\ds = K(n) \tfrac{1}{2} r^{1/2} 
	\int_L^\infty  (1 + g)^{(n - p_\vgamma)/2 - 2}(  1 + g \widehat{\sigma}_\vgamma^2)^{-(n-1)/2} dg,
\end{equation}

\noindent where $\widehat{\sigma}_\vgamma^2 = 1 - R_\vgamma^2$ is the MLE
for $\sigma^2$ for model (\ref{eq:linearModel}) when $\mX$ is replaced with $\mX_\vgamma$ under the standardization described in Section 2.
Using the substitution 
$x = r/(g - L)$ and some minor algebraic
manipulation leads to
%$$
%x = r/(g - L),
%\quad 
%g = L + r/x
%\quad 
%d g = -r/x^2 dx
%\quad 
%\lim_{g\to L_+} x = \infty
%\quad 
%\lim_{g\to \infty} x = 0
%$$
%
$$
%\begin{array}{rl}
\ds \mbox{BF}_{{rob}}(\vgamma)
%& \ds = \tfrac{1}{2} r^{3/2} 
%\int_0^\infty x^{-2}   (r + r/x)^{(n - p_\vgamma - %4)/2}(  1 + \widehat{\sigma}_\vgamma^2  (L + %r/x))^{-(n-1)/2}  dx
%\\ 
%& 
\ds = \tfrac{1}{2} r^{ - p_\vgamma/2} (\widehat{\sigma}_\vgamma^2)^{-(n-1)/2}
\int_0^\infty x^{(p_\vgamma-1)/2} 
(1 + x)^{(n - p_\vgamma - 4)/2}
\left( 1 + \tfrac{(1 + \widehat{\sigma}_\vgamma^2 L)x}{(1 + L)\widehat{\sigma}_\vgamma^2}  \right)^{-(n-1)/2}  dx.
%\end{array} 
$$



\noindent 
Using Equation 3.197(5) of \cite{Gradshteyn2007}, i.e. (\ref{eq:31975}), 
%\begin{equation}\label{eq:31975}
%\ds 
%\int_0^\infty x^{\lambda - 1}(1 + x)^\nu (1 + \alpha x)^\mu dx
%=\mbox{Beta}(\lambda,-\mu-\nu-\lambda){}_2F_1(-\mu,\lambda;-\mu-\nu; 1 - \alpha),
%\end{equation}
%\noindent (which holds provided $-(\mu  + \nu) > \lambda > 0$).
%More specifically we use 
with the mappings
$$
\lambda \leftrightarrow \frac{p_\vgamma+1}{2},
\quad 
\nu \leftrightarrow \frac{n - p_\vgamma - 4}{2},
\quad 
\alpha \leftrightarrow \frac{(1 + \widehat{\sigma}_\vgamma^2 L)}{(1 + L)\widehat{\sigma}_\vgamma^2},
\quad \mbox{and} \quad 
\mu \leftrightarrow -\frac{n-1}{2},
$$

\noindent the conditions required by (\ref{eq:31975}) are satisfied
provided $\alpha \in (-1,1)$ (which is a relatively restrictive condition). 
%Checking the conditions we have $\lambda > 0$ since $p_\vgamma \ge 0$.
%The second condition implies 
%$$
%\begin{array}{l}
%\ds -(\mu  + \nu) > \lambda
%\\
%\ds \qquad \Rightarrow \qquad \frac{n-1}{2} - \frac{n - p_\vgamma - 4}{2} > \frac{p_\vgamma+1}{2}
%\\
%\ds \qquad \Rightarrow \qquad \frac{p_\vgamma + 3}{2}   > \frac{p_\vgamma+1}{2}
%
%\end{array} 
%$$
%
%\noindent which always holds. 
This leads to
\begin{equation}\label{eq:yGivenGammaRobust}
	%\begin{array}{rl}
	\ds \mbox{BF}_{{rob}}(\vgamma)
	%& \ds = \tfrac{1}{2} r^{ - p_\vgamma/2} (\widehat{\sigma}_\vgamma^2)^{-(n-1)/2}
	%\mbox{Beta}\left( \frac{p_\vgamma+1}{2}, 1 \right)
	%{}_2F_1\left(\frac{n-1}{2},\frac{p_\vgamma+1}{2};\frac{p_\vgamma + 3}{2}; 
	%\frac{(1 + L)\widehat{\sigma}_\vgamma^2  - (1 + \widehat{\sigma}_\vgamma^2 L)}{r\widehat{\sigma}_\vgamma^2} \right)
	%\\
	%& \ds 
	= \left( \tfrac{n + 1}{ p_\vgamma + 1} \right)^{ - p_\vgamma/2} \tfrac{(\widehat{\sigma}_\vgamma^2)^{-(n-1)/2}}{p_\vgamma+1}
	{}_2F_1\left( \tfrac{n-1}{2}, \tfrac{p_\vgamma+1}{2}; \tfrac{p_\vgamma+3}{2}  ; 
	\tfrac{(1  - 1/\widehat{\sigma}_\vgamma^2)(p_\vgamma + 1)}{1 + n}  \right),
	%\end{array} 
\end{equation}


\noindent 
which is the same expression as Equation 26 of \cite{Bayarri2012}
modulo notation.

The expression (\ref{eq:yGivenGammaRobust}) is difficult to deal with numerically for two reasons. Firstly, if either of the first 
two arguments of the ${}_2F_1$ function are large relative to the third this will often lead to numerical overflow problems. Secondly,
and more problematically, when $\widehat{\sigma}_\vgamma^2$ becomes small the last argument
of ${}_2F_1$ function can become less than $-1$ which falls outside the radius of convergence
of the ${}_2F_1$ function. The {\tt BayesVarSel} package which implements
this choice of prior deals with these problems using numerical
quadrature.

Instead suppose we begin with the substitution $x = g - L$ which after minor algebraic manipulation
leads to
$$
%\begin{array}{rl}
\ds \mbox{BF}_{{rob}}(\vgamma)
%& \ds = \tfrac{1}{2} r^{1/2} 
%\int_0^\infty  (1 + L + x)^{(n - p_\vgamma - 4)/2}(  1 + \widehat{\sigma}_\vgamma^2(x + L) )^{-(n-1)/2} dx,
%\\
%& \ds 
= \tfrac{1}{2} r^{1/2} \left( \widehat{\sigma}_\vgamma^2\right)^{-(n-1)/2} 
\int_0^\infty  (r + x)^{(n - p_\vgamma-4)/2}
\left(  \tfrac{1 +  \widehat{\sigma}_\vgamma^2L}{\widehat{\sigma}_\vgamma^2} +  x \right)^{-(n-1)/2} dx.
%\end{array} 
$$

\noindent Employing Equation 3.197(1) of \cite{Gradshteyn2007}, i.e.,
$$
\int_0^\infty x^{\nu - 1}(\beta + x)^{-\mu}(x + \gamma)^{-\varrho} dx
= \beta^{-\mu}
\gamma^{\nu - \varrho} 
\mbox{Beta}(\nu,\mu - \nu + \varrho)
{}_2F_1(\mu,\nu;\mu+\varrho; 1 - \gamma/\beta),
$$

\noindent (which holds provided $\nu>0$, $\mu > \nu - \varrho$), with the mappings
$$
\nu \leftrightarrow 1,
\quad 
\beta \leftrightarrow \frac{1 +  \widehat{\sigma}_\vgamma^2L}{\widehat{\sigma}_\vgamma^2},
\quad 
\mu \leftrightarrow (n-1)/2
\quad 
\gamma \leftrightarrow r
\quad \mbox{and} \quad 
\varrho \leftrightarrow -(n - p_\vgamma-4)/2,
$$

\noindent The conditions of the integral result easily hold.
%
%\noindent The condition $\nu>0$ holds. The condition
%$\mu > \nu - \varrho$ requires
%$$
%\begin{array}{l}
%(n-1)/2 > 1 + (n - p_\vgamma-4)/2
%\quad \Rightarrow \quad p_\vgamma > - 1,
%\end{array}  
%$$
%\noindent so that the second condition also holds. 
%Hence,
%$$
%\begin{array}{rl}
%\ds p_{rob}(\vy|\vgamma)
%& \ds = K(n) \tfrac{1}{2} r^{1/2} \left( \widehat{\sigma}_\vgamma^2\right)^{-(n-1)/2} 
%\left( \frac{1 +  \widehat{\sigma}_\vgamma^2L}{\widehat{\sigma}_\vgamma^2} \right)^{-(n-1)/2}
%r^{1 + (n - p_\vgamma-4)/2} 
%\\
%& \ds \qquad \times 
%\mbox{Beta}\left( 1, \frac{n-1}{2} - 1 - \frac{n - p_\vgamma - 4}{2} \right)
%{}_2F_1\left( \frac{n-1}{2}, 1; \frac{n-1}{2} - \frac{n - p_\vgamma - 4}{2}; 1 - \frac{1 + L}{\frac{1 +  \widehat{\sigma}_\vgamma^2L}{\widehat{\sigma}_\vgamma^2}} \right),
%
%\\
%& \ds = K(n) \frac{1}{2} r^{(n - p_\vgamma-1)/2} 
%\left(  1 +  \widehat{\sigma}_\vgamma^2L  \right)^{-(n-1)/2}
%\mbox{Beta}\left( 1, \frac{p_\vgamma+1}{2} \right)
%{}_2F_1\left( \frac{n-1}{2}, 1; \frac{p_\vgamma + 3}{2}; \frac{1  - \widehat{\sigma}_\vgamma^2}{1 +  \widehat{\sigma}_\vgamma^2L} \right),
%
%\\
%
%& \ds = K(n) \frac{r^{(n - p_\vgamma-1)/2}}{1 + p_\vgamma} 
%\left(  1 +  \widehat{\sigma}_\vgamma^2L  \right)^{-(n-1)/2}
%{}_2F_1\left( \frac{n-1}{2}, 1; \frac{p_\vgamma + 3}{2}; \frac{1  - \widehat{\sigma}_\vgamma^2}{1 +  \widehat{\sigma}_\vgamma^2L} \right),
%
%\end{array} 
%$$
%$$
%\ds p_{{rob}}(\vy|\vgamma)
%\ds = \frac{K(n)}{2}\left(\frac{1+n}{1 + p_\vgamma}  \right)^{1/2} %(\widehat{\sigma}_\vgamma^2)^{-(n-1)/2}
%\int_0^\infty  (1 + L + h)^{(n - p_\vgamma)/2 - 2}\left[  \frac{1 + %L\widehat{\sigma}_\vgamma^2}{\widehat{\sigma}_\vgamma^2} + h \right]^{-(n-1)/2} dh.
%$$
%
%\noindent 
Hence, after some algebraic manipulation and applying Result 1, and letting
$\widetilde{R}_\vgamma^2 = R_\vgamma^2/(1 + L\widehat{\sigma}_\vgamma^2)$ we obtain
\begin{equation}\label{eq:yGivenGammaRobust2}
	%\begin{array}{rl}
	\ds \mbox{BF}_{{rob}}(\vgamma)
	%& \ds = \left( \frac{1 + n}{1 + p_\vgamma} \right)^{(n - p_\vgamma - 1)/2} \frac{\left( 1 + L\widehat{\sigma}_\vgamma^2 \right)^{-(n - 1)/2}}{1 + p_\vgamma}
	%{}_2F_1\left(  
	%\frac{n-1}{2}, 1; \frac{p_\vgamma+3}{2}; \frac{1 - \widehat{\sigma}_\vgamma^2}{1 + L\widehat{\sigma}_\vgamma^2}
	% \right)
	%\\
	%& \ds 
	= \left( \frac{1 + n}{1 + p_\vgamma} \right)^{(n - p_\vgamma - 1)/2} \frac{\left( 1 + L\widehat{\sigma}_\vgamma^2 \right)^{-(n - 1)/2}}{2 \widetilde{R}_\vgamma^2(1 - \widetilde{R}_\vgamma^2)} 
	\frac{
		\mbox{pbeta}\left( 
		\widetilde{R}_\vgamma^2,
		\frac{p_\vgamma +1}{2},
		\frac{n - p_\vgamma - 2}{2} 
		\right)
	}{
		\mbox{dbeta}\left( 
		\widetilde{R}_\vgamma^2,
		\frac{p_\vgamma +1}{2},
		\frac{n - p_\vgamma - 2}{2} 
		\right)
	}.
	%\end{array}
\end{equation}


%Then
%$(1 + L\widehat{\sigma}_\vgamma^2) = %R_\vgamma^2/\widetilde{R}_\vgamma^2$.


\noindent 
This expression is numerically far easier to evaluate efficiently and accurately in a numerically stable manner. Due to simplifications
we have $0\le \widehat{\sigma}_\vgamma^2<1$, we also have $L>0$ so that the last argument
of the ${}_2F_1$ above is bounded in the unit interval.  


%$$
%\int_0^\infty x^{\nu - 1}(\beta + x)^{-\mu}(x + \gamma)^{-\varrho} dx
%= \beta^{-\mu}\gamma^{\nu - %\varrho}\mbox{Beta}(\nu,\mu-\nu + \varrho)
%{}_2F_1(\mu,\nu;\mu + \varrho; 1 - %\gamma/\beta)
%$$

%$$
%\beta = \frac{1 + %L\widehat{\sigma}_\vgamma^2}{\widehat{\sigma}_\vgamma^2}
%$$
%$$
%\gamma = 1 + L
%$$
%$$
%\nu = 1
%$$
%$$
%\mu = \frac{n-1}{2}
%$$
%$$
%\varrho = - (n - p_\vgamma - 4)/2
%$$




\subsection{Beta-prime prior} 

\noindent
Next we will consider the prior 
\begin{equation}\label{eq:betaPrime}
	\ds p_{bp}(g) = \frac{g^{b}(1 + g)^{-(a+b+2)}}{\mbox{Beta}(a+1,b+1)},
\end{equation}

\noindent proposed by \cite{Maruyama2011} where $g>0$, $a>-1$ and $b>-1$. 
This is a Pearson Type VI or beta-prime distribution. More specifically, 
$g\sim \mbox{Beta-prime}(b+1,a+1)$ using the usual parametrization of 
the beta-prime distribution \citep{Johnson1995}.
Then combining (\ref{eq:yGivenG}) with (\ref{eq:betaPrime}) the quantity $p(\vy|\vgamma)$ 
can be expressed as the integral
$$
%\begin{array}{rl}
\ds p_{bp}(\vy|\vgamma) 
%& \ds = \int_0^\infty                                         
%\frac{g^{b}(1 + g)^{-a-b-2}}{\mbox{Beta}(a+1,b+1)}
%K(n) (1 + g)^{(n - p - 1)/2}\left[ 1 + g(1-R^2) \right]^{-(n-1)/2}
%dg
%\\
%& \ds 
=
\frac{K(n)}{\mbox{Beta}(a+1,b+1)}
\int_0^\infty             
g^{b}(1 + g)^{(n - p_\vgamma - 1)/2 - (a + b + 2)}  (1 + g (1-R_\vgamma^2) )^{-(n-1)/2}  
dg.
%\end{array}
$$

\noindent If we choose 
%$b$ such that
%$a+b+2 = (n - p_\vgamma - 1)/2$, implying
$b = (n - p_\vgamma - 5)/2 - a$, then the exponent of the $(1 + g)$ term in the equation above is zero.
Using Equation 3.194 (iii) of \cite{Gradshteyn2007}, i.e.,
$$
\int_0^\infty \frac{ x^{\mu - 1} }{(1 + \beta x)^\nu} dx = \beta^{-\mu} \mbox{Beta}(\mu,\nu - \mu),
$$

\noindent provided $\mu,\nu>0$ and $\nu>\mu$, we obtain
\begin{equation}\label{eq:marginalLikelihoodBetaPrime}
	\begin{array}{rl}
		\ds \mbox{BF}_{bp}(\vy|\vgamma) 
		%& \ds =
		%\frac{K(n)}{\mbox{Beta}(a+1,b+1)}
		%\int_0^\infty g^{b} \left[ 1 + g(1-R^2) \right]^{-(n-1)/2}  
		%dg
		%\\ [2ex]
		& \ds 
		=   
		\frac{\mbox{Beta}(p/2 + a + 1,b + 1)}{\mbox{Beta}(a+1,b+1)} (1-R_\vgamma^2)^{-(b + 1)}
		%\\ [2ex]
		%& \ds = \widetilde{K}(n,a)
		%
		%\Gamma(p/2 + a + 1)\Gamma(a + b + 2)
		%(\widehat{\sigma}^2)^{-(b + 1)}
	\end{array}
\end{equation}

\noindent which is a simplification of the Bayes factor proposed by
\cite{Maruyama2011}.


Note that (\ref{eq:marginalLikelihoodBetaPrime}) is proportional
to a special case of the prior structure considered by \cite{Maruyama2011}
who refer to this as a model selection criterion (after Zellner's $g$ prior). This choice of $b$ also ensures that $g = O(n)$ so that $\mbox{tr}\{\mbox{Var}(\vbeta | g, \sigma^2)\} = O(1)$, preventing Bartlett's paradox. 
Note that in comparison to previously discussed priors
marginal likelihood only involves gamma functions which
are well behaved from a numerical analysis perspective. 
\cite{Maruyama2011} showed the prior (\ref{eq:betaPrime}) leads to model
selection consistency.
For derivation of the above properties and further discussion see \cite{Maruyama2011}.




\subsection{BIC via cake priors}  

\noindent
\cite{OrmerodEtal2017} develops cake priors which allow for arbitrarily diffuse priors
while avoiding Bartlett's paradox leading Bayes factors equal to the exponential of minus half
the BIC. Cake priors
can be thought of as a Jefferys prior in the limit
as the prior becomes increasingly diffuse
and enjoy nice theoretical properties including model
selection consistency. \cite{OrmerodEtal2017} 
departs from the prior structure (\ref{eq:priorStructure}) and instead uses
\begin{equation}\label{eq:proirs2}
	\ds \alpha|\sigma^2,g \sim N(0,g\sigma^2), \quad 
	\ds \vbeta_\vgamma|\sigma^2,g \sim N\left( \vzero,g\sigma^2\left( \tfrac{1}{n}\mX_\vgamma^T\mX_\vgamma\right)^{-1}\right)
	\quad \mbox{and} \quad
	p(g|\vgamma_j) = \delta(g; h^{1/(1 + p_\vgamma)})
\end{equation}

\noindent where $h$ is a common prior hyperparameter for all models. After marginalizing
out $\alpha$, $\vbeta$, $\sigma^2$ and $g$ the null based Bayes factor 
for model $\vgamma$ is of the form
$$
\begin{array}{rl}
\ds \log\mbox{BF}(\vgamma;h)
=
-\tfrac{n}{2}\log\left( 1 - \tfrac{h^{1/(1+p_\vgamma)}}{1+h^{1/(1+p_\vgamma)}} R_\vgamma^2 \right) 
- \tfrac{p_\vgamma}{2}\log\left(n + h^{-1/(1+p_\vgamma)} \right).
\end{array}
$$

\noindent Taking $h\to\infty$ we obtain a null based Bayes factor of
\begin{equation}\label{eq:marginalLikelihoodCake}
	\ds \mbox{BF}(\vgamma)
	=
	\exp\left[ \,
	-\tfrac{n}{2}\log\left( 1 - R_\vgamma^2 \right) 
	- \tfrac{p_\vgamma}{2}\log\left(n \right) \,
	\right] = \exp\left[ \, -\tfrac{1}{2}\mbox{BIC}(\vgamma) \,\right]
\end{equation}

\noindent where $\mbox{BIC}(\vgamma) = n\log\left( 1 - R_\vgamma^2 \right) + p_\vgamma \log(n)$. 
%Note that as $h\to\infty$ the parameter %posteriors become
%$$\alpha|\vy,\vgamma \sim %t_n(0,\widehat{\sigma}_{\vgamma}^2/n), \quad
%\vbeta_{\vgamma}|\vy,\vgamma \sim t_n( %\widehat{\vbeta}_{\vgamma}, \widehat{\sigma}_{\vgamma}^2 \left(\mX_\vgamma^T\mX_\vgamma  \right)^{-1} ),
%\quad \mbox{and} \quad  
%\sigma^2|\vy,\vgamma \sim \mbox{IG}\left( \tfrac{n}{2}, \tfrac{n}{2}\widehat{\sigma}_{\vgamma}^2 \right),
%$$

%\noindent 
%where $\widehat{\vbeta}_{\vgamma}$
%and $\widehat{\sigma}_{\vgamma}^2$ are the
%MLEs corresponding to model $\vgamma$.


\section{Implementation}
\label{sec:implementation}

\noindent
Key to the feasibility of the model selection and averaging is an efficient implementation of these procedures. We employ two main 
strategies to achieve computational efficiency (i) efficient software implementation using
highly optimized software libraries; and (ii) efficient calculation of
$R$-squared values for all models based on using a Gray code and appropriate
matrix algebraic simplifications.
For ease of use we 
implemented an {\tt R} package called {\tt blma}.
The internals of {\tt blma} are implemented
in {\tt C++} and use the {\tt R} packages \texttt{Rcpp} and \texttt{RcppEigen} to enhance
computational performance. The library {\tt OpenMP} was used to
exploit parallel computation.

There are two main special functions used in the paper -- the 
Gaussian hypergeometric function, and the Appell hypergeometric function
of two variables. During the 
implementation process we tried several packages which implemented the
Gaussian hypergeometric function.
We found that the {\tt R} package {\tt gsl} \citep{Hankin2006} was the most accurate, numerically
stable implementation amongst the packages we tried. The {\tt R} package {\tt appell}
implements the Appell
hypergeometric function \citep{Bove2013}. We also developed our own numerical quadrature routine to evaluate the Appell
hypergeometric function to check our results.



\subsection{Gray code} 
\label{sec:GrayCode}

\noindent
The Gray code was originally developed by Frank Gray in 1947 \cite[][Section 22.3]{PressEtal2007} to aid in detecting errors in analog to digital conversions in
communications systems. It is a sequence of binary numbers whose key feature is that
one and only one binary digit is different between binary numbers in the sequence. 
Gray codes can be constructed using a sequence of ``reflect'' and ``prefix'' steps.
Let $\mGamma_1 = (0,1)^T \in \{0,1\}^{2\times 1}$ be the first Gray code matrix and let $\mGamma_k$ be the $k$th Gray code matrix. Then we can obtain the $(k+1)$th Gray code matrix given $\mGamma_k$ via 
$$
\ds \mGamma_{k+1} = \left[\begin{array}{cc}
\vzero & \mGamma_k \\
\vone  & \mbox{reflect}(\mGamma_k)
\end{array} \right]
$$ 

\noindent where $\mbox{reflect}(\mGamma_k)$ is the matrix obtained by reversing the order of rows of $\mGamma_k$, and the $\vzero$ and $\vone$ are vectors of zeros and ones
of length $2^k$ respectively. In {\tt C} and {\tt C++} these Gray codes can be efficiently constructed
using bit-shift operations on binary strings in such a way that $\mGamma_{k}$
matrices are never computed and stored explicitly.

Gray codes allow the enumeration of the entire model space in an order which only adds
or removes one covariate from the previous model at a time. We can then use standard matrix
inverse results to perform rank
one updates and downdates in the calculation of the $R^2$, $(\mX^T\mX)^{-1}$ and
$\widehat{\vbeta}$ values for each model in the
model space.


\subsection{Model updates and downdates} 

\noindent
Both updates and downdates depend on the fact that
the inverse of a real symmetric matrix can be written as
\begin{eqnarray}
	\ds \left[ \begin{array}{cc}
		\mA   & \mB \\
		\mB^T & \mC
	\end{array} \right]^{-1}
	&  = &
	\ds \left[ \begin{array}{cc}
		\mI & \vzero \\
		-\mC^{-1}\mB^T &  \mI
	\end{array} \right]
	\left[ \begin{array}{cc}
		\widetilde{\mA} & \vzero \\
		\vzero & \mC^{-1}
	\end{array} \right]
	\left[ \begin{array}{cc}
		\mI    & -\mB\mC^{-1}\\
		\vzero & \mI
	\end{array} \right] \label{eq:blockdiag1}\\
	&  = &
	\ds\left[
	\begin{array}{cc}
		\widetilde{\mA}
		& - \widetilde{\mA}\mB\mC^{-1} \\
		-\mC^{-1}\mB^T\widetilde{\mA}
		& \mC^{-1} + \mC^{-1}\mB^T\widetilde{\mA}\mB\mC^{-1}
	\end{array}\right]\label{eq:blockdiag2}
\end{eqnarray}

\noindent where $\widetilde{\mA} = \left(\mA-\mB\mC^{-1}\mB^T\right)^{-1}$
provided all inverses in (\ref{eq:blockdiag1}) and
(\ref{eq:blockdiag2}) exist. 
For both the update and downdate formula we assume that the quantities
$\mX^T\vy$, $\mX^T\mX$ have been precalculated, and that the $(\mX_{\vgamma_i}^T\mX_{\vgamma_i})^{-1}$, 
$\widehat{\vbeta}_{\vgamma_i}$ and $R_{\vgamma_i}^2$ values have been computed from the previous step.

We want to update the model inverse matrix, coefficient vector and $R^2$ values for the model $\vgamma_{i+1}$ where $\mX_{\vgamma_{i+1}}$ is the matrix given by $\mX_{\vgamma_i}$ with a column $\vz$ inserted into the appropriate position.
For clarity of exposition we will assume that the column $\vz$ is located in the last column of $\mX_{\vgamma_{i+1}}$, i.e., $\mX_{\vgamma_{i+1}} = [\mX_{\vgamma_{i}},\vz]$. This can be achieved, if necessary, by appropriate permuting  columns of various matrices.

The updates for the model inverse matrix, coefficient estimates, and $R^2$ values can be obtained by following the steps bellow.
\begin{enumerate}
	\item Calculate $\widehat{\vz} = (\mX_{\vgamma_i}^T\mX_{\vgamma_i})^{-1}\mX_{\vgamma_i}^T\vz$, 
	$\kappa 
	%= 
	%1/(\vz^T(\mI - \mX_{\vgamma_i}(\mX_{\vgamma_i}^T\mX_{\vgamma_i})^{-1}\mX_{\vgamma_i}^T)\vz) 
	= 1/(n - \vz^T\widehat{\vz})$, and  $s = \vy^T(\vz - \widehat{\vz})$.
	
	\item The model inverse matrix can be updated via  
	%The update for the $(\mX_{\vgamma_{i+1}}^T\mX_{\vgamma_{i+1}})^{-1}$ using $(\mX_{\vgamma_{i}}^T\mX_{\vgamma_{i}})^{-1}$ is
	%given by the following. 
	$$
	\begin{array}{rl}
	%\left[ \begin{array}{cc}
	%\mX^T\mX & \mX^T\vz \\
	%\vz^T\mX & \vz^T\vz \\
	%\end{array} \right]^{-1}
	(\mX_{\vgamma_{i+1}}^T\mX_{\vgamma_{i+1}})^{-1}
	%& \ds = 
	%\left[ \begin{array}{cc}
	%(\mX^T\mX)^{-1} + \kappa(\mX^T\mX)^{-1}\mX^T\vz\vz^T\mX(\mX^T\mX)^{-1}  & %-(\mX^T\mX)^{-1}\mX^T\vz \kappa \\
	%-\kappa \vz^T\mX(\mX^T\mX)^{-1}              
	%& \kappa 
	%\end{array} \right]
	%\\
	%& \ds = 
	%\left[ \begin{array}{cc}
	%(\mX^T\mX)^{-1} + \kappa\widehat{\vz}\widehat{\vz}^T  & - \widehat{\vz} \kappa \\
	%-\kappa \widehat{\vz}^T             
	%& \kappa 
	%\end{array} \right]
	%\\
	&\ds = 
	\left[ \begin{array}{cc}
	(\mX_{\vgamma_{i}}^T\mX_{\vgamma_{i}})^{-1}    & \vzero \\
	\vzero             
	& 0
	\end{array} \right] + \kappa \left[ \begin{array}{r}
	\widehat{\vz} \\
	-1 \\
	\end{array} \right] \left[ \begin{array}{r}
	\widehat{\vz} \\
	-1 \\
	\end{array} \right]^T.
	\end{array} 
	$$
	
	\item
	The coefficient estimators 
	%are given by
	$ 
	\ds \widehat{\vbeta}_{\vgamma_{i}} = (\mX_{\vgamma_{i}}^T\mX_{\vgamma_{i}})^{-1}\mX_{\vgamma_{i}}^T\vy$,
	and $\ds \widehat{\vbeta}_{\vgamma_{i+1}}  = (\mX_{\vgamma_{i+1}}^T\mX_{\vgamma_{i+1}})^{-1}\mX_{\vgamma_{i+1}}^T\vy$.  
	Then using the block inverse formula we have
	the relation
	$$
	\begin{array}{rl}
	\ds \widehat{\vbeta}_{\vgamma_{i+1}}
	%& \ds = \left[ \begin{array}{c}
	%\widehat{\vbeta} \\
	%0 
	%\end{array} \right] + \kappa  \left[ \begin{array}{cc}
	%(\mX^T\mX)^{-1}\mX^T\vz \left\{ \vz^T\mX(\mX^T\mX)^{-1}\mX^T\vy - \vz^T\vy\right\}  \\
	%\vz^T\vy - \vz^T\mX(\mX^T\mX)^{-1}\mX^T\vy
	%\end{array} \right]
	%\\
	& \ds 
	= \left[ \begin{array}{c}
	\widehat{\vbeta}_{\vgamma_{i}} \\
	0 
	\end{array} \right] - \kappa s  \left[ \begin{array}{r}
	\widehat{\vz}   \\
	- 1
	\end{array} \right].
	\end{array} 
	$$
	
	\item The $R^2$ value let 
	$R_{\vgamma_{i}}^2 = \tfrac{1}{n} \vy^T\mX_{\vgamma_{i}}(\mX_{\vgamma_{i}}^T\mX_{\vgamma_{i}})^{-1}\mX_{\vgamma_{i}}^T\vy$.
	Then using the block inverse formula we have
	$$
	\begin{array}{rl}
	\ds 
	R_{\vgamma_{i+1}}^2 
	%& \ds = \tfrac{1}{n} \vy^T\mX(\mX^T\mX)^{-1}\mX^T\vy
	%+ \tfrac{\kappa}{n}\left[ 
	%\widehat{\vy}^T\vz\vz^T\widehat{\vy}
	%- 2\widehat{\vy}^T\vz\vz^T\vy 
	%+ \vy^T\vz\vz^T\vy 
	%\right]
	= R_{\vgamma_{i}}^2
	+ \frac{\kappa s^2}{n}.
	
	\end{array}
	$$
	
	%\item
	%\noindent The model determinants are given by
	%$D = |\mX^T\mX|$
	%and
	%$D_{\mbox{\tiny update}} = |\mC^T\mC|$.
	%Using the block determinant formula we have
	%$D_{\mbox{\tiny update}} = D/c$.
\end{enumerate}

\noindent Presuming relevant summary quantities have been precomputed
the above updates costs $O(p_{\vgamma_{i}}^2 + n)$ time.

Suppose want to downdate the model summary quantities for the model
$\vgamma_{i+1}$ where $\mX_{\vgamma_{i+1}}$ is the matrix given by 
$\mX_{\vgamma_i}$ with a column $\vz$ removed from the appropriate position.
Similarly as for updates for clarity of exposition we will assume that
$\vz$ will be removed from the last column of $\mX_{\vgamma_i}$, i.e., we assume that
$\mX_{\vgamma_{i}} = [\mX_{\vgamma_{i+1}}, \vz]$. 
Again, this can be achieved by permuting the columns of various matrices.
Then the downdates for model summary values are given by the following steps.
\begin{enumerate}
	\item 
	%The downdate for the model inverse matrix to obtain $(\mX_{\vgamma_{i+1}}^T\mX_{\vgamma_{i+1}})^{-1}$
	%from $(\mX_{\vgamma_{i}}^T\mX_{\vgamma_{i}})^{-1}$ can be found using the block-inverse formula.
	Suppose we partition the matrix
	$(\mX_{\vgamma_{i}}^T\mX_{\vgamma_{i}})^{-1}$ so that
	$$
	\ds (\mX_{\vgamma_{i}}^T\mX_{\vgamma_{i}})^{-1} 
	= \left[ \begin{array}{cc}
	\mA   & \vb \\
	\vb^T & c \\
	\end{array} \right].
	%= 
	%\left[ \begin{array}{cc}
	%\mX^T\mX & \mX^T\vz \\
	%\vz^T\mX & \vz^T\vz \\
	%\end{array} \right]^{-1}
	$$
	
	\noindent Calculate the model inverse matrix  
	by   $(\mX_{\vgamma_{i+1}}^T\mX_{\vgamma_{i+1}})^{-1} = \mA - c^{-1}\vb\vb^T$.
	
	\item Calculate
	$\widehat{\vz} = (\mX_{\vgamma_{i+1}}^T\mX_{\vgamma_{i+1}})^{-1}\mX_{\vgamma_{i+1}}^T\vz$,
	$\kappa = 1/(n - \vz^T\widehat{\vz})$,
	and $s = \vy^T(\vz - \widehat{\vz})$.
	
	\item 
	%Let $\widehat{\vbeta}_{\vgamma_{i+1}} = (\mX_{\vgamma_{i+1}}^T\mX)^{-1}\mX_{\vgamma_{i+1}}^T\vy$
	%and $\widehat{\vbeta}_{\vgamma_{i}} = %(\mX_{\vgamma_{i}}^T\mX_{\vgamma_{i}})^{-1}\mX_{\vgamma_{i}}^T\vy$.
	%Then
	The coefficient estimates downdate can be obtained
	via
	$$
	\widehat{\vbeta}_{\vgamma_{i+1}} = \left[ \widehat{\vbeta}_{\vgamma_{i}} \right]_{-|{\vgamma_{i}}|} + \kappa s\widehat{\vz},
	$$
	
	\noindent where $[ \widehat{\vbeta}_{\vgamma_{i}}]_{-|{\vgamma_{i}}|}$
	removes the last column from $\widehat{\vbeta}_{\vgamma_{i}}$.
	
	\item 
	%Let $R_{\vgamma_{i}}^2 = \tfrac{1}{n}\vy^T\mX_{\vgamma_{i}}(\mX_{\vgamma_{i}}^T\mX_{\vgamma_{i}})^{-1}\mX_{\vgamma_{i}}^T\vy$
	%and $R_{\vgamma_{i+1}}^2 = \tfrac{1}{n} \vy^T\mX_{\vgamma_{i+1}}(\mX_{\vgamma_{i+1}}^T\mX_{\vgamma_{i+1}})^{-1}\mX_{\vgamma_{i+1}}^T\vy$.
	The $R^2$ downdate can be obtained
	via
	$$
	R_{\vgamma_{i+1}}^2 = R_{\vgamma_{i}}^2 - \frac{\kappa s^2}{n}.
	$$
	
	
	%\item Let  $D = |\mC^T\mC|$ and $D_{\mbox{\tiny downdate}} = |\mX^T\mX|$ then
	%$D_{\mbox{\tiny downdate}} = cD$.
\end{enumerate}

\noindent Again, presuming relevant summary quantities have been precomputed
the updates for all of the above quantities costs $O(p_{\vgamma_{i}}^2 + n)$ time.





\section{Numerical results}
\label{sec:numerical_g_prior}

We will now compare the different Bayes factors under different hyperpriors on $g$ that we have explored. 
Firstly we will look at these Bayes factors by comparing them directly.
We will then compare the results based on exact Bayesian linear model
averaging on some available datasets.


\subsection{Numerical comparison of $g$ hyperpriors}

Note that
each of the Bayes factors is a function of three quantities $R^2$, $p_\vgamma$ 
and $n$. Figure \ref{fig:bayesfactors} illustrates various log Bayes factors
over a grid of $p_\vgamma$ values from 1 to 20 and $R^2\in\{0.1,0.5,0.9\}$
and $n \in \{100,500,1000\}$. In the context of Bayesian hypothesis testing
values above the $y$-axis value 0 indicate that the alternative model is
preferred, while lines below 0 indicate the null model is preferred. Note that
cake priors (BIC) have the strongest penalty for larger $p_\vgamma$, followed
by the beta-prime prior (ZE), the robust prior, hyper-$g/n$ prior and lastly
the hyper-$g$ prior. Increasing $n$ and/or $R^2$ leads to all of the different
Bayes factors becoming increasingly close to one another. We also see that 
the {\tt appell()} function becomes unstable as $n$ and/or $R^2$ becomes large.
For the Bayes factor corresponding to the hyper-$g/n$ prior our approximation
tracks very closely to the methods using the {\tt appell()} function and our
numerical quadrature approach.

\begin{figure}[h!]
	\centering
	\includegraphics[width=0.99\linewidth]{BayesFactors}
	\caption{Cake prior or BIC (black), 
		beta-prime prior (blue), 
		hyper-$g$ prior (red),
		robust prior (green),
		hyper-$g/n$ ({\tt appell} - solid orange),
		hyper-$g/n$ (quadrature - dashed orange), 
		and hyper-$g/n$ (approximation - dotted orange). The grey line corresponds to the Bayes factor equal to 1. Above the grey line the alternative model is preferred, below the grey line the null model is preferred.}
	\label{fig:bayesfactors}
\end{figure}

\subsection{Settings for {\tt R} packages} 

We will now compare three different popular {\tt R} implementations of Bayesian model averaging on 
several small datasets. We compare the {\tt R} packages {\tt BAS} \citep{Clyde2017}, 
{\tt BayesVarSelect} \citep{Garcia-Donato2016}, and  {\tt BMS} \citep{Zeugner2015}. For each method
we assumed a uniform prior on the model space, i.e. $p(\vgamma)\propto 2^{-p}$. We used the
setting implied by the following commands for each of these methods.
\begin{itemize}
	\item {\tt BAS}: We used the command
	\begin{verbatim}
	bas.lm(y~X,prior=prior.val,modelprior=uniform())
	\end{verbatim}
	
	where \verb|prior.val| takes the value \verb|"hyper-g"|, \verb|"hyper-g-laplace"| or \verb|"hyper-g-n"|.
	These correspond to a direct implementation of (\ref{eq:hyperGmarginal}),
	a Laplace approximation of (\ref{eq:hyperGmarginalIntegral}), and the 
	Laplace approximation of (\ref{eq:hyperGonNmarginalIntegral}) respectively. The
	value $a=3$ is implicitly used.
	
	\item {\tt BayesVarSelect}: We used the command
	\begin{verbatim}
	Bvs(formula="y~.",data=data.frame(y=y,X=X),prior.betas=prior.val,
	prior.models="Constant",time.test=FALSE,n.keep=50000)
	\end{verbatim}
	
	
	\noindent where \verb|prior.val| takes the value \verb|"Liangetal"| or \verb|"Robust"|.
	These 
	correspond to a direct implementation of (\ref{eq:hyperGmarginal}) with $a=3$, and a hybrid approach which
	uses (\ref{eq:yGivenGammaRobust}) directly and numerical quadrature based on (\ref{eq:marginalLikelihoodRobust}) if this fails respectively. 
	Again, the value
	$a=3$ is implicitly used.
	
	\item {\tt BMS}: We used the command
	\begin{verbatim}
	bms(cbind(y,X),nmodel=50000,mcmc="enumerate",g="hyper=3",
	mprior="uniform")	
	\end{verbatim}
	
	\noindent which uses a direct implementation of (\ref{eq:hyperGmarginal}) for the hyper-$g$
	prior with $a=3$.
\end{itemize}

\noindent 
The syntax for {\tt blma} is relatively straightforward:

\begin{verbatim}
blma(vy, mX, prior, mprior, cores = 1L)
\end{verbatim}

\noindent where the arguments of {\tt blma}
are explained below.
\begin{itemize}
	\item {\tt vy} -- a vector of length $n$ of responses (this vector does not need to be standardized).
	
	\item {\tt mX} -- a design matrix with $n$ rows and $p$ columns (the columns of ${\tt mX}$ do not need to be standardized).
	
	\item {\tt prior} -- the choice of mixture $g$-prior used to perform Bayesian model averaging. The choices available include:
	\begin{itemize}
		\item {\tt "BIC"} -- the Bayesian information criterion obtained by using the cake prior 
		of \cite{OrmerodEtal2017}. 
		
		\item {\tt "ZE"} -- special case of the prior structure in \cite{Maruyama2011}.
		
		\item {\tt "liang\_g1"} -- the mixture $g$-prior of \cite{Liang2008} with prior hyperparameter $a=3$
		evaluated directly using (\ref{eq:hyperGmarginal}) where the Gaussian hypergeometric function is evaluated using the {\tt gsl} library. Note: this option can lead to numerical problems and is only meant to be used for comparative purposes.
		
		\item {\tt "liang\_g2"} -- the mixture $g$-prior of \cite{Liang2008} with prior hyperparameter $a=3$
		evaluated directly using (\ref{eq:hyperGmarginal2}).
		
		\item {\tt "liang\_g\_n\_appell"} -- the mixture $g/n$-prior of \cite{Liang2008} with prior hyperparameter $a=3$ evaluated using the {\tt appell R} package.
		
		\item {\tt "liang\_g\_approx"} -- the mixture $g/n$-prior of \cite{Liang2008} with prior hyperparameter $a=3$ using the approximation (\ref{eq:hyperGonNmarginalApprox}) for $p_\vgamma >2$ and
		numerical quadrature (see below) ofr $p_\vgamma\in \{1,2\}$.
		
		\item {\tt "liang\_g\_n\_quad"} -- the mixture $g/n$-prior of \cite{Liang2008} with prior hyperparameter $a=3$ evaluated using a composite trapezoid rule.
		
		\item {\tt "robust\_bayarri1"} -- the robust prior of \cite{Bayarri2012}
		using default prior hyperparameter choices evaluated directly using 
		(\ref{eq:yGivenGammaRobust}) with the {\tt gsl} library.
		
		\item {\tt "robust\_bayarri2"} -- the robust prior of \cite{Bayarri2012}
		using default prior hyperparameter choices evaluated directly using
		(\ref{eq:yGivenGammaRobust2}).
		
	\end{itemize}
	\item {\tt mprior} -- the prior to be imposed on the model space. The choices available include:
	\begin{itemize}
		\item {\tt "uniform"} -- corresponds to the prior $p(\vgamma) = 2^{-p}$ where $p$ is the number of columns of $\mX$, .i.e.,
		a uniform prior on the model space.
		
		\item {\tt "beta-binomial"} -- corresponds to a prior of the form
		$$
		\ds p(\vgamma) = \prod_{j=1}^p \rho^{\gamma_j} (1 - \rho)^{1 - \gamma_j} \qquad \mbox{and} \qquad \rho \sim \mbox{Beta}(a,b),
		$$
		
		\noindent where $\rho$ is the prior probability a variable is included in the mode, and $a$ and $b$ are fixed prior hyperparameters. After marginalizing out $\rho$ we have
		$$
		p(\vgamma) = \frac{\mbox{Beta}(a + |\vgamma|,b + p - |\vgamma|)}{\mbox{Beta}(a,b)},
		$$
		
		\noindent which is a beta-binomial distribution. Note $a=b=1$ corresponds to a uniform prior on the prior
		variable inclusion probability. The values of $a$ and $b$ should be set to be the
		first and second elements of the {\tt modelpriorvec} argument respectively (see below).
		
		\item {\tt "bernoulli"} -- corresponds to a prior of the form 
		$$
		p(\vgamma) = \prod_{j=1}^p \rho_j^{\gamma_j} (1 - \rho_j)^{1 - \gamma_j}
		$$
		
		\noindent where the $\rho_j\in(0,1)$. The $\rho_j$ values are specified by
		{\tt modelpriorvec} (see below). Using $\rho_j = 1/2$, $1\le j\le p$ corresponds to 
		{\tt mprior=="uniform"}.
	\end{itemize}
	
	\item {\tt modelpriorvec} -- A vector of additional parameters. If {\tt mprior=="uniform"} this argument is ignored.
	If {\tt mprior=="beta-binomial"} this should be a postive vector of length 2 corresponding to the shape parameters
	of a Beta distribution (the values $a$ and $b$ above). If {\tt mprior=="bernoulli"} this should be a vector of length $p$ 
	with values on the interval $(0,1)$.
	
	\item {\tt cores} -- the number of computer cores to use.
\end{itemize}

\noindent 
The object returned is a 
list containing:
\begin{itemize}
	\item 
	{\tt vR2} -- the vector $R$-square values for each model; 
	
	\item 
	{\tt vp\_gamma} -- the vector of number of covariates for each model;
	
	\item 
	{\tt vlogp} -- the vector of logs of the marginal likelihoods of each model; and
	
	\item 
	{\tt vinclusion\_prob} -- the vector of posterior inclusion probabilities for each of the covariates. 
\end{itemize}

\noindent 
Note that we do not return the fitted values of 
$\widehat{\vbeta}_{\vgamma}$ which should only be calculated
for a subset of models. We also do not return $\mGamma$, the Gray code matrix 
which we provide a separate function to calculate. We made the decisions
not to return these quantities to reduce the memory overhead.

A short example fitting the {\tt USCrime} data described in Section 
\ref{sec:BLMA} is found below.

\begin{verbatim}
library(blma); library(MASS)
dat <- UScrime
dat[,-c(2,ncol(UScrime))] <- log(dat[,-c(2,ncol(UScrime))])
vy <- dat$y
mX <- data.matrix(cbind(dat[1:15]))
colnames(mX) <- c("log(AGE)","S","log(ED)","log(Ex0)","log(Ex1)",
"log(LF)","log(M)","log(N)","log(NW)","log(U1)","log(U2)","log(W)",
"log(X)","log(prison)","log(time)") 
blma_result <- blma(vy, mX, prior="ZE")
\end{verbatim}

\noindent Results for the above example are summarised as part of the result within Section 
\ref{sec:BLMA}.

\subsection{Bayesian linear model averaging on data}\label{sec:BLMA}

We considered several small datasets to illustrate our methodology. These datasets
can be found in the {\tt R} packages {\tt MASS} \citep{Venables2002} and 
{\tt Ecdat} \citep{Croissant2016}. Table \ref{tab:datasets} summarizing the sizes,  
sources, and response variable used for each dataset used. 
We chose {\tt USCrime} data because it is used in most papers in
the area and is small enough so that n\"aive implementations using special functions will
not lead to numerical issues. The 
%{\tt Hitters} nad
{\tt Kakadu} dataset is chosen to
be large enough to begin to strain the resources of a typical 2018 laptop so that
relative differences in speeds between different packages becomes apparent. Finally, the
{\tt Kakadu} dataset is chosen to lead numerical instability in the direct evaluation of Bayes
factors for some of the priors on $g$ considered in this paper.

\begin{table}[h]
	\begin{center}
		\begin{tabular}{l|r|r|l|l}
			Dataset	& $n$ & $p$ & Response & {\tt R} package \\ 
			\hline 
			UScrime 	& 47 & 15 & y & {\tt MASS} \\  
			%Bodyfat	& 244  & 13 &  \\ 
			%	\hline 
			%Hitters	& 263 & 19 & Salary & {\tt ISLR} \\ 
			%	\hline 
			%Wage	& 3000 & 17 &  {\tt ISLR}  \\
			VietNamI	& 27765 & 11 & lnhhexp & {\tt Ecdat}  \\ 
			Kakadu	& 1827 & 22 & income & {\tt Ecdat}   \\  
		\end{tabular} 
	\end{center}
	\caption{A summary of the datasets used in the paper and their respective {\tt R} packages.}
	\label{tab:datasets}
\end{table}

For each of the datasets some minimal preprocessing was used.
We first used the {\tt R} command {\tt na.omit()} to remove samples containing missing predictors. 
For {\tt USCrime} all variables except the predictor {\tt S} were log-transformed. For all datasets
the {\tt R} command {\tt model.matrix()} was used to construct the design matrix using all 
variables except for the response as predictors.

Tables \ref{tab:UScrime}, \ref{tab:VietNamI}, and \ref{tab:Kakadu} summarise the times and variable 
inclusion probabilities, i.e., $\bE(\vgamma|\vy)$, for all of the mixture $g$-prior structures we have considered here under a uniform prior
on the model space.  
All times are based on running {\tt R} code on a dedicated server with 48 cores,
each running at 2.70GHz, with a total of 512GB of RAM.
The {\tt BVS} package in the table refers to the {\tt BayesVarSelect} {\tt R} package where we have used a this acronym to save space in the tables. 




\begin{sidewaystable}[h!]
	\begin{center}
		{\scriptsize 
			\begin{tabular}{c|r|r|rrrrrr|rrrr|rrr}
				Package & blma   & blma   & BAS    & BAS     & BVS    & BMS    & blma & blma & BAS & blma & blma & blma & BVS & blma & blma \\ 
				Prior   & BIC    & ZE     & $g$    & $g$     & $g$    & $g$    & $g$  & $g$ &  $g/n$ & $g/n$ & $g/n$ & $g/n$ & Robust & Robust & Robust \\ 
				Method  & (\ref{eq:marginalLikelihoodCake})  & (\ref{eq:marginalLikelihoodBetaPrime}) 
				& (\ref{eq:hyperGmarginal}) & Laplace & (\ref{eq:hyperGmarginal}) & (\ref{eq:hyperGmarginal}) & (\ref{eq:hyperGmarginal}) & (\ref{eq:hyperGmarginal2}) & Laplace & 
				{\tt appell} & quad. & (\ref{eq:hyperGonNmarginalApprox}) & (\ref{eq:yGivenGammaRobust}) & (\ref{eq:yGivenGammaRobust}) & (\ref{eq:yGivenGammaRobust2}) \\ 
				\hline
				1 & 70.87 & 65.51 & 65.93 & 65.99 & 64.74 & 65.93 & 65.93 & 65.93 & 65.14 & 65.10 & 65.10 & 65.72 & 64.74 & NaN & 64.74 \\ 
				2 & 19.06 & 22.88 & 25.52 & 25.54 & 24.51 & 25.52 & 25.52 & 25.52 & 22.93 & 22.91 & 22.91 & 22.47 & 24.51 & NaN & 24.51 \\ 
				3 & 92.07 & 86.91 & 86.23 & 86.28 & 85.59 & 86.23 & 86.23 & 86.23 & 86.54 & 86.51 & 86.51 & 87.24 & 85.59 & NaN &  85.59 \\ 
				4 & 72.53 & 69.65 & 69.20 & 69.22 & 69.02 & 69.20 & 69.20 & 69.20 & 69.52 & 69.51 & 69.51 & 69.89 & 69.02 & NaN &  69.02 \\ 
				5 & 37.01 & 42.36 & 44.61 & 44.61 & 44.08 & 44.61 & 44.61 & 44.61 & 42.53 & 42.52 & 42.52 & 41.88 & 44.08 & NaN &  44.08 \\ 
				6 & 15.82 & 20.18 & 23.06 & 23.08 & 22.04 & 23.06 & 23.06 & 23.06 & 20.27 & 20.26 & 20.26 & 19.73 & 22.04 & NaN &  22.04 \\ 
				7 & 27.06 & 32.43 & 34.55 & 34.55 & 34.08 & 34.55 & 34.55 & 34.55 & 32.59 & 32.59 & 32.59 & 32.00 & 34.08 & NaN &  34.08 \\ 
				8 & 60.64 & 56.91 & 57.34 & 57.39 & 56.47 & 57.34 & 57.34 & 57.34 & 56.66 & 56.63 & 56.63 & 57.07 & 56.47 & NaN &  56.47 \\ 
				9 & 36.92 & 35.81 & 37.66 & 37.71 & 36.35 & 37.66 & 37.66 & 37.66 & 35.64 & 35.61 & 35.61 & 35.71 & 36.35 & NaN &  36.35 \\ 
				10 & 21.92 & 24.35 & 27.06 & 27.10 & 25.78 & 27.06 & 27.06 & 27.06 & 24.31 & 24.29 & 24.29 & 24.00 & 25.78 & NaN &  25.78 \\ 
				11 & 55.84 & 50.19 & 51.25 & 51.32 & 49.66 & 51.25 & 51.25 & 51.25 & 49.79 & 49.75 & 49.75 & 50.38 & 49.66 & NaN &  49.66 \\ 
				12 & 17.39 & 21.57 & 24.46 & 24.48 & 23.40 & 24.46 & 24.46 & 24.46 & 21.65 & 21.63 & 21.63 & 21.12 & 23.40 & NaN &  23.40 \\ 
				13 & 99.92 & 99.69 & 99.50 & 99.51 & 99.54 & 99.50 & 99.50 & 99.50 & 99.66 & 99.66 & 99.66 & 99.72 & 99.54 & NaN &  99.54 \\ 
				14 & 90.27 & 84.92 & 83.87 & 83.92 & 83.45 & 83.87 & 83.87 & 83.87 & 84.57 & 84.55 & 84.55 & 85.32 & 83.45 & NaN &  83.45 \\ 
				15 & 17.63 & 22.55 & 25.49 & 25.51 & 24.52 & 25.49 & 25.49 & 25.49 & 22.67 & 22.65 & 22.65 & 22.05 & 24.52 & NaN &  24.52 \\ 
				\hline
				%Nan \% & 0.00 & 0.00 & 0.00 & 0.00     & 0.00      &  0.00 & 0.00 & 0.00 & 0.00         & 0.00 & 0.00 & 0.00    & 0.00       & 2.56 & 0.00 \\	
				%Time (s) & 0.10 & 0.18 & 0.62 & 0.31 &         & 24.60 & 3.50 & 0.37 & 0.21 & 126.41  & 47.85 & 0.22 & 397.39  &  183.18 & 0.31    \\	
				Time (s) & 0.11 & 0.10 & 1.07 & 0.51 & 1358.61 & 44.73 & 0.12 & 0.10 & 0.30 &  12.59  & 40.36 & 0.25 & 618.59  &   31.81 & 0.11  \\
				\hline		
			\end{tabular}
		}
	\end{center}
	\caption{Variable inclusion probabilities (as a percentage) and computational times (in seconds) for the {\tt UScrime} dataset.
		The first to third line indicates the package, mixture $g$-prior and evaluation method used respectively. Bracketed terms
		refer to equations in the paper. NaN entries indicate numerical issues for the prior/implementation pair. The acronym BVS refers to the {\tt BayesVarSelect} package.}
	\label{tab:UScrime}
\end{sidewaystable}

\begin{sidewaystable}[h!]
	\begin{center}
		{\scriptsize 
			\begin{tabular}{c|r|r|rrrrrr|rrrr|rrr}
				Package & blma   & blma   & BAS    & BAS     & BVS    & BMS    & blma & blma & BAS & blma & blma & blma & BVS & blma & blma \\ 
				Prior   & BIC    & ZE     & $g$    & $g$     & $g$    & $g$    & $g$  & $g$ &  $g/n$ & $g/n$ & $g/n$ & $g/n$ & Robust & Robust & Robust \\ 
				Method  & (\ref{eq:marginalLikelihoodCake})  & (\ref{eq:marginalLikelihoodBetaPrime}) 
				& (\ref{eq:hyperGmarginal}) & Laplace & (\ref{eq:hyperGmarginal}) & (\ref{eq:hyperGmarginal}) & (\ref{eq:hyperGmarginal}) & (\ref{eq:hyperGmarginal2}) & Laplace & 
				{\tt appell} & quad. & approx. & (\ref{eq:yGivenGammaRobust}) & (\ref{eq:yGivenGammaRobust}) & (\ref{eq:yGivenGammaRobust2}) \\ 
				\hline
				1 & 100.00 & 100.00 & 100.00 & 100.00 & NaN & NaN & NaN & 100.00 & 100.00 & NaN & 100.00 & 100.00 & NaN & 100.00 & 100.00 \\ 
				2 & 100.00 & 100.00 & 100.00 & 100.00 & NaN & NaN & NaN & 100.00 & 100.00 & NaN & 100.00 & 100.00 & NaN & 100.00 &  100.00 \\ 
				3 & 1.21 & 3.16 & 8.65 & 8.65 & NaN & NaN & NaN & 8.64 & 7.17 & NaN & 7.16 & 7.65 & NaN  & 4.77 &  4.77 \\ 
				4 & 100.00 & 100.00 & 100.00 & 100.00 & NaN & NaN & NaN & 100.00 & 100.00 &NaN  & 100.00 & 100.00 & NaN  & 100.00 &  100.00 \\ 
				5 & 100.00 & 100.00 & 100.00 & 100.00 & NaN & NaN & NaN & 100.00 & 100.00 & NaN & 100.00 & 100.00 & NaN & 100.00 &  100.00 \\ 
				6 & 100.00 & 100.00 & 100.00 & 100.00 & NaN & NaN & NaN & 100.00 & 100.00 & NaN & 100.00 & 100.00 & NaN & 100.00 &  100.00 \\ 
				7 & 0.62 & 1.72 & 5.33 & 5.33 & NaN &NaN  & NaN & 5.32 & 4.30 & NaN & 4.29 & 4.63 & NaN & 2.70 &  2.70 \\ 
				8 & 96.07 & 98.32 & 99.35 & 99.35 & NaN & NaN & NaN & 99.35 & 99.21 & NaN & 99.20 & 99.26 & NaN & 98.86 &  98.86 \\ 
				9 & 3.28 & 8.16 & 20.69 & 20.69 & NaN & NaN & NaN & 20.66 & 17.46 & NaN & 17.42 & 18.52 & NaN & 12.02 & 12.02 \\ 
				10 & 100.00 & 100.00 & 100.00 & 100.00 & NaN & NaN & NaN & 100.00 & 100.00 & NaN & 100.00 & 100.00 & NaN & 100.00 &  100.00 \\ 
				11 & 100.00 & 100.00 & 100.00 & 100.00 & NaN & NaN & NaN & 100.00 & 100.00 & NaN & 100.00 & 100.00 & NaN & 100.00 &  100.00 \\ 
				\hline
				%Nan \% & 0.00 & 0.00 & 0.00 & 0.00   &       &  ?? & 75.00 & 0.00 & 0.00         & 44.29 & 0.00 & 0.00    & *        & 0.00 & 0.00 \\	\hline	
				%Time (s) & 0.03 & 0.03 & 1.01 & 0.27 &       &  6.94 & 0.05 & 0.04 & 0.30 & 65.19 & 8.79 & 0.03    & *        & 14.69 & 0.04 \\	\hline	
				Time (s) & 0.03 & 0.02 & 0.88 & 0.33  &       &  5.89 & 0.02 & 0.02 & 0.08 & 84.73 & 2.69 & 0.10    & *        & 2.18 & 0.01 \\	\hline	
			\end{tabular}
		}
	\end{center}
	\caption{Variable inclusion probabilities (as a percentage) and computational times (in seconds) for the {\tt VietNamI} dataset.
		The first to third line indicates the package, mixture $g$-prior and evaluation method used respectively. Bracketed terms
		refer to equations in the paper. NaN entries indicate numerical issues for the prior/implementation pair. The acronym BVS refers to the {\tt BayesVarSelect} package.}
	\label{tab:VietNamI}
\end{sidewaystable}

\begin{sidewaystable}[h!]
	\begin{center}
		{\scriptsize 
			\begin{tabular}{c|r|r|rrrrrr|rrrr|rrr}
				Package & blma   & blma   & BAS    & BAS     & BVS    & BMS    & blma & blma & BAS & blma & blma & blma & BVS & blma & blma \\ 
				Prior   & BIC    & ZE     & $g$    & $g$     & $g$    & $g$    & $g$  & $g$ &  $g/n$ & $g/n$ & $g/n$ & $g/n$ & Robust & Robust & Robust \\ 
				Method  & (\ref{eq:marginalLikelihoodCake})  & (\ref{eq:marginalLikelihoodBetaPrime}) 
				& (\ref{eq:hyperGmarginal}) & Laplace & (\ref{eq:hyperGmarginal}) & (\ref{eq:hyperGmarginal}) & (\ref{eq:hyperGmarginal}) & (\ref{eq:hyperGmarginal2}) & Laplace & 
				{\tt appell} & quad. & approx. & (\ref{eq:yGivenGammaRobust}) & (\ref{eq:yGivenGammaRobust}) & (\ref{eq:yGivenGammaRobust2}) \\ 
				\hline
				1 & 11.96 & 20.36 & 34.62 & 34.64      &  NaN      & 34.69 & 34.69 & 34.69 & 31.98     &  NaN   & 32.04 & 32.96   &  NaN       & 26.46 & 26.46 \\ 
				2 & 43.60 & 47.24 & 50.36 & 50.34      &  NaN      & 50.34 & 50.34 & 50.34 & 49.79     &  NaN   & 49.78 & 49.98   &  NaN       & 48.73 & 48.73 \\ 
				3 & 3.00 & 7.49 & 16.97 & 16.99        &  NaN      & 17.10 & 17.10 & 17.10 & 15.02     &  NaN   & 15.13 & 15.80   &  NaN       & 11.20 & 11.20 \\ 
				4 & 37.14 & 42.02 & 46.85 & 46.88      &  NaN      & 46.87 & 46.87 & 46.87 & 46.02     &  NaN   & 46.01 & 46.31   &  NaN       & 44.28 & 44.28 \\ 
				5 & 81.87 & 86.11 & 90.49 & 90.50      &  NaN      & 90.41 & 90.41 & 90.41 & 89.92     &  NaN   & 89.85 & 90.07   &  NaN       & 88.59 & 88.59 \\ 
				6 & 16.83 & 26.67 & 41.70 & 41.69      &  NaN      & 41.83 & 41.83 & 41.83 & 38.98     &  NaN   & 39.10 & 40.05   &  NaN       & 33.27 & 33.27 \\ 
				7 & 3.22 & 8.89 & 21.41 & 21.43        &  NaN      & 21.53 & 21.53 & 21.53 & 18.86     &  NaN   & 18.95 & 19.83   &  NaN       & 13.82 & 13.82 \\ 
				8 & 4.30 & 11.09 & 23.57 & 23.59       &  NaN      & 23.66 & 23.66 & 23.66 & 21.20     &  NaN   & 21.26 & 22.09   &  NaN       & 16.34 & 16.34 \\ 
				9 & 2.62 & 7.19 & 16.97 & 16.98        &  NaN      & 17.09 & 17.09 & 17.09 & 14.97     &  NaN   & 15.07 & 15.76   &  NaN       & 11.04 & 11.04 \\ 
				10 & 52.53 & 77.78 & 90.97 & 90.99     &  NaN      & 90.81 & 90.81 & 90.81 & 89.59     &  NaN   & 89.44 & 89.98   &  NaN       & 85.92 & 85.92 \\ 
				11 & 92.51 & 93.73 & 94.75 & 94.79     &  NaN      & 94.58 & 94.58 & 94.58 & 94.68     &  NaN   & 94.49 & 94.53   &  NaN       & 94.35 & 94.35 \\ 
				12 & 99.82 & 99.94 & 99.97 & 99.97     &  NaN      & 99.97 & 99.97 & 99.97 & 99.97     &  NaN   & 99.97 & 99.97   &  NaN       & 99.96 & 99.96 \\ 
				13 & 2.45 & 6.60 & 15.70 & 15.72       &  NaN      & 15.84 & 15.84 & 15.84 & 13.81     &  NaN   & 13.92 & 14.57   &  NaN       & 10.13 & 10.13 \\ 
				14 & 8.10 & 19.91 & 38.61 & 38.63      &  NaN      & 38.66 & 38.66 & 38.66 & 35.36     &  NaN   & 35.39 & 36.55   &  NaN       & 28.37 & 28.37 \\ 
				15 & 8.17 & 18.51 & 35.17 & 35.19      &  NaN      & 35.24 & 35.24 & 35.24 & 32.19     &  NaN   & 32.24 & 33.29   &  NaN       & 25.87 & 25.87 \\ 
				16 & 62.99 & 75.30 & 83.41 & 83.42     &  NaN      & 83.30 & 83.30 & 83.30 & 82.39     &  NaN   & 82.29 & 82.68   &  NaN       & 79.98 & 79.98 \\ 
				17 & 3.27 & 8.53 & 19.41 & 19.43       &  NaN      & 19.54 & 19.54 & 19.54 & 17.20     &  NaN   & 17.31 & 18.07   &  NaN       & 12.85 & 12.85 \\ 
				18 & 54.75 & 74.93 & 86.65 & 86.65     &  NaN      & 86.55 & 86.55 & 86.55 & 85.31     &  NaN   & 85.22 & 85.74   &  NaN       & 81.95 & 81.95 \\ 
				19 & 100.00 & 100.00 & 100.00 & 100.00 &  NaN      & 100.00 & 100.00 & 100.00 & 100.00 &  NaN   & 100.00 & 100.00 &  NaN       & 100.00 & 100.00 \\ 
				20 & 26.63 & 44.11 & 62.58 & 62.60     &  NaN      & 62.56 & 62.56 & 62.56 & 59.88     &  NaN   & 59.83 & 60.83   &  NaN       & 53.58 & 53.58 \\ 
				21 & 100.00 & 100.00 & 100.00 & 100.00 &  NaN      & 100.00 & 100.00 & 100.00 & 100.00 &  NaN   & 100.00 & 100.00 &  NaN       & 100.00 & 100.00 \\ 
				22 & 4.95 & 13.22 & 29.03 & 29.05      &  NaN      & 29.12 & 29.12 & 29.12 & 26.04     &  NaN   & 26.09 & 27.14    & NaN        & 19.87 & 19.87 \\ 
				\hline
				%	Nan \% &   &   &   &       & *      &   &   &   &           &   &   &      & *        &   &   \\ 	
				%	\hline
				%Time(s) & 38.46 & 64.31 & 18.89  & 12.06  &       & 2114.16  & 319.58  &   96.230 &  17.56         &  28325.40 & 34798.648  &   75.11   & 4921.23       & 53107.45  & 77.86  \\ 	
				Time(s) & 15.43 & 16.18 & 14.85  &  9.53   &       & 1735.66  & 34.925  &   17.55 &  10.82         &  25008.93 & 
				5425.11  &   18.06   & 4606.92       &  4275.55  & 21.03  \\ 
				\hline
			\end{tabular}
		}
	\end{center}
	\caption{Variable inclusion probabilities (as a percentage) and computational times (in seconds) for the {\tt Kakadu} dataset.
		The first to third line indicates the package, mixture $g$-prior and evaluation method used respectively. Bracketed terms
		refer to equations in the paper. NaN entries indicate numerical issues for the prior/implementation pair. The acronym BVS refers to the {\tt BayesVarSelect} package. Note that 
		the {\tt BayesVarSelect} method ran out of RAM for this example.}
	\label{tab:Kakadu}
\end{sidewaystable}


For Table \ref{tab:UScrime} we see that all of the ``exact'' methods agree with one another to the first 2 decimal 
places. We note that the Laplace approximation is quite accurate and appears superior to  the method
``(\ref{eq:hyperGonNmarginalApprox})'' for the mixture $g/n$-prior. However, for both of these approximation
methods the discrepancies to their exact counterparts is roughly the same size, or perhaps even less, than
the differences between each of the choices of mixture $g$-priors. In terms of speed {\tt BAS} and {\tt BMLA} are
the fastest packages and roughly comparable in speed. Both {\tt BMS} and {\tt BayesVarSelect} are not as fast.
For the mixture $g$-prior we suspect that the package {\tt BAS} replies on Laplace's method for models
where direct evaluation of (\ref{eq:hyperGmarginal}) becomes numerically problematic, which would explain
differences between the {\tt BAS} and {\tt blma} packages for the {\tt Kakadu} dataset.
%Overall our package {\tt BLMA} offers mixture $g$-priors not offered by {\tt BAS}, is arguably more
%accurate when $n$ is large, and is of comparable speed. 

\section{Conclusion}
\label{sec:conclusion}

We have reviewed the prior structures that lead to closed form expressions for Bayes factors for
linear models. We have described ways that each of these priors, except for the hyper-$g/n$ 
prior can be evaluated in a numerically stable manner and have implemented a package 
{\tt blma} for performing full exact Bayesian model averaging using
this methodology. 

We are currently working on several extensions to this work. Firstly, we are working on a
parallel implementation of the package which will allow for exact Bayesian inference for
problems roughly the size $p\approx 30$. Secondly, we are currently implementing  Markov
Chain Monte Carlo (MCMC) and population based MCMC methods for exploring the model space
when $p>30$. Lastly, we are deriving exact expressions for parameter posterior distributions
under some of the prior structures we have considered here.


Our package is competitive with {\tt BAS} and {\tt BMS} in terms of computational speed,
is numerically more stable and accurate, and offers some different priors structures not
offered in {\tt BAS}. Our package is much faster than {\tt BayesVarSelect} and is also
numerically more stable and accurate, and represents an advance in the implementation of
exact Bayesian linear model averaging.

