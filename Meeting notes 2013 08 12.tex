\documentclass{amsart}
\begin{document}
\section{Meeting with John Ormerod, August 12 2013}
John had just returned from the JSM 2013 in Montreal.

\subsection{In relation to splines}
How to choose $\lambda$, the smoothing penalty? Why do we fit it as a variance
component as it is estimated?

There is a bias-variance trade off, and there is uncertainty in $\lambda$, which
is estimated from the data. To account for this uncertainty, we fit it as a
random effect. For more detail, see Semiparametric Regression by Rupert, Wand
and Carroll. The copy in the library is currently out.

\subsection{Vague priors}
Vague priors give us confidence that we are not assuming too much when
specifying Bayesian models, but priors are not invariant under transformation.
This means that priors which are vague in one parameterisation may not be vague
at all in another, and should make use cautious when specifying our Bayesian
models. We can't be ``asleep at the wheel'' and assume that just because we've
specified a vague prior, everything is okay.

\subsection{Doubly intractable models}
These are Bayesian models for which we don't know the normalising constant or
joint distribution.

% Insert diagram here. How do I do this?
If we have a zero-inflated Poisson model, I was worried that there will probably
be a strong dependence between the Bernoulli and Poisson r.v.s, and hence a
product density approximation may perform poorly. John said that it may, but it's
still the thing we're going to try first. John was not worried, because all of the
Bernoulli trials are independent, and so the CLT kicks in and helps us estimate
the mean of the proportion of zeroes well. The variance is another matter, but at
least we're going to be very sure about the mean.

\subsection{Model selection}
MCMC explores model space, asymptotics helps as model space has peaks and
valleys. The true model peak will get larger and larger as we continue to
sample from the Markov chain. This is by the Law of Large Numbers.

\subsection{How should I learn numerics}
John recommended that I work through Chapters 1-5, 10 of Numerical Recipes
in C.

\subsection{Stop reading, start implementing OR mathematics is not a spectator sport}
I showed John that I had worked through the example in Machine Learning: A
Probabilistic Perspective on using Variational Bayes to fit a univariate
Gaussian model. He said that everyone has to start somewhere, and that when
he was doing his PhD, in order to figure out how to fit splines better he first
had to understand how to fit splines!

He said that I was right to start at the start, and that in order to understand
how variational approximations work, I should implement every one of them in
his Explaining Variational Approximations 2008 paper. And that I should also
be calculating the lower bounds. I must admit that I'm not looking forward to
the lower bounds part, because that algebra looks incredibly tedious! But I don't
recall anyone saying that doing a PhD in mathematical statistics would be a
walk in the park.

\subsection{Send John your JAGS code}
John thinks he asked for this before. But this time, he definitely asked for it.
He wants to use JAGS instead of WinBUGS for his Monte Carlo methods part of the
course.

\subsection{pvalues are meaningless OR Santa Claus is your parents, and the tooth fairy will not put money under your pillow}
John came up with the following rather disquieting example:

Consider $x_i \sim N(\mu, \sigma^2)$. Let $H_0: \mu = \mu_0$ versus
$H_1: \mu > \mu_0$. Use everyone's favorite test statistic
\[
t = \frac{\mu - \mu_0}{S/\sqrt{n}}
\]

Under $H_0$, $t \sim t_{n-1}$. Now, consider the case where the true value of 
$\mu_0$ is not $\mu_0$, but in fact $\mu_0 + \epsilon$ for some fixed $\epsilon$. 
As $n \to \infty$, the numerator goes to $\epsilon \sqrt{n} \to \infty$, while the
denominator $S \to 0$. So $t \to \infty$, and the pvalue goes to $0$ as the sample 
size increases.

This has profound and disturbing implications -- you can prove any hypothesis you 
like is statistically significant simply by making your sample size sufficiently 
large. This casts doubt on, for example, most of the public health literature, 
where sample sizes are typically in the tens or hundreds of thousands, and 
borderline pvalues are used as a matter of routine.

This also highlights the fact that pvalue fishing is at its worst a form of 
scientific fraud.

\end{document}