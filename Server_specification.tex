\documentclass{amsart}
\author{Mark Greenaway}
\title{Server specifications}
\begin{document}
\section{Budget}
The range \$15k-\$25k was mentioned

\subsection{Main use cases}
The server will be used principally by the statistics research group, and in particular the bioinformatics researchers. I spoke to Sheila and Dario, in addition to talking to
John and Jean. The following computational issues were raised:

\begin{enumerate}
\item{CPUs of our own} -- One of the main issues at present is that our jobs have to
compete with whoever else is using the school's servers. This often results in our
jobs getting ``niced'' (there's nothing nice about it) down to the point where they
take an incredibly long time to complete. The more CPUs the better, but for our
purposes, RAM and especially fast storage are much more important than CPUs.

\item{Mapping} -- This uses large quantities of RAM, gigabytes of it. We should not
consider getting a server configured with any less than 128 gigabytes of RAM.

\item{Processing large data sets} -- The researchers regularly process data in the
terabyte range. To process these volumes of data quickly we will use a solution like
RAID striping across multiple hard disks. The basic idea is much like multicore processing,
if one disk is too slow, use several in parallel. This isn't very hard to set up, and
also has the advantage that the multiple disks are combined together into one very large
file system, so we will no longer have to juggle files between disks that are almost full.
We can simply combine all of our disks together, and let the operating system take care
of the details for us.

To ensure that these disks performance well in this configuration, we need a controller that 
will support this many hard disks at full speed, in this configuration. We could use 
software RAID, or possibly consider a solution involving hardware RAID at the controller 
level. The current state of the art seems to be SATA-3 controllers, but I'm happy to go with 
Paul's guidance on that.

Of all the things we need from a new server, I think we need fast storage the most, so this
should be prioritised.
\end{enumerate}

\subsection{Future proofing}
As you wisely asked me to make sure that whatever we buy continues to be useful in the
future, I would also suggest we get a 10 gigabit Ethernet network adapter. These are
expensive, but well worth it. In the future, if we want to expand, we can simply
buy more machines and link them with 10 gigabit Ethernet. That way, we can build
a computing cluster with fast interconnections between the nodes.

Of course, 10 gigabit Ethernet adapters aren't much use without the appropriate 10 gigabit
switch and other machines equipped with 10 gigabit Ethernet adapters to talk to them, but
we can get those things piece by piece if we need to.

We should also choose a server allows us to install more hard disks if need be, in case the
volume of data grows even more than it already has done.

\end{document}