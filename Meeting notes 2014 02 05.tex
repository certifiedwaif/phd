\documentclass{amsart}
\begin{document}
I had just returned from a month at the AMSI Summer School, so there had been no progress since this time.

\section{Shila's project}
Ask Shila where she is with her project, and if she still needs help from you.

\section{Nancy's bipolar data}
John did a lot of work on this while he was on ``holiday''. He broke the data set into two subsets,
those with Bipolar12, and those with Depression. He then analysed the Alchohol interference
at 12 months in these two subsets. The Alcohol interference field contains
1 Yes
0 No, not applicable
NA Not stated

John analysed the data using logistic regression and classification trees.
He included the sociodemographic variables age, gender, employment and education.
The important variables that he identified were:
drinking to cope - 3, sometimes or often
smoking
drinking $\times$ emotional support
There are positive main effects for emotional support and drinking, but a negative
interaction between emotional support and drinking, which is counterintuitive.

There is also an interesting interaction between age and tangible support,
where if age was low and tangible support was available, alchohol interference
was more likely.

Exercise and sleep were also important.

\subsection{To Do}
Check over what John has done, and think about how to continue it. Think about how to incorporate survey weighting.

\section{Approximate inference}
Collapsed Variational Bayes 2010 applied to Hidden Markov Models, Latent Dictionary Analysis. John would like to apply this
technique to Gaussian mixture models, and see if there's any advantage to the factored variational approximation to
Gaussian mixture models.

Marginal Variational Bayes 2012

Stochastic Variational Bayes $q(\theta; \xi)$. This leads to stochastic approximation - control variate,
gradient descent, BFGS approaches. BFGS approaches use the gradient to approximate the inverse of the
Hessian. The standard method of calculating an inverse takes $O(p^3)$, whereas this takes $O(p^2)$.

This method allows you to produce a sequence which solves
$$
\text{min}_{\xi} E_{\theta} f(\theta; \xi)
$$

without having to evaluate the underlying integral.

\end{document}
