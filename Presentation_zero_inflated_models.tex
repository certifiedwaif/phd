\documentclass{beamer}

\usepackage{graphicx}
% include.tex

% \newcommand{\expit}[1]{\text{expit} #1}
% \newcommand{\logit}[1]{\text{logit} #1}

\def \R {{\mathbb{R}}}
\def \vbeta {{\boldsymbol \beta}}
\def \vnu {{\bf \nu}}
\def \vy {{\bf y}}
\def \vx {{\bf x}}
\def \vu {{\bf u}}
\def \vr {{\bf r}}
\def \vp {{\bf p}}
\def\vectorfontone{\bf}
\def\vone{{\bf 1}}
\def\vzero{{\bf 0}}
\def \vmu {{\boldsymbol \mu}}
\def \vnu {{\bf \nu}}
\def \vmuqbeta {{\vmu_{q(\vbeta)}}}
\def \vmubeta {{\vmu_{\vbeta}}}
\def \Sigmaqbeta {{\Sigma_{q(\vbeta)}}}
\def \Sigmabeta {{\Sigma_{\vbeta}}}
\def \va {{\bf a}}
\def \vtheta {{\bf \theta}}
\def \mX {{\bf X}}
\def \mZ {{\bf Z}}
\def \mR {{\bf R}}
\def \mC {{\bf C}}
\def \mI {{\bf I}}
\def \mLambda {{\boldsymbol \Lambda}}
\def \mSigma {{\boldsymbol \Sigma}}
\def \B {{\text{B}}}

\def\ds{{\displaystyle}}

\def\diag{{\mbox{diag}}}
\def\bbE{\mathbb{E}}


\title{Zero-inflated models}
\author{Mark Greenaway}

\mode<presentation>
{ \usetheme{boxes} }

\begin{document}
% 1. Front slide
\begin{frame}
\titlepage
% Details about myself here?
\end{frame}

% 2. Intro
\begin{frame}
\frametitle{Introduction}
Zero inflated data arises in many areas of application, such as physical
activity data, number of hospital visits and number of insurance claims per
year.

We will work with zero-inflated count data.
\end{frame}
% 3. Univariate model
\begin{frame}
\frametitle{Univariate model formulation}
We start by building a relatively simple zero-inflated count model.

\begin{align*}{ll}
X_i &= R_i Y_i \\
R_i &\sim \text{Bernoulli}(\rho) \\
Y_i &\sim \text{Poisson}(\lambda) \\
\rho &\sim \text{Beta}(a_\rho, b_\rho) \\
\lambda &\sim \text{Gamma}(a_\lambda, b_\lambda)
\end{align*}

\end{frame}

% 4. \rho = 9/10, \lambda = 5
% Example data 0 0 0 5 10
\begin{frame}[fragile]
\frametitle{Example data}
Take, for example, $\rho = \frac{1}{2}, \lambda = 5$.

\begin{verbatim}
0 7 3 4 5 3 2 6 5 0 0 1
0 0 5 0 2 3 6 4 0 5 4 0
7 0 0 0 7 0 6 6 0 3 0 5
0 4 0 0 0 2 3 0 3 4 5 0
8 0
\end{verbatim}

% Histogram
\includegraphics[width=100mm, height=100mm]{code/univariate_data_histogram.pdf}
% Density
\end{frame}

% 5. How to fit, and advantages and disadvantages of each approach
% - Maximum likelihood
% - MCMC
% - VB
\
\begin{frame}
\frametitle{Comparison of fitting techniques}
Maximum likelihood
Frequentist
\begin{tabular}{ll}
Pro & Con \\
Standard optimisation techniques can be used & Biased for mixed models
\end{tabular}

MCMC
\begin{tabular}{ll}
Pro & Con \\
Very accurate & Slow \\
& May not converge at all
\end{tabular}

Variational Bayes
\begin{tabular}{ll}
Pro & Con \\
Fast & May lose accuracy, variance \\
Still quite accurate & Solution may be intractable
\end{tabular}

\end{frame}

% 6. Overview of Variational Bayes
\begin{frame}
\frametitle{An overview of Variational Bayes}
\begin{itemize}
\item Approximate the full posterior $p(\theta)$ with an approximation $q(\theta)$
\item Minimise the KL divergence between $p(\theta)$ and $q(\theta)$
\item Theory guarantees that $p(\theta) < q(\theta)$ and that $q(\theta)$ will
increase with each VB step
\end{itemize}
\end{frame}

% 7a. Variational Bayes solution to ZIP
% - q-densities
\begin{frame}
Choose a factored approximation of the form
$$
q(\theta) = q(\lambda) q(\rho) \prod_{i=1}^n q(r_i)
$$
where
$$
q(\lambda) = \text{Gamma}(a_{q(\lambda)}, b_{q(\lambda)})
q(\rho) = \text{Beta}(a_{q(\rho)}, b_{q(\rho)})
q(r_i) = \text{Bernoulli}(p_i)
$$

% - Algorithm
We iteratively update the parameters of each approximate distribution
in turn until the lower bound of the approximation converges.

This could be thought of as a generalisation of Expectation Maximisation.
\end{frame}


% 8. Results
% - Lower bound convergence
% - Accuracies
% 7b. Define accuracy
\begin{frame}
\frametitle{Mean field updates}
% Are they going to be happy with that?
Most of the mean field updates are straightforward, as the priors are conjugate.
$$
q(\lambda) = \text{Gamma}(\alpha_\lambda + \vone^T \vx, \beta_\lambda + \vone^T\vp)
q(\rho) = \text{Beta}(\alpha_\rho + \vone^T\vr, \beta_\rho + \vone^T(\vone - \vr))
$$
$$
q(r_i) = \text{Bernoulli}(\text{expit}(\eta_i))
$$
where
$$
\eta_i = - \frac{\alpha_\lambda^*}{\beta_\lambda^*} + \Psi(\alpha_\rho^*) - \Psi(\beta_\rho^*)
$$
\end{frame}

\begin{frame}
\frametitle{Results/Accuracy}
% Definition of accuracy
Accuracy is defined as the difference in $L_1$ norm between the true posterior distribution and
the approximate posterior distribution of the variational approximation.

Excellent accuracy for the univariate approximation, over 99\% in all of the cases that I looked at.
% Graph of lower bound
%\includegraphics{univariate_lower_bound_convergence.pdf}
\end{frame}

% 9. Extension to linear model
% 10. Overview GVA
% 11. Algorithm
% 12. Results?
% 13. What next
% 14. Conclusion
% 15. References
\begin{frame}
\frametitle{Extension to multivariate/regression models}
Extension from univariate, model formulation
$$
q(\theta) = q(\beta) q(\rho) \prod_{i=1}^n q(r_i)
$$
where
$q(\beta) \sim N(\mu, \Sigma)$ and
$q(\sigma_u^2) \sim IG(\alpha_{\sigma_u^2}, \beta_{\sigma_u^2})$

\begin{itemize}
\item Mixed models -- one model to rule them all
\item The need for better approaches - MCMC with existing software can take minutes to
converge, if they converge at all. That might not sound so bad to you, but how do you
do model selection?
\end{itemize}
\end{frame}

\begin{frame}
\begin{itemize}
\item Lack of conjugacy means mean field updates won't be analytically tractable.
\item We try Gaussian Variational Approximations instead, assume that $\beta, \vu \sim N(\vmu, \Sigma)$
and approximate as closely as we can
\item Computation - a work in progress
\item Initial signs are that this approach will work
\end{itemize}
\end{frame}

\begin{frame}
\frametitle{References}
\begin{itemize}
\item Explaining variational approximations
\item Gaussian variational approximations
\item General Design Mixed Models
\end{itemize}
\end{frame}

\end{document}
