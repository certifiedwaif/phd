% \documentclass{amsart}[12pt]
% \documentclass{book}[12pt]
\RequirePackage{rotating}
\documentclass[PhD,stats]{usydthesis}[12pt]
% \documentclass{ociamthesis}

%\addtolength{\oddsidemargin}{-.75in}%
%\addtolength{\evensidemargin}{-.75in}%
%\addtolength{\textwidth}{1.5in}%
%\addtolength{\textheight}{1.3in}%
%\addtolength{\topmargin}{-.8in}%
%\addtolength{\marginparpush}{-.75in}%

% \setlength{\bibsep}{0pt plus 0.3ex}

%\addtolength{\oddsidemargin}{-.5in}%
%\addtolength{\evensidemargin}{-.5in}%
%\addtolength{\textwidth}{1in}%
%\addtolength{\textheight}{-.3in}%
%\addtolength{\topmargin}{-.8in}%

\usepackage{geometry}
\geometry{verbose,a4paper,tmargin=1in,bmargin=1.5in,lmargin=1in,rmargin=1in}

%\setlength\parindent{0pt}
%\setlength{\parskip}{1em}

\usepackage{float}
\usepackage[authoryear]{natbib}
\usepackage[doublespacing]{setspace}
\usepackage{graphicx}
\usepackage{algorithm,algorithmic}
\usepackage{color}
\usepackage{verbatim}
\usepackage{subfigure}
\usepackage{amsfonts}
\usepackage{color}
\usepackage{latexsym,amssymb,amsmath,amsfonts}
%\usepackage{tabularx}
\usepackage{theorem}
\usepackage{verbatim,array,multicol,palatino}
\usepackage{graphics}
\usepackage{fancyhdr}
\usepackage{url}
%\usepackage[all]{xy}
\usepackage{cancel}
\usepackage{graphicx,rotating,booktabs}

\title{Numerically Stable Approximate Bayesian Methods for Generalized Linear
       Mixed Models and Linear Model Selection}

\author{Mark Greenaway}

% include.tex

% \newcommand{\expit}[1]{\text{expit} #1}
% \newcommand{\logit}[1]{\text{logit} #1}

\def \R {{\mathbb{R}}}
\def \vbeta {{\boldsymbol \beta}}
\def \vnu {{\bf \nu}}
\def \vy {{\bf y}}
\def \vx {{\bf x}}
\def \vu {{\bf u}}
\def \vr {{\bf r}}
\def \vp {{\bf p}}
\def\vectorfontone{\bf}
\def\vone{{\bf 1}}
\def\vzero{{\bf 0}}
\def \vmu {{\boldsymbol \mu}}
\def \vnu {{\bf \nu}}
\def \vmuqbeta {{\vmu_{q(\vbeta)}}}
\def \vmubeta {{\vmu_{\vbeta}}}
\def \Sigmaqbeta {{\Sigma_{q(\vbeta)}}}
\def \Sigmabeta {{\Sigma_{\vbeta}}}
\def \va {{\bf a}}
\def \vtheta {{\bf \theta}}
\def \mX {{\bf X}}
\def \mZ {{\bf Z}}
\def \mR {{\bf R}}
\def \mC {{\bf C}}
\def \mI {{\bf I}}
\def \mLambda {{\boldsymbol \Lambda}}
\def \mSigma {{\boldsymbol \Sigma}}
\def \B {{\text{B}}}

\def\ds{{\displaystyle}}

\def\diag{{\mbox{diag}}}
\def\bbE{\mathbb{E}}


\usepackage{latexsym,amssymb,amsmath,amsfonts}
%\usepackage{tabularx}
\usepackage{theorem}
\usepackage{verbatim,array,multicol,palatino}
\usepackage{graphicx}
\usepackage{graphics}
\usepackage{fancyhdr}
\usepackage{algorithm,algorithmic}
\usepackage{url}
%\usepackage[all]{xy}



\def\approxdist{\stackrel{{\tiny \mbox{approx.}}}{\sim}}
\def\smhalf{\textstyle{\frac{1}{2}}}
\def\vxnew{\vx_{\mbox{{\tiny new}}}}
\def\bib{\vskip12pt\par\noindent\hangindent=1 true cm\hangafter=1}
\def\jump{\vskip3mm\noindent}
\def\etal{{\em et al.}}
\def\etahat{{\widehat\eta}}
\def\thick#1{\hbox{\rlap{$#1$}\kern0.25pt\rlap{$#1$}\kern0.25pt$#1$}}
\def\smbbeta{{\thick{\scriptstyle{\beta}}}}
\def\smbtheta{{\thick{\scriptstyle{\theta}}}}
\def\smbu{{\thick{\scriptstyle{\rm u}}}}
\def\smbzero{{\thick{\scriptstyle{0}}}}
\def\boxit#1{\begin{center}\fbox{#1}\end{center}}
\def\lboxit#1{\vbox{\hrule\hbox{\vrule\kern6pt
      \vbox{\kern6pt#1\kern6pt}\kern6pt\vrule}\hrule}}
\def\thickboxit#1{\vbox{{\hrule height 1mm}\hbox{{\vrule width 1mm}\kern6pt
          \vbox{\kern6pt#1\kern6pt}\kern6pt{\vrule width 1mm}}
               {\hrule height 1mm}}}


%\sloppy
%\usepackage{geometry}
%\geometry{verbose,a4paper,tmargin=20mm,bmargin=20mm,lmargin=40mm,rmargin=20mm}


%%%%%%%%%%%%%%%%%%%%%%%%%%%%%%%%%%%%%%%%%%%%%%%%%%%%%%%%%%%%%%%%%%%%%%%%%%%%%%%%
%
% Some convenience definitions
%
% \bf      -> vector
% \sf      -> matrix
% \mathcal -> sets or statistical
% \mathbb  -> fields or statistical
%
%%%%%%%%%%%%%%%%%%%%%%%%%%%%%%%%%%%%%%%%%%%%%%%%%%%%%%%%%%%%%%%%%%%%%%%%%%%%%%%%

% Sets or statistical values
\def\sI{{\mathcal I}}                            % Current Index set
\def\sJ{{\mathcal J}}                            % Select Index set
\def\sL{{\mathcal L}}                            % Likelihood
\def\sl{{\ell}}                                  % Log-likelihood
\def\sN{{\mathcal N}}                            
\def\sS{{\mathcal S}}                            
\def\sP{{\mathcal P}}                            
\def\sQ{{\mathcal Q}}                            
\def\sB{{\mathcal B}}                            
\def\sD{{\mathcal D}}                            
\def\sT{{\mathcal T}}
\def\sE{{\mathcal E}}                            
\def\sF{{\mathcal F}}                            
\def\sC{{\mathcal C}}                            
\def\sO{{\mathcal O}}                            
\def\sH{{\mathcal H}} 
\def\sR{{\mathcal R}}                            
\def\sJ{{\mathcal J}}                            
\def\sCP{{\mathcal CP}}                            
\def\sX{{\mathcal X}}                            
\def\sA{{\mathcal A}} 
\def\sZ{{\mathcal Z}}                            
\def\sM{{\mathcal M}}                            
\def\sK{{\mathcal K}}     
\def\sG{{\mathcal G}}                         
\def\sY{{\mathcal Y}}                         
\def\sU{{\mathcal U}}  


\def\sIG{{\mathcal IG}}                            


\def\cD{{\sf D}}
\def\cH{{\sf H}}
\def\cI{{\sf I}}

% Vectors
\def\vectorfontone{\bf}
\def\vectorfonttwo{\boldsymbol}
\def\va{{\vectorfontone a}}                      %
\def\vb{{\vectorfontone b}}                      %
\def\vc{{\vectorfontone c}}                      %
\def\vd{{\vectorfontone d}}                      %
\def\ve{{\vectorfontone e}}                      %
\def\vf{{\vectorfontone f}}                      %
\def\vg{{\vectorfontone g}}                      %
\def\vh{{\vectorfontone h}}                      %
\def\vi{{\vectorfontone i}}                      %
\def\vj{{\vectorfontone j}}                      %
\def\vk{{\vectorfontone k}}                      %
\def\vl{{\vectorfontone l}}                      %
\def\vm{{\vectorfontone m}}                      % number of basis functions
\def\vn{{\vectorfontone n}}                      % number of training samples
\def\vo{{\vectorfontone o}}                      %
\def\vp{{\vectorfontone p}}                      % number of unpenalized coefficients
\def\vq{{\vectorfontone q}}                      % number of penalized coefficients
\def\vr{{\vectorfontone r}}                      %
\def\vs{{\vectorfontone s}}                      %
\def\vt{{\vectorfontone t}}                      %
\def\vu{{\vectorfontone u}}                      % Penalized coefficients
\def\vv{{\vectorfontone v}}                      %
\def\vw{{\vectorfontone w}}                      %
\def\vx{{\vectorfontone x}}                      % Covariates/Predictors
\def\vy{{\vectorfontone y}}                      % Targets/Labels
\def\vz{{\vectorfontone z}}                      %

\def\vone{{\vectorfontone 1}}
\def\vzero{{\vectorfontone 0}}

\def\valpha{{\vectorfonttwo \alpha}}             %
\def\vbeta{{\vectorfonttwo \beta}}               % Unpenalized coefficients
\def\vgamma{{\vectorfonttwo \gamma}}             %
\def\vdelta{{\vectorfonttwo \delta}}             %
\def\vepsilon{{\vectorfonttwo \epsilon}}         %
\def\vvarepsilon{{\vectorfonttwo \varepsilon}}   % Vector of errors
\def\vzeta{{\vectorfonttwo \zeta}}               %
\def\veta{{\vectorfonttwo \eta}}                 % Vector of natural parameters
\def\vtheta{{\vectorfonttwo \theta}}             % Vector of combined coefficients
\def\vvartheta{{\vectorfonttwo \vartheta}}       %
\def\viota{{\vectorfonttwo \iota}}               %
\def\vkappa{{\vectorfonttwo \kappa}}             %
\def\vlambda{{\vectorfonttwo \lambda}}           % Vector of smoothing parameters
\def\vmu{{\vectorfonttwo \mu}}                   % Vector of means
\def\vnu{{\vectorfonttwo \nu}}                   %
\def\vxi{{\vectorfonttwo \xi}}                   %
\def\vpi{{\vectorfonttwo \pi}}                   %
\def\vvarpi{{\vectorfonttwo \varpi}}             %
\def\vrho{{\vectorfonttwo \rho}}                 %
\def\vvarrho{{\vectorfonttwo \varrho}}           %
\def\vsigma{{\vectorfonttwo \sigma}}             %
\def\vvarsigma{{\vectorfonttwo \varsigma}}       %
\def\vtau{{\vectorfonttwo \tau}}                 %
\def\vupsilon{{\vectorfonttwo \upsilon}}         %
\def\vphi{{\vectorfonttwo \phi}}                 %
\def\vvarphi{{\vectorfonttwo \varphi}}           %
\def\vchi{{\vectorfonttwo \chi}}                 %
\def\vpsi{{\vectorfonttwo \psi}}                 %
\def\vomega{{\vectorfonttwo \omega}}             %


% Matrices
%\def\matrixfontone{\sf}
%\def\matrixfonttwo{\sf}
\def\matrixfontone{\bf}
\def\matrixfonttwo{\boldsymbol}
\def\mA{{\matrixfontone A}}                      %
\def\mB{{\matrixfontone B}}                      %
\def\mC{{\matrixfontone C}}                      % Combined Design Matrix
\def\mD{{\matrixfontone D}}                      % Penalty Matrix for \vu_J
\def\mE{{\matrixfontone E}}                      %
\def\mF{{\matrixfontone F}}                      %
\def\mG{{\matrixfontone G}}                      % Penalty Matrix for \vu
\def\mH{{\matrixfontone H}}                      %
\def\mI{{\matrixfontone I}}                      % Identity Matrix
\def\mJ{{\matrixfontone J}}                      %
\def\mK{{\matrixfontone K}}                      %
\def\mL{{\matrixfontone L}}                      % Lower bound
\def\mM{{\matrixfontone M}}                      %
\def\mN{{\matrixfontone N}}                      %
\def\mO{{\matrixfontone O}}                      %
\def\mP{{\matrixfontone P}}                      %
\def\mQ{{\matrixfontone Q}}                      %
\def\mR{{\matrixfontone R}}                      %
\def\mS{{\matrixfontone S}}                      %
\def\mT{{\matrixfontone T}}                      %
\def\mU{{\matrixfontone U}}                      % Upper bound
\def\mV{{\matrixfontone V}}                      %
\def\mW{{\matrixfontone W}}                      % Variance Matrix i.e. diag(b'')
\def\mX{{\matrixfontone X}}                      % Unpenalized Design Matrix/Nullspace Matrix
\def\mY{{\matrixfontone Y}}                      %
\def\mZ{{\matrixfontone Z}}                      % Penalized Design Matrix/Kernel Space Matrix

\def\mGamma{{\matrixfonttwo \Gamma}}             %
\def\mDelta{{\matrixfonttwo \Delta}}             %
\def\mTheta{{\matrixfonttwo \Theta}}             %
\def\mLambda{{\matrixfonttwo \Lambda}}           % Penalty Matrix for \vnu
\def\mXi{{\matrixfonttwo \Xi}}                   %
\def\mPi{{\matrixfonttwo \Pi}}                   %
\def\mSigma{{\matrixfonttwo \Sigma}}             %
\def\mUpsilon{{\matrixfonttwo \Upsilon}}         %
\def\mPhi{{\matrixfonttwo \Phi}}                 %
\def\mOmega{{\matrixfonttwo \Omega}}             %
\def\mPsi{{\matrixfonttwo \Psi}}                 %

\def\mone{{\matrixfontone 1}}
\def\mzero{{\matrixfontone 0}}

% Fields or Statistical
\def\bE{{\mathbb E}}                             % Expectation
\def\bP{{\mathbb P}}                             % Probability
\def\bR{{\mathbb R}}                             % Reals
\def\bI{{\mathbb I}}                             % Reals
\def\bV{{\mathbb V}}                             % Reals

\def\vX{{\vectorfontone X}}                      % Targets/Labels
\def\vY{{\vectorfontone Y}}                      % Targets/Labels
\def\vZ{{\vectorfontone Z}}                      %

% Other
\def\etal{{\em et al.}}
\def\ds{\displaystyle}
\def\d{\partial}
\def\diag{\text{diag}}
%\def\span{\text{span}}
\def\blockdiag{\text{blockdiag}}
\def\tr{\text{tr}}
\def\RSS{\text{RSS}}
\def\df{\text{df}}
\def\GCV{\text{GCV}}
\def\AIC{\text{AIC}}
\def\MLC{\text{MLC}}
\def\mAIC{\text{mAIC}}
\def\cAIC{\text{cAIC}}
\def\rank{\text{rank}}
\def\MASE{\text{MASE}}
\def\SMSE{\text{SASE}}
\def\sign{\text{sign}}
\def\card{\text{card}}
\def\notexp{\text{notexp}}
\def\ASE{\text{ASE}}
\def\ML{\text{ML}}
\def\nullity{\text{nullity}}

\def\logexpit{\text{logexpit}}
\def\logit{\mbox{logit}}
\def\dg{\mbox{dg}}

\def\Bern{\mbox{Bernoulli}}
\def\sBernoulli{\mbox{Bernoulli}}
\def\sGamma{\mbox{Gamma}}
\def\sInvN{\mbox{Inv}\sN}
\def\sNegBin{\sN\sB}

\def\dGamma{\mbox{Gamma}}
\def\dInvGam{\mbox{Inv}\Gamma}

\def\Cov{\mbox{Cov}}
\def\Mgf{\mbox{Mgf}}

\def\mis{{mis}} 
\def\obs{{obs}}

\def\argmax{\operatornamewithlimits{\text{argmax}}}
\def\argmin{\operatornamewithlimits{\text{argmin}}}
\def\argsup{\operatornamewithlimits{\text{argsup}}}
\def\arginf{\operatornamewithlimits{\text{arginf}}}


\def\minimize{\operatornamewithlimits{\text{minimize}}}
\def\maximize{\operatornamewithlimits{\text{maximize}}}
\def\suchthat{\text{such that}}


\def\relstack#1#2{\mathop{#1}\limits_{#2}}
\def\sfrac#1#2{{\textstyle{\frac{#1}{#2}}}}


\def\comment#1{
\vspace{0.5cm}
\noindent \begin{tabular}{|p{14cm}|}  
\hline #1 \\ 
\hline 
\end{tabular}
\vspace{0.5cm}
}


\def\mytext#1{\begin{tabular}{p{13cm}}#1\end{tabular}}
\def\mytextB#1{\begin{tabular}{p{7.5cm}}#1\end{tabular}}
\def\mytextC#1{\begin{tabular}{p{12cm}}#1\end{tabular}}

\def\jump{\vskip3mm\noindent}

\def\KL{\text{KL}}
\def\N{\text{N}}
\def\Var{\text{Var}}

\def \E {\mathbb{E}}
\def \BigO {\text{O}}
\def \IG {\text{IG}}
\def \Beta {\text{Beta}}



\newcommand{\mgc}[1]{{\color{blue}#1}}
\newcommand{\joc}[1]{{\color{red}#1}}


\usepackage{titlesec}


\titleformat{\section}{\normalfont\Large\bfseries}{\thesection}{1em}{ }
\titleformat{\subsection}{\normalfont\large\bfseries\leftskip 0ex}{\thesubsection}{1em}{ }
\titleformat{\subsubsection}{\normalfont\large\bfseries\leftskip 0ex}{\thesubsubsection}{1em}{}
%\titleformat{\paragraph}[runin]{\normalfont\normalsize\bfseries}{\theparagraph}{1em}{}
%\titleformat{\subparagraph}[runin]{\normalfont\normalsize\bfseries}{\thesubparagraph}{1em}{}


%\usepackage{indentfirst}
%\let\@afterindentfalse\@afterindenttrue
%\@afterindenttrue
 
\titlespacing\section{0pt}{12pt plus 4pt minus 2pt}{2pt plus 2pt minus 2pt}
\titlespacing\subsection{0pt}{12pt plus 4pt minus 2pt}{2pt plus 2pt minus 2pt}
\renewcommand{\baselinestretch}{1.6}


%\makeatletter
%\renewcommand\section{\@startsection {section}{1}{\z@}%
%	{-3.5ex \@plus -1ex \@minus -.2ex}%
%	{2.3ex \@plus.2ex}%
%	{\normalfont\Large\bfseries}}% from \Large
%\renewcommand\subsection{\@startsection{subsection}{2}{\z@}%
%	{-3.25ex\@plus -1ex \@minus -.2ex}%
%	{1.5ex \@plus .2ex}%
%	{\normalfont\Large\bfseries}}% from \large
%\renewcommand\subsubsection{\@startsection{subsubsection}{3}{\z@}%
%	{-3.25ex\@plus -1ex \@minus -.2ex}%
%	{1.5ex \@plus .2ex}%
%	{\normalfont\large\bfseries}}% from \normalsize

\begin{document}

\makeatletter
%\renewcommand\section{\@startsection {section}{1}{\z@}%
%                                  {-0.1ex \@plus -0.1ex \@minus -.1ex}%
%                                  {0.1ex \@plus.1ex}%
%                                  {\normalfont\Large\bfseries}}% from \Large
 
 
%\renewcommand\subsection{\@startsection{subsection}{2}{\z@}%
%                                     {-1.25ex\@plus -1ex \@minus -.2ex}%
%                                     {0.1ex \@plus .2ex}%
%                                     {\normalfont\Large\bfseries}}% from \large
%\renewcommand\subsubsection{\@startsection{subsubsection}{3}{\z@}%
 %                                    {-1.25ex\@plus -1ex \@minus -.2ex}%
%                                     {1.5ex \@plus .2ex}%
%                                     {\normalfont\large\bfseries}}% from \normalsize


\makeatother

\maketitle

\tableofcontents
\listoffigures
% \listoftables

% Abstract
\section{Abstract}

Bayesian models offer great flexibility, but can be computationally demanding
to fit. The gold standard for fitting Bayesian models, when posterior
distributions are not available analytically, are Monte Carlo Markov Chain
methods. However these can be slow and prone to convergence problems.
Approximate methods of fitting Bayesian models allow these models to be fit
using deterministic algorithms in substantially less time.  Variational Bayes
(VB) is a method for approximating the posterior distributions of the model
parameters sometimes with only a slight loss of accuracy.

In this thesis, we consider two important problems -- zero inflated mixed
models and variable selection for linear models. Zero-inflated models have many
applications in areas such as manufacturing and public health, but pose
numerical issues when fitting them to data. We apply a variational
approximation to Zero-Inflated Poisson mixed models with Gaussian distributed
random effects using a combination of Variational Bayes and the Gaussian
Variational Approximation. We demonstrate that this approximation is accurate
and fast on a number of simulated and benchmark data sets. We also incorporate
a novel parameterisation of the covariance of the Gaussian Variational
Approximation using the Cholesky factor of the precision matrix, similiar to
\cite{Tan2016}, and discuss the computational advantages of this
parameterisation due to the sparsity of the precision matrix for mixed models
and resolve associated numerical difficulties.

The second problem we address is variable selection, a task of central
importance in modern statistics. Here, Bayesian model selection has the
advantage of incorporating the uncertainty of the model selection process
itself which propagates to the estimates of the model  parameters. Linear
regression models with Gaussian priors are ubiquitous in applied statistics due
to their ease of fitting and interpretation. We use the popular $g$-prior
\cite{Zellner1986} for model selection of linear models with normal priors
where $g$ is a prior hyperparameter. This raises the question of how best to
choose $g$. \cite{Liang2008} show that a fixed choice of $g$ leads to problems,
such as Bartlett's Paradox and the Information Paradox. These paradoxes, and
other problems, can be avoided by putting a prior on $g$. Using several popular
priors on $g$, we derive exact expressions for the model selection Bayes
Factors in terms of special functions depending only on  the sample size,
number of covariates and correlation of the model being considered. We show
that these expressions are accurate, fast to evaluate and numerically stable.
An R package \texttt{blma} for doing Bayesian linear model averaging using
these exact expressions has been  released on GitHub.

For data sets with a small number of covariates, it is computationally feasible
to perform exact model averaging. As the number of covariates increases the
model space becomes too large to explore exhaustively.  Recently,
\cite{Rockova2017} introduced Particle EM, a population-based method for
efficiently exploring a subset of the model space with high posterior
probability. The population-based method allows the method to seek multiple
local modes, and captures greater total posterior mass from the model space
than choosing a single model would. We extend this method using Particle
Variational Approximation and the exact posterior marginal likelihood
expressions to derive a computationally efficient algorithm for model selection
on data sets with a large number of covariates. We demonstrate the algorithm's
performance on a number of data sets for different combinations of $g$-prior,
model selection prior and population size. We also compare our method to the
existing methods such as lasso, SCAD, MCP and PEM in terms of model selection
performance,  and show that our method outperforms these. We also show that
total posterior mass increases and mean marginal variable error decreases, as
the number of models in the population increases.
% Draw attention to speed 8s on 20 cores
for n = 600, p = 7200 sized problem. Our algorithm performs very well relative
to previous algorithms in the literature, completing in 8 seconds on a model
selection problem with a sample size of 600 and 7200 covariates.

%! TEX root = thesis.tex


% Abstract
\chapter*{Abstract}

Bayesian models offer great flexibility, but can be computationally demanding
to fit. The gold standard for fitting Bayesian models, when posterior
distributions are not available analytically, are Monte Carlo Markov Chain
methods. However, these can be slow and prone to convergence problems.
Approximate methods of fitting Bayesian models allow these models to be fit
using deterministic algorithms in substantially less time.  Variational Bayes
(VB) is a method for approximating the posterior distributions of the model
parameters sometimes with only a slight loss of accuracy.
In this thesis, we consider two important problems --  variable selection for linear models, and
zero inflated mixed
models. 

The first problem we address is variable selection, a task of central
importance in modern statistics. Here, Bayesian model selection has the
advantage of incorporating the uncertainty of the model selection process
itself which propagates to the estimates of the model  parameters. Linear
regression models with Gaussian priors are ubiquitous in applied statistics due
to their ease of fitting and interpretation. We use the popular $g$-prior
\cite{Zellner1986} for model selection of linear models with normal priors
where $g$ is a prior hyperparameter. This raises the question of how best to
choose $g$. \cite{Liang2008} show that a fixed choice of $g$ leads to problems,
such as Bartlett's Paradox and the Information Paradox. These paradoxes, and
other problems, can be avoided by putting a prior on $g$. Using several popular
priors on $g$, we derive exact expressions for the model selection Bayes
Factors in terms of special functions depending only on  the sample size,
number of covariates and correlation of the model being considered. We show
that these expressions are accurate, fast to evaluate, and numerically stable.
An R package \texttt{blma} for doing Bayesian linear model averaging using
these exact expressions has been  released on GitHub.

For data sets with a small number of covariates, it is computationally feasible
to perform exact model averaging. As the number of covariates increases the
model space becomes too large to explore exhaustively.  Recently,
\cite{Rockova2017} introduced Particle EM (PEM), a population-based method for
efficiently exploring a subset of the model space with high posterior
probability. The population-based method allows the method to seek multiple
local modes, and captures greater total posterior mass from the model space
than choosing a single model would. We extend this method using Particle
Variational Approximation and the exact posterior marginal likelihood
expressions to derive a computationally efficient algorithm for model selection
on data sets with a large number of covariates. We demonstrate the algorithm's
performance on a number of data sets for different combinations of $g$-prior,
model selection prior and population size. We also compare our method to the
existing methods such as lasso, SCAD, and MCP penalized regression methods, and
PEM in terms of model selection performance,  and show that our method
outperforms these. We also show that total posterior mass increases and mean
marginal variable error decreases, as the number of models in the population
increases.
% Draw attention to speed 8s on 20 cores
%For $n = 600$, $p = 7200$ sized problem. 
Our algorithm performs very well relative to previous algorithms in the
literature, completing in 8 seconds on a model selection problem with a sample
size of 600 and 7200 covariates.

The second problem we address is zero-inflated models have many applications in
areas such as manufacturing and public health, but pose numerical issues when
fitting them to data. We apply a variational approximation to zero-inflated
Poisson mixed models with Gaussian distributed random effects using a
combination of VB and the Gaussian Variational Approximation (GVA). We
demonstrate that this approximation is accurate and fast on a number of
simulated and benchmark data sets. We also incorporate a novel parameterisation
of the covariance of the GVA using the Cholesky factor of the precision matrix,
similar to \cite{Tan2018}, and discuss the computational advantages of this
parameterisation due to the sparsity of the precision matrix for mixed models
and resolve associated numerical difficulties.


\chapter{Introduction}

% Outline that John suggested
\section{Motivation}

The advent of digital computers and the internet have lead to an explosion in
the volume of data being collected. With technological progress marching on,
this trend seems only set to continue and accelerate. In the future, as
technology continues to advance more data will be able to be stored and
processed, and so this trend of increasing volumes of data is set to continue
\citep{Gandomi2015}. But this data is only of value if it can be analysed and
understood.

This incredible increase in the volume of data has introduced new computational
difficulties in processing and modelling such large amounts of data, so-called
\emph{Big Data}, which is so large that it is difficult to process on one
computer. This data raises new challenges which modern statisticians must be
ready to meet. Approaches to modelling data are needed which can handle large
volumes of data in a computationally efficient manner while retaining the
probabilistic underpinning of classical statistics and statistical machine
learning, providing a rigorous underlying theory for inference. This realisation
has created an explosion of interest in \emph{Data Science}, incorporating ideas from
both statistics and computer science in recent years. Machine learning problems
are being tackled with algorithms which use probability models for the data --
motivating the development of the new field of statistical learning which
combines many of the best elements of statistics and machine learning
\citep{James:2014:ISL:2517747, MacKay:2002:ITI:971143,
hastie01statisticallearning, Murphy:2012:MLP:2380985}.

\section{Choosing an inferential paradigm}

How one proceeds given the above needs can be addressed through an inferential
paradigm. The most common of these are  the frequentist and Bayesian statistical
paradigms. The difference between frequentist and Bayesian approaches begins
with a difference in philosophy. Frequentists define an event's probability as
its' relative frequency after a large number of trials. While Bayesians view
probability as our reasonable expectation about an event, representing our state
of knowledge about the event.

There are many practical reasons to choose Bayesian approaches to modelling
data. It is flexible in modelling statistical complications, such as missing or
hierarchical data, and complicated models can be built by chaining together
multiple levels of simple models. These models can then be fit to data by
calculating the posterior probability of the parameters using Bayes' Theorem,
\begin{equation}\label{eq:bayes_theorem}
	p(\vtheta | \vy) = \frac{p(\vy | \vtheta) p(\vtheta)}{\int p(\vy | \vtheta) p(\vtheta) d \vtheta}
\end{equation}

\noindent where $\vy$ is a vector of observed data, $\vtheta$ are the model
parameters, $p(\vy|\vtheta)$ is the likelihood function, $p(\vtheta)$ is a
prior distribution on $\vtheta$, and $p(\vy)=\int
p(\vy|\vtheta)p(\vtheta)d\vtheta$.  Here the integral is performed over the
range of $\vtheta$. If a subset of $\vtheta$ are discrete random variables
then the integral over these parameters is replaced with a combinatorial sum
over all possible values of these discrete random variables.

There are many models which are difficult to fit under the frequentist paradigm,
as the likelihood can be difficult to maximise for complex models. Furthermore,
as the Bayesian paradigm treats each of the parameters in a model as uncertain,
the full uncertainty associated with all of the parameters can be estimated via
the uncertainty in the posterior distribution. This approach avoids many of the
pitfalls of statistical inference encountered with the frequentist approach
using significance testing and p-values \citep{Cox2005}.

The ability to build a model one component at a time and have the uncertainty
propagate through the model makes Bayesian modelling  particularly appropriate
for mixed effects and hierarchical models. In particular, uncertainty regarding
model selection is taken into account in the context of model selection. Thus
for the two classes of problems we consider in this thesis the Bayesian approach
is more suitable.

\section{Research problems}

In this section, we introduce the major problems that will be addressed in this
thesis. The themes of flexible modelling of data using Generalised Linear Mixed
Models and model selection of linear models with normal priors  will be
explored.

\subsection{Exponential family and the canonical form of linear regression
models}

The concept of the exponential family of probability distributions was first
introduced by \cite{Koopman1935} and \cite{pitman_1936}. The canonical form of a
regression model from the exponential family is
\begin{equation}\label{eq:exponential_family}
	p(\vy | \vtheta) = h(\vy) \exp \{ \vtheta^\top T(\vy) - b(\vtheta) \}
\end{equation}
for a parameter vector $\vtheta \in \mathbf{\Theta}$, and observed data $\vy$.
The sufficient statistic $T$ and $h$ are functions of the observed data, while
the cumulant function $b(\vtheta)$ is a function of the parameter $\vtheta$. The
cumulant function is the logarithm of the normalisation constant.

Many commonly used probability distributions of practical interest, such as the
Gaussian, Bernoulli, Poisson, Exponential and Gamma probability distributions,
can be expressed as an exponential family by making an appropriate choice of
$h$, $T$ and $b$ functions. The exponential family of distributions have several
appealing statistical and computational properties which derive from the
convexity of the parameter space $\Theta$ for which the exponential family
distribution is defined, and the convexity of the cumulant function
\citep{Jordan2010}. The mean of an exponential family distribution can be
obtained by calculating the first derivative of the cumulant function and then
evaluating at zero, while the variance can be obtained by calculating the second
derivatives of the cumulant function and evaluating at zero.

The exponential family of distributions allow us to extend linear models to more
general situations where the response variable is not normally distributed but
may be categorical, discrete or continuous and the relationship between the
response and the explanatory variables need not be of simple linear form.  By
choosing the parameterisation $\vtheta = \mX \vbeta$ where $\mX$ is the matrix
of observed covariates in $\R^{n \times p}$ and $\vbeta$ are regression
parameters in $\R^p$, for $n$ the sample size and $p$ the number of covariates,
a canonical form of generalised linear regression models may be written as
\begin{equation}\label{eq:glm}
	\log p(\vy | \vtheta) = \vy^\top \mX \vbeta - \vone^\top b(\mX \vbeta) + \vone^\top c(\vy)
\end{equation}
where $c(\vtheta)$ is the log of $h(\vy)$ from 
(\ref{eq:exponential_family}). A choice of $b(x) = e^x$ corresponding to the
Poisson family of distributions specifies a Poisson linear model appropriate for
modelling count data, while a choice of $b(x) = \log(1 + e^x)$ corresponding to
the logistic family of distributions specifies a logistic linear model
appropriate to modelling binary data.

\subsection{Generalised Linear Mixed Models}

Generalised Linear Mixed Models, an extension of Generalised Linear Models to
include both fixed and random effects, are applicable to many complicated
modelling situations.

Linear and generalised linear regression models are the standard tools used by
applied statisticians to explain the relationship between an outcome variable
and one or more explanatory variables. They provide a general method  to analyse
quantified relationships between variables within a data set in an easily
interpretable way. A standard assumption is that the outcomes are independent,
and that the effect of the explanatory variables on the outcome is fixed. But if
the outcomes are dependent and this assumption is not met, then linear and
generalised linear models can be extended to linear mixed models. These allow us
to incorporate dependencies amongst the  observations via the assumption of a
more complicated covariance structure, including random effects for  different
subgroups or longitudinal data and other extensions such as splines. This
additional flexibility makes their application popular in many fields, such as
public health, psychology and agriculture \citep{Kleinman2004, Lo2015, Kachman2000}.

In the frequentist paradigm, model parameters are fixed and uncertainty enters
the model through random errors, which have an associated covariance. The data
is modelled as a combination of these fixed parameters and random errors. In
the Bayesian paradigm, the uncertainty in the parameters and the data is
accounted for by the likelihood function.

\subsubsection{A Canonical Form for Generalised Linear Mixed Models}

The generalised form for linear models in  (\ref{eq:glm}) can easily be
extended to include random effects.  Following the conventions for Generalised
Design of \cite{Zhao2006}, we adopt the canonical form for Generalised Linear
Mixed Models exponential family with Gaussian random effects take the general
form
$$
\begin{array}{rl}\label{eq:glmm}
	p(\vy | \vbeta, \vu) &= \exp{\{ \vy^\top (\mX \vbeta + \mZ \vu) - \vone^\top b(\mX \vbeta + \mZ \vu) + \vone^\top c(\vy) \}}, \\
	\vu | \mG &\sim \N(\vzero, \mG),
\end{array}
$$

\noindent 
where the fixed effects are denoted by the vector $\vbeta$, the random effects
are denoted by $\vu$ and $\mG$ is the covariance matrix of random effects. The
covariance structure in $\mG$ is usually chosen to capture the dependencies of
interest between the observations in the application, such as the dependency
between repeated observations on an individual within a longitudinal study,
% TODO(Mark) Rewrite this sentence
the dependency between observations within a cluster in a hierarchical model or the
spatial dependency between observations are close to within another within a
spatial model.
%
The design matrix for the fixed effects is denoted by $\mX$ and
the design matrix for the random effects are denoted by $\mZ$.

Random effects are very flexible in the variety of models they allow us to fit
to our data. Through specification of the covariance structures in the matrix
$\mG$ with the appropriate data in the design matrix $\mZ$, complicated
dependencies amongst the responses $\vy$ can be specified, allowing modelling of
longitudinal data, fitting smoothing splines to the data and modelling spatial
relationships between responses. This allows us to fit hierarchical models with
random intercepts and slopes, capturing levels of variation within groups within
the data \citep{Gelman2007}.
% TODO: Not happy with how this paragraph is written. I can express this idea
% better.

% FIXME: Is this the best place for this?
While mixed models are very useful for gaining insight into a data set, fitting
them can be computationally challenging. For all but the simplest situations,
fitting these models involves computing high-dimensional integrals which are
often analytically and computationally intractable. The standard technique for
fitting Bayesian versions of these models is to use Monte Carlo Markov Chains
techniques. Thus, an approximation must be used in order to fit these models
within a reasonable time frame. Our approach to this problem is outlined in
Chapter 4.

\section{Splines and smoothing}

While linear models are statistically convenient to work with and easy to
interpret once fitted, the relationship between the response and explanatory
variables may not always be linear in practice. Thus a generalisation of linear
models to nonlinear situations is needed that still retains the beneficial
statistical and interpretive properties of linear models as much as possible.
The most general form of the univariate regression problem is
$ y_i = f(x_i) $
where $f: \R \to \R$ is unknown, and we wish to estimate it.  Fully
nonparametric regression is a difficult problem to solve, but the problem can
be simplified by prespecifying the points at which the function may change
curvature, which we refer to as \emph{knots}.

\newpage

% \subsubsection{Penalised spline}
% \subsubsection{B-splines}
\subsection{B-Splines}

% This is taken from the Wikipedia page on the subject. Yet somehow, I've managed to avoid including anything
% that's interesting or useful about B-Splines.

There are many families of basis functions which can be conveniently used for
function approximation, including orthogonal polynomials. The B-spline basis
\citep{DeBoor1972} is numerically stable and efficient to computationally
evaluate. A B-Spline is a piecewise polynomial function of degree $< n$ in a
variable $x$. It is defined over a domain $\kappa_0 \leq x \leq \kappa_m, m=n$.
The points where $x = \kappa_j$ are known as knots or break points. The number
of internal knots is equal to the degree of the polynomial if there are no knot
multiplicities. The knots must be in ascending order. The number of knots is
the minimum for the degree of the B-spline, which has a non-zero value in the
range between the first and last knot. Each piece of the function is a
polynomial of degree less than $n$ between and including adjacent knots. A
B-Spline is continuous at its' knots. When all internal knots are distinct its
derivatives are also continuous up to the derivative of degree $n - 1$. If
internal knots coincide at a given value of $x$, the continuity of derivative
order is reduced by 1 for each additional knot.

For any given set of knots, the B-spline for approximating a given function is
a unique linear combination of basis functions recursively defined as
$$
\begin{array}{rl}
	B_{i, 0}(x) & := \begin{cases}                                                                                                        
	1           & \text{if } \kappa_i \leq x < \kappa_{i+1};  \quad \text{and}                                                                                         \\
	0           & \text{otherwise,}                                                                                                        
	\end{cases}
	% B_{i, k}(x) & := \frac{x - \kappa_i}{\kappa_{i + 1} - \kappa_i} Q_{i, k-1} (x) + 
	% 									\frac{\kappa_{i + k + 1} - x}{\kappa_{i + k + 1} - \kappa_{i + 1}} Q_{i, k-1} (x). 
\end{array}
$$

\noindent for $i = 1, \ldots, K + 2M -1$ and
$$
\begin{array}{rl}
	B_{i, k}(x; \vkappa) & \ds = \frac{x - \kappa_i}{\kappa_{i + k} - \kappa_i} Q_{i, k-1} (x; \vkappa) + 
										\frac{\kappa_{i + k + 1} - x}{\kappa_{i + k + 1} - \kappa_{i + 1}} Q_{i, k-1} (x; \vkappa)
\end{array}
$$

\noindent for $i = 1, \ldots, K + 2 M - m$ with
% B-Splines
$$
Q_{m, i}(x; \kappa) =
\begin{cases}
B_{m, i}(x; \kappa),& \kappa_{i + m} > \kappa_i; \quad \mbox{and} \\
0, & \text{otherwise}.
\end{cases}
$$

We define the B-Spline basis in this way so that the definition remains correct
in the case where knots are repeated in $\vkappa$. We choose piecewise cubic
splines as cubics are numerically well behaved while still capturing the
curvature of functions we wish to approximate well
\citep{Press:2007:NRE:1403886}. Thus we select the knot sequence $\vkappa$ to be
$$
a = \kappa_1 = \kappa_2 = \kappa_3 = \kappa_4 < \kappa_5 < \ldots < \kappa_{K+5} = \kappa_{K+6} = \kappa_{K+7} = \kappa_{K+8} = b.
$$

There are many ways of choosing knots for applied statistical problems. A
typical approach is to choose the internal knots using the sample quantiles of
the data set being examined.
A common choice is to select 
$\min(n_U/4, 35)$ internal knots
where $n_U$ is the unique number of $x_i$'s.

\subsection{O'Sullivan Splines}
% $B_{ik} = B_k (x)$
% $B_x = [B_1(x), \ldots, B_K+4(x)]$
% Divided difference notation?
% Lagrange's interpolating polynomials?
% Semiparametric regression / Connection to mixed models

In this section, we follow the discussion of semiparametric regression in
\cite{ruppert_wand_carroll_2003}.  Using a mixed models setup to fit spline
models protects against overfitting, we construct a $\mZ$ matrix with the
appropriate B-Spline function evaluations in each of row of the matrix, where
each column in the matrix corresponds to one of the knots we have selected.

O'Sullivan introduced a class of penalised splines based on the B-spline basis
functions in \cite{OSullivan1986} which are a direct generalisation of
smoothing splines. Let $B_1$, \ldots, $B_{K+4}$ be the cubic B-spline basis
functions defined by the knots $\kappa_1$ to $\kappa_{K+4}$. O'Sullivan splines
are splines which are penalised using the penalty matrix $\mOmega$. Let
$\mOmega$ be the $(K+4) \times (K+4)$ matrix where the $(k, k')-th$ element is
\[ \mOmega_{k k'} = \int_a^b B''_k(x) B''_{k'}(x) dx. \] Then the O'Sullivan
spline estimate of the true function $f$ at the point $x$ is
\begin{equation*}
\widehat{f}_O(x; \lambda) = \mB_x \widehat{\vnu}_O,
\end{equation*}
where $\widehat{\vnu}_O = (\mB^\top \mB + \lambda \mOmega)^{-1} \mB^\top \vy$,
as shown in \cite{ruppert_wand_carroll_2003}.

The matrix $\mOmega$ is defined in this way to penalise oscillation, which is
measured by the second derivative. This penalty differs from the penalty for
``penalised B-Splines'' or P-splines in that the P-spline penalty matrix is
$\mD_2^\top \mD_2$ where $\mD_2$ is the second-order differencing matrix.

\section{Variable selection}

It is often the case in applied statistics that many covariates are available,
but it is unknown a priori which covariates explain the response variable of
interest. An automatic method of exploring which model among many possible
candidate models incorporating these covariates explains the response variable
best would relieve the burden of having to fit and compare the performance of
many such models manually.

The problem of selecting a statistical model from a set of candidate models
given a data set, hence referred to as \emph{model selection}, is one of the
most important problems encountered in practice by applied statisticians. It is
one of the central tasks of science, and there is a correspondingly large
literature on the subject -- \cite{Claeskens:1251912, NengjunYi2013,
Johnstone2009} together give a comprehensive overview.

The problem of model selection for normal linear models is particularly well
studied, owing to the popularity and importance of normal linear models in
applications. While new types of model are continually being developed, linear
models with normal priors remain a popular and essential modelling tool owing to
the ease of fitting these models, statistical inference on the parameters and,
most importantly, the ease which these models can be interpreted. But for a data
set with a moderate or large number of parameters, the question is immediately
raised of which covariates we should include in our model. One of the problems
that we address in this thesis is \emph{variable selection} on linear models
with normal priors.

The bias-variance trade-off is one of the central issues in statistical learning
\citep{Murphy:2012:MLP:2380985, Bishop:2006:PRM:1162264,
hastie01statisticallearning}. The guise this issue takes in model selection is
balancing the quality of the model fit against the complexity of the model, in
an attempt to find a compromise between over-fitting and under-fitting, in the
hope that the model fit will generalise well beyond the training data we have
observed to the general population and that we haven't simply learned the noise
in the training set.

There have been many approaches to model selection proposed, including criteria
based approaches, approaches based on functions of the
residual sum of squares, penalised regression such as the lasso and $L_1$
regression, and Bayesian modelling approaches. Model selection is a difficult
problem in high-dimensional spaces in general because as the dimension of the
space increases, the number of possible models increases combinatorially
\citep{Schelldorfer2010}. Many model selection algorithms use heuristics in an
attempt to search the model space more efficiently but still find an optimal or
near-optimal model within a reasonable period of time. A major motivation for
this field of research is the need for a computationally feasible approach to
performing model selection on large scale problems where the number of
covariates is large.


% Non-Bayesian
\subsection{Frequentist approaches to model selection}
\subsubsection{Information Criteria}
% Need to define gamma first before referring to it. Is this the right place for this?

Let $\vgamma$ be a $p$-dimensional vector of indicators, where a $1$ in the
$j$th position indicates that the $j$th covariate is included in the model,
while a $0$ indicates it is excluded. Thus $\vgamma$ defines a model with
covariates drawn from a $p$ column data matrix $\mX$.

In a frequentist context, there are many functions which can be used to judge
which model is best, such as Akaike's Information Criteria (AIC) and the Bayesian 
Information Criteria (BIC). These are functions $f\colon \vgamma \to \R^+$ which allow the
models under consideration to be ranked, and the best model chosen from those
available. Thus the optimal model selected by an information criteria is
$\vgamma^* = \min_\vgamma f(\vgamma)$. These functions typically attempt to
balance log-likelihood against the complexity of the model, achieving a
compromise between each.

% \mgc{AIC, BIC, DIC, Mallow's $C_p$}

Information Criteria are frequently used to compare models. Letting $\vgamma$
denote the candidate model, Information Criteria take the form ``negative twice
times the log-likelihood plus a term penalising for complexity of the mode''
$$
	\text{Information Criteria} = -2 \log p(\vy | \widehat{\vtheta}_\vgamma) + \text{complexity penalty},
$$

\noindent where $\widehat{\vtheta}_{\gamma}$ is the maximum likelihood estimate of
the model parameters $\vtheta$ for the model $\vgamma$ and $\log p(\vy |
\widehat{\vtheta_\gamma})$ is the log-likelihood of that model with that
parameter estimate and the complexity penalty is a function of the sample size
$n$ and the number of parameters $p$ of the model. Information criteria attempt
to successfully compromise between goodness of fit and model complexity.

The most popular of the Information Criteria is the 
AIC \citep{Akaike1974}. AIC calculates an estimate of the information lost
when a given model is used to represent the process that generates the data and
so is an estimator of the Kullback-Leibler divergence of the true model from the
fitted model. The AIC of the model $\vgamma$ is defined as
$$
	\text{AIC}(\vgamma) = -2 \log p(\vy | \widehat{\vtheta}_\vgamma) + 2 p_\vgamma,
$$

\noindent where $p_\vgamma$ is the number of parameters in the model $\vgamma$.
The model with the lowest AIC is selected as the 'best'.

Of a similar form as the AIC, but derived via a more Bayesian framework is the
BIC. The BIC approximates the posterior
probability of the candidate model $\vgamma$. The BIC is defined as
$$
	\text{BIC}(\vgamma) = -2 \log p(\vy | \widehat{\vtheta}_\vgamma) + p_\vgamma \log(n).
$$

\noindent This is a more severe penalty for model complexity than in the
Akaike's Information Criteria when $n$ is greater than $8$. BIC can be shown to
be approximately equivalent to model selection using Bayes Factors in certain
contexts \citep{Kass1993}.

Alternatively, the process of model selection can be made implicit in the model fitting
process itself, ridge regression \citep{Casella1980}, of which the well-known
lasso is a special case \citep{Tibshirani1996}. As \cite{Breiman1996} and
\cite{Efron2013} showed, while  the standard formulation of a linear model is
unbiased, the goodness of fit of these models is numerically  unstable. Breiman
showed that by introducing a penalty on the size of the regression coefficients
such as  in ridge regression, this numerical instability can be avoided. This
reduces the variances of the coefficient estimates, at the expense of
introducing some bias -- which is another instance of the bias-variance
trade-off.

\subsubsection{Penalised regression}

Penalised regression methods trade introducing some bias in the estimator for
reducing the variance and thus fitting a more parsimonious model. The major
advantages are that a model with fewer covariates will be correspondingly
easier to interpret, and that the variance of the regression co-efficient
estimator will be less. In penalised regression, the regression coefficients
are subjected to a penalty or constraint. This is typically expressed as the
minimisation of the sum of a goodness of fit function such as squared Euclidean
distance and a penalty function $$ \widehat{\vbeta}_{\text{penalised}} =
\argmin_\vbeta \|\vy - \mX \vbeta\|_2^2 + \text{penalty}(\vbeta).  $$

From a Bayesian perspective, the penalty can be considered as a prior
distribution on the regression coefficients where smaller values of $\vbeta$ are
given more weight than larger ones. Here the penalised estimate of the
regression coefficients is the mode of their posterior distribution.

\subsubsection{Ridge regression}

Ridge regression is a penalised regression method, introduced in
\cite{Hoerl1970}. The penalty on the regression coefficients is the Euclidean
norm of the regression coefficients. This penalty shrinks the estimated
coefficients towards zero. The ridge regression coefficients can thus be
estimated by solving the constrained optimisation problem

$$
\widehat{\vbeta}_{\text{ridge}} = \argmin_\vbeta \|\vy - \mX \vbeta\|_2^2 \quad \text{ subject to } \quad \|\vbeta\|_2 \leq \lambda
$$

\noindent 
where $\lambda$ is a pre-specified free parameter specifying the amount of
regularisation. This constrained optimisation problem can be transformed by the
method of Lagrange multipliers into the sum of the residual sum of squares and
the product of the Lagrange multiplier and the constraint, which acts as a
penalty on the Euclidean norm of the regression coefficients.

% Both functions are quadratic in \vbeta

\subsubsection{Lasso regression}

Lasso regression is a penalised regression method developed in
\cite{Tibshirani1996}, which was directly inspired by ridge regression.  The
penalty is the $l_1$ norm of the coefficient vector.  The lasso regression
coefficients can be estimated by solving the constrained optimisation problem
$$
\widehat{\vbeta}_{\text{lasso}} = \argmin_\vbeta \|\vy - \mX \vbeta\|_2^2 \text{ subject to } \|\vbeta\|_1 \leq \lambda,
$$

\noindent 
where $\lambda$ is a pre-specified free parameter specifying the amount of
regulation. Similarly to the constrained optimisation problem for ridge
regression, the constrained optimisation problem can be transformed by the
method of Lagrange multipliers into the sum of the residual sum of squares and
the product of the Lagrange multiplier and the constraint, which acts as a
penalty on the $l_1$ norm of the regression coefficients. It follows from
Minkowski's inequality that the function above is convex, and thus the
optimisation problem is convex, and can be solved using standard methods from
convex optimisation \citep{Boyd2010}.  The constraint on the $l_1$ norm has the
effect of shrinking the coefficients, and setting some of them to zero. This
forces the models fit by lasso regression to be sparse, providing model
selection as part of the model-fitting process.

A disadvantage of lasso regression is that the constraint on the regression
coefficients depends on the free tuning parameter which must be selected a
priori or through cross-validation. But a much greater issue is that the model
selection process intrinsic to lasso regression does not take into account the
uncertainty of the model selection process itself, particularly the selection of
$\lambda$, as Bayesian model selection methods do.

\subsection{Bayesian approaches to Model Selection}

Parallel to the frequentist approaches, model selection can be performed using
a Bayesian approach. This can be done, for example, by using Bayes Factors  to
compare the posterior likelihoods of the candidate models to see which is most
probable given the observed data\citep{Kass1993}. Rather than selecting one
candidate model, several models can be combined together using Bayesian model
averaging \citep{Hoeting1999, Raftery1997, Fernandez2001,
Papaspiliopoulos2016}.

\subsubsection{Variable selection}

A special case of model selection is variable selection, where the focus is on
selecting individual covariates, rather than entire models. Variable selection
approaches search over the variables in the model space for the best covariates
to include in the candidate model. Due to the large number of possible
combinations of covariates -- typically $2^p$ where $p$ is the number of
covariates, such searches are often stochastic. This approach can either be
Fully Bayesian or Empirically Bayesian \citep{Cui2008}.  This search can be
driven by posterior probabilities  \citep{Casella2006}, or by Gibbs sampling
approaches such in \cite{George1993}. These two approaches of model selection and
variable selection can be combined  \citep{Geweke1996}. Variable selection can
also be accomplished by selecting the median probability model, consisting of
those models whose posterior inclusion probability is at least $1/2$
\citep{Barbieri2004}.

A challenge to applying this method of model selection is that exact model
fitting may be computationally infeasible for models involving even moderate
numbers of observations and covariates, and popular alternatives for fitting
Bayesian models such as Monte Carlo Markov Chains (MCMC) are still extremely computationally intensive.

% \subsubsection{Linear regression with normal prior and $g$ hyperprior}

% Zellner's g-prior

% \[
% 	\vbeta_\vgamma | \sigma^2, \mathcal{M}_\vgamma \sim \N(\vzero, g \sigma^2 (\mX^\top \mX)^{-1})
% \]

% \noindent which scales the Fisher information $\sigma^2 (\mX^\top \mX)^{-1}$ by $g$, was first introduced in
% \cite{Zellner1986}. It is widely used for variable selection in linear models with normal priors owing to its'
% computational efficiency in evaluating marginal likelihoods and model selection and conceptually simple
% interpretation.

% This model specification performs shrinkage on the regression co-efficients. As stated in \cite{Hastie2015},
% it is best to `bet on sparsity': ``Use a procedure that does well in sparse  problems, since no procedure does
% well in dense problems.''. With different choices of hyper- prior on $g$, this shrinkage can be made to behave
% like Bayesian versions of the lasso, ridge regression or generalisations of these \citep{Hahn2015}.  Thus this
% model specification has the advantage of performing well on model selection problems where the true model is
% sparse in the sense that the number of true non-zero covariates $p$ is less than the number of samples $n$.

% \begin{itemize}
% % \item Sparsity
% % \item Ease of interpretation from fewer regression co-efficients
% % \item Computational convenience
% \item Convex optimisation problem
% % \item Does not take model selection uncertainty into account
% \item Bias-Variance trade-off -- trade some bias in the estimator for a decrease in variance
% \item Deal with collinearity
% \end{itemize}


\section{Approximate Bayesian inference}

When the prior and model chosen for a Bayesian model is conjugate, the
posterior distribution is available in closed form and can be easily
calculated.  When the prior is non-conjugate, the integral in Equation
\ref{eq:bayes_theorem} to calculate the posterior distribution  is typically
intractable and so numerical methods must be used to calculate it
approximately.  The gold standard for Bayesian inference is to use MCMC methods
such as Metropolis-Hastings or Gibbs sampling.  But these methods are
computationally intensive, to the point where they are simply impractical in
Big Data situations where $n$ or $p$ are large. Moreover, they can be prone to
convergence problems.  Thus there is a need for approximate Bayesian inference
methods which are less computationally intensive while being almost as
accurate.

 
\subsection{Variational Bayes}
\label{sec:vb}

We now introduce Variational Bayes (VB), the popular approximate inference method for
Bayesian models. It is used to accelerate Bayesian model fitting by tens or
hundreds of times, with sometimes only minor loss in accuracy for some models.
This method plays a central role in this thesis, particularly in the third and
fourth chapters.

% John says this sounds repetitive.
% As described previously, Bayesian models offer many advantages in flexibility and ease of interpretation. But
% such models may be computationally difficult or intractable to fit.  The calculation of the true posterior
% distribution is often either computationally intractable or no closed form exists for the posterior
% distribution and so an approximation is required.

As described previously, Bayesian models may be computationally difficult or
intractable to fit. The calculation of the true posterior distribution for the
model is often either computationally intractable or no closed form exists for
the posterior distribution. We may be able to gain much of the same insight from
a given data set by fitting an accurate approximation  of the model, allowing us
to summarise the data and perform statistical inference. Variational approximation aims
to approximate a true, possibly intractable probability distribution $p(x)$ by a
simpler, more tractable distribution $q(x)$ of known form.

Variational approximation often takes the form minimising the Kullback-Leibler
divergence between the true posterior $p(\vtheta|\vy)$ and an approximating
distribution $q(\vtheta)$, sometimes called a $q$-density. For an introduction,
see \cite{Ormerod2010}.

% John says this is repetitive
% The density function of a random vector $\vu$ is denoted by $p(\vu)$.  The conditional density function of a
% random vector $\vu$ given $\vv$ is denoted by $p(\vu|\vv)$. Consider a generic Bayesian model with parameter
% vector $\vtheta \in \Theta$. Throughout this section we assume that $\vy$ and $\vtheta$ are continuous random
% vectors.

The KL divergence between the probability distributions $p$ and $q$ is defined
as
$$
	\KL(q || p) \equiv \int q(\vtheta) \log \left [ \frac{q(\vtheta)}{p(\vtheta | \vy)} \right ] d \vtheta.
$$

Suppose that a class of candidate approximating distributions $q(\vtheta)$ is
parameterised by a vector variational parameters $\vxi$ and write
$q(\vtheta)\equiv q(\vtheta;\vxi)$. We attempt to find an  optimal approximating
distribution $q^*(\vtheta)$ such that
$$
	\ds q^*(\vtheta) = \argmin_{\vxi \in \Xi} \,  \text{KL} \{ {q(\vtheta;\vxi) || p(\vtheta|\vy)} \}.
$$

\noindent If $\vtheta$ is partitioned into $M$ partitions $\vtheta_1$,
$\vtheta_2$, \ldots, $\vtheta_M$ then a simple form of approximation to adopt is
the factored approximation of the form
$$
	q(\vtheta) = \prod_{i=1}^M q(\vtheta_i)
$$

\noindent where each of the density $q(\vtheta_i)$ is a member of a parametric
family of density functions.  This form of approximation is computationally
convenient, but assumes that the partitions of $\vtheta$ are completely
independent of one another.

The optimal mean field update for each of the parameters $\vtheta_i$ can be
shown to be
$$
	q^*(\vtheta_i) \propto \exp\left[ \E_q \left\{ \log p(\vy; \vtheta)\right\} \right].
$$

\noindent For details of the proof, and a more thorough introduction to the
topic of variational approximations, see \cite{Ormerod2010}. It can easily be
shown that
$$
	\ds \log p(\vy) = \int q(\vtheta;\vxi) \log \left[ \frac{p(\vy|\vtheta)p(\vtheta)}{q(\vtheta;\vxi)} \right] d\vtheta + \text{KL}(q(\vtheta;\vxi)||p(\vtheta|\vy)).
$$

\noindent As the Kullback-Leibler divergence is strictly positive, the first
term on the right hand side is a lower bound on the marginal log-likelihood
which we will define by
$$
	\ds \log \underline{p}(\vy;\vxi) \equiv \int q(\vtheta;\vxi)  \log \left[ \frac{p(\vy|\vtheta)p(\vtheta)}{q(\vtheta;\vxi)} \right] d\vtheta
$$

\noindent and maximizing $\log \underline{p}(\vy;\vxi)$ with respect to $\vxi$
is equivalent to minimizing $\text{KL}(q(\vtheta;\vxi)||p(\vtheta|\vy))$.

\noindent  The term $\log
\underline{p}(\vy;\vxi)$ is referred to as the variational lower bound.

When the optimal distributions for each $q_i^*(\vtheta_i)$ are calculated, they
yield a set of equations, sometimes called the consistency conditions, which
need to be  satisfied simultaneously. These yield a series of mean field updates
for the parameters of each approximating distribution. By executing the mean
field update equations in turn for each parameter in the model, the variational
lower bound for the model $\underline{p}(\vtheta; \vy)$ is iteratively
increased. It can be shown that by calculating $q_i^*(\vtheta_i)$ for  a
particular $i$ with the remaining $q_j^*(\vtheta_j)$, $j\ne i$ fixed, results in
a monotonic increase in the variational lower bound, and thus a monotonic
decrease in the Kullback-Leibler divergence between $p(\vtheta|\vy)$ and
$q(\vtheta)$.

The variational lower bound is maximised iteratively. On each iteration, the
value of each parameter in the model is calculated as the expectation of the
full likelihood relative to the other parameters in the model, which is
referred to as the mean field update. This is done for each parameter in the
model in turn until the variational lower bound's increase is negligible and
convergence is achieved. Note that this approach can be extended to a wide
range of models such as semiparametric models as has been formalized by
\cite{Rohde2015}.

This approach works well for classes of models where all of the parameters are
conjugate. For more general classes of models, the mean field updates are not
analytically tractable and general gradient-based optimisation methods must be
used, such as for the Gaussian Variational Approximation \citep{Ormerod2012}.
These methods are generally difficult to apply in practice, as the problems can
involve the optimisation of many parameters over high-dimensional, constrained
spaces whose constraints cannot be simply expressed.

% TODO - merge this into the rest of the text
Recently, several stochastic Variational Bayes approaches to approximation
problems of this type have emerged.  \cite{Gershman2012} used a uniform
weighted mixture of isotropic Gaussians to approximate complex posterior
distributions. The variational lower bound is approximated with first and
second-order Taylor series expansions, and then optimised with L-BFGS.  In
\cite{Kingma2013}, the expectations in the expression for the variational lower
bound are approximated using Monte Carlo integration. The variational lower
bound is reparameterised in terms of an auxiliary noise variable such as a
standard normal, to reduce the variance of the Monte Carlo estimate.
\cite{Tan2018} takes an approach closest to the one we will adopt, using a
Gaussian Variational Approximation.  By parameterising the covariance matrix of
the Gaussian using Cholesky factors of the precision matrix, the covariance
matrix is guaranteed to be sparse due to the conditional independence between
fixed and random effects of the mixed model. The variational lower bound can be
rewritten so that it does not depend on the variational parameters.  By making
a transformation in terms of a noise variable to standardise the variational
parameters, efficient gradient estimators can be derived, then estimated using
subsampling of the data set. Sampling from the fixed normal distribution on
each iteration rather than a multivariate normal depending on the variational
parameters in the current iteration reduces the variance of the estimator.
Subsampling of the data set and sampling from the noise variable make the
fitting algorithm doubly stochastic.

Other approximate Bayesian inference techniques exist in the literature, such
as Laplace approximation \citep{Tierney1986},   integrated nested Laplace
approximation \citep{Rue2009}, and Expectation Propagation \citep{Minka2013}.
These have been applied to the problem of fitting count models
\citep{Barber2016, KimWand2017}.  But Expectation Propagation requires very
difficult algebra to complete the derivations required for the updates, and can
exhibit convergence problems. Laplace approximation relies on a Gaussian
approximation to the log of the posterior found by Taylor expanding around the
mode, which performs poorly when the true posterior is not symmetric, as is the
case for Poisson regression models.
% General Design Bayesian Generalized Linear Mixed Model, as in \citep{zhao06}.
% This allows us to incorporate within-subject correlation, and smoothing
% splines (as in \citep{Wand2008}) in our models.

% Idea: We can use an approximation of the from q(\beta, \u, \Sigma) q(\rho)
% \Product q(r_i) and use GVA on q(\beta, \u, \Sigma) and mean field updates on
% \rho and r_i

% \subsection{Semiparametric Mean Field Variational Bayes}


% \subsubsection{Definitions}

% Two strategies for selecting a suitable class of approximating distributions $q$ such that
% an optimal distribution $q^*(\vtheta)$ can be found are
% (A) specifying the parametric form of $q$; or 
% (B) choosing $q$ to be of the factored form $q(\vtheta) = \prod_{i=1}^M q(\theta_i)$.
% For the second alternative,
% it can be shown (see Ormerod \& Wand, 2010, for example) that the optimal form of the
% approximating distributions $q_i$ for each parameter are of the form
% \begin{equation}\label{eq:consistency}
% 	q_i^*(\theta_i) \propto \exp{\{ \bE_{-q(\theta_i)} \log p(\vy, \vtheta) \}},  \quad 1\le i\le M.
% \end{equation}

%This approach works well for classes of models where all of the parameters are conjugate. For more general
%classes of models, mean field updates are not analytically tractable and general gradient-based optimisation
%methods must be used, as for the Gaussian Variational Approximation (see \citep{ormerod09}) used in this paper.
%These methods are generally difficult to apply in practice, as the problems can involve the optimisation of
%many parameters over high-dimensional, constrained spaces whose constraints cannot be simply expressed.

\subsection{Gaussian Variational Approximation}

In cases where there is a strong dependence between partitions of $\vtheta$,
such as between the parameters $\vmu$ and $\mSigma$ in a hierarchical Gaussian
model, a factored approximation may not approximate the true distribution
accurately. In this case, an alternate form of approximation may be used with
the parameters considered together to take their dependence into account. One
such form of approximation is the Gaussian Variational Approximation
\citep{Ormerod2012}, which assumes that the distribution of the parameters being
approximated is multivariate Gaussian. The covariance matrix of the Gaussian
allows the approximation to capture the dependence amongst the elements of
$\vtheta$, which increases the accuracy of the variational approximation
relative to the factored approximation. This will be the approach used in
Chapter 4.

\subsection{Laplace Method of approximation}
\label{sec:laplace_approximation}

Laplace's method of approximation, as described in \cite{butler_2007} or
\cite{MacKay:2002:ITI:971143}, is used to approximate integrals of a unimodal
function $f$ with negative second derivative at the mode, indicating that the
function is decreasing rapidly away from this point. The essential idea is that
if the function is decreasing rapidly away from the mode, the bulk of the area
under the function will be within a neighbourhood of the mode. Thus, the integral
of the function can be well approximated by an integral over the neighbourhood
of the mode. How large that neighbourhood needs to be is estimated using how
fast the function is changing at the mode $x_m$, which is estimated by
$|f''(x_m)|$.

Consider an exponential integral of the form
$$
	\int_a^b e^{M f(x)} dx
$$

\noindent where $f(x)$ is twice differentiable and $f''(x_m) < 0$, $M \in \R$
and $a, b \in \R \cup \{-\infty, \infty\}$. Let $f(x)$ have a unique mode at
$x_m$. Then, Taylor expanding about $x_m$, we have
$$
	f(x) = f(x_m) + f'(x_m) (x - x_m) + \frac{1}{2} f''(x_m) (x - x_m)^2 + \BigO\left((x - x_m)^3\right).
$$

\noindent As $f$ has a global maximum at $x_m$, the first derivative of $f$ is
zero at $x_m$. Thus, the function $f(x)$ may be approximated by
$$
	f(x) \approx f(x_m) - \frac{1}{2} |f''(x_m)| (x - x_m)^2
$$

\noindent for $x$ sufficiently close to $x_m$, as the second derivative is
negative at $x_m$. This ensures the the approximation of the integral
$$
	\int_a^b e^{M f(x)} dx \approx e^{M f(x_m)} \int_a^b e^{-M |f''(x_m)|(x - x_m)^2} dx
$$

\noindent is accurate. The integral on the right-hand side of the equality is a
Gaussian integral, and thus we find that
$$
	\int_a^b e^{M f(x)} dx \approx \sqrt{\frac{2 \pi}{M |f''(x_m)|}} e^{M f(x_m)}.
$$

 % See the Relative error section of the Wikipedia page, for instance.
\noindent Thus, we have approximated our integral by a closed form expression.
The error in the approximation is $\BigO(1/M)$. The approximation can be made
more accurate by using a Taylor expansion beyond second order.

\subsubsection{Extending to multiple dimensions}

This approach to approximating integrals extends naturally to multiple
dimensions. Consider the second order Taylor expansion of $\log f(\vtheta): \R^p
\to \R$ around the mode $\vtheta_m \in \R^p$ given by
$$
\begin{array}{rl}
\ds \log f(\vtheta) 
& \ds \approx f(\vtheta_m) + (\vtheta - \vtheta_m)^\top \nabla \log f(\vtheta_m) + \tfrac{1}{2} (\vtheta - \vtheta_m)^\top \mH_{\log f}(\vtheta_m) (\vtheta - \vtheta_m)
\\
& \ds  \qquad 
+ \BigO(\|\vtheta - \vtheta_m\|^3).
\end{array} 
$$

\noindent 
where $\nabla \log f(\vtheta_m)$ is the gradient of the log-likelihood at
$\vtheta_m$ and $\mH_{\log f}(\vtheta_m)$ is the Hessian matrix of the
log-likelihood at $\vtheta_m$. Assuming that $\vtheta_m$ is a stationary point
of $\log f$, then $\nabla f(\vtheta_m) = \vzero$ and so
$$
\log f(\vtheta) \approx f(\vtheta_m) + \frac{1}{2} (\vtheta - \vtheta_m)^\top \mH_{\log f}(\vtheta_m) (\vtheta - \vtheta_m) + \BigO(\|\vtheta - \vtheta_m\|^3)
$$

\noindent at such a point. The quadratic form in $\vtheta$ in the approximate
expression for the log likelihood above leads to a Gaussian approximation for
the likelihood 
$$
\N(\vtheta_m, -\mH_{\log f}(\vtheta_m)^{-1}).
$$ 

\noindent 
The approximation
is crude but can be quite accurate if the likelihood is symmetric and unimodal, which is often the case when the sample size is large.

\subsection{Other methods: Expectation propagation}

Expectation Propagation is an approximate Bayesian inference method, first
proposed in \cite{Minka2001}.  It relies on minimising the reverse KL
divergence $\text{KL}(p || q)$ between the true and approximating distributions
$p$ and $q$. A factorised form of the distribution
\[
	q(\vtheta) = \prod_{i=1}^n q(\vtheta_i)
\]

\noindent 
is assumed. In general, fully minimising the KL divergence between $p$ and $q$
is intractable, so Expectation Propagation approximates this by minimising the
KL divergence of each of the factors individually.  It does this by cycling
through each of the factors matching the sufficient statistics of each,
incorporating the information already in the other factors.
%\[
%	\bE_{-q_i} [\text{KL}(p || q)]
%\]
The factors are cycled through several times until convergence is achieved.

While promising, unlike
with Variational Bayes, there is no guarantee of convergence, and there is still
much work to be done before it is as mature as other approximation methods like
Variational Bayes and Laplace approximation.

A linear model with normal priors allows exact inference on the regression and
model selection parameters in closed form, which might appear to negate the
benefits of a variational approximation to the model. However, the performance
of our variational approximation should remain similar if the priors are
altered to cater for complications such as robustness, while exact Bayesian
inference calculations are no longer possible in closed form in these
situations.

% TODO: How to introduce this? Is this really the best place to put it?
%\cite{Zellner1986} suggested a particular form of conjugate Normal-Gamma family
%where the Bayes factors have a relatively simple form, incorporating a parameter
%$g$ to control mixing between the model fit from the %data and a prior
%specification of model fit. This immediately raises the question of how $g$
%should be chosen, and whether it should be fixed or %have a prior specification.
%\cite{Liang2008} showed that fixed choices of $g$ lead to paradoxes such as
%Bartlett's Paradox and the Information Paradox, and so a prior specification
%should be preferred. There are many ways of choosing a prior on $g$. Using a
%mixture of $g$-priors has the advantage of adapting the degree of shrinkage to
%the prior model dependent on the data.

\section{Our contributions}
% \begin{itemize}
% 	\item Gaussian Variational Approximation to Zero-Inflated Models. Parameterisation of the covariance structure.

\noindent 
In this section, we briefly outline the major contributions in this thesis.

\begin{itemize}



% \item Exact inference for some regression parameters for regression model with
% Maruyama and George prior on $g$.

\item A popular choice of Bayesian model selection is to use regression models
with $g$-priors. For the Beta Prime prior \citep{Maruyama2011} we were able to
derive closed form expressions for the posterior distributions of most of the
parameters of the model in terms of the hypergeometric function.

% \item Exact moments for $\vbeta$ for the regression model with the Maruyama
% and George prior.

% We were unable to derive a closed expression for the non-intercept regression
% parameters $\vbeta$, so instead we focused on the first and second moments. We
% were able to obtain closed form expressions for these, again in terms of the
% hypergeometric function. We were also able to obtain an approximation for the
% regression co-efficients which is multivariate normal using the Laplace
% approximation method, showing that the first and second moments of $\vbeta$
% characterise the posterior distribution well.

% \item Posterior distribution of $g$ for different g-priors - Liang's hyper-$g$
% prior, Liang's hyper-$g/n$ prior, Bayarri's robust $g$ prior and the Maruyama
% and George Beta-Prime prior.

\item An important consideration in model selection is being able to compare
models against one another.  Calculation of the Bayes Factors for comparing
models requires being able to compute the posterior distribution of $g$. In our
second chapter, we derive closed form expressions in terms of special functions
for the posterior distributions of $g$ for a number of choices of $g$ prior from
the literature: Liang's hyper-$g$ prior, Liang's hyper-$g/n$ prior
\citep{Liang2008}, Bayarri's robust $g$ prior \citep{Bayarri2012} and the
Beta-Prime \citep{Maruyama2011} prior.

% \item CVA

\item Exact inference for model selection for linear models with normal priors
    is computationally feasible when the number of  covariates is small, with
    $p$ below 40. But exhaustively exploring the search space is not efficient,
    and often not computationally feasible for a larger number of covariates.
    To deal with this situation, in our fourth chapter, we adopt a population-
    based technique inspired by Ro\v{c}kov\'{a}'s work on population-based EM
    \citep{Rockova2017} to efficiently explore the posterior model space.
    Instead, approximate methods can be used to search the parts of the model
    space for which the posterior model likelihood is the highest. In our
    third chapter, we propose a population-based algorithm, which works by
    adding or removing a covariate at a time to each of the fitted models in
    the population. We implement this algorithm for a number of model selection
    priors from the literature: the Liang's hyper-$g$ prior, the Liang's
    hyper-$g/n$ prior \citep{Liang2008}, Bayarri's robust $g$ prior
    \citep{Bayarri2012} and the Maruyama and George Beta-Prime prior
    \citep{Maruyama2011}.


\item We are able to implement this algorithm efficiently by using rank-one
updates and downdates and the closed forms of the posteriors for the model
selection priors that we consider. The population-based approach allows us to
estimate the uncertainty in the model selection process.


\item Generalised Linear Mixed Models are an appealing way to model data, as
they are flexible enough to model a range of data types and situations. But the
Bayesian versions of these models typically require computationally demanding
MCMC, which can also be prone to convergence problems. Instead, we consider
approximate Bayesian inference techniques, which are computationally efficient
and deterministic.

\item It is desirable to use normal priors for the regression coefficients of
these models, as these are easily interpreted. But for Generalised Linear Mixed
Models with a non-normal response, these priors are non-conjugate, making VB
difficult to apply as the required mean field updates are intractable. We apply
Gaussian Variational Bayes -- an extension to Variational Bayes, to fit a
multivariate normal distribution to the regression coefficients of our models.

\item In our fourth chapter, we present a Gaussian Variational Approximation to
a zero-inflated Poisson mixed model which can flexibly incorporate both fixed
and random effects. This allows us to use our model fitting algorithm to fit
complicated models to the data incorporating random intercepts and slopes and
additive models using O'Sullivan-penalised splines.  The model is fit by
optimising the conditional likelihood of the Gaussian component of the model
given the parameters governing zero-inflation and the covariance matrix
$\mSigma$.

\item We present a new parameterisation for the covariance matrix of the
Gaussian based on the Cholesky factorisation of the precision matrix, and detail
computation and numerical advantages of this factorisation, owing to its
sparsity when the form of the covariance matrix of the Gaussian is known due to
knowledge of the random effects in the model.

\end{itemize}

\chapter{Zero-inflated models}

% \begin{abstract}
\noindent We consider variational inference for zero--inflated Poisson regression models using a latent
variable representation. The model is extended to include random effects which allow simple incorporation of
spline and other modelling structures. Several variational approximations to the resulting set of models are
presented, including a novel approach based on the inverse covariance matrix rather than the covariance matrix
of the approximate posterior density for the random effects. This parameterisation improves upon the
computational cost and numerical stability of previous methods. We demonstrate these approximations on
simulated and real data sets.
% \end{abstract}
 
% \noindent Keywords: Approximate Bayesian inference ; mixed model ; Markov chain Monte Carlo ; Stan ; penalized splines.

% \joc{
% 	Comments: 
% 	\begin{itemize}
% 		\item Need to organize in terms of a flow of ideas. What are we approximating?		      		      		      		      
% 		\item I believe that we are using a semiparametric mean field variational Bayes approach discussed by Rohde \& Wand (2015).
% 		      However, I am not sure that we are using the their formalisms. (see page 3-6 of Rohde and Wand 2015).
% 	\end{itemize}	
% }

\section{Introduction}
\label{sec:introduction}

% \mgc{This section is too short}

Count data with a large number of zero counts arises in many areas of application, such as data arising from
physical activity studies, insurance claims, hospital visits or defects in manufacturing processes. Zero
inflation is a frequent cause of overdispersion in Poisson data, and not accounting for the extra zeroes may
lead to biased parameter estimates. These models have been used for many applications, including defects in
manufacturing in \cite{lambert1992}, horticulture in \cite{BIOM:BIOM1030} and \cite{BIOM:BIOM1030}, length
of stay data from hospital admissions in \cite{BIMJ:BIMJ200390024}, psychology in \cite{JOFP:rethink},
pharmaceutical studies in \cite{Min01042005}, traffic accidents on roadways in \cite{Shankar1997829} and
longitudinal studies in \cite{LeeWangScottYauMcLachlan2006}.

The strength of this approach derives from modelling the zero and non-zero count data seperately as a mixture
of distributions for the zero and non-zero components, allowing analysis of both the proportion of zeroes in
the data set and the conditions for the transition from zero observations to non-zero observations. When
combined with a multivariate mixed model regression framework, an extremely rich class of models can be fit
allowing a broad range of applications to be addressed. Often the transition from zero to non-zero has a
direct interpretation in the area of application, and is interesting in its' own right.

Bayesian estimation methods for zero-inflated models was developed in \cite{Ghosh2006} using MCMC implemented
with WinBUGS, and in \cite{Vatsa2014} using a Variational Bayes solution to the inverse zero- inflated
Poisson regression problem. While simple forms of these models are easy to fit with standard  maximum
likelihood techniques, more general models incorporating random effects, splines and missing data  typically
have no closed form solutions and hence present a greater computational challenge to fit.

In this chapter, we build upon the earlier work on Bayesian zero-inflated models by \cite{Ghosh2006} and
\cite{Vatsa2014}. While simple forms of these models are easy to fit with standard maximum likelihood
techniques, more general models incorporating random effects, splines and missing data typically have no
closed form solutions and hence present a greater computational challenge to fit.

Fitting these models is typically done with Monte Carlo Markov Chain techniques, but these techniques can be
computationally intensive and prone to convergence problems.  Other fitting methods such as the Variational
Bayes approach above can be inflexible, not allowing complicated models incorporating random effects, splines
and missing data.

Other approximate Bayesian inference techniques exist in the literature, such as
Laplace approximation \cite{Tierney1986}/ integrated nested Laplace approximation \cite{Rue2009} and Expectation Propagation \cite{Minka2013}, and these have been applied to the problem of fitting
count models
\cite{Barber2016}
\cite{KimWand2017}.
But Expectation Propagation requires very difficult algebra to complete the derivations required for the
updates, and is slow to compute. And Laplace approximation relies on a Gaussian approximation to the
log-likelihood found by Taylor expanding around the mode, which performs poorly when the true posterior is
not symmetric, as is the case for Poisson regression models.

We build upon a latent variable representation of these models to allow a tractable semiparametric mean field
Variational Bayes approximation to be derived. Semiparametric Mean Field Variational Bayes is an approximate
Bayesian inference method as detailed in \cite{Ormerod2010} and \cite{Rohde2015}, which allows us to fit
close approximations to these models using a deterministic  algorithm which converges much more quickly.

We allow a flexible regression modelling approach incorporating both fixed and random effects by using a
Gaussian Variational Approximation as defined in \cite{Ormerod2012} on the regression parameters to allow a
non- conjugate Gaussian prior to be used, making the resulting Gaussian posterior distribution of the
regression parameters easy to interpret. We adopt a Mean Field  Variational Bayes (VB) approach on the other
parameters in the model to derive the rest of the approximation.

% This makes sense in a paper, but not in a thesis chapter.
The focus of this chapter is on developing methods of fitting flexible ZIP regression models accurately, and
showing the advantages of our methods to previously presented methods. We also investigate stability problems
that can arise when using naive versions of these methods, and the modifications to the fitting methods we
devised to mitigate these problems. In Section \ref{sec:model} we define our model and provide a framework for
our approach incorporating regression modelling and random effects. In Section \ref{sec:gaussian} we focus on
several approaches to fitting the Gaussian component of our model. In Section \ref{sec:param}, we present new
parameterisations for use in these algorithms which offers substantial advantages in accuracy, numerical
stability and computational speed. In Section \ref{sec:results} we perform numerical experiments on simulated
data which show how our approach offers computational advantages over existing approaches -- in terms of both
speed and stability. In Section \ref{sec:application} we show an application of our pure Poisson model fitting
method to a hierarchical model studying the effect of ethnicity on the rate of police stops, and an
application of our zero-inflated Poisson model fitting method to a multi-level longitudinal study of pest
control in apartments. Finally, in Section \ref{sec:discussion} we conclude with a discussion of the results.
An appendix contains details of the derivation of the variational lower bound for our model.

\section{Zero--inflated models}
\label{sec:model}

In this section we present a Bayesian zero-inflated Poisson model for count data with extra zeroes. After
introducing the latent variable representation of Bayesian zero-inflated models, we first extend this to a
model incorporating fixed effects regression modelling, and extend the model again to a more flexible mixed
model approach incorporating both fixed and random effects.

\subsection{Modelling zero-inflated Poisson data}

We consider a sample of counts $y_i$, $1 \le i\le n$, where there are an excessive number of zeros for a
Poisson model, but the sample is otherwise well--modelled by a Poisson distribution. There are two main
parameterizations for modelling such data. The first approach models the probability of a zero by $\rho$ and
adjusts for counts greater than zero. This model uses the probability distribution
$$
\begin{array}{rll}
	P(Y_j = y_i) = \begin{cases}
	\rho + e^{-\lambda}, &y_i = 0\\
	\left( \frac{1 - \rho}{1 - e^{-\lambda}} \right) \frac{\lambda^{y_i} e^{-\lambda}} {y_i!}, &y_i \ge 1.
	\end{cases}
\end{array}
$$

A second approach using latent variables views the data as the product of two data--generating processes, a
Bernoulli process that determines whether the data is definitely zero, and a second process where data is
generated from a Poisson distribution which may be zero.

Note that this allows zeros to be generated from the model in one of two ways -- either from the Bernoulli
process generating a zero or from the Bernoulli process generating a Poisson sample which is then zero. A
latent variable representation of this parameterization introduces the latent variables $r_i$ which equal $1$
when $y_i>0$ and $0$ otherwise. This leads to the specification
$$
\begin{array}{rl}
	P(Y_i=y_i|r_i) = & \frac{\exp(-\lambda r_i)(\lambda r_i)^{y_i}}{y_i!} \quad \mbox{and} \\
	r_i \sim & \mbox{Bernoulli}(1-\rho)
\end{array}
$$
where $\text{Bernoulli}(\pi)$ denotes the probability distribution $\pi^k (1 - \pi)^{1-k}$.

Let $p$ be the dimension of the space of fixed effects, $m$ be the number of individuals in the random effects
and $b$ be the block size for each of those individuals. We use $\vone_p$ and $\vzero_p$ to denote the $p
\times 1$ column vectors with all entries equal to 1 or 0, respectively.

Let $\vy$ be the $n \times 1$ vector. The norm of a column vector $\vv$, defined to be $\sqrt{\vv^\top \vv}$,
is  denoted by $\|\vv\|$. For a $p \times 1$ vector $\va$, we let $\diag{(\va)}$ denote the $p \times p$
matrix with the elements of $\va$ along its' diagonal.

We denote the design matrix of fixed effects with dimensions $n \times p$ as $\mX$ is , and the design matrix
of random  effects with dimensions $n \times m b$ as $\mZ$. The combined design matrix $\mC$ is formed by
appending the columns of $\mX$ to the columns of $\mZ$, giving $\mC = [ \mX, \mZ ]$.

Let $\vtheta$ is the vector of all parameters.
Let $\vbeta$ be the $p \times 1$ column vector of fixed
effects, and $\vu$ the $m b \times 1$ column vector of random effects. $\vnu$ is the
concatenation of these vectors $[\vbeta^\top, \vu^\top]$.
% Let $\vp$ be the $n \times 1$ column vector of probabilities that each observation in $\vy$ is
% non-zero.

Let $\mSigma$ be the covariance matrix of the random effects $\vu$,
and 
$\mPsi$ the covariance matrix prior on $\mSigma$.
These matrices are all of dimension $(p + m b) \times (p + m b)$.

$\text{expit}(x)$ denotes the function $\tfrac{1}{1 + \exp(-x)}$ which is the inverse of the logit
function.

We can extend the model naturally to a multiple covariate regression model by using a log link function on the
response variable and replacing the parameter $\lambda$ in the model above with $\vx_i^\top \vbeta$ to specify
the mean, where $\vx_i,\vbeta \in \R^p$, with $\vx_i$ the vector of observed predictors and $\vbeta$ the
vector of regression coefficients. Letting $\vr = (r_1,\ldots,r_n)$, the model becomes
$$%\label{eq:main}
	\begin{array}{rl}
		\log p(\vy|\vr, \vbeta) 
		    & = \vy^\top \mR (\mX\vbeta)                           
		- \vr^\top \exp{(\mX\vbeta)} 
		- \vone^\top \log{\Gamma{(\vy + \vone)}}, \quad \mbox{ and }\\ [1ex]
		r_i | \rho & \sim \text{Bernoulli}(1-\rho), \quad 1 \leq i \leq n \\
	\end{array}
$$

\noindent where $\mX$ is the $n\times p$ matrix whose $i$th row equals $\vx_i$ and $\mR = \diag{(\vr)}$.

\subsection{Extending to mixed models, incorporating random effects}

To be able to construct multivariate models with as much generality as possible, we wish to specify the full
model as a General Design Bayesian Generalized Linear Mixed Model, as in \cite{Zhao2006}. This allows for a
very rich class of models, which can incorporate such features as random intercepts and slopes, within-subject
correlation and smoothing splines, as in \cite{Wand2008}, into our models.

The zero-inflated model regression model introduced above can be extended to a flexible mixed model by
incorporating the latent variable $\vr$ which controls the mixture of the zero and non-zero components from
the zero-inflated model above into a Poisson mixed model likelihood.

When the indicator $\vr_{ij} = 0$, the likelihood is $1$ for $\vy_{ij} = 0$ and $0$ for all $\vy_{ij} > 0$,
and when the indicator $\vr_{ij} = 1$, the likelihood is a Poisson mixed model regression likelihood for
$\vy_{ij}$. $\vr_{ij}$ is a Bernoulli indicator with probability $\rho$, allowing a proportion of zero-
inflation in the observed data to be specified.

The $j$th predictor/response pair for the $i$th group is denoted by $(\vx_{ij}, \vy_{ij}), 1 \leq j \leq n_i, 1 \leq i \leq m$, where $\vx_{ij} \in \R$, and the $\vy_{ij}$ are nonnegative integers.

For each $1 \leq i \leq m$, define the $n_i \times 1$ vectors $\vy_{ij} = [\vy_{i 1}, \ldots, \vy_{i
n_i}]^\top$ as the response vector. Vectors $\vy_1, \ldots, \vy_m$ are assumed to be independent of each other.

We develop a zero-inflated regression model incorporating both fixed effects $\vbeta$ and random effects
$\vu$. The log-likelihood for one observation is then
$$
	\begin{array}{rl}
		\log p(\vy_{ij} | \vr_{ij}, \vbeta, \vu) & = \vy_{ij} \vr_{ij} (\vx_i^\top \vbeta + \vz_{ij}^\top \vu) - \vr_{ij} \exp (\vx_{ij}^\top \vbeta + \vz_{ij}^\top \vu) - \log \Gamma (\vy_{ij} + 1), \\
		\vr_{ij} | \rho                  & \sim \text{Bernoulli}(\rho), 1 \leq i \leq m, 1 \leq j \leq n, \text{ and }                                                              \\
		\rho                        & \sim \text{Beta}(\alpha, \beta).                                                                                              \\
	\end{array}
$$

\noindent We now extend this to multiple observations. Let $\mC = [\mX, \mZ]$ and $\vnu = [\vbeta^\top, \vu^\top]^\top$. The multivariate model with multiple observations is then
\begin{equation}\label{eq:main}
	\begin{array}{rl}
		\log{p(\vy|\vr, \vbeta, \vu)} & = \vy^\top \mR (\mC\vnu) - \vr^\top \exp{(\mC\vnu)} - \vone^\top \log{\Gamma{(\vy + \vone)}}, \quad \mbox{ and } \\ [1ex]
		r_i                           & \sim \text{Bernoulli}(\rho), 1 \leq i \leq n                                                                     \\
	\end{array}
\end{equation}

% \joc{(The prior structure will depend on the structure of the random effects model)}
\noindent with priors
\begin{align*}
	\log{p(\mSigma_{\vu \vu})} & = \text{Inverse Wishart}(\mPsi, v),    \\
	\rho                       & \sim \Beta(\alpha, \beta),             \\
	\vbeta|\sigma^2_\vbeta     & \sim \N_p(\vzero, \sigma^2_\vbeta \mI) \text{ and } \\
	\vu|\mG       & \sim \N_{mb}(\vzero, \mG)              
\end{align*}

\noindent where $\mX$ is $n \times p$, $\mZ$ is $n \times mb$ and $\mSigma_{\vu \vu}$ is $mb \times mb$ and
$\mPsi$ is $b \times b$. The covariance of $\Cov(\vu) \equiv \blockdiag_{1 \leq i \leq m} (\mSigma) \equiv
\mI_m \otimes \mSigma$. $\text{Inverse Wishart}(\mPsi, v)$ denotes the probability distribution
$$\tfrac{|\mPsi|^\frac{v}{2}}{2^{\frac{vp}{2}} \Gamma_p{\left(\tfrac{v}{2}\right)}} |\mX|^{-\tfrac{v + p + 1}{2}}
\exp{\left\{-\tfrac{1}{2} \tr{(\mPsi \mX^{-1})}\right\}}$$ where $\Gamma_p{(x)}$ denotes the multivariate gamma function and $\tr$
is the trace function.

The covariance matrix of random effects $\mSigma$ will depend on the mixed model being fit. In the random
intercept case, $\mSigma = \sigma_u^2 \mI$ while in the random slopes case
\[
	\mSigma = 
	\begin{pmatrix}
		\sigma_{\vu_1}^2                                 & \rho_{\vu_1 \vu_2} \sigma_{\vu_1} \sigma_{\vu_2} \\
		\rho_{\vu_1 \vu_2} \sigma_{\vu_1} \sigma_{\vu_2} & \sigma_{\vu_2}^2                                 
	\end{pmatrix}
\]
where $\sigma_{\vu_1}^2$ is the variance of the random intercepts, $\sigma_{\vu_2}^2$ is the variance of the
random slopes and $\rho_{\vu_1 \vu_2}$ is the correlation between the random intercepts and random slopes.

% \joc{Shouldn't we specify the structure of $\mSigma_{\vu \vu})$ later
% which is different for the random intercept, slope and spline cases?}
% \joc{(Perhaps it is wroth spelling out all of the various random effects structures that we will be using. Consider templating from Zhao \etal (2006).))}

In the spline case, we use the cubic spline basis $1$, $x$, $x^3$, $(x - \kappa_1)^3_+$, \ldots, $(x -
\kappa_K)^3_+$, where $K$ is the number of knots. $\mSigma$ is a $K + 2$ banded matrix, where $K$ is the
number of knots. Banded matrices are highly sparse, and matrix operations can be performed on them in
$\BigO(n)$ time. The matrix $\mSigma$ is symmetric, with contents
\[
	\mSigma =
	\begin{pmatrix}
		\sigma^2_{\text{intercept}} & \ldots                      &                             &                               &                                          & \text{symmetric}              \\
		\rho_{\text{intercept} x} & \sigma^2_x & \ldots\\
		\rho_{\text{intercept} x^2} & \rho_{x x^2} & \sigma^2_{x^2} & \ldots \\
		\rho_{\text{intercept} x^3} & \rho_{x x^3} & \rho_{x^2 x^3} & \sigma^2_{x^3} & \ldots \\
		0                           & \rho_{x (x - \kappa_1)^3_+} & \rho_{x^2 (x-\kappa_1)_+^3} & \rho_{x^3 (x - \kappa_1)_+^3} & \sigma^2_{(x - \kappa_1)_+^3}            & \ldots                        \\
		0                           & 0                           & \rho_{x^2 (x-\kappa_2)_+^3} & \rho_{x^3 (x-\kappa_2)_+^3}   & \rho_{(x-\kappa_1)_+^3 (x-\kappa_2)_+^3} & \sigma^2_{(x - \kappa_2)_+^3} \\
		0 & 0 & 0 & \rho_{x^3 (x - \kappa_3)_+^3} & \rho_{(x - \kappa_1)_+^3 (x - \kappa_3)_+^3} & \rho_{(x - \kappa_2)_+^3 (x - \kappa_3)_+^3}
	\end{pmatrix}.
\]

\subsection{Variational Bayes Approximation to the Zero-Inflated Poisson model}

We choose a factored variational approximation for the model of the form 
\[
	q(\vnu, \sigma_{\vu}^2,\vr_0, \rho) = q(\vnu) q(\mSigma_{\vu \vu}) q(\vr_0) q(\rho)
\]

\noindent 
where we define $\vr_0 = \{ r_i : y_i = 0 \}$. \\
% The distributions we select are?
$q(\vnu) = \N(\vmu, \mLambda)$, \\
$q(\sigma_{\vu}^2) = \text{Inverse Wishart}\left(\mPsi + \sum_{i=1}^m (\vmu_i \vmu_i^\top + \mLambda_{\vu_i \vu_i}), v + m \right)$ \mbox{and } \\
$q(r_i) = \text{Bernoulli}{(p_i)}$ with
$$p_i = 
\begin{cases}
\text{expit}\left\{ \psi{(\alpha_{q(\rho)})} - \psi{(\beta_{q(\rho)})} - \exp{(c_i^\top\vmu + \frac{1}{2} c_i^\top \mLambda c_i)} \right\},& \text{ when } \vy_i = 0\\
1,& \text{ otherwise.}
\end{cases}$$

%$\propto \exp{\left \{-r_i \bE_{-r_i} [\exp{(c_i^\top\vnu)}] + r_i [\psi(\alpha_\rho) - \psi(\beta_\rho)] \right \} }.\\$

The optimal approximation 
% \joc{(reword: the ``optimal approximation'' might be called the true posterior)}
for $\vr$ is
\[
\begin{array}{rl}
	q(\vr) & \propto \exp \left \{ \bE_{-q(\vr)}\vy^\top\mR(\mC\vmu) - \vr^\top\exp{(\mC\vnu)}-\frac{1}{2} \vnu^\top \mSigma_{\vu \vu} \vnu \right \}                                                  \\ [1ex]
	       & = \exp{ \left[ \vy^\top\mR\mC \vmu - \vr^\top \exp{\{\mC \vmu + \frac{1}{2} \text{diag}(\mC \mLambda \mC^\top)\}} - \frac{1}{2} \vmu^\top \mD \vmu - \frac{1}{2} \text{tr}(\mLambda \mD ) \right] } 
\end{array}
\]
\noindent where $\mD = \left\{ (\mPsi + \sum_{i=1}^m \vmu_i \vmu_i^\top + \mLambda_{\vu_i\vu_i}) / (v - p - 1) \right\}^{-1}$. 

We observe that this expression is close in form to the likelihood of a Poisson regression model with random
effects. Poisson regression models are non-conjugate with normal priors, and hence the mean field updates for
the regression parameters do not have closed form expressions. But by assuming a multivariate normal
distribution for the regression co-efficients parameterised by $\vmu$ and $\mLambda$, the model can still be
fit using a Gaussian Variational Approximation for $\vbeta$ and $\vu$ jointly. Techniques for efficiently
fitting these models are described in \cite{Ormerod2012}, \cite{Challis2013} and \cite{Opper2009}. Gaussian
variational approximations have also been shown to have good asymptotic properties in \cite{Sinica2017}. The
model can be fit using Algorithm \ref{alg:algorithm_one} below.

\begin{algorithm}
	\caption[Algorithm 1]{Iterative scheme for obtaining the parameters in the
		optimal densities $q^*(\vmu, \mLambda)$, $q^*(\mSigma_{\vu \vu})$ and $q^*(\rho)$}
	\label{alg:algorithm_one}
	\begin{algorithmic}
		\REQUIRE{$\alpha_{q(\rho)} \leftarrow \alpha_\rho + \vone^\top\vp, p_{q(\mSigma_{\vu \vu})} \leftarrow p + 1$} \\[1ex]
		\WHILE{the increase in $\log{\underline{p}}(\vy;q)$ is significant}
		% \vmu, \mLambda
			\STATE Optimise $\vmu$ and $\mLambda$ using $\vy, \mC, \vp$ and $\mSigma_{\vu \vu}$ \\[1ex]
			% \vp
			\STATE $\beta_{q(\rho)} \leftarrow \beta_\rho + n - \vone^\top\vp$ \\[1ex]
			\STATE $\veta \leftarrow -\exp \left \{ \mC \vmu + \frac{1}{2} \diag{(\mC\mLambda\mC^\top)} \right \} + (\psi{(\alpha_{q{(\rho)}})} - \psi{(\beta_{q{(\rho)}})}) \vone_n$ \\[1ex]
			\STATE $\vp_{q(\vr_0)} \leftarrow \text{expit}{(\eta)}$ \\[1ex]
			% \mSigma_{\vu \vu}
			\STATE $\mPsi_{q(\mSigma_{\vu \vu})} \leftarrow \mPsi + \sum_{i=1}^m (\vmu_i \vmu_i^\top + \mLambda_{{\vu}_i {\vu}_i})$ \\[1ex]
			\STATE $\mSigma_{\vu\vu} \leftarrow \{\mPsi_{q(\mSigma_{\vu \vu})}/(v - d - 1)\}^{-1}$
		\ENDWHILE
	\end{algorithmic}
\end{algorithm}
						
\section{Optimising the Gaussian Part of the Model}
\label{sec:gaussian}

The most computationally and numerically difficult part of Algorithm \ref{alg:algorithm_one} above is
optimising the mean and covariance of the Gaussian approximation to the regression co-efficients $[\vbeta,
\vu]^\top$. In this section, we compare the accuracy, stability and speed of four different algorithms for
fitting the Gaussian component of our model, $q(\vmu, \mLambda)$ in Algorithm \ref{alg:algorithm_one}.  We
compare these approaches for accuracy, computational complexity and stability.

Our first attempts at implementation of some of these algorithms were prone to numerical stability problems
when initialised from some starting points. We also discuss the modifications we made to these algorithms to
enhance their numerical stability.
	
\subsection{Laplace-Variational Approximation}
The Laplace-Variational Approximation method is based on Laplace's method of approximating integrals, as
introduced in Section \ref{sec:laplace_approximation}. The variational lower bound  is approximated by a
Gaussian centred at its' mode.

% Laplace's method of
% approximation uses the second order Taylor expansion of the full log likelihood of the  model around the
% mode to find a Gaussian approximation to the full posterior. Taylor expanding the full log likelihood
% once around the mode yields the following approximation.
This yields the approximation to the variational lower bound
% The algorithm is very quick to execute, but the resulting approximate posterior
% distributions are not as accurate as those produced by the other algorithms considered in this article.
% NR
% Detail the function and its derivatives
\begin{align*}
	\log \underline{p}(\vmu, \mLambda; \vy) \approx \vy^\top\mP\mC\vmu - \vp^\top\exp \left (\mC \vmu \right ) - \tfrac{1}{2} \vmu^\top \mSigma^{-1} \vmu. 
\end{align*}
		
\noindent This expression can be iteratively optimised with respect to $\vmu$ and $\mLambda$ using the
Newton-Raphson method, with the derivatives for $\vmu$ and $\mLambda$ given in Appendix
\ref{sec:appendix_derivatives_laplace}. The steps of the algorithm are shown in Algorithm \ref{alg:laplace_alg}.
		
Upon implementing the algorithm and performing numerical experiments, we observed numerical issues which had to be dealt with in 
order for the algorithm to successfully complete.
% Describe the iteration
We implemented checks for error conditions, and steps to recover from the error conditions should
they arise.

If during an iteration of the Laplace-Variational approximation algorithm the inversion  of $\mLambda$
fails, or the diagonal elements of $\mLambda$ become negative when $\mLambda$ must be positive-definite,
then $\vmu$ and $\mLambda$ were reverted to the previous iteration's $\vmu$ and $\mLambda$ values and
the algorithm was terminated.

If after calculating the gradient of the Gaussian Variational lower bound with respect to $\vmu$, any of
its' elements were NaN or $\infty$, then $\vmu$ and $\mLambda$ were reverted to the previous iteration's
$\vmu$ and $\mLambda$ values and the algorithm was terminated.
		
\begin{algorithm}
	\caption{Laplace scheme for optimising $\log \underline{p}(\vmu, \mLambda; \vy)$}
	\label{alg:laplace_alg}
	\begin{algorithmic}
		\REQUIRE $\mH \leftarrow [- \mC^\top \text{diag}(\vp e^{(\mC \vmu)}) \mC - \mSigma^{-1}]^{-1}$.
		% Fit \vmu, \mLambda using Laplace approximation
		\WHILE{the increase in $\log \underline{p}(\vmu, \mLambda; \vy)$ is significant}
		% \vmu, \mLambda
		\STATE $\mLambda \leftarrow \left \{\mP \mC^\top \text{diag}(\exp{(\mC \vmu)}) \mC + \mSigma^{-1} \right \}^{-1}$ \\ [1ex] 
		If $\mLambda$ cannot be inverted, or any diagonal element of $\mLambda$ is negative, revert to previous
		$\mLambda$ and break \\ [1ex]
		\STATE $\mH
		\leftarrow [- \mC^\top \text{diag}(\vp e^{(\mC \vmu)}) \mC - \mSigma^{-1}]^{-1}$ \\ [1ex]
		If any element of $\mH$ is NaN or $\infty$,
		break
		\STATE $\vmu \leftarrow \vmu + \mLambda g$ \\ [1ex]
		\ENDWHILE
	\end{algorithmic}
\end{algorithm}
		
\subsection{Gaussian Variational Approximation}
		
% Detail techniques used for fitting models.
The full variational likelihood for a Generalised Linear Mixed model is computationally difficult to compute,
requiring the evaluation of a high dimensional integral. However, \cite{Ormerod2012} devised an accurate
approximation to the full variational likelihood, the Gaussian Variational Lower Bound, which only requires
the evaluation of a substantially simpler univariate integral.
	
To optimise the Gaussian component of the lower bound in each iteration of Algorithm \ref{alg:algorithm_one},
optimal $\vmu$ and $\mLambda$ values must be found while keeping the other variational parameters fixed. The
variational lower bound is not necessarily unimodal if $\vp$ and $\mSigma$ are free to vary, leading to
potential difficulty in optimising to the global maximum. However, for fixed $\vp$ and $\mSigma$, the
variational lower bound is log-concave with respect to $\vmu$ and $\mLambda$, and so standard optimisation
methods such as L-BFGS-B as described in, for example, \cite{Liu1989} \cite{Nocedal2006}, work well. This
leads to an extremely accurate approximation of the true posterior at the expense of some additional
computational effort. Care must be taken in the parameterisation of $\mLambda$, as it is both of high
dimension $(p + mb)^2$ and constrained to be semi- positive definite. We present and compare two approaches to
parameterising the covariance matrix $\mLambda$ below.
	
\subsubsection{Covariance parameterisation $\mLambda = \mR^\top \mR$}

We fit the Gaussian component of our approximation in Algorithm \ref{alg:algorithm_one} by maximising  the
variational lower bound
\begin{align*}
	\log \underline{p}(\vmu, \mLambda; \vy) & = \quad \vy^\top\mP \mC \vmu - \vp^\top \exp\{\mC \vmu + \tfrac{1}{2} \text{diag}(\mC \mLambda \mC^\top)\} - \tfrac{1}{2} \vmu^\top \mSigma^{-1} \vmu - \tfrac{1}{2} \tr{(\mSigma^{-1} \mLambda)} + \log{|\mR|} \\
	                                        & \quad + \tfrac{1}{2} \log{|\mSigma^{-1}|}  + \tfrac{p}{2}                                                                              
\end{align*}
\noindent with respect to $\vmu$ and $\mLambda$, keeping $\vp$, $\mSigma$ and $\rho$ fixed.
		
The first variant of the Gaussian Variational Approximation algorithm that we present optimises the
Gaussian variational lower bound of the log likelihood with respect to $\vmu$ and the Cholesky decomposition
$\mR$ of $\mLambda$, that is, $\mLambda = \mR \mR^\top$. This ensures that $\mLambda$ remains positive
definite, and reduces the number of parameters we have to optimise over in order to optimise $\mLambda$
to the $(p + 1) p / 2$, as $\mR$ is lower triangular.	We refer to this as the covariance
parameterisation. The resulting function
% This algorithm trades the computational complexity of
% numerically evaluating an integral for greatly increased accuracy in the approximating posterior
% distribution. 
\begin{align*}
	\log \underline{p}(\vmu, \mLambda; \vy) & = \quad \vy^\top\mP \mC \vmu - \vp^\top \exp\{\mC \vmu + \tfrac{1}{2} \text{diag}(\mC \mLambda \mC^\top)\} - \tfrac{1}{2} \vmu^\top \mSigma^{-1} \vmu - \tfrac{1}{2} \tr{(\mSigma^{-1} \mLambda)} + \log{|\mR|} \\
	                                        & \quad + \tfrac{1}{2} \log{|\mSigma^{-1}|} + \tfrac{p}{2},                                                                              
\end{align*}
can be optimised with L-BFGS-B using the derivatives in Appendix \ref{sec:appendix_derivatives_gva}.

\begin{figure}[p]
	\includegraphics[width=0.95 \textwidth]{mX_mZ_mLambda.pdf}
	\caption{Inverse Covariance matrix of approximate posterior for $\vnu$ -- Fixed effects before random effects}
	\label{fig:covfixedrandom}
\end{figure}
	
\begin{figure}[p]
	\includegraphics[width=0.95 \textwidth]{mZ_mX_mLambda.pdf}
	\caption{Inverse Covariance matrix of approximate posterior for $\vnu$ -- Random effects before fixed effects}
	\label{fig:covrandomfixed}
\end{figure}
						      				      			      			      			      	
\begin{figure}[p]
	\includegraphics[width=0.95 \textwidth]{mX_mZ_cholesky.pdf}
	\caption{Cholesky factor of Inverse Covariance matrix of approximate posterior for $\vnu$ -- Fixed effects before random effects}
	\label{fig:cholfixedrandom}
\end{figure}
	
\begin{figure}[p]
	\includegraphics[width=0.95 \textwidth]{mZ_mX_cholesky.pdf}
	\caption{Cholesky factor of Inverse Covariance matrix of approximate posterior for $\vnu$ -- Random effects before fixed effects}
	\label{fig:cholrandomfixed}
\end{figure}
	
\subsubsection{Precision parameterisation $\mLambda = (\mR^\top \mR)^{-1}$}
		
\noindent The second variant of the Gaussian Variational Approximation algorithm is similiar to the first, but
instead of optimising the Gaussian variational lower bound with respect to $\vmu$ and the Cholesky factor
$\mR$ of $\mLambda$, we instead optimise the Cholesky factor of the inverse of $\mLambda$ i.e. $\mLambda =
(\mR \mR^\top)^{-1}$.

The Gaussian variational lower bound in this parameterisation is
\begin{align*}
	\log \underline{p}(\vmu, \mLambda; \vy) & = \quad \vy\mP\mC \vmu - \vp^\top \exp\{\mC \vmu + \tfrac{1}{2} \text{diag}(\mC \mLambda \mC^\top)\} - \tfrac{1}{2} \vmu^\top \mSigma^{-1} \vmu - \tfrac{1}{2} \tr{(\mSigma^{-1} \mLambda)} \\
	                                        & \quad+ \tfrac{1}{2} \log{|\mSigma^{-1}|} + \tfrac{p}{2} - \log{|\mR|}.                                             
\end{align*}
		
\noindent The derivative with respect to $\vmu$ is the same as that in the first variant of the algorithm, but 
as the parameterisation of $\mLambda$ has changed, the  derivative with respect to $\mLambda$ becomes
\begin{align*}
	\frac{\partial \log \underline{p}(\vmu, \mLambda; \vy)}{\partial \mLambda}
	  & = \hphantom{-}(\mLambda^{-1} + \mH)(-\mLambda \mR \mLambda) \\
	  & = -(\mI + \mH\mLambda)\mR\mLambda                           \\
	  & = - (\mR\mLambda + \mH\mLambda\mR\mLambda)                  
\end{align*} 
		
\noindent where $\mH = (\mP \mC)^\top \text{diag}(\exp(\mC \vmu + \frac{1}{2} \mC \mLambda \mC^\top)) \mP \mC - \mSigma^{-1}$.
		
\subsubsection{GVA fixed point} 	% Fixed point update of \mLambda
This variant of the algorithm uses Newton-Raphson-like fixed point updates on the Gaussian variational lower
bound. We optimise the same variational lower bound as in the covariance parameterisation above, using the
derivatives below. The steps are presented in Algorithm \ref{alg:algorithm_nr} where   the derivatives are as
presented in Appendix \ref{sec:appendix_derivatives_gva_fixed_point}. As this algorithm involves a simple
Newton-Raphson style update step, it is computationally simple to implement, but potentially unstable as there
is no adaptation of step size, as in L-BFGS-B.

For efficiency, the inversion of $\mLambda$ within the algorithm was implemented using the block inverse 
formula, where	the matrix was partitioned
\[
	\mLambda =
	\begin{pmatrix}
		\mLambda_{\vbeta \vbeta} & \mLambda_{\vbeta \vu} \\
		\mLambda_{\vbeta \vu}^\top & \mLambda_{\vu \vu}
	\end{pmatrix}
\]
where $\mLambda_{\vbeta \vbeta}$ is the $p \times p$ approximation of the fixed effects covariance, $\mLambda_{\vbeta \vu}$ is the $p \times mb$
approximation of the covariances between the fixed and random effects and $\mLambda_{\vu \vu}$ is the $mb \times mb$
approximation of the random effects covariance.

Sometimes in the course of  executing the algorithm, we observed numerical issues which had to be dealt
with in order for the algorithm to successfully complete. If the block $\mLambda_{\vu \vu}$ cannot be inverted on an
iteration, we reverted to $\vmu$ and $\mLambda$ from the previous iteration. If after updating $\vmu$
any element is NaN, we reverted to the $\vmu$ and $\mLambda$ from the previous iteration. This greatly
improved the stability of the algorithm.

\begin{algorithm}
	\begin{algorithmic}
		\REQUIRE $g = \mP \mC (\vy - \mC^\top \exp(\mC \vmu + \frac{1}{2} \text{diag}{(\mC \mLambda \mC^\top)})) - \mSigma^{-1} \vmu$.
		% Fit \vmu, \mLambda using Laplace approximation
		\WHILE{the increase in $\log{\underline{p}}(\vmu, \mLambda; \vy)$ is significant}
			\STATE $\vg \leftarrow \mC^\top \vp [\vy - \{ \exp(\mC \vmu + \tfrac{1}{2} \text{diag}(\mC \mLambda \mC^\top)) \}] - \mSigma^{-1} \vmu$ \\
			\STATE $\mH \leftarrow -\mC^\top \text{diag}(\vp^\top \exp(\mC \vmu + \tfrac{1}{2} \text{diag}(\mC \mLambda \mC^\top))) - \mSigma^{-1}$ \\
			\STATE $\mLambda \leftarrow (-\mH)^{-1} \text{ using block inversion on } \mH$ \\
			If the inversion fails, revert to previous $\vmu$ and $\mLambda$ and exit the loop \\
			\STATE $\vmu \leftarrow \vmu + \mLambda \vg$ \\
			If any element of $\vmu$ is $\infty$ or NaN, revert to previous $\vmu$ and $\mLambda$ and exit the loop
		\ENDWHILE
	\end{algorithmic}
	\caption{The GVA Newton-Raphson fixed point iterative scheme for obtaining the optimal $\vmu$ and $\mLambda$
		given $\vy$, $\mC$ and $\vp$.}
	\label{alg:algorithm_nr}
\end{algorithm}
		
% Splines
		
\section{Parameterisations and computational cost of Gaussian Variational Approximation approaches}
\label{sec:param}
\subsection{Covariance parameterisation of $\mLambda$}

To ensure symmetry of $\mLambda$, we parameterise the optimisation problem in terms of $\mLambda$'s
Cholesky factor  $\mLambda = \mR^\top \mR$. We optimise over the space $(\vmu, \overline{\mR})$, where $\vmu
\in \R^{p + m}b$ and $\overline{\mR}$ is a lower-triangular $(p + mb) \times (p + mb)$ matrix. Then
		
\begin{equation*}
	\mR_{ij} =
	\begin{cases}
		\exp(\overline{\mR}_{ij}), & i = j             \\
		\overline{\mR}_{ij},       & i > j             \\
		0,                         & \text{otherwise}, 
	\end{cases}
\end{equation*}
		
\noindent exponentiating the diagonal to ensure positive-definiteness of $\mR$. We parameterise $\mLambda$
as $\mLambda = \mR \mR^\top$ so that is is guaranteed to be symmetric, and we only have $p(p-1)/2$ 
parameters to deal with instead of $p^2$ parameters, some of which are constrained. 

This parameterisation can lead to numeric overflow when the diagonals of $\overline{\mR}$ become moderately
large, which can lead to singular matrices when attempting to invert $\mLambda$. We dealt with this by
defining a piecewise function which is exponential for arguments less than $t$, and quadratic for arguments
greater than or equal to $t$
$$
f(r_{ij}) =
\begin{cases}
	e^{r_{ij}}, r_{ij} < t                   \\
	a r_{ij}^2 + b r_{ij} + c, r_{ij} \geq t 
\end{cases}
$$
and then choosing the co-efficients $a$, $b$ and $c$ such that the function, first and second derivatives would
agree at $r_{ij} = t$.

To find the co-efficients $a$, $b$ and $c$, we solved the system of equations formed by repeatedly 
differentiating the quadratic at $r_{ij} =  t$ and equating it with $e^t$
$$
\begin{array}{lllll}
	e^t & = & a t^2 & + \quad b t & + \quad c \\
	e^t & = &       & \quad 2a t  & + \quad b \\
	e^t & = &       &             & \quad 2a  \\
\end{array}
$$
to obtain $a = e^t / 2$, $b = (1 - t) e^t$ and $c = \{1 - t^2/2 - (1 - t) t\} e^t$.

We also addressed the overflow problem by working with the Cholesky factorisation of $\mLambda^{-1}$
rather than $\mLambda$, allowing us to solve a system of equations rather than invert and multiply by a
matrix, which is also faster and more numerically stable. We use knowledge of the regression  model we are
fitting to specify a sparse matrix structure, greatly reducing the dimension of   the problem and thus
improving both computational speed and numeric accuracy.

% \noindent By noticing that the lower rows of the product depend on the higher rows of the Cholesky factor, we
% observe that by re-ordering the fixed and random effects in $\mLambda$ so that the , we arrive at a covariance structure which is sparse in the first diagonal block. Thus the Cholesky factor of $\mLambda$ that we optimise over is as sparse as possible. This reduces the dimension of the optimisation problem we have to solve from
% $\BigO(np^2)$ to $\BigO(np)$.
	
Any symmetric matrix $\mLambda$ can be written as a product of its' Cholesky factors, $\mLambda =
\mR \mR^\top$ where $\mR$ is lower triangular. $\mR$ is unique if $\mR_{ii} \geq 0$.
	
\begin{align*}
	&\begin{pmatrix}
	\mR_{11}          & 0                                    & 0                                     \\
	\mR_{21}          & \mR_{22}                             & 0                                     \\
	\mR_{31}          & \mR_{32}                             & \mR_{33}                              
	\end{pmatrix}
	\begin{pmatrix}
	\mR_{11}          & \mR_{21}                             & \mR_{31}                              \\
	0                 & \mR_{22}                             & \mR_{32}                              \\
	0                 & 0                                    & \mR_{33}                              
	\end{pmatrix}
	\\
	=& \begin{pmatrix}
	\mR_{11}^2        &                                      & \text{symmetric}                      \\
	\mR_{21}\mR_{11} & \mR_{21}^2 + \mR_{22}^2 \\
	\mR_{31} \mR_{11} & \mR_{31}\mR_{21} + \mR_{32} \mR_{22} & \mR_{31}^2 + \mR_{32} ^2 + \mR_{33}^2 
	\end{pmatrix}.
\end{align*}

\noindent We exploit this structure. By interchanging the fixed and random effects in the design matrix $\mC = [\mX \mZ]$ to $\mC = [\mZ \mX]$, and re-ordering the dimensions of $\vmu, \mLambda$ and $\mSigma$ in the same manner, the independence between the
blocks relating to the random effects in $\mZ$ induce sparsity in the Cholesky factor $\mR$ of
$\mLambda^{-1}$, as can be seen in Figures \ref{fig:covfixedrandom}, \ref{fig:covrandomfixed},
\ref{fig:cholfixedrandom} and \ref{fig:cholrandomfixed}. Thus the Gaussian $q(\vnu) \sim \N(\vmu, \mLambda)$ can be optimised over a space of dimension $\frac{1}{2} p (p + 1) + pq + \frac{1}{2} q (q + 1)$ rather than dimension
$\frac{1}{2} (p + mq) (p + mq + 1)$ as in the dense parameterisation. This leads to subtantial performance gains
when $m$ is large, as is typically the case in problems of practical importance such as longitudinal or 
clinical trials with many subjects or the application presented in Section \ref{sec:application}.
		
By re-ordering the fixed and random effects in $\mLambda$, we end up with a covariance structure which is 
sparse in the first diagonal block.

\subsection{Precision parameterisation}

We optimise over the space $(\vmu, \overline{\mR})$ as before, but now 
		
\begin{equation*}
	\mR_{ij} =
	\begin{cases}
		\exp(-\overline{\mR}_{ij}), & i = j             \\
		\overline{\mR}_{ij},        & i > j             \\
		0,                          & \text{otherwise}.
	\end{cases}
\end{equation*}
	
\noindent This new choice of parameterisation allows us to calculate $\frac{1}{2} \text{diag}(\mC \mLambda
\mC^\top)$ by solving the linear systems $\mR \va = \mC_{i}, i=1, \ldots, n$ for   $\va$ and then calculating
$\va^\top\va$ where $\mC_{i} = $ the $i$th row of $\mC$, rather than calculating $\text{diag}(\mC \mLambda
\mC^\top)$ directly.
	
% TODO: We fixed this using safe-exp
The implementation of these algorithms was not without its' challenges, chiefly numerical issues encountered during testing and verification of the accuracy of the model fitting. Using the exponential function to parameterise the main diagonal coupled with L-BFGS-B's unconstrained line search and \texttt{optim()}'s lack of robustness to infinities lead to many overflow problems which may have been lessened or dealt with entirely by using a function with a less aggressive growth rate, such as an appropriate piecewise quadratic.
	
The main computational cost is the evaluation of the variational lower bound and its' derivatives. By
virtue of their dimension, the expressions involving $\mLambda$ dominate the computational cost. The key
term is $\frac{1}{2} \diag(\mC \mLambda \mC^\top)$. This can be naively calculate by ignoring the
symmetry in the expression and simply calculating the product $\mC \mLambda \mC^\top$, taking $2 n
\times (p + m b)^2$ floating point operations, and retaining the diagonal entries of the result. This is
obviously wasteful, as the off--diagonal entries of the resulting product that has just been computed
are immediately discarded.
	
By parameterising $\mLambda$ in terms of its' Cholesky factors and realising that
	
\[
	\mC \mLambda \mC^\top = \mC \mR \mR^\top \mC^\top
\]
	
\noindent and that
	
\[
	\diag(\mC \mLambda \mC^\top)_{ii} = \mC_{i .} \mR \mR^\top \mC_{i .}^\top, 1 \leq i \leq n
\]
	
\noindent we can calculate the products $\mC_{i .} \mR, 1 \leq i \leq n$, using $n \times \frac{1}{2}(p + m
b)(p + m b   + 1)$ floating point operations, and storing the results of the $i$th product in the $i$th
element of the   vector $\va$ and then calculate $\diag(\mC \mLambda \mC^\top) = \va^\top \va$.
	
Moreover, mixed models typically have sparse design matrices, allowing us to encode $\mR$ as a sparse matrix, and	further reduce   this depending on the model. For example, in the random intercept case, only the diagonals of the random effects block need to be non-zero, and hence the above expression can be calculated in
$\BigO(n)$ floating point operations.
	
% $\log |\mR|$ can be calculated using only $p + m b$ floating point operations, as $\mR$ is lower triangular.
	
For the precision parameterisation, we observe that in this parameterisation
\[
	\diag(\mC \mLambda \mC^\top)_{ii} = \mC_{i .} \mR^{-\top} \mR^{-1} \mC_{i .}^\top, 1 \leq i \leq n,
\]
	
\noindent and so by solving $\mR \va = \mC_{i .}^\top$ for $\va$ for all $i$ at a cost of $n \times
\frac{1}{2} (p + m b) (p + m b + 1)$ floating point operations, and then calculating
\[
	\diag(\mC \mLambda \mC^\top)_{ii} = \va^\top \va, 1 \leq i \leq n,
\]
	
\noindent we can then calculate $\diag(\mC \mLambda \mC^\top)$.
	
As above, by using our knowledge of the model being fit we can encode $\mR$ sparsely to decrease the required
computation still further. In the random intercept model case, the computational cost will drop to $n \times
\{m + \frac{1}{2} p (p + 1) + p \times m b\}$.
			
\subsubsection{Justification of Cholesky factor of the precision matrix as the parameterisation}
The Gaussian Variational Approximation is fit by maximising the Gaussian Variational Lower Bound, which is
parameterised by a mean vector $\vmu$ and a covariance matrix $\mLambda$. The most straightforward
parameterisation of $\vmu$ is the natural parameterisation. But the covariance matrix has many possible
parameterisations. Covariance matrices are positive semi-definite, and hence symmetric, so they have a unique
Cholesky factorisation. Parameterising the covariance matrix in terms of the Cholesky factor allows us to
represent the square covariance matrix using only a lower triangular matrix with half as many non-zero
elements. Thus the Cholesky factor is a convenient way to parameterise covariance matrices.

While the most obvious choice for parameterising the covariance matrix is simply the covariance matrix itself,
this is not the only available parameterisation. Another choice worth considering is parameterising the
covariance matrix using the Cholesky factor of the precision matrix.

The variational lower bound of a Gaussian Variational Approximation takes the form
\begin{align*}
\log \underline{p}(\vy; \vmu, \mLambda) =& \vy^\top \mC \vmu - \vone^\top B(\mC \vmu, \text{diag}(\mC \mLambda \mC^\top)) + \vone^\top c(\vy) \\
&- \tfrac{1}{2} \vmu^\top \mSigma^{-1} \vmu - \tfrac{1}{2} \tr(\mSigma^{-1} \mLambda) \\
&+ \tfrac{1}{2} \log |\mLambda| - \tfrac{1}{2} \log |\mSigma| + \tfrac{d}{2}
\end{align*}

Let $\mOmega = \mLambda^{-1}$, the precision matrix. Then if we reparameterise the variational lower bound in
terms of $\mOmega$ it becomes
\begin{align*}
F(\mOmega) =& \vy^\top \mC \vmu - \vone^\top B(\mC \vmu, \text{diag}(\mC \mOmega^{-1} \mC^\top)) + \vone^\top c(\vy) \\
&- \tfrac{1}{2} \vmu^\top \mSigma^{-1} \vmu - \tfrac{1}{2} \tr(\mSigma^{-1} \mOmega^{-1}) \\
&- \tfrac{1}{2} \log |\mOmega| - \tfrac{1}{2} \log |\mSigma| + \tfrac{d}{2}
\end{align*}
When the variational lower bound is optimised, by the first-order optimality conditions, $\tfrac{\partial
F}{\partial \mOmega_{jk}} = \vzero$. Then using matrix calculus and the properties of the trace operator
\begin{align*}
\tfrac{\partial F}{\partial \mOmega_{jk}} &= -\tfrac{1}{2} \tr(\mOmega^{-1} \tfrac{\partial \mOmega}{\partial \mOmega_{jk}}) + \tfrac{1}{2} \tr(\mSigma^{-1} \mOmega^{-1} \tfrac{\partial \mOmega}{\partial \mOmega_{jk}} \mOmega^{-1})
-\tfrac{1}{2} \tr\{\mC^\top \text{diag}(B^{(2)}(\mC \vmu, \text{diag}(\mC \mOmega^{-1} \mC^\top))) \mC \mOmega^{-1} \tfrac{\partial \mOmega}{\partial \mOmega_{jk}} \mOmega^{-1}\} \\
&=-\tfrac{1}{2} [ \tr(\mOmega^{-1} \mOmega \mOmega^{-1} \tfrac{\partial \mOmega}{\partial \mOmega_{jk}}) - \tr(\mSigma^{-1} \mOmega^{-1} \tfrac{\partial \mOmega}{\partial \mOmega_{jk}} \mOmega^{-1})
+ \tr\{\mOmega^{-1} \mC^\top \text{diag}(B^{(2)}(\mC \vmu, \text{diag}(\mC \mOmega^{-1} \mC^\top))) \mC \mOmega^{-1} \tfrac{\partial \mOmega}{\partial \mOmega_{jk}}\} ] \\
&= -\tfrac{1}{2} \tr[\mOmega^{-1}\{ \mOmega - \mC^\top \text{diag}(B^{(2)}(\mC \vmu, \text{diag}(\mC \mOmega^{-1} \mC^\top))) \mC - \mSigma^{-1} \} \mOmega^{-1} \tfrac{\partial \mOmega}{\partial \mOmega_{jk}}]
\end{align*}

As $\mOmega^{-1} \ne \vzero$ and $\tfrac{\partial \mOmega}{\partial \mOmega_{jk}} \ne \vzero$, this implies
$\mOmega = \mC^\top \text{diag}(B^{(2)}) \mC + \mSigma^{-1}$. Thus the sparsity of $\mOmega$ depends on the
structure of $\mC$ and $\mSigma$, which depends on the model specified.

Another argument in favour of parameterising using the precision matrix is that the covariance matrix contains
the marginal covariances between the elements of $\vnu$, while the precision matrix contains the conditional
covariances betweenm those elements. In generalised linear mixed models, fixed and random effects are
conditionally independent, implying sparsity in the precision matrix although not necessarily in the
covariance matrix.

Still another advantage of this parameterisation is its' greater numerical accuracy.
Matrix multiplication and back substitution are both equally numerically accurate and stable --- see
\cite{Golub:1996:MC:248979} \S2.7.8 and \S3.1.2 or \cite{trefethen97} Lecture 17. Moreoever, as the precision
matrix will be sparse due to the specification of the mixed model/conditional independence, implying the
numerical accuracy of the inversion will be higher as there are fewer non-zero entries in the Cholesky factor
of the precision matrix than of the Cholesky factor of the covariance matrix. Thus parameterising the
variational lower bound in terms of the precision matrix will have the same or higher numerical accuracy than
parameterising in terms of the covariance matrix.

\section{Numerical results}
\label{sec:results}
		
The accuracy of each model fitting algorithms presented in Section \ref{sec:gaussian} was assessed by
comparing the approximating distribution of each parameter with the posterior distribution of Monte Carlo
Markov Chain samples of that parameter. 1 million Monte Carlo Markov Chain samples were generated using
\texttt{RStan}, as described in \cite{Carpenter2016} and \cite{StanDevelopmentTeam2016}. The accuracy of
examples of random intercept, random slope and spline models were evaluated using this method.
		
\subsection{Simulated data}
		
For each of these simulations, the model is as presented in Section \ref{sec:model}.
		
Several common application scenarios were simulated and their accuracy evaluated. A random intercept
model was simulated with $\vbeta = (2, 1)^\top$, $\rho = 0.5$, $m = 20$, $n_i = 10$ and $b = 1$. The
results are presented in Table \ref{tab:accuracy_int}. A random slope model was simulated with $\vbeta =
(2, 1)^\top$, $\rho = 0.5$, $m = 20$, $n_i = 10$ and $b = 2$. The results are presented in Table
\ref{tab:accuracy_slope}. Spline model was fit to a data set generated from the function $3 + 3
\sin{(\pi x)}$ on the interval $[-1, 1]$. The resulting model fits are presented in Figure
\ref{fig:spline}.
		
To assess the speed of each approach, a test case was constructed of a random slope model with $m=50$
groups, each containing $n_i = 100$ individuals. A model was then fit to this data set ten times using
each algorithm, and the results averaged. They are presented in Table \ref{tab:application_slope_speed}.

\begin{table}
	\begin{tabular}{|l|rr|}
		\hline
		Algorithm & Mean (seconds) & Standard deviation (seconds) \\
		\hline
		Laplace's method & $0.37$ & $0.07$ \\
		GVA covariance parameterisation & $2.04$ & $1.24$ \\
		% Why is this slower?
		GVA inverse parameterisation & $0.38$ & $0.66$ \\
		GVA fixed point & $0.05$ & $0.07$ \\
		\hline
	\end{tabular}
	\caption{Table of results - Speed}
	\label{tab:application_slope_speed}
\end{table}

% The stability of the algorithms was confirmed by running them on 10,000 different data sets that were randomly
% generated after having initialised the random number generator with different seeds.
		
Median accuracy of the algorithms was assessed by running them on 100 randomly generated data sets. The	results are presented in Figure \ref{fig:median_accuracy_intercept} and Figure
\ref{fig:median_accuracy_slope}.
		
% Figure: Median accuracy graph intercept
\begin{figure}
	\begin{center}
		\includegraphics[width=0.95\textwidth]{code/results/median_accuracy_combined_intercept.pdf}
		\caption{Boxplots of accuracies of the parameter estimates for a random intercept model after 100 repeated
							runs on simulated data. We see that the accuracy of the parameter estimates is quite stable,
							and the median accuracies are high.}
		\label{fig:median_accuracy_intercept}
	\end{center}
\end{figure}
		
% Figure: Median accuracy graph slope
\begin{figure}
	\includegraphics[width=0.95\textwidth]{code/results/median_accuracy_combined_slope.pdf}
	\caption{Boxplots of accuracies of the parameter estimates for a random slope model after 100 repeated
							runs on simulated data. We see that the accuracy of the parameter estimates is quite stable,
							and the median accuracies are high.}
	\label{fig:median_accuracy_slope}
\end{figure}
		
% Table of accuracy results - intercept model
\begin{table}
	\begin{tabular}{|l|rrrr|}
		\hline
		                   & Laplace's Method & GVA $(\mLambda = \mR \mR^\top)$ & GVA NP $(\mLambda = (\mR \mR^\top)^{-1})$ & GVA FP \\
		\hline
		$\vbeta_1$         & $85\%$           & $90\%$                          & $91\%$                                    & $90\%$ \\ 
		$\vbeta_2$         & $76\%$           & $98\%$                          & $99\%$                                    & $99\%$ \\ 
		Mean of $\vu$      & $81\%$           & $94\%$                          & $94\%$                                    & $94\%$ \\
		$\sigma^2_{\vu_1}$ & $66\%$           & $66\%$                          & $66\%$                                    & $66\%$ \\ 
		$\rho$             & $99\%$           & $99\%$                          & $99\%$                                    & $99\%$ \\ 
		\hline
	\end{tabular}
	\caption{Table of accuracy - Random intercept model}
	\label{tab:accuracy_int}
\end{table}
		
\begin{table}
	\begin{tabular}{|l|rrrr|}
		\hline
		                   & Laplace's Method & GVA $(\mLambda = \mR \mR^\top)$ & GVA $(\mLambda = (\mR \mR^\top)^{-1})$ & GVA FP \\
		\hline
		$\vbeta_1$         & $67\%$             & $88\%$                            & $88\%$                                   & $88\%$   \\
		$\vbeta_2$         & $70\%$             & $89\%$                            & $88\%$                                   & $89\%$   \\
		Mean of $\vu$      & $70\%$             & $91\%$                            & $91\%$                                   & $91\%$   \\
		$\sigma^2_{\vu_1}$ & $71\%$             & $73\%$                            & $73\%$                                   & $73\%$   \\
		$\sigma^2_{\vu_2}$ & $68\%$             & $69\%$                            & $69\%$                                   & $69\%$   \\
		$\rho$             & $91\%$             & $90\%$                            & $90\%$                                   & $90\%$   \\
		\hline
	\end{tabular}
	\caption{Table of accuracy - Random slope model}
	\label{tab:accuracy_slope}
\end{table}
		
% \begin{table}
% \caption{Table of accuracy - Splines}
% \label{tab:accuracy_spline}
% \begin{tabular}{|l|l|}
% \hline
% Approximation & Accuracy \\
% \hline
% Laplace's Method & 0.969 \\
% GVA & 0.969 \\
% GVA NP & 0.969 \\
% GVA NR & 0.969 \\
% \hline
% \end{tabular}
% \end{table}
		
\begin{figure}
	\label{fig:spline}
	\caption{Comparison of VB and MCMC spline fits with the true function}
	\includegraphics[width=0.95 \textwidth]{code/results/accuracy_plots_spline_gva2.pdf}
\end{figure}
		
% Graphs - exactly what sort of graphs do we need?
% Median accuracy
% Increase in lower bound
% MCMC posterior, with approximating posterior for at least one or two of the
% key parameters, such as, say, vbeta[2]
		
\subsection{Numerical stability of the parameterisation}

In the process of performing numerical experiments, we discovered that our model fitting software was
prone to numeric overflow due to the log link in our model and the exponentiation of the diagonals of
the Cholesky factors in the covariance parameterisation of the Gaussian Variational Approximation of
$\vnu$.

We dealt with this difficulty by developing a 'safe exponential' parameterisation for the diagonals of
the Cholesky factors. The parameterisation is exponential up to a threshold $t$, and then quadratic
beyond that threshold.

The stability of this scheme was tested by calculating the accuracy of the approximations fit with a
range of safe exponential thresholds, the results of which are presented in Figure
\ref{fig:stability_accuracy}. The variational approximation was found to be stable, with the accuracy
largely insensitive to the choice of threshold.

\begin{figure}
	\includegraphics[width=0.95 \textwidth]{code/stability_intercept.pdf}
	\label{fig:stability_accuracy}
	\caption{Accuracy of approximation of parameters versus the safe exponential threshold}
\end{figure}

We repeated our numerical experiments with the new parameterisation, varying the threshold within
reasonable bounds and found that the numerical experiments no longer resulted in overflow, and that the
numerical accuracy of the approximation was still very good.

The stability of the GVA algorithm with the parameterisation $\mLambda = (\mR^\top \mR)^{-1}$ depends on
the threshold chosen for the safe exponential function. When the threshold is set to $2$, the algorithm
is stable for all starting points within the grid except $6$. When the threshold is set to $\infty$,
equivalent to using the naive $\exp$ parameterisation, the algorithm encounters numerical errors for
every starting point on the  grid.
	
\subsection{Stability of the GVA precision parameterisation algorithm for different starting points}
		
The numerical stability of each fitting algorithm in Section \ref{sec:gaussian} was assessed by
initialising each algorithm from a range of different starting points. Errors due to numerical
instability and the fitted $\vmu$ were recorded for each starting point.
		
A data set of 100 individuals in ten groups ($m=10$) was generated from a model with a fixed intercept
and slope, and a random intercept. $\vmu$ was initialised from a grid of points on the interval $[-4.5,
5]$ for intercept and slope, spaced $0.1$ apart. The error counts are presented in Table
\ref{tab:stability_results}. Plots of the starting locations which resulted in numerical errors when the
fitting algorithm was run are presented in \ref{fig:stability_locations_gva}.
		
\begin{table}
	\begin{tabular}{|l|r|}
		\hline
		Algorithm                            & Error count \\
		\hline
		Laplace's algorithm                  & $12$          \\
		GVA $\mLambda = \mR^\top \mR$        & $1306$       \\
		GVA $\mLambda = (\mR^\top \mR)^{-1}$ & $6$           \\
		GVA NR fixed point                   & $992$         \\
		\hline
	\end{tabular}
	\caption{Count of numerical errors for each algorithm during stability tests}
	\label{tab:stability_results}
\end{table}

The GVA algorithm with the $\mLambda = (\mR \mR^\top)^{-1}$ parameterisation was less prone to
instability due to starting point when the safe exponential parameterisation was used then when it was
not used, as can be seen from Figure \ref{fig:stability_locations_gva}. % FIXME: Add error counts
		
\begin{figure}
	\includegraphics[width=0.45 \textwidth]{code/safe_exp_stability.pdf}
	\includegraphics[width=0.45 \textwidth]{code/no_safe_exp_stability.pdf}
	\label{fig:stability_locations_gva}
	\caption{Starting locations which caused the GVA fitting algorithm to fail with numeric errors. The true model had fixed parameters $\vbeta = (2, 1)^\top$ and random intercepts. There were ten groups in the hierarchical model each	with ten individuals $(m=10, n_i=10)$. In the left figure the starting points which lead to numeric errors when the safe exponential was used are shown, while in the right figure the starting points which lead to numeric errors when the safe exponential was not used are plotted.}
\end{figure}

\subsection{Stability of the GVA fixed point algorithm for different starting points}
The naive fixed point algorithm was extremely unstable for many starting points, as can be seen from
Figure \ref{fig:stability_locations_nr}. The variant of the algorithm which checked whether the
inversion of the $\mLambda_{\vu \vu}$ block of $\mLambda$ was performed successfully was much more
stable, and did not suffer from any numeric errors at all over the range of starting points we tested.
The algorithm is able to abort safely, and allow the Variational Bayes algorithm to update the other
parameters before trying to fit the Gaussian component of the model again until the correct parameters
are accurately estimated.

\begin{figure}
	\includegraphics[width=0.95 \textwidth]{code/local_solutions_gva_nr_error_locations_no_protections.pdf}
	\label{fig:stability_locations_nr}
	\caption{Starting locations which caused the NR fitting algorithm to fail with numeric errors. The true model had fixed parameters $\vbeta = (2, 1)^\top$ and random intercepts. There were ten groups in the
	hierarchical model each	with ten individuals $(m=10, n_i=10)$}
\end{figure}

\section{Application}
\label{sec:application}

\subsection{Poisson example without zero-inflated component -- Police stops}
\label{sec:police_stops}
The data set used for this example was the police stop example from Chapter 15 of \cite{Gelman2007}.
The model fit was
$$
	\vy_{ep}        \sim \text{Poisson}(n_{ep} e^\vnu)
$$
where $\vnu = {\beta_0 + \beta_e \text{ethnicity}_e + \alpha_{c} \text{crime} + \vu_p}$, with priors
\begin{align*}
	\valpha				& \sim \N(0, \sigma_\valpha^2),	\\
	\vbeta        & \sim \N(0, \sigma_\vbeta^2), \text{ and }\\
	\vu_p         & \sim \N(0, \sigma_\vu^2)
\end{align*}
where $p$ is the $p$-th precinct, and $e$ is the $e$-th ethnicity (blacks, hispanics or whites), and $c$ is
the $c$-th category of crime (violent crimes, weapons crimes, propery crimes or drug crimes). The random
intercepts $u_p$ allow for variation in the base rates of stops across precincts, the co-efficients $\beta_j$
measure the effect of ethnicity on the rate of police stops and the co-efficients $\alpha_k$ measure the
effect of each type of crime on the rate. The model finds the relationship between the number of police stops
in each precinct and  ethnicity for each type of crime.

The model was fit using the GVA algorithm with the $\mLambda = (\mR^\top \mR)^{-1}$ parameterisation, using
the prior $a_\rho = 3$, $b_\rho = 1$ on $\rho$. Accuracy of the approximation was assessed by comparing the
fitted distribution for each parameter to a kernel density estimate of the parameter's distribution from 1
million samples from the equivalent model fitted using Stan. The results are presented in Table
\ref{tab:application_police_stops}. Figure \ref{fig:police_stops}

\begin{figure}
\includegraphics[width=0.95 \textwidth]{code/results/accuracy_plots_application2_GVA_inv_par-highlights-nup.pdf}
\caption{Accuracy of parameter estimates for police stops}
\label{fig:police_stops}
\end{figure}

% Table of results
\begin{table}
	\begin{tabular}{|l|rrrr|}
		\hline
		Covariate                     & Posterior Mean & Lower 95\% CI & Upper 95\% CI & Accuracy \\
		\hline
		Intercept [African-Americans] & $4.04$          & $3.98$           & $4.07$          & 82\%   \\
		$\beta_2$ [hispanics]         & $-0.45$         & $-0.46$          & $-0.43$         & 98\%   \\
		$\beta_3$ [whites]            & $-1.38$         & $-1.40$          & $-1.37$         & 99\%   \\
		$\alpha_1$ [weapons crimes]   & $0.58$          & $0.57$           & $0.59$          & 89\%   \\
		$\alpha_2$ [property crimes]  & $-0.19$         & $-0.21$          & $-0.17$         & 92\%   \\
		$\alpha_3$ [drug crimes]      & $-0.75$        & $-0.77$          & $-0.73$         & 95\%   \\
		Random intercept              & $1.32$         & $-0.19$         & $2.20$         & 87\%   \\
		$\sigma^2_{\vu}$              & $8.57$          & $1.02$           & $24.35$         & 49\%     \\
		\hline
	\end{tabular}
	\caption{Table of results - Police stops}
	\label{tab:application_police_stops}
\end{table}

% Table of speeds
\begin{table}
	\begin{tabular}{|ll|}
		\hline
		Algorithm & Time  in seconds \\
		\hline
		Laplace & $0.24$ \\
		GVA covariance parameterisation & $6.02$ \\
		GVA inverse paramaterisation & $3.92$ \\
		GVA fixed point & $0.18$ \\
		\hline
	\end{tabular}
	\caption{Table of speeds - Police stops}
	\label{tab:police_stop_speeds}
\end{table}

% TODO: You need to describe the data set and the model.
\subsection{Zero--inflated example -- Cockroaches in apartments}
\label{sec:cockroaches}
The model described in this section was fit  to the cockroach data set from Section 6.7 of
\cite{Gelman2007}, taken from a study on the effect of integrated pest management in controlling
cockroach levels in urban apartments. The data set contains data on 160 treatment and 104 control
apartments, along with the response $y_i$ in each apartment of the number of cockroaches caught in a set
of traps. The apartments had the traps deployed for different numbers of days, referred to as trap days,
which was handled by using a log offset \cite{Agresti2002}. The predictors in the data set included the
pre-treatment roach level, a treatment indicator, the time of the observation and an indicator for
whether the apartment is in a senior building restricted to the elderly.
		
In the example application presented in this paper, the zero component represents an apartment completely free of roaches, while the non-zero component represents an apartment where roaches have been able to live and reproduce, possibly in spite of pest control treatment aimed at preventing them from doing so.

The model fit was
$$
	y_i = \begin{cases}
	0, \phantom{-} \text{if} \phantom{-} R_i = 0, \\
	\text{Poisson}(e^{\mX_i \vbeta + \mZ_i \vu}), \phantom{-} \text{if} \phantom{-} R_i = 1
	\end{cases}
$$
with priors
\begin{align*}
	R_i &\sim \text{Bernoulli}(\rho), \\
	\rho &\sim \text{Beta}(a, b), \\
	\vbeta &\sim \text{N}(\vzero, \sigma^2_\vbeta \mI), \\
	\vu &\sim \text{N}(\vzero, \mSigma) \text{ and } \\
	\mSigma &\sim \text{Inverse-Wishart}(\mPsi, v)
\end{align*}
with prior parameters $a = 1$, $b = 1$, $\sigma^2_\vbeta = 10^5$, $\mPsi = 10^{-5} \mI$ and $v = 2$.
These priors were chosen to be vaguely informative for the variance components and a uniform prior for
the zero-inflation proportion latent variable $\rho$. The fixed effects covariates included in the
model were time in days and time in days $\times$ pest control treatment. A random intercept to account
for variation between the apartment buildings was included.
		
The GVA algorithm with the $\mLambda = (\mR^\top \mR)^{-1}$ parameterisation was used to fit a random
intercept model to the Roaches data set provided \cite{Gelman2007}. The fitted co-efficients and accuracy
results are presented in Table \ref{tab:application_roaches}.
		
%       lci  uci
% 1  3.179 3.157 3.201
% 2 -0.046 -0.053 -0.039
% 3 -0.420 -0.434 -0.406
% 1 -0.976 -1.015 -0.936
% 2 -0.309 -0.323 -0.295
% 3 -0.947 -0.963 -0.930
% 4 -2.129 -2.384 -1.874
% 5 -3.230 -3.490 -2.970
% 6 -3.099 -3.404 -2.794
% 7 -1.290 -1.326 -1.255
% 8 -0.956 -0.991 -0.921
% 9 -2.404 -2.600 -2.209
% 10 -1.076 -1.123 -1.029
% 11 -1.079 -1.107 -1.052
% 12 -1.681 -1.737 -1.624
		
%> round(cbind(fit1$vmu, lci, uci), 3)
% fit1$a_rho
% [1] 377.2375
% > fit1$b_rho
% [1] 152.7625
		
\begin{table}
	\begin{tabular}{|l|rrrr|}
		\hline
		Covariate          & Posterior Mean & Lower 95\% CI & Upper 95\% CI & Accuracy \\
		\hline
		Intercept          & $3.42$						& $3.2$ 					& $3.65$          & $95\%$     \\
		Time               & $-0.14$        & $-0.05$       & $-0.02$       & $96\%$     \\
		Time:Treatment     & $-0.31$        & $-0.43$       & $-0.14$       & $96\%$     \\
		Random intercept   & $-1.60$        & $-1.71$       & $-1.49$       & $90\%$     \\
		$\sigma^2_{\vu_1}$ & $3.29$           & $2.02$          & $8.48$          & $63\%$     \\
		$\rho$             & $0.51$           & $0.50$          & $0.55$          & $62\%$     \\
		\hline
	\end{tabular}
	\caption{The posterior means, 95\% credible intervals and accuracy of the fixed and random
						effects, $\sigma_{\vu_1}^2$ and $\rho$ for the Roach model.}
	\label{tab:application_roaches}
\end{table}

\begin{table}
	\begin{tabular}{|lr|}
	\hline
	Algorithm & Time in seconds \\
	\hline
	Laplace & $0.49$ \\
	GVA & $1.08$ \\
	GVA inv. param & $0.81$ \\
	GVA fixed point & $0.11$ \\
	\hline
	\end{tabular}
	\label{tab:application_roaches_runtime}
	\caption{The runtimes in seconds for fitting algoritms when fitting the roach model.}
\end{table}
		
\begin{figure}
	\centering
	% \includepdf[width=75mm,height=75mm,pages={1,2,3,16},nup=2x2]{code/results/accuracy_plots_application_GVA2.pdf}
	\begin{tabular}{@{}c@{\hspace{.5cm}}c@{}}
	\includegraphics[width=0.95 \textwidth]{code/results/accuracy_plots_application_GVA_inv_param-highlights-nup.pdf}
	\end{tabular}
	\caption{Accuracy graphs for roach model}
	\label{fig:accuracy_roach}
\end{figure}
		
\subsection{Example - Biochemists}
\label{sec:biochemists}
The model described in this section was fit to the biochemistry data set analysed by
\cite{10.2307/2579146}. The sample was taken from 915 biochemistry graduate students. The outcome
$\vy_i$ is the number of articles published in the last three years of the PhD. The covariates were the
gender of the student, coded $1$ for female and $0$ for male, the marital status of the student ($1$ for
married, $0$ for unmarried), the number of children under age six and the prestige of the PhD program.

In this example application, the zero component represents the number of biochemists who did not publish
any articles during the last three years of their PhD. Examination of the data reveals that this number
is higher than would be expected if the data followed a purely Poisson distribution -- 30\% of
biochemistry graduate students published no articles in their final years whereas a Poisson distribution
would predict only 18\%. This justifies our choice of model.

The model fit was
$$
	y_i = \begin{cases}
	\begin{array}{ll}
	0, & \text{if} R_i = 0, \\
	\text{Poisson}(e^\vnu), & \text{if} \phantom{-} R_i = 1,
	\end{array}
	\end{cases}
$$

\noindent where $\vnu = \vbeta_1 + \vbeta_2 \text{female} + \vbeta_3 \text{married} + \vbeta_4 \text{children under age 6} + \vbeta_5 \text{PhD}$, with priors
\begin{align*}
R_i &\sim \text{Bernoulli}(\rho), \\
\rho &\sim \text{Beta}(A, B) \text { and } \\
\vbeta &\sim \text{N}(0, \sigma_\vbeta^2 \mI)
\end{align*}

\noindent with $A=1$, $B=1$ and $\sigma_\vbeta^2 = 10,000$. The model was fit using the GVA inverse parameterisation algorithm. The resulting model fit is presented in Table \ref{tab:biochemists_results}
The accuracy of the parameter estimates is presented in Figure
\ref{fig:biochemists}. As this is a fixed effects model with a large number of samples relative to the
number of parameters being fit, we are able to estimate all of the parameters with great accuracy.

\begin{figure}
\includegraphics[width=0.95 \textwidth]{code/results/accuracy_plots_application_biochemists_GVA_inv_param-nup.pdf}
\label{fig:biochemists}
\caption{Accuracy of the approximations of the parameters fit to the biochemists data}
\end{figure}

\begin{table}
	\begin{tabular}{|l|rrrr|}
		\hline
		Covariate          & Posterior Mean & Lower 95\% CI & Upper 95\% CI & Accuracy \\
		\hline
		Intercept & $0.86$ & $0.65$ & $1.06$ &  $95\%$ \\
		Female & $-0.18$ & $-0.29$ & $-0.08$ &  $95\%$ \\
		Married & $0.06$ & $-0.05$ & $0.18$ & $96\%$ \\
		Children under age 6 & $-0.08$ & $-0.15$ & $-0.01$ & $97\%$ \\
		PhD & $0.03$ & $-0.02$ & $-0.01$ & $97\%$ \\
		\hline
	\end{tabular}			
	\label{tab:biochemists_results}
	\caption{The posterior means, 95\% credible intervals and accuracy of the fixed effects for the 
						Biochemists model.}
\end{table}

\begin{table}
	\begin{tabular}{|l|r|}
	\hline
	Algorithm & Time in seconds \\
	\hline
	Laplace & $1.01$ \\
	GVA & $0.96$ \\
	GVA inv. param & $0.68$ \\
	GVA fixed point & $0.41$ \\
	\hline
	\end{tabular}
	\label{tab:biochemists_runtime}
	\caption{The runtimes in seconds for fitting algoritms when fitting the Biochemists model.}
\end{table}

\subsection{Example - Owls}
\label{sec:owls}
The model described in this section was fit to the Owls data set from taken from \cite{zuur_mixed_2009}.
The sample was 599 observations of owls grouped across 25 nests.The fixed covariates
fit in the model were food treatment (Deprived or Satiated) and arrival time, a continuous covariate.
The variation between the 25 different nests sampled from was modelled by a random intercept
$\vu$.

The model fit was
$$
	y_i = \begin{cases}
	\begin{array}{ll}
	0, & \text{if} R_i = 0, \\
	\text{Poisson}(e^\vnu), & \text{if} \phantom{-} R_i = 1,
	\end{array}
	\end{cases}
$$
where $\vnu = {\vbeta_2 \text{Food Treatment = Satiated} + \vbeta_3 \text{Arrival Time} + \vu_n}$ and $n$ is the $n$-th nest, with priors
\begin{align*}
R_i &\sim \text{Bernoulli}(\rho), \\
\rho &\sim \text{Beta}(A, B), \\
\vbeta &\sim \text{N}(\vzero, \sigma_\vbeta^2 \mI), \\
\vu &\sim \text{N}(0, \sigma_\vu^2) \text { and } \\
\sigma_\vu^2 &\sim \text{Inverse-Gamma}(s, t).
\end{align*}
where $\sigma_\vbeta^2=10,000$, $A=1$, $B=1$, $s=10^{-2}$ and $r=10^{-2}$.

The model was fit using the GVA inverse parameterisation algorithm. The accuracy of the parameter
estimates is shown in Figure \ref{fig:owls}. The variance component
The runtime of the algorithms is shown in Table
\ref{tab:owls_times}. We draw attention to the difference in run-times between the covariance and
inverse parameterisations. The inverse parameterisation is significantly faster -- $1.88$ seconds versus
$5.66$ seconds for the covariance parameterisation.

\begin{figure}
	\includegraphics[width=0.95 \textwidth]{code/results/accuracy_plots_application_owls_GVA_inv_param-highlights-nup.pdf}
	\caption{Accuracy of the approximations of the parameters fit to the Owls data}
	\label{fig:owls}
\end{figure}

\begin{table}
	\begin{tabular}{|l|rrrr|}
		\hline
		Covariate          & Posterior Mean & Lower 95\% CI & Upper 95\% CI & Accuracy \\
		\hline
		Satiated & $-0.22$ & $-0.21$ & $-0.21$ & $97\%$ \\
		Arrival Time & $-0.07$ & $-0.07$ & $-0.07$ & $73\%$ \\
		Random intercept (nest) & $0.34$ & $-5.28$ & $5.96$ & $84\%$ \\
		$\sigma_{\vu_1}^2$ & $7.90$ & $3.21$ & $468.12$ & $94\%$ \\
		$\rho$ & $0.74$ & $0.70$ & $0.77$ & $99\%$ \\
		\hline
	\end{tabular}			
	\label{tab:owls_results}
	\caption{The posterior means, 95\% credible intervals and accuracy of the fixed and random
						effects, $\sigma_{\vu_1}^2$ and $\rho$ for the Owls model.}
\end{table}

\begin{table}
	\begin{tabular}{|l|r|}
	\hline
	Algorithm & Time in seconds \\
	\hline
	Laplace & $0.71$ \\
	GVA covariance parameterisation & $5.66$ \\
	GVA inverse parameterisation & $1.88$ \\
	GVA fixed point & $0.18$ \\
	\hline
	\end{tabular}
	\caption{The runtimes of the fitting algorithms for the Owls model in seconds.}
	\label{tab:owls_times}
\end{table}


% % \include{Chapter_2_zero_inflated_models_formula_sheet}
%! TEX root = thesis.tex
% %\maketitle

\chapter{Numerical aspects of calculating Bayes factors for linear models using
	mixture $g$-priors
	}



\noindent
In this chapter, we consider the numerical evaluation of Bayes factors for
linear models using different mixture g-priors. In particular, we consider
hyperpriors for $g$ leading to closed-form expressions for the Bayes factor
including the hyper-$g$ and hyper-$g/n$ priors of \cite{Liang2008}, the
beta-prime prior of \cite{Maruyama2011}, the robust prior of
\cite{Bayarri2012}, and the Cake prior of \cite{OrmerodEtal2017}. In
particular, we describe how each of these Bayes factors, except for Bayes
factor under the hyper-$g/n$ prior, can be evaluated in efficient, accurate and
numerically stable manner. We also derive a closed form expression for the
Bayes factor under the hyper-$g/n$ for which we develop a convenient numerical
approximation. We implement an R package for Bayesian linear model averaging,
and discuss some associated computational issues. We illustrate the advantages
of our implementation over several existing packages on several small datasets.


\vfill
{\footnotesize
\noindent	
	This chapter corresponds to the collaborative paper: \\
	Greenaway M.J. \& Ormerod J.T (2018).
	Numerical aspects of calculating Bayes factors for linear models using mixture $g$-priors. Submitted to the Journal of Computational and Graphical Statistics.
}

\newpage 

 
\section{Introduction}

 
There has been a large amount of research in recent years into the appropriate
choice of suitable and meaningful priors for linear regression models in the
context of Bayesian model selection and averaging. Specification of the prior
structure of these models must be made with great care in order for Bayesian
model selection and averaging procedures to have good theoretical properties.
A key problem in this context occurs when the models have differing dimensions
and non-common parameters where inferences are typically highly sensitive to
the choice of priors for the non-common parameters due to the
Jeffreys-Lindley-Bartlett paradox
\citep{Lindley1957,Bartlett1957,OrmerodEtal2017}.  Furthermore, this
sensitivity does not necessarily vanish as the sample size grows
\citep{Kass1995,Berger2001}.  

Bayes factors in the context of linear model selection 
\citep{Zellner1980,
	Zellner1980b,
	Mitchell1988,
	George1993,
	Fernandez2001,
	Liang2008,
	Maruyama2011,
	Bayarri2012}
have received an enormous amount of attention. A landmark paper in this field
is \cite{Liang2008}.  \cite{Liang2008} considers a particular prior structure
for the model parameters.  In particular they consider a Zellner's $g$-prior
\citep{Zellner1980,Zellner1986} for the regression coefficients where $g$ is a
prior hyperparameter. The parameter $g$ requires special consideration. If $g$
is set to a large constant most of the posterior mass is placed on the null
model, a phenomenon sometimes referred to as Bartlett's paradox.  Due to this
problem they discuss previous approaches which set $g$ to a constant, e.g.,
setting $g=n$ \citep{Kass1995b},  $g=p^2$ \citep{Foster1994}, and
$g=\max(n,p^2)$ \citep{Fernandez2001}. However, \cite{Liang2008} showed that
all of these choices lead to what they call the information paradox, where the
posterior probability of the true model does not tend to 1 as the sample size
grows. Finally, \cite{Liang2008} also consider a local and global empirical
Bayes (EB) procedure for selecting $g$. In these cases \cite{Liang2008} show
that these EB procedures are model selection consistent except when the true
model is the null model (the model containing the intercept only). 

The above problems suggest that a hyperprior should be placed on $g$.
\cite{Bayarri2012} also discuss in some depth desirable properties priors
should have in the context of linear model averaging and selection.  In this
chapter we review the prior structures, specifically the hyperpriors on $g$,
that lead to closed form expressions of Bayes factors for linear models.  These
include the hyper-$g$ prior of \cite{Liang2008}, the beta-prime prior of
\cite{Maruyama2011}, and the robust prior of \cite{Bayarri2012}, and most
recently the Cake prior of \cite{OrmerodEtal2017} leads to a Bayes factor which
is a simple function of the Bayesian Information Criterion (BIC). We concern
ourselves with the efficient, accurate and numerically stable evaluation of
Bayes factors, Bayesian model averaging, and Bayesian model selection  for
linear models under the above choices of prior structures for the model
parameters.


Our main contributions in this chapter are as follows.
\begin{enumerate}

    \item To the above list of hyperpriors on $g$ leading to closed form Bayes
        factors we add the  hyper-$g/n$ prior of \cite{Liang2008} for which we
        derive a new closed form expression for the Bayes factor in terms of
        the Appell hypergeometric function.
	
    \item We derive an alternative expression for the Bayes factor when using
        the robust prior of \cite{Bayarri2012} in terms of the Gaussian
        hypergeometric function.
	
    \item We describe how the  Bayes factors corresponding to the hyper-$g$
        prior of \cite{Liang2008} and robust prior of \cite{Bayarri2012} can be
        calculated in an efficient, accurate and numerically stable manner
        without the need for special software or approximation.
	
    \item We derive a reasonably accurate approximation for the Appell
        hypergeometric function which can be calculated in an efficient and
        numerically stable manner when the number of non-zero coefficients in a
        particular model is strictly greater than 2.
	
    \item We make available a highly efficient and {\it numerically stable}
        {\tt R} package called {\tt blma} available for exact Bayesian linear
        model averaging using the above prior structures which is available for
        download from the following web address.
	
	\begin{center}
		\url{http://github.com/certifiedwaif/blma}
	\end{center}

\end{enumerate}

\noindent We demonstrate the advantages of our implementation of exact Bayesian
model averaging over some existing {\tt R} packages using several small
datasets.


The chapter is organised as follows. Section \ref{sec:bma} describes Bayesian
model averaging and model selection for linear models. Section \ref{sec:model}
outlines and justifies our chosen model and prior structure for the linear
regression model parameters. Section \ref{sec:hyperpriors} derives closed form
expressions for various marginal likelihoods using different hyperpriors for
$g$ and, wherever possible, describes how these may be evaluated well
numerically.  In Section \ref{sec:implementation}, we discuss details of our
implementation which made our implementation computationally feasible.  In
Section \ref{sec:numerical_g_prior} we perform a series of numerical
experiments to show the advantages of our approach. 
%Finally, in Section \ref{sec:conclusion} we provide a conclusion and discuss
%future directions.

\section{Bayesian linear model selection and averaging}
\label{sec:bma}

Suppose $\vy = (y_1,\ldots,y_n)^T$ is a response vector of length $n$, $\mX$ is
an $n \times p$ matrix of covariates where we anticipate a linear relationship
between $\vy$ and $\mX$, but do not know which of the columns of $\mX$ are
important to the prediction of $\vy$.  Bayesian model averaging seeks to
improve prediction by averaging over multiple predictions over different
choices of combinations of predictors.

We consider the linear model for predicting $\vy$ with design matrix $\mX$ via
\begin{equation}
	\label{eq:linearModel}
	\vy | \alpha, \vbeta, \sigma^2 \sim \N_n(\vone\alpha + \mX \vbeta, \sigma^2 \mI),
\end{equation} 


\noindent where $\alpha$ is the model intercept, $\vbeta$ is a coefficient
vector of length $p$, $\sigma^2$ is the residual variance, and $\mI$ is the $n
\times n$ identity matrix.  Without loss of generality, to simplify later
calculations, we will standardize $\vy$ and $\mX$ so that $\overline{y} = 0$,
$\|\vy\|^2 = \vy^T\vy = n$, $\mX_j^T\vone = 0$,  and $\|\mX_j\|^2 = n$ where
$\mX_j$ is the $j$th column of $\mX$. 


Suppose that we wish to perform Bayesian model selection, model averaging or
hypothesis testing where we are interested in comparing how different subsets
of predictors (which correspond to different columns of the matrix $\mX$) have
on the response $\vy$. To this end, let $\vgamma \in \{0, 1\}^p$ be a binary
vector of indicators for the inclusion of the $p$th column of $\mX$ in the
model where $\mX_\vgamma$ denotes the design matrix formed by including only
the $j$th column of $\mX$ when $\gamma_j = 1$, and excluding it otherwise. 

In order to keep our exposition as general as possible we will assume a prior
structure of $p(\alpha,\vbeta_{\vgamma}|\vgamma)p(\vgamma)$ but, for the time
being, we will leave the specific form of $p(\alpha,\vbeta_{\vgamma}|\vgamma)$
and $p(\vgamma)$ unspecified.  We adopt a prior on $\vbeta_{-\vgamma}$  of the
form
\begin{equation}
	\label{eq:spikeAndSlab}
	\ds p(\vbeta_{-\vgamma}|\vgamma) = \prod_{j=1}^p \delta(\beta_j;0)^{1-\gamma_j},
\end{equation} 

\noindent where $\delta(x;a)$ is the Dirac delta function with location $a$.
The prior on $\vbeta_{-\vgamma}$ in (\ref{eq:spikeAndSlab}) is the spike in a
spike and slab prior where the prior on $\vbeta_{\vgamma}$ is assumed to be
flat (the slab). There are several variants of the spike and slab prior
initially used in \cite{Mitchell1988} and later refined in \cite{George1993}.
The above structure implies that $p(\vbeta_{-\vgamma}|\vy)$ is a point mass at
$\vzero$ and leads to algebraic and computational simplifications for
components of $\vbeta$ when corresponding elements of $\vgamma$ are zero.
Thus, $\gamma_j=0$ is equivalent to excluding the corresponding predictor
$\mX_j$ from the model.


Exact Bayesian model averaging revolves around the posterior probability of a
model $\vgamma$ using Bayes theorem
\begin{equation*}
\ds p(\vgamma|\vy) = \frac{p(\vy|\vgamma)p(\vgamma)}{\sum_{\vgamma'} p(\vy|\vgamma')p(\vgamma')} = \frac{p(\vgamma)\mbox{BF}(\vgamma)}{\sum_{\vgamma'} p(\vgamma')\mbox{BF}(\vgamma')}
\quad \mbox{where} \quad 
p(\vy|\vgamma) = \int p(\vy,\vtheta|\vgamma) \, d\vtheta,
\end{equation*}

\noindent letting $\vtheta = (\alpha,\vbeta,\sigma^2)$, using $\sum_{\vgamma}$
to denote a combinatorial sum over all $2^p$ possible values of $\vgamma$, and
$\mbox{BF}(\vgamma) = p(\vy|\vgamma)/p(\vy|\vzero)$ is the null based Bayes
factor for model $\vgamma$.  Note that the Bayes factor is a statistic commonly
used in Bayesian hypothesis testing \citep{Kass1995,OrmerodEtal2017}.
Prediction is based on the the posterior distributions of $\alpha$ and $\vbeta$
where $p(\vbeta|\vy) = \sum_{\vgamma} p(\vbeta|\vy,\vgamma) \cdot
p(\vgamma|\vy)$ (with similar expressions for $\alpha$ and $\sigma^2$).  The
posterior expectation of $\vgamma$ is given by $\bE(\vgamma|\vy) =
\sum_{\vgamma} \vgamma \cdot p(\vgamma|\vy)$.

If one is required to select a single model, say $\vgamma^*$, two common
choices are the highest posterior model (HPM) which uses $\vgamma^* =
\vgamma_{\mbox{\tiny HPM}} = \argmax_\vgamma \{ \, p(\vy|\vgamma) \, \}$, or
the median posterior model (MPM) where $\vgamma^*$ is obtained by rounding each
element of $\bE(\vgamma|\vy)$ to the nearest integer.  The MPM has predictive
optimality properties \citep{Barbieri2004}.  If the MPM is used for model
selection the quantity $\bE(\vgamma|\vy)$ is sometimes referred to as the
posterior (variable) inclusion probability (PIP) vector.

Ignoring for the moment the problems associated with specifying
$p(\alpha,\vbeta_{-\vgamma},\vgamma)$, all of the above quantities are
conceptually straightforward. In practice the computation of the quantities
$p(\vgamma|\vy)$, $p(\vbeta|\vy)$ and $\bE(\vgamma|\vy)$ are only feasible for
small values of $p$ (say around $p=30$). For large values of $p$ we need to
pursue alternatives to exact inference.



\section{Prior specification for linear model parameters}
\label{sec:model}

We will specify the prior $p(\alpha,\vbeta,\sigma^2|\vgamma)$ as follows
\begin{equation}
	\label{eq:priorStructure}
	\begin{array}{c}
		\ds p(\alpha) \propto 1,  
		\qquad 
		\vbeta_\vgamma | \sigma^2, g, \vgamma \sim \N_p(\vzero, g \sigma^2 (\mX_\vgamma^T \mX_\vgamma)^{-1}),
		\quad \text{ and }  \quad 
		\ds p(\sigma^2) \propto (\sigma^2)^{-1},                      
	\end{array}
\end{equation} 

\noindent where we have introduced a new prior hyperparameter $g$.  For the
time being we will defer specification of $p(g)$ and $p(\vgamma)$.  We will now
justify each element of the above prior structure.

The priors on $\alpha$ and $\sigma^2$ are improper Jeffreys priors and have
been justified in \cite{Berger1998}. In the context of Bayesian model
selection, model averaging or hypothesis testing $\alpha$ and $\sigma^2$ appear
in all models so that when comparing models the proportionality constants in
the corresponding Bayes factors cancel.

The prior on $\vbeta_\vgamma$ is Zellner's $g$-prior \citep[see for
example,][]{Zellner1986} with prior hyperparameter $g$. This family of priors
for a Gaussian regression model where the prior covariance matrix of
$\vbeta_\vgamma$ is taken to be a multiple of $g$ with the Fisher information
matrix for $\vbeta$.  This places the most prior mass for $\vbeta_\vgamma$ on
the section of the parameter space where the data is least informative, and
makes the marginal likelihood of the model scale-invariant. Furthermore, this
choice of prior removes a log-determinant of $\mX_\vgamma^T\mX_\vgamma$ term
from the expression for the marginal likelihood, which is an additional
computational burden to calculate.  The prior on $\vbeta_\vgamma$ combined with
the prior on $\vbeta_{-\vgamma}$ in (\ref{eq:priorStructure}) constitutes one
variant of the spike and slab prior for $\vbeta$.

An alternative choice of prior on $\vbeta_\vgamma$ was proposed by
\cite{Maruyama2011}. Let $p_{\vgamma} = |\vgamma|$, the number of non-zero
elements in $\vgamma$. We will now describe their prior on $\vbeta_\vgamma$ for
the case where for the case $p_{\vgamma} < n - 1$. Let $\mU\mLambda\mU^T$ be an
eigenvalue decomposition of $\mX_\vgamma^T\mX_\vgamma$ where $\mU$ is an
orthonormal $p_{\vgamma} \times p_{\vgamma}$ matrix, and $\mLambda =
\mbox{diag}(\lambda_1,\ldots,\lambda_{p_{\vgamma}})$ is a diagonal matrix of
eigenvalues with $\lambda_1\ge\ldots,\ge \lambda_{p_{\vgamma}}>0$. Then
\cite{Maruyama2011} propose a prior for $\vbeta_\vgamma$ of the form
\begin{equation} \label{eq:priorBetaMG} \vbeta_\vgamma | \sigma^2, g \sim
\N(\vzero, \sigma^2 (\mU\mW\mU^\top)^{-1}),   \end{equation} 

\noindent where $\mW = \mbox{diag}(w_1,\ldots,w_{p_{\vgamma}})$ with $ w_j =
\lambda_j/[\nu_j(1 + g) - 1]$ for some prior hyperparameters $\nu_q < \ldots <
\nu_1$. \cite{Maruyama2011} suggest as a default choice for the $\nu_j$'s to
use $\nu_j = \lambda_j/\lambda_{p_{\vgamma}}$, for $1\le j \le p_{\vgamma}$.
This choice down-weights the prior on the rotated parameter space of $(\mU
\vbeta)_j$ when the corresponding eigenvalue $\lambda_j$ is large, which leads
to prior standard errors that are approximately the same size. Note that when
$\nu_1 = \ldots = \nu_{p_{\vgamma}} = 1$ the prior (\ref{eq:priorBetaMG})
reduces to the prior for $\vbeta$ in (\ref{eq:priorStructure}). 

The choice between (\ref{eq:priorBetaMG}) and the prior for $\vbeta$ in
(\ref{eq:priorStructure}) represents a trade-off over computational efficiency
and desirable statistical properties. We choose (\ref{eq:priorStructure})
because it avoids the computational burden of calculating an eigenvalue or a
singular value decomposition of a $p_{\vgamma}\times p_{\vgamma}$ matrix for
every model considered, which typically can be computed in $O(p_{\vgamma}^3)$
floating point operations.  It also means that we can exploit efficient matrix
updates to traverse the entire model space in a computationally efficient
manner allowing this to be done feasibly when $p$ is less than around 30 on a
standard 2017 laptop (see Section \ref{sec:implementation} for details).


The marginal likelihood for the model  (\ref{eq:linearModel}) and under prior
structure (\ref{eq:priorStructure}). 
%Integrating out $\alpha$ and $\vbeta$ from $p(\vy,\alpha,\vbeta|\sigma^2,g)$
%we find \begin{equation}\label{eq:yGivenSigma2andG} \begin{array}{rl} \ds
%p(\vy|\sigma^2,g) & \ds = \int \exp\left[ - \tfrac{n}{2}\log(2\pi\sigma^2) -
%\tfrac{1}{2\sigma^2}\|\vy - \vone\alpha - \mX\vbeta\|^2 -
%\tfrac{p}{2}\log(2\pi g\sigma^2) + \tfrac{1}{2}\log|\mX^T\mX| -
%\tfrac{1}{2g\sigma^2}\vbeta^T\mX^T\mX\vbeta  \right] d\alpha d\vbeta \\ & \ds
%= \int \exp\left[ - \tfrac{n}{2}\log(2\pi\sigma^2) - \tfrac{n}{2\sigma^2} -
%\tfrac{n\alpha^2}{2\sigma^2} + \sigma^{-2}\vy^T\mX\vbeta -
%\tfrac{1}{2\sigma^2}(1 + g^{-1})\vbeta^T\mX^T\mX\vbeta - \tfrac{p}{2}\log(2\pi
%g\sigma^2) + \tfrac{1}{2}\log|\mX^T\mX| \right] d\alpha d\vbeta \\ & \ds =
%\int \exp\left[ - \tfrac{n-1}{2}\log(2\pi\sigma^2) - \tfrac{1}{2}\log(n) -
%\tfrac{n}{2\sigma^2} + \sigma^{-2}\vy^T\mX\vbeta - \tfrac{1}{2\sigma^2}(1 +
%g^{-1})\vbeta^T\mX^T\mX\vbeta - \tfrac{p}{2}\log(2\pi g\sigma^2) +
%\tfrac{1}{2}\log|\mX^T\mX| \right]  d\vbeta \\ & \ds = \exp\left[ -
%\tfrac{n-1}{2}\log(2\pi\sigma^2) - \tfrac{1}{2}\log(n) - \tfrac{p}{2}\log(1 +
%g) - \tfrac{n}{2 \sigma^2} \left( 1 - \tfrac{g}{1 + g} R^2 \right)  \right],
%\end{array} \end{equation}
%
%\noindent \joc{ Derivation of the above expression uses the identity $ \int
%\exp\left\{ -\tfrac{1}{2}\vx^T\mA\vx + \vb^T\vx \right\} d \vx =
%|2\pi\mSigma|^{1/2} \exp\left\{ \tfrac{1}{2}\vmu^T\mSigma^{-1}\vmu \right\} $
%where $\vmu = \mA^{-1}\vb$, and $\mSigma = \mA^{-1}$.  It also uses the
%identities: $|c\mA| = c^d|\mA|$ and $|\mA^{-1}| = |\mA|^{-1}$ when $\mA
%\in\R^{d\times d}$.  }
Integrating out $\alpha$, $\vbeta$, and $\sigma^2$ from
$p(\vy,\alpha,\vbeta,\sigma^2|g,\vgamma)$ we obtain
\begin{equation}\label{eq:yGivenG} \begin{array}{rl} \ds p(\vy|g,\vgamma)
        %& \ds = \int \exp\left[ - \tfrac{n-1}{2}\log(2\pi) -
    %\tfrac{1}{2}\log(n) - \tfrac{p}{2}\log(1 + g) - \left( \tfrac{n-1}{2} +
    %1\right)\log(\sigma^2) - \left( \tfrac{n}{2} \tfrac{1 + g(1-R^2)}{1 + g}
    %\right)\sigma^{-2} \right]  d\sigma^2 \\
        %& 
        \ds = K(n) (1 + g)^{(n - p_\vgamma - 1)/2}(1 + g (1 -
    R_\vgamma^2))^{-(n-1)/2}, \end{array} \end{equation}

\noindent where $K(n) = [\Gamma( (n-1)/2 )]/[\sqrt{n}(n\pi)^{(n-1)/2}]$, and
$R_\vgamma^2 =
\vy^T\mX_\vgamma^T(\mX_\vgamma^T\mX_\vgamma)^{-1}\mX_\vgamma^T\vy/n$ is the the
usual R-squared statistic for model $\vgamma$.  This is the same expression as
\cite{Liang2008} Equation (5) after simplification. Note that when $\vgamma =
\vzero$, i.e., the null model, then $p_\vgamma = 0$, and $R_\vgamma^2 = 0$
leading to the simplification $p(\vy|g,\vzero) = K(n)$ for all $g$. Hence,
$p(\vy|\vzero) = K(n)$ provided the hyperprior for $g$ is a proper density. We
will now discuss the specification of $g$.


\section{Hyperpriors on $g$}
\label{sec:hyperpriors}

Here we outline some of the choices of hyperpriors for $g$ used in the
literature, their properties, and where possible how to implement these in an
efficient, accurate, and numerically stable manner. We cover the the hyper-$g$
and hyper-$g/n$ priors of \cite{Liang2008}, the beta-prime prior of
\cite{Maruyama2011}, the robust prior of \cite{Bayarri2012}, and the Cake prior
of \cite{OrmerodEtal2017}.  We also considered the prior structure implied by
\cite{Zellner1980}, but were able to make no meaningful progress on existing
methodology for this case.

We show that many of the hyperpriors on $g$ result in Bayes factors which can
be expressed in terms of the Gaussian hypergeometric function denoted
${}_2F_1(\,\cdot\,,\,\cdot\,;\,\cdot\,;\,\cdot\,)$ \citep[see for example
Chapter 15 of ][]{Abramowitz1972}.  The Gaussian hypergeometric function is
notoriously prone to overflow and numerical instability \citep{Pearson2017}.
When such numerical issues arise \cite{Liang2008} derive a Laplace
approximation to ${}_2F_1$ implemented in the {\tt R} package {\tt BAS}.  Key
to achieving accuracy, efficiency and numerical stability for several different
mixture $g$-priors is the following result.

 
\noindent 
{\bf Result 1:} {\it For $x\in(0,1)$, $c>1$, and $b +1 > c$ we have}
\begin{equation}\label{eq:logGuassHypergeometric2}
	\ds {}_2F_1(a+b,1;a+1;x) = \frac{a}{x(1 - x)}   \frac{\mbox{pbeta}(x,a,b)}{\mbox{dbeta}(x,a,b)},
\end{equation}

\noindent 
{\it where} $\mbox{pbeta}(x;a,b)$ {\it and} $\mbox{dbeta}(x;a,b)$ {\it are the cdf and pdf of the beta 
	distribution respectively.}

 
\noindent 
{\bf Proof:} Using identity 2.5.23 of \cite{Abramowitz1972} the cdf of the beta distribution
can be written as
\begin{equation*}
\mbox{pbeta}(x;a,b) = \frac{x^a}{a\mbox{Beta}(a,b)} \cdot {}_2F_1(a,1-b;a+1;x) 
\end{equation*}

\noindent where 
$\mbox{Beta}(a,b)$ is the beta function.
Using the Euler transformation
${}_2 F_1(a,b;c,x) = (1 - x)^{c-a-b} {}_2 F_1(c-a,c-b;c,x)$,
and the fact that ${}_2 F_1(a,b;c,x)={}_2 F_1(b,a;c,x)$,  we obtain
$$
\mbox{pbeta}(x;a,b) = \frac{x^a(1 - x)^{b}}{a\mbox{Beta}(a,b)} \cdot {}_2F_1(a+b,1;a+1;x). 
$$

\noindent Lastly, after rearranging we obtain Result 1.
\vspace{-0.5cm}\begin{flushright}$\Box$\end{flushright}
%$$
%{}_2F_1(a+b,1;a+1;x)  = \frac{\mbox{pbeta}(x;a,b)a\mbox{Beta}(a,b)}{x^a(1 - x)^b} = \frac{a}{x(1-x)}\frac{\mbox{pbeta}(x;a,b)}{\mbox{dbeta}(x;a,b)}
%$$

\noindent Numerical overflow can be avoided since standard libraries exist for
evaluating $\mbox{pbeta}(x,a,b)$ and $\mbox{dbeta}(x,a,b)$ on the log scale.
Recently, \cite{Nadarajah2015} stated an equivalent result originally derived
in \cite{PrudnikovEtal1986}. 

\subsection{The hyper-$g$ prior}

\noindent Initially, \cite{Liang2008} suggest the hyper $g$-prior where
\begin{equation}\label{eq:hyperG}
	\ds p_{g}(g) = \frac{a - 2}{2}(1 + g)^{-a/2},
\end{equation}

\noindent for $a>2$ and $g>0$. Combining (\ref{eq:yGivenG}) with
(\ref{eq:hyperG}), we have
\begin{equation}\label{eq:hyperGmarginalIntegral}
	p_{g}(\vy|\vgamma) = K(n) \frac{a - 2}{2}  \int_0^\infty 
	\left( 1 + g \right)^{-a/2}
	(1 + g)^{(n-p_\vgamma-1)/2} \left[ 1 + g (1 - R_\vgamma^2) \right]^{-(n-1)/2}  dg.
\end{equation}

\noindent After applying 3.197(5) of \cite{Gradshteyn2007}, i.e.,
\begin{equation}\label{eq:31975}
	\ds 
	\int_0^\infty x^{\lambda - 1}(1 + x)^\nu (1 + \alpha x)^\mu dx
	=\mbox{Beta}(\lambda,-\mu-\nu-\lambda){}_2F_1(-\mu,\lambda;-\mu-\nu; 1 - \alpha),
\end{equation}
\noindent (which holds provided $-(\mu  + \nu) > \lambda > 0$), leads to
\begin{equation}\label{eq:hyperGmarginal}
	\ds \mbox{BF}_{g}(\vgamma) = \frac{p_{g}(\vy|\vgamma)}{p_{g}(\vy|\vzero)} =  \left( \frac{a - 2}{p_\vgamma + a - 2} \right) \cdot {}_2F_1\left( \frac{n-1}{2}, 1; \frac{p_\vgamma + a}{2}; R_\vgamma^2 \right).
\end{equation}

\noindent Using Result 1 the Bayes factor under the hyper-$g$ prior can be
written as
\begin{equation}\label{eq:hyperGmarginal2}
	\ds \mbox{BF}_{g}(\vgamma) 
	=  
	\frac{a - 2}{2 R_\vgamma^2(1 - R_\vgamma^2)} 
	\frac{\mbox{pbeta}\left(R_\vgamma^2,\tfrac{p_\vgamma + a - 2}{2},\tfrac{n-p_\vgamma - a+1}{2}\right)}{
		\mbox{dbeta}\left(R_\vgamma^2,\tfrac{p_\vgamma + a - 2}{2},\tfrac{n-p_\vgamma - a+1}{2}\right)}.
\end{equation}

\noindent Unfortunately, \cite{Liang2008} also showed that
(\ref{eq:hyperGmarginal}) is not model selection consistent when the true model
is the null model (the model only containing the intercept) and so alternative
hyperpriors for $g$ should be used.

\subsection{The hyper-$g/n$ prior}

Given the problems with the hyper-$g$ prior, \cite{Liang2008} proposed a
modified variant of the hyper-$g$ prior which uses
\begin{equation}\label{eq:hyperGonN}
	\ds p_{g/n}(g) = \frac{a - 2}{2n}\left( 1 + \frac{g}{n} \right)^{-a/2},
\end{equation}

\noindent which they call the hyper-$g/n$ prior where again $a>2$ and $g>0$.
They show that this prior leads to model selection consistency.  Combining
(\ref{eq:yGivenG}) with (\ref{eq:hyperGonN}), and using the transform $g = u/(1
- x)$, the quantity $p(\vy|\vgamma)$ can be expressed as the integral
\begin{equation}\label{eq:hyperGonNmarginalIntegral}
	\begin{array}{rl}
		p_{g/n}(\vy|\vgamma) 
		%& \ds 
		%= K(n) \frac{a - 2}{2n}  \int_0^\infty 
		%\left( 1 + \frac{g}{n} \right)^{-a/2}
		%(1 + g)^{(n-p_\vgamma-1)/2} \left[ 1 + g (1 - R_\vgamma^2) \right]^{-(n-1)/2}  dg
		%\\ [2ex]
		& \ds = K(n) \frac{a - 2}{2n}  \int_0^1 
		(1 - u)^{p/2 + a/2 - 2  } \left(  1 - u \left(1  -  \tfrac{1}{n} \right) \right)^{-a/2} \left(  1 - u R^2\right)^{-(n-1)/2} du.
	\end{array} 
\end{equation}

\noindent  Employing Equation 3.211 of \cite{Gradshteyn2007}, i.e.,
$$
\int_0^1 x^{\lambda-1}(1 - x)^{\mu - 1}(1 - u x)^{-\delta}(1 - vx)^{-\sigma} dx = \mbox{Beta}(\mu,\lambda) F_1(\lambda,\delta,\sigma,\lambda+\mu;u,v) 
%F_1(a,b_1,b_2,c; x,y) = \frac{\Gamma(c)} {\Gamma(a)\Gamma(c-a)} 
%\int_0^1 t^{a-1} (1-t)^{c-a-1} (1-xt)^{-b_1} (1-yt)^{-b_2} \, dt,
$$

\noindent provided $\lambda>0$ and $\mu>0$ where $F_1$ is the Appell
hypergeometric function in two variables \citep{Weisstein2009} leads to
\begin{equation}\label{eq:hyperGonNmarginal}
	\ds \mbox{BF}_{g/n}(\vgamma) =  \frac{a - 2}{n(p_\vgamma + a - 2)} F_1\left( 1, \frac{a}{2}, \frac{n-1}{2}; \frac{p_\vgamma + a}{2}; 1  -  \frac{1}{n}, R_\vgamma^2 \right),
\end{equation}

\noindent which is to our knowledge a new expression for the Bayes factor under
the hyper $g/n$-prior.


Unfortunately, the expression (\ref{eq:hyperGonNmarginal}) is extremely
difficult to evaluate numerically since the second last argument of the above
$F_1$ is asymptotically close to the radius of convergence of the $F_1$
function.  \cite{Liang2008} again suggest Laplace approximation for this choice
of prior. We now derive an alternative approximation.  Using the fact that
$$
F_1(1,b_1,b_2,c; 1,y) 
= (c - 1)
\int_0^1  (1-t)^{c-b_1-2} (1-yt)^{-b_2} \, dt
= (c - 1) \frac{\, _2F_1(1,b_2;c-b_1;y)}{c-b_1-1}
$$

\noindent and the approximation $F_1(1,b_1,b_2,c; 1-1/n,y)  \approx
F_1(1,b_1,b_2,c; 1,y)$ (which should be reasonable for large $n$), for
$p_\vgamma > 2$ we obtain
\begin{equation}\label{eq:hyperGonNmarginalApprox}
	%\begin{array}{rl}
	\ds \mbox{BF}_{g/n}(\vgamma) 
	%& \ds =  \frac{a - 2}{n(p_\vgamma - 2)} 
	%\, _2F_1\left( \frac{n-1}{2}, 1;  \frac{p_\vgamma}{2}; R_\vgamma^2 \right)
	%\\
	%& \ds 
	\approx    
	\frac{a - 2}{2n R_\vgamma^2(1 - R_\vgamma^2)}   \frac{
		\mbox{pbeta}\left( R_\vgamma^2, \frac{p_\vgamma-2}{2}, \frac{n-p_\vgamma+1}{2} \right)
	}{
		\mbox{dbeta}\left( R_\vgamma^2, \frac{p_\vgamma-2}{2}, \frac{n-p_\vgamma+1}{2} \right)
	}.
	%\end{array}
\end{equation}

\noindent For the cases where $p\in \{1,2\}$ we will use numerical quadrature.
When $p=0$, we also have that $R_\vgamma^2= 0$ so $\mbox{BF}_{g/n}(\vgamma) =
1$.  Figure \ref{fig:gonnapprox} illustrates the differences between ``exact''
values of the $\mbox{BF}_{g/n}$ (obtained using numerical quadrature) as a
function of $n$, $p_\vgamma$, and $R^2$. From this figure we see that the
approximation has a good relative error except for values close to 1 when the
approximation overestimates the true value of the log Bayes factor. We found
numerical quadrature to be more reliable than using
(\ref{eq:hyperGonNmarginal}) evaluated using the {\tt appell} function in the
package {\tt Appell}.

\begin{figure}
	\centering
	\includegraphics[width=0.9\linewidth]{gOnNapprox}
	\caption{On the left side panels are plotted the values of log of $\mbox{BF}_{g/n}$ (light versions of the
		colours) and their corresponding approximation (dark version of the colours) 
		as a function of $n$, $p$ over a the range $R^2\in(0,0.999)$. Right side panels display
		the exact values of log of $\mbox{BF}_{g/n}$ minus the corresponding approximations.}
	\label{fig:gonnapprox}
\end{figure}


\subsection{Robust prior}  

\noindent Next we will consider the robust hyperprior for $g$ as proposed by
\cite{Bayarri2012} designed to have several nice theoretical properties
outlined there. Using the default parameter choices the hyperprior for $g$ used
by \cite{Bayarri2012} corresponds to:
\begin{equation}\label{eq:robustPrior}
	p_{{rob}}(g) = \tfrac{1}{2}r^{1/2} (1 + g)^{-3/2},
\end{equation}

\noindent for $g>L$  where $L = r - 1$ and $r = (1 + n)/(1 + p_\vgamma)$.
Combining (\ref{eq:yGivenG}) with (\ref{eq:robustPrior}) leads to an expression
for $p(\vy|\vgamma)$ of the form
\begin{equation}\label{eq:marginalLikelihoodRobust}
	\ds p_{rob}(\vy|\vgamma)
	\ds = K(n) \tfrac{1}{2} r^{1/2} 
	\int_L^\infty  (1 + g)^{(n - p_\vgamma)/2 - 2}(  1 + g \widehat{\sigma}_\vgamma^2)^{-(n-1)/2} dg,
\end{equation}

\noindent where $\widehat{\sigma}_\vgamma^2 = 1 - R_\vgamma^2$ is the MLE for
$\sigma^2$ for model (\ref{eq:linearModel}) when $\mX$ is replaced with
$\mX_\vgamma$ under the standardization described in Section 2.  Using the
substitution $x = r/(g - L)$ and some minor algebraic manipulation leads to
%$$
%x = r/(g - L),
%\quad 
%g = L + r/x
%\quad 
%d g = -r/x^2 dx
%\quad 
%\lim_{g\to L_+} x = \infty
%\quad 
%\lim_{g\to \infty} x = 0
%$$
%
$$
%\begin{array}{rl}
\ds \mbox{BF}_{{rob}}(\vgamma)
%& \ds = \tfrac{1}{2} r^{3/2} 
%\int_0^\infty x^{-2}   (r + r/x)^{(n - p_\vgamma - %4)/2}(  1 + \widehat{\sigma}_\vgamma^2  (L + %r/x))^{-(n-1)/2}  dx
%\\ 
%& 
\ds = \tfrac{1}{2} r^{ - p_\vgamma/2} (\widehat{\sigma}_\vgamma^2)^{-(n-1)/2}
\int_0^\infty x^{(p_\vgamma-1)/2} 
(1 + x)^{(n - p_\vgamma - 4)/2}
\left( 1 + \tfrac{(1 + \widehat{\sigma}_\vgamma^2 L)x}{(1 + L)\widehat{\sigma}_\vgamma^2}  \right)^{-(n-1)/2}  dx.
%\end{array} 
$$



\noindent 
Using Equation 3.197(5) of \cite{Gradshteyn2007}, i.e. (\ref{eq:31975}), 
%\begin{equation}\label{eq:31975}
%\ds 
%\int_0^\infty x^{\lambda - 1}(1 + x)^\nu (1 + \alpha x)^\mu dx
%=\mbox{Beta}(\lambda,-\mu-\nu-\lambda){}_2F_1(-\mu,\lambda;-\mu-\nu; 1 - \alpha),
%\end{equation}
%\noindent (which holds provided $-(\mu  + \nu) > \lambda > 0$).
%More specifically we use 
with the mappings
$$
\lambda \leftrightarrow \frac{p_\vgamma+1}{2},
\quad 
\nu \leftrightarrow \frac{n - p_\vgamma - 4}{2},
\quad 
\alpha \leftrightarrow \frac{(1 + \widehat{\sigma}_\vgamma^2 L)}{(1 + L)\widehat{\sigma}_\vgamma^2},
\quad \mbox{and} \quad 
\mu \leftrightarrow -\frac{n-1}{2},
$$

\noindent the conditions required by (\ref{eq:31975}) are satisfied provided
$\alpha \in (-1,1)$ (which is a relatively restrictive condition). 
%Checking the conditions we have $\lambda > 0$ since $p_\vgamma \ge 0$.
%The second condition implies 
%$$
%\begin{array}{l}
%\ds -(\mu  + \nu) > \lambda
%\\
%\ds \qquad \Rightarrow \qquad \frac{n-1}{2} - \frac{n - p_\vgamma - 4}{2} > \frac{p_\vgamma+1}{2}
%\\
%\ds \qquad \Rightarrow \qquad \frac{p_\vgamma + 3}{2}   > \frac{p_\vgamma+1}{2}
%
%\end{array} 
%$$
%
%\noindent which always holds. 
This leads to
\begin{equation}\label{eq:yGivenGammaRobust}
	%\begin{array}{rl}
	\ds \mbox{BF}_{{rob}}(\vgamma)
	%& \ds = \tfrac{1}{2} r^{ - p_\vgamma/2} (\widehat{\sigma}_\vgamma^2)^{-(n-1)/2}
	%\mbox{Beta}\left( \frac{p_\vgamma+1}{2}, 1 \right)
	%{}_2F_1\left(\frac{n-1}{2},\frac{p_\vgamma+1}{2};\frac{p_\vgamma + 3}{2}; 
	%\frac{(1 + L)\widehat{\sigma}_\vgamma^2  - (1 + \widehat{\sigma}_\vgamma^2 L)}{r\widehat{\sigma}_\vgamma^2} \right)
	%\\
	%& \ds 
	= \left( \tfrac{n + 1}{ p_\vgamma + 1} \right)^{ - p_\vgamma/2} \tfrac{(\widehat{\sigma}_\vgamma^2)^{-(n-1)/2}}{p_\vgamma+1}
	{}_2F_1\left( \tfrac{n-1}{2}, \tfrac{p_\vgamma+1}{2}; \tfrac{p_\vgamma+3}{2}  ; 
	\tfrac{(1  - 1/\widehat{\sigma}_\vgamma^2)(p_\vgamma + 1)}{1 + n}  \right),
	%\end{array} 
\end{equation}


\noindent which is the same expression as Equation 26 of \cite{Bayarri2012}
modulo notation.

The expression (\ref{eq:yGivenGammaRobust}) is difficult to deal with
numerically for two reasons. Firstly, if either of the first two arguments of
the ${}_2F_1$ function are large relative to the third this will often lead to
numerical overflow problems. Secondly, and more problematically, when
$\widehat{\sigma}_\vgamma^2$ becomes small the last argument of ${}_2F_1$
function can become less than $-1$ which falls outside the radius of
convergence of the ${}_2F_1$ function. The {\tt BayesVarSel} package which
implements this choice of prior deals with these problems using numerical
quadrature.

Instead suppose we begin with the substitution $x = g - L$ which after minor
algebraic manipulation leads to
$$
%\begin{array}{rl}
\ds \mbox{BF}_{{rob}}(\vgamma)
%& \ds = \tfrac{1}{2} r^{1/2} 
%\int_0^\infty  (1 + L + x)^{(n - p_\vgamma - 4)/2}(  1 + \widehat{\sigma}_\vgamma^2(x + L) )^{-(n-1)/2} dx,
%\\
%& \ds 
= \tfrac{1}{2} r^{1/2} \left( \widehat{\sigma}_\vgamma^2\right)^{-(n-1)/2} 
\int_0^\infty  (r + x)^{(n - p_\vgamma-4)/2}
\left(  \tfrac{1 +  \widehat{\sigma}_\vgamma^2L}{\widehat{\sigma}_\vgamma^2} +  x \right)^{-(n-1)/2} dx.
%\end{array} 
$$

\noindent Employing Equation 3.197(1) of \cite{Gradshteyn2007}, i.e.,
$$
\int_0^\infty x^{\nu - 1}(\beta + x)^{-\mu}(x + \gamma)^{-\varrho} dx
= \beta^{-\mu}
\gamma^{\nu - \varrho} 
\mbox{Beta}(\nu,\mu - \nu + \varrho)
{}_2F_1(\mu,\nu;\mu+\varrho; 1 - \gamma/\beta),
$$

\noindent (which holds provided $\nu>0$, $\mu > \nu - \varrho$), with the
mappings
$$
\nu \leftrightarrow 1,
\quad 
\beta \leftrightarrow \frac{1 +  \widehat{\sigma}_\vgamma^2L}{\widehat{\sigma}_\vgamma^2},
\quad 
\mu \leftrightarrow (n-1)/2
\quad 
\gamma \leftrightarrow r
\quad \mbox{and} \quad 
\varrho \leftrightarrow -(n - p_\vgamma-4)/2,
$$

\noindent The conditions of the integral result easily hold.
%
%\noindent The condition $\nu>0$ holds. The condition $\mu > \nu - \varrho$
%requires $$ \begin{array}{l} (n-1)/2 > 1 + (n - p_\vgamma-4)/2 \quad
%\Rightarrow \quad p_\vgamma > - 1, \end{array}  $$ \noindent so that the
%second condition also holds.  Hence, $$ \begin{array}{rl} \ds
%p_{rob}(\vy|\vgamma) & \ds = K(n) \tfrac{1}{2} r^{1/2} \left(
%\widehat{\sigma}_\vgamma^2\right)^{-(n-1)/2} \left( \frac{1 +
%\widehat{\sigma}_\vgamma^2L}{\widehat{\sigma}_\vgamma^2} \right)^{-(n-1)/2}
%r^{1 + (n - p_\vgamma-4)/2} \\ & \ds \qquad \times \mbox{Beta}\left( 1,
%\frac{n-1}{2} - 1 - \frac{n - p_\vgamma - 4}{2} \right) {}_2F_1\left(
%\frac{n-1}{2}, 1; \frac{n-1}{2} - \frac{n - p_\vgamma - 4}{2}; 1 - \frac{1 +
%L}{\frac{1 +  \widehat{\sigma}_\vgamma^2L}{\widehat{\sigma}_\vgamma^2}}
%\right),
%
%\\ & \ds = K(n) \frac{1}{2} r^{(n - p_\vgamma-1)/2} \left(  1 +
%\widehat{\sigma}_\vgamma^2L  \right)^{-(n-1)/2} \mbox{Beta}\left( 1,
%\frac{p_\vgamma+1}{2} \right) {}_2F_1\left( \frac{n-1}{2}, 1; \frac{p_\vgamma
%+ 3}{2}; \frac{1  - \widehat{\sigma}_\vgamma^2}{1 +
%\widehat{\sigma}_\vgamma^2L} \right),
%
%\\
%
%& \ds = K(n) \frac{r^{(n - p_\vgamma-1)/2}}{1 + p_\vgamma} \left(  1 +
%\widehat{\sigma}_\vgamma^2L  \right)^{-(n-1)/2} {}_2F_1\left( \frac{n-1}{2},
%1; \frac{p_\vgamma + 3}{2}; \frac{1  - \widehat{\sigma}_\vgamma^2}{1 +
%\widehat{\sigma}_\vgamma^2L} \right),
%
%\end{array} $$ $$ \ds p_{{rob}}(\vy|\vgamma) \ds =
%\frac{K(n)}{2}\left(\frac{1+n}{1 + p_\vgamma}  \right)^{1/2}
%%(\widehat{\sigma}_\vgamma^2)^{-(n-1)/2} \int_0^\infty  (1 + L + h)^{(n -
%p_\vgamma)/2 - 2}\left[  \frac{1 +
    %L\widehat{\sigma}_\vgamma^2}{\widehat{\sigma}_\vgamma^2} + h
%\right]^{-(n-1)/2} dh.  $$
%
%\noindent 
Hence, after some algebraic manipulation and applying Result 1, and letting
$\widetilde{R}_\vgamma^2 = R_\vgamma^2/(1 + L\widehat{\sigma}_\vgamma^2)$ we
obtain
\begin{equation}\label{eq:yGivenGammaRobust2}
	%\begin{array}{rl}
	\ds \mbox{BF}_{{rob}}(\vgamma)
	%& \ds = \left( \frac{1 + n}{1 + p_\vgamma} \right)^{(n - p_\vgamma - 1)/2} \frac{\left( 1 + L\widehat{\sigma}_\vgamma^2 \right)^{-(n - 1)/2}}{1 + p_\vgamma}
	%{}_2F_1\left(  
	%\frac{n-1}{2}, 1; \frac{p_\vgamma+3}{2}; \frac{1 - \widehat{\sigma}_\vgamma^2}{1 + L\widehat{\sigma}_\vgamma^2}
	% \right)
	%\\
	%& \ds 
	= \left( \frac{1 + n}{1 + p_\vgamma} \right)^{(n - p_\vgamma - 1)/2} \frac{\left( 1 + L\widehat{\sigma}_\vgamma^2 \right)^{-(n - 1)/2}}{2 \widetilde{R}_\vgamma^2(1 - \widetilde{R}_\vgamma^2)} 
	\frac{
		\mbox{pbeta}\left( 
		\widetilde{R}_\vgamma^2,
		\frac{p_\vgamma +1}{2},
		\frac{n - p_\vgamma - 2}{2} 
		\right)
	}{
		\mbox{dbeta}\left( 
		\widetilde{R}_\vgamma^2,
		\frac{p_\vgamma +1}{2},
		\frac{n - p_\vgamma - 2}{2} 
		\right)
	}.
	%\end{array}
\end{equation}
%Then
%$(1 + L\widehat{\sigma}_\vgamma^2) = %R_\vgamma^2/\widetilde{R}_\vgamma^2$.

\noindent This expression is numerically far easier to evaluate efficiently and
accurately in a numerically stable manner. Due to simplifications we have $0\le
\widehat{\sigma}_\vgamma^2<1$, we also have $L>0$ so that the last argument of
the ${}_2F_1$ above is bounded in the unit interval.  
%$$
%\int_0^\infty x^{\nu - 1}(\beta + x)^{-\mu}(x + \gamma)^{-\varrho} dx
%= \beta^{-\mu}\gamma^{\nu - %\varrho}\mbox{Beta}(\nu,\mu-\nu + \varrho)
%{}_2F_1(\mu,\nu;\mu + \varrho; 1 - %\gamma/\beta)
%$$

%$$
%\beta = \frac{1 + %L\widehat{\sigma}_\vgamma^2}{\widehat{\sigma}_\vgamma^2}
%$$
%$$
%\gamma = 1 + L
%$$
%$$
%\nu = 1
%$$
%$$
%\mu = \frac{n-1}{2}
%$$
%$$
%\varrho = - (n - p_\vgamma - 4)/2
%$$

\subsection{Beta-prime prior} 

\noindent Next we will consider the prior 
\begin{equation}\label{eq:betaPrime}
	\ds p_{bp}(g) = \frac{g^{b}(1 + g)^{-(a+b+2)}}{\mbox{Beta}(a+1,b+1)},
\end{equation}

\noindent proposed by \cite{Maruyama2011} where $g>0$, $a>-1$ and $b>-1$.  This
is a Pearson Type VI or beta-prime distribution. More specifically, $g\sim
\mbox{Beta-prime}(b+1,a+1)$ using the usual parametrization of the beta-prime
distribution \citep{Johnson1995}.  Then combining (\ref{eq:yGivenG}) with
(\ref{eq:betaPrime}) the quantity $p(\vy|\vgamma)$ can be expressed as the
integral
$$
%\begin{array}{rl}
\ds p_{bp}(\vy|\vgamma) 
%& \ds = \int_0^\infty                                         
%\frac{g^{b}(1 + g)^{-a-b-2}}{\mbox{Beta}(a+1,b+1)}
%K(n) (1 + g)^{(n - p - 1)/2}\left[ 1 + g(1-R^2) \right]^{-(n-1)/2}
%dg
%\\
%& \ds 
=
\frac{K(n)}{\mbox{Beta}(a+1,b+1)}
\int_0^\infty             
g^{b}(1 + g)^{(n - p_\vgamma - 1)/2 - (a + b + 2)}  (1 + g (1-R_\vgamma^2) )^{-(n-1)/2}  
dg.
%\end{array}
$$

\noindent If we choose 
%$b$ such that $a+b+2 = (n - p_\vgamma - 1)/2$, implying
$b = (n - p_\vgamma - 5)/2 - a$, then the exponent of the $(1 + g)$ term in the
equation above is zero.  Using Equation 3.194 (iii) of \cite{Gradshteyn2007},
i.e.,
$$
\int_0^\infty \frac{ x^{\mu - 1} }{(1 + \beta x)^\nu} dx = \beta^{-\mu} \mbox{Beta}(\mu,\nu - \mu),
$$

\noindent provided $\mu,\nu>0$ and $\nu>\mu$, we obtain
\begin{equation}\label{eq:marginalLikelihoodBetaPrime}
	\begin{array}{rl}
		\ds \mbox{BF}_{bp}(\vy|\vgamma) 
		%& \ds =
		%\frac{K(n)}{\mbox{Beta}(a+1,b+1)}
		%\int_0^\infty g^{b} \left[ 1 + g(1-R^2) \right]^{-(n-1)/2}  
		%dg
		%\\ [2ex]
		& \ds 
		=   
		\frac{\mbox{Beta}(p/2 + a + 1,b + 1)}{\mbox{Beta}(a+1,b+1)} (1-R_\vgamma^2)^{-(b + 1)}
		%\\ [2ex]
		%& \ds = \widetilde{K}(n,a)
		%
		%\Gamma(p/2 + a + 1)\Gamma(a + b + 2)
		%(\widehat{\sigma}^2)^{-(b + 1)}
	\end{array}
\end{equation}

\noindent which is a simplification of the Bayes factor proposed by
\cite{Maruyama2011}.


Note that (\ref{eq:marginalLikelihoodBetaPrime}) is proportional to a special
case of the prior structure considered by \cite{Maruyama2011} who refer to this
as a model selection criterion (after Zellner's $g$ prior). This choice of $b$
also ensures that $g = O(n)$ so that $\mbox{tr}\{\mbox{Var}(\vbeta | g,
\sigma^2)\} = O(1)$, preventing Bartlett's paradox. 
% Note that in comparison to previously discussed priors marginal likelihood
% only involves gamma functions which are well behaved from a numerical
% analysis perspective. 
Note that in comparison to the priors we have previously discussed, this choice
of prior yields a marginal likelihood that can be expressed entirely with gamma
functions, which are well-behaved numerically.  \cite{Maruyama2011} showed the
prior (\ref{eq:betaPrime}) leads to model selection consistency.  For
derivation of the above properties and further discussion see
\cite{Maruyama2011}.

\subsection{BIC via Cake priors}  

\noindent
% \cite{OrmerodEtal2017} develops Cake priors which allow for arbitrarily
% diffuse priors while avoiding Bartlett's paradox leading Bayes factors equal
% to the exponential of minus half the BIC.
\cite{OrmerodEtal2017} developed the Cake prior, which allows arbitrarily
diffuse priors while avoiding Bartlett's paradox.  Cake priors can be thought
of as a Jefferys prior in the limit as the prior becomes increasingly diffuse
and enjoy nice theoretical properties including model selection consistency.
\cite{OrmerodEtal2017} departs from the prior structure
(\ref{eq:priorStructure}) and instead uses
\begin{equation}\label{eq:proirs2}
	\ds \alpha|\sigma^2,g \sim \N(0,g\sigma^2), \quad 
	\ds \vbeta_\vgamma|\sigma^2,g \sim N\left( \vzero,g\sigma^2\left( \tfrac{1}{n}\mX_\vgamma^T\mX_\vgamma\right)^{-1}\right)
	\quad \mbox{and} \quad
	p(g|\vgamma_j) = \delta(g; h^{1/(1 + p_\vgamma)})
\end{equation}

\noindent where $h$ is a common prior hyperparameter for all models. After
marginalizing out $\alpha$, $\vbeta$, $\sigma^2$ and $g$ the null based Bayes
factor for model $\vgamma$ is of the form
$$
\begin{array}{rl}
\ds \log\mbox{BF}(\vgamma;h)
=
-\tfrac{n}{2}\log\left( 1 - \tfrac{h^{1/(1+p_\vgamma)}}{1+h^{1/(1+p_\vgamma)}} R_\vgamma^2 \right) 
- \tfrac{p_\vgamma}{2}\log\left(n + h^{-1/(1+p_\vgamma)} \right).
\end{array}
$$

\noindent Taking $h\to\infty$ we obtain a null based Bayes factor of
\begin{equation}\label{eq:marginalLikelihoodCake}
	\ds \mbox{BF}(\vgamma)
	=
	\exp\left[ \,
	-\tfrac{n}{2}\log\left( 1 - R_\vgamma^2 \right) 
	- \tfrac{p_\vgamma}{2}\log\left(n \right) \,
	\right] = \exp\left[ \, -\tfrac{1}{2}\mbox{BIC}(\vgamma) \,\right]
\end{equation}

\noindent where $\mbox{BIC}(\vgamma) = n\log\left( 1 - R_\vgamma^2 \right) +
p_\vgamma \log(n)$. 
%Note that as $h\to\infty$ the parameter %posteriors become
%$$\alpha|\vy,\vgamma \sim %t_n(0,\widehat{\sigma}_{\vgamma}^2/n), \quad
%\vbeta_{\vgamma}|\vy,\vgamma \sim t_n( %\widehat{\vbeta}_{\vgamma},
%\widehat{\sigma}_{\vgamma}^2 \left(\mX_\vgamma^T\mX_\vgamma  \right)^{-1} ),
%\quad \mbox{and} \quad  \sigma^2|\vy,\vgamma \sim \mbox{IG}\left(
%\tfrac{n}{2}, \tfrac{n}{2}\widehat{\sigma}_{\vgamma}^2 \right), $$

%\noindent 
%where $\widehat{\vbeta}_{\vgamma}$
%and $\widehat{\sigma}_{\vgamma}^2$ are the
%MLEs corresponding to model $\vgamma$.


\section{Implementation}
\label{sec:implementation}

\noindent Key to the feasibility of the model selection and averaging is an
efficient implementation of these procedures. We employ two main strategies to
achieve computational efficiency (i) efficient software implementation using
highly optimized software libraries; and (ii) efficient calculation of
$R$-squared values for all models based on using a Gray code and appropriate
matrix algebraic simplifications.  For ease of use we implemented an {\tt R}
package called {\tt blma}.  The internals of {\tt blma} are implemented in {\tt
C++} and use the {\tt R} packages \texttt{Rcpp} and \texttt{RcppEigen} to
enhance computational performance. The library {\tt OpenMP} was used to exploit
parallel computation.

There are two main special functions used in the paper -- the Gaussian
hypergeometric function, and the Appell hypergeometric function of two
variables. During the implementation process we tried several packages which
implemented the Gaussian hypergeometric function.  We found that the {\tt R}
package {\tt gsl} \citep{Hankin2006} was the most accurate, numerically stable
implementation amongst the packages we tried. The {\tt R} package {\tt Appell}
implements the Appell hypergeometric function \citep{Bove2013}. We also
developed our own numerical quadrature routine to evaluate the Appell
hypergeometric function to check our results.

\subsection{Gray code} 
\label{sec:GrayCode}

\noindent The Gray code was originally developed by Frank Gray in 1947
\cite[][Section 22.3]{PressEtal2007} to aid in detecting errors in analog to
digital conversions in communications systems. It is a sequence of binary
numbers whose key feature is that one and only one binary digit is different
between binary numbers in the sequence.  Gray codes can be constructed using a
sequence of ``reflect'' and ``prefix'' steps.  Let $\mGamma_1 = (0,1)^T \in
\{0,1\}^{2\times 1}$ be the first Gray code matrix and let $\mGamma_k$ be the
$k$th Gray code matrix. Then we can obtain the $(k+1)$th Gray code matrix given
$\mGamma_k$ via 
$$
\ds \mGamma_{k+1} = \left[\begin{array}{cc}
\vzero & \mGamma_k \\
\vone  & \mbox{reflect}(\mGamma_k)
\end{array} \right]
$$ 

\noindent where $\mbox{reflect}(\mGamma_k)$ is the matrix obtained by reversing
the order of rows of $\mGamma_k$, and the $\vzero$ and $\vone$ are vectors of
zeros and ones of length $2^k$ respectively. In {\tt C} and {\tt C++} these
Gray codes can be efficiently constructed using bit-shift operations on binary
strings in such a way that $\mGamma_{k}$ matrices are never computed and stored
explicitly.

Gray codes allow the enumeration of the entire model space in an order which
only adds or removes one covariate from the previous model at a time. We can
then use standard matrix inverse results to perform rank one updates and
downdates in the calculation of the $R^2$, $(\mX^T\mX)^{-1}$ and
$\widehat{\vbeta}$ values for each model in the model space.

\subsection{Model updates and downdates} 

\noindent Both updates and downdates depend on the fact that the inverse of a
real symmetric matrix can be written as

\begin{eqnarray}
	\ds \left[ \begin{array}{cc}
		\mA   & \mB \\
		\mB^T & \mC
	\end{array} \right]^{-1}
	&  = &
	\ds \left[ \begin{array}{cc}
		\mI & \vzero \\
		-\mC^{-1}\mB^T &  \mI
	\end{array} \right]
	\left[ \begin{array}{cc}
		\widetilde{\mA} & \vzero \\
		\vzero & \mC^{-1}
	\end{array} \right]
	\left[ \begin{array}{cc}
		\mI    & -\mB\mC^{-1}\\
		\vzero & \mI
	\end{array} \right] \label{eq:blockdiag1}\\
	&  = &
	\ds\left[
	\begin{array}{cc}
		\widetilde{\mA}
		& - \widetilde{\mA}\mB\mC^{-1} \\
		-\mC^{-1}\mB^T\widetilde{\mA}
		& \mC^{-1} + \mC^{-1}\mB^T\widetilde{\mA}\mB\mC^{-1}
	\end{array}\right]\label{eq:blockdiag2}
\end{eqnarray}

\noindent where $\widetilde{\mA} = \left(\mA-\mB\mC^{-1}\mB^T\right)^{-1}$
provided all inverses in (\ref{eq:blockdiag1}) and (\ref{eq:blockdiag2}) exist.
For both the update and downdate formula we assume that the quantities
$\mX^T\vy$, $\mX^T\mX$ have been precalculated, and that the
$(\mX_{\vgamma_i}^T\mX_{\vgamma_i})^{-1}$, $\widehat{\vbeta}_{\vgamma_i}$ and
$R_{\vgamma_i}^2$ values have been computed from the previous step.

We want to update the model inverse matrix, coefficient vector and $R^2$ values
for the model $\vgamma_{i+1}$ where $\mX_{\vgamma_{i+1}}$ is the matrix given
by $\mX_{\vgamma_i}$ with a column $\vz$ inserted into the appropriate
position.  For clarity of exposition we will assume that the column $\vz$ is
located in the last column of $\mX_{\vgamma_{i+1}}$, i.e., $\mX_{\vgamma_{i+1}}
= [\mX_{\vgamma_{i}},\vz]$. This can be achieved, if necessary, by appropriate
permuting  columns of various matrices.

The updates for the model inverse matrix, coefficient estimates, and $R^2$
values can be obtained by following the steps bellow.
\begin{enumerate}
	\item Calculate $\widehat{\vz} = (\mX_{\vgamma_i}^T\mX_{\vgamma_i})^{-1}\mX_{\vgamma_i}^T\vz$, 
	$\kappa 
	%= 
	%1/(\vz^T(\mI - \mX_{\vgamma_i}(\mX_{\vgamma_i}^T\mX_{\vgamma_i})^{-1}\mX_{\vgamma_i}^T)\vz) 
	= 1/(n - \vz^T\widehat{\vz})$, and  $s = \vy^T(\vz - \widehat{\vz})$.
	
	\item The model inverse matrix can be updated via  
	%The update for the $(\mX_{\vgamma_{i+1}}^T\mX_{\vgamma_{i+1}})^{-1}$ using $(\mX_{\vgamma_{i}}^T\mX_{\vgamma_{i}})^{-1}$ is
	%given by the following. 
	$$
	\begin{array}{rl}
	%\left[ \begin{array}{cc}
	%\mX^T\mX & \mX^T\vz \\
	%\vz^T\mX & \vz^T\vz \\
	%\end{array} \right]^{-1}
	(\mX_{\vgamma_{i+1}}^T\mX_{\vgamma_{i+1}})^{-1}
	%& \ds = 
	%\left[ \begin{array}{cc}
	%(\mX^T\mX)^{-1} + \kappa(\mX^T\mX)^{-1}\mX^T\vz\vz^T\mX(\mX^T\mX)^{-1}  & %-(\mX^T\mX)^{-1}\mX^T\vz \kappa \\
	%-\kappa \vz^T\mX(\mX^T\mX)^{-1}              
	%& \kappa 
	%\end{array} \right]
	%\\
	%& \ds = 
	%\left[ \begin{array}{cc}
	%(\mX^T\mX)^{-1} + \kappa\widehat{\vz}\widehat{\vz}^T  & - \widehat{\vz} \kappa \\
	%-\kappa \widehat{\vz}^T             
	%& \kappa 
	%\end{array} \right]
	%\\
	&\ds = 
	\left[ \begin{array}{cc}
	(\mX_{\vgamma_{i}}^T\mX_{\vgamma_{i}})^{-1}    & \vzero \\
	\vzero             
	& 0
	\end{array} \right] + \kappa \left[ \begin{array}{r}
	\widehat{\vz} \\
	-1 \\
	\end{array} \right] \left[ \begin{array}{r}
	\widehat{\vz} \\
	-1 \\
	\end{array} \right]^T.
	\end{array} 
	$$
	
	\item
	The coefficient estimators 
	%are given by
	$ 
	\ds \widehat{\vbeta}_{\vgamma_{i}} = (\mX_{\vgamma_{i}}^T\mX_{\vgamma_{i}})^{-1}\mX_{\vgamma_{i}}^T\vy$,
	and $\ds \widehat{\vbeta}_{\vgamma_{i+1}}  = (\mX_{\vgamma_{i+1}}^T\mX_{\vgamma_{i+1}})^{-1}\mX_{\vgamma_{i+1}}^T\vy$.  
	Then using the block inverse formula we have
	the relation
	$$
	\begin{array}{rl}
	\ds \widehat{\vbeta}_{\vgamma_{i+1}}
	%& \ds = \left[ \begin{array}{c}
	%\widehat{\vbeta} \\
	%0 
	%\end{array} \right] + \kappa  \left[ \begin{array}{cc}
	%(\mX^T\mX)^{-1}\mX^T\vz \left\{ \vz^T\mX(\mX^T\mX)^{-1}\mX^T\vy - \vz^T\vy\right\}  \\
	%\vz^T\vy - \vz^T\mX(\mX^T\mX)^{-1}\mX^T\vy
	%\end{array} \right]
	%\\
	& \ds 
	= \left[ \begin{array}{c}
	\widehat{\vbeta}_{\vgamma_{i}} \\
	0 
	\end{array} \right] - \kappa s  \left[ \begin{array}{r}
	\widehat{\vz}   \\
	- 1
	\end{array} \right].
	\end{array} 
	$$
	
	\item The $R^2$ value let 
	$R_{\vgamma_{i}}^2 = \tfrac{1}{n} \vy^T\mX_{\vgamma_{i}}(\mX_{\vgamma_{i}}^T\mX_{\vgamma_{i}})^{-1}\mX_{\vgamma_{i}}^T\vy$.
	Then using the block inverse formula we have
	$$
	\begin{array}{rl}
	\ds 
	R_{\vgamma_{i+1}}^2 
	%& \ds = \tfrac{1}{n} \vy^T\mX(\mX^T\mX)^{-1}\mX^T\vy
	%+ \tfrac{\kappa}{n}\left[ 
	%\widehat{\vy}^T\vz\vz^T\widehat{\vy}
	%- 2\widehat{\vy}^T\vz\vz^T\vy 
	%+ \vy^T\vz\vz^T\vy 
	%\right]
	= R_{\vgamma_{i}}^2
	+ \frac{\kappa s^2}{n}.
	
	\end{array}
	$$
	
	%\item
	%\noindent The model determinants are given by
	%$D = |\mX^T\mX|$
	%and
	%$D_{\mbox{\tiny update}} = |\mC^T\mC|$.
	%Using the block determinant formula we have
	%$D_{\mbox{\tiny update}} = D/c$.
\end{enumerate}

\noindent Presuming relevant summary quantities have been precomputed the above
updates costs $O(p_{\vgamma_{i}}^2 + n)$ time.

Suppose want to downdate the model summary quantities for the model
$\vgamma_{i+1}$ where $\mX_{\vgamma_{i+1}}$ is the matrix given by
$\mX_{\vgamma_i}$ with a column $\vz$ removed from the appropriate position.
Similarly as for updates for clarity of exposition we will assume that $\vz$
will be removed from the last column of $\mX_{\vgamma_i}$, i.e., we assume that
$\mX_{\vgamma_{i}} = [\mX_{\vgamma_{i+1}}, \vz]$.  Again, this can be achieved
by permuting the columns of various matrices.  Then the downdates for model
summary values are given by the following steps.
\begin{enumerate}
	\item 
	%The downdate for the model inverse matrix to obtain $(\mX_{\vgamma_{i+1}}^T\mX_{\vgamma_{i+1}})^{-1}$
	%from $(\mX_{\vgamma_{i}}^T\mX_{\vgamma_{i}})^{-1}$ can be found using the block-inverse formula.
	Suppose we partition the matrix
	$(\mX_{\vgamma_{i}}^T\mX_{\vgamma_{i}})^{-1}$ so that
	$$
	\ds (\mX_{\vgamma_{i}}^T\mX_{\vgamma_{i}})^{-1} 
	= \left[ \begin{array}{cc}
	\mA   & \vb \\
	\vb^T & c \\
	\end{array} \right].
	%= 
	%\left[ \begin{array}{cc}
	%\mX^T\mX & \mX^T\vz \\
	%\vz^T\mX & \vz^T\vz \\
	%\end{array} \right]^{-1}
	$$
	
	\noindent Calculate the model inverse matrix  
	by   $(\mX_{\vgamma_{i+1}}^T\mX_{\vgamma_{i+1}})^{-1} = \mA - c^{-1}\vb\vb^T$.
	
	\item Calculate
	$\widehat{\vz} = (\mX_{\vgamma_{i+1}}^T\mX_{\vgamma_{i+1}})^{-1}\mX_{\vgamma_{i+1}}^T\vz$,
	$\kappa = 1/(n - \vz^T\widehat{\vz})$,
	and $s = \vy^T(\vz - \widehat{\vz})$.
	
	\item 
	%Let $\widehat{\vbeta}_{\vgamma_{i+1}} = (\mX_{\vgamma_{i+1}}^T\mX)^{-1}\mX_{\vgamma_{i+1}}^T\vy$
	%and $\widehat{\vbeta}_{\vgamma_{i}} = %(\mX_{\vgamma_{i}}^T\mX_{\vgamma_{i}})^{-1}\mX_{\vgamma_{i}}^T\vy$.
	%Then
	The coefficient estimates downdate can be obtained
	via
	$$
	\widehat{\vbeta}_{\vgamma_{i+1}} = \left[ \widehat{\vbeta}_{\vgamma_{i}} \right]_{-|{\vgamma_{i}}|} + \kappa s\widehat{\vz},
	$$
	
	\noindent where $[ \widehat{\vbeta}_{\vgamma_{i}}]_{-|{\vgamma_{i}}|}$
	removes the last column from $\widehat{\vbeta}_{\vgamma_{i}}$.
	
	\item 
	%Let $R_{\vgamma_{i}}^2 = \tfrac{1}{n}\vy^T\mX_{\vgamma_{i}}(\mX_{\vgamma_{i}}^T\mX_{\vgamma_{i}})^{-1}\mX_{\vgamma_{i}}^T\vy$
	%and $R_{\vgamma_{i+1}}^2 = \tfrac{1}{n} \vy^T\mX_{\vgamma_{i+1}}(\mX_{\vgamma_{i+1}}^T\mX_{\vgamma_{i+1}})^{-1}\mX_{\vgamma_{i+1}}^T\vy$.
	The $R^2$ downdate can be obtained
	via
	$$
	R_{\vgamma_{i+1}}^2 = R_{\vgamma_{i}}^2 - \frac{\kappa s^2}{n}.
	$$
	
	
	%\item Let  $D = |\mC^T\mC|$ and $D_{\mbox{\tiny downdate}} = |\mX^T\mX|$ then
	%$D_{\mbox{\tiny downdate}} = cD$.
\end{enumerate}

\noindent Again, presuming relevant summary quantities have been precomputed
the updates for all of the above quantities costs $O(p_{\vgamma_{i}}^2 + n)$
time.

\section{Numerical results}
\label{sec:numerical_g_prior}

We will now compare the different Bayes factors under different hyperpriors on
$g$ that we have explored.  Firstly we will look at these Bayes factors by
comparing them directly.  We will then compare the results based on exact
Bayesian linear model averaging on some available datasets.

\subsection{Numerical comparison of $g$ hyperpriors}

Note that each of the Bayes factors is a function of three quantities $R^2$,
$p_\vgamma$ and $n$. Figure \ref{fig:bayesfactors} illustrates various log
Bayes factors over a grid of $p_\vgamma$ values from 1 to 20 and
$R^2\in\{0.1,0.5,0.9\}$ and $n \in \{100,500,1000\}$. In the context of
Bayesian hypothesis testing values above the $y$-axis value 0 indicate that the
alternative model is preferred, while lines below 0 indicate the null model is
preferred. Note that Cake priors (BIC) have the strongest penalty for larger
$p_\vgamma$, followed by the beta-prime prior (ZE), the robust prior,
hyper-$g/n$ prior and lastly the hyper-$g$ prior. Increasing $n$ and/or $R^2$
leads to all of the different Bayes factors becoming increasingly close to one
another. We also see that the {\tt appell()} function becomes unstable as $n$
and/or $R^2$ becomes large.  For the Bayes factor corresponding to the
hyper-$g/n$ prior our approximation tracks very closely to the methods using
the {\tt appell()} function and our numerical quadrature approach.

\begin{figure}[h!]
	\centering
	\includegraphics[width=0.99\linewidth]{BayesFactors}
	\caption{Cake prior or BIC (black), 
		beta-prime prior (blue), 
		hyper-$g$ prior (red),
		robust prior (green),
		hyper-$g/n$ ({\tt appell} - solid orange),
		hyper-$g/n$ (quadrature - dashed orange), 
		and hyper-$g/n$ (approximation - dotted orange). The grey line corresponds to the Bayes factor equal to 1. Above the grey line the alternative model is preferred, below the grey line the null model is preferred.}
	\label{fig:bayesfactors}
\end{figure}

\subsection{Settings for {\tt R} packages} 

We will now compare three different popular {\tt R} implementations of Bayesian
model averaging on several small datasets. We compare the {\tt R} packages {\tt
BAS} \citep{Clyde2017}, {\tt BayesVarSelect} \citep{Garcia-Donato2016}, and
{\tt BMS} \citep{Zeugner2015}. For each method we assumed a uniform prior on
the model space, i.e. $p(\vgamma)\propto 2^{-p}$. We used the setting implied
by the following commands for each of these methods.
\begin{itemize}
	\item {\tt BAS}: We used the command
	\begin{verbatim}
	bas.lm(y~X,prior=prior.val,modelprior=uniform())
	\end{verbatim}
	
    where \verb|prior.val| takes the value \verb|"hyper-g"|,
    \verb|"hyper-g-laplace"| or \verb|"hyper-g-n"|.  These correspond to a
    direct implementation of (\ref{eq:hyperGmarginal}), a Laplace approximation
    of (\ref{eq:hyperGmarginalIntegral}), and the Laplace approximation of
    (\ref{eq:hyperGonNmarginalIntegral}) respectively. The value $a=3$ is
    implicitly used.
	
	\item {\tt BayesVarSelect}: We used the command
	\begin{verbatim}
	Bvs(formula="y~.",data=data.frame(y=y,X=X),prior.betas=prior.val,
	prior.models="Constant",time.test=FALSE,n.keep=50000)
	\end{verbatim}
	
	
    \noindent where \verb|prior.val| takes the value \verb|"Liangetal"| or
    \verb|"Robust"|.  These correspond to a direct implementation of
    (\ref{eq:hyperGmarginal}) with $a=3$, and a hybrid approach which uses
    (\ref{eq:yGivenGammaRobust}) directly and numerical quadrature based on
    (\ref{eq:marginalLikelihoodRobust}) if this fails respectively.  Again, the
    value $a=3$ is implicitly used.
	
	\item {\tt BMS}: We used the command
	\begin{verbatim}
	bms(cbind(y,X),nmodel=50000,mcmc="enumerate",g="hyper=3",
	mprior="uniform")	
	\end{verbatim}
	
    \noindent which uses a direct implementation of (\ref{eq:hyperGmarginal})
    for the hyper-$g$ prior with $a=3$.
\end{itemize}

\noindent The syntax for {\tt blma} is relatively straightforward:

\begin{verbatim}
blma(vy, mX, prior, mprior, cores = 1L)
\end{verbatim}

\noindent where the arguments of {\tt blma}
are explained below.
\begin{itemize}
    \item {\tt vy} -- a vector of length $n$ of responses (this vector does not
        need to be standardized).
	
    \item {\tt mX} -- a design matrix with $n$ rows and $p$ columns (the
        columns of ${\tt mX}$ do not need to be standardized).
	
    \item {\tt prior} -- the choice of mixture $g$-prior used to perform
        Bayesian model averaging. The choices available include:
	\begin{itemize}
        \item {\tt "BIC"} -- the Bayesian information criterion obtained by
            using the Cake prior of \cite{OrmerodEtal2017}. 
		
        \item {\tt "ZE"} -- special case of the prior structure in
            \cite{Maruyama2011}.
		
        \item {\tt "liang\_g1"} -- the mixture $g$-prior of \cite{Liang2008}
            with prior hyperparameter $a=3$ evaluated directly using
            (\ref{eq:hyperGmarginal}) where the Gaussian hypergeometric
            function is evaluated using the {\tt gsl} library. Note: this
            option can lead to numerical problems and is only meant to be used
            for comparative purposes.
		
        \item {\tt "liang\_g2"} -- the mixture $g$-prior of \cite{Liang2008}
            with prior hyperparameter $a=3$ evaluated directly using
            (\ref{eq:hyperGmarginal2}).
		
        \item {\tt "liang\_g\_n\_appell"} -- the mixture $g/n$-prior of
            \cite{Liang2008} with prior hyperparameter $a=3$ evaluated using
            the {\tt appell R} package.
		
        \item {\tt "liang\_g\_approx"} -- the mixture $g/n$-prior of
            \cite{Liang2008} with prior hyperparameter $a=3$ using the
            approximation (\ref{eq:hyperGonNmarginalApprox}) for $p_\vgamma >2$
            and numerical quadrature (see below) ofr $p_\vgamma\in \{1,2\}$.
		
        \item {\tt "liang\_g\_n\_quad"} -- the mixture $g/n$-prior of
            \cite{Liang2008} with prior hyperparameter $a=3$ evaluated using a
            composite trapezoid rule.
		
        \item {\tt "robust\_bayarri1"} -- the robust prior of
            \cite{Bayarri2012} using default prior hyperparameter choices
            evaluated directly using (\ref{eq:yGivenGammaRobust}) with the {\tt
            gsl} library.
		
        \item {\tt "robust\_bayarri2"} -- the robust prior of
            \cite{Bayarri2012} using default prior hyperparameter choices
            evaluated directly using (\ref{eq:yGivenGammaRobust2}).
		
	\end{itemize}
    \item {\tt mprior} -- the prior to be imposed on the model space. The
        choices available include:
	\begin{itemize}
        \item {\tt "uniform"} -- corresponds to the prior $p(\vgamma) = 2^{-p}$
            where $p$ is the number of columns of $\mX$, .i.e., a uniform prior
            on the model space.
		
        \item {\tt "beta-binomial"} -- corresponds to a prior of the form
		$$
		\ds p(\vgamma) = \prod_{j=1}^p \rho^{\gamma_j} (1 - \rho)^{1 - \gamma_j} \qquad \mbox{and} \qquad \rho \sim \mbox{Beta}(a,b),
		$$
		
        \noindent where $\rho$ is the prior probability a variable is included
        in the mode, and $a$ and $b$ are fixed prior hyperparameters. After
        marginalizing out $\rho$ we have
		$$
		p(\vgamma) = \frac{\mbox{Beta}(a + |\vgamma|,b + p - |\vgamma|)}{\mbox{Beta}(a,b)},
		$$
		
        \noindent which is a beta-binomial distribution. Note $a=b=1$
        corresponds to a uniform prior on the prior variable inclusion
        probability. The values of $a$ and $b$ should be set to be the first
        and second elements of the {\tt modelpriorvec} argument respectively
        (see below).
		
		\item {\tt "bernoulli"} -- corresponds to a prior of the form 
		$$
		p(\vgamma) = \prod_{j=1}^p \rho_j^{\gamma_j} (1 - \rho_j)^{1 - \gamma_j}
		$$
		
        \noindent where the $\rho_j\in(0,1)$. The $\rho_j$ values are specified
        by {\tt modelpriorvec} (see below). Using $\rho_j = 1/2$, $1\le j\le p$
        corresponds to {\tt mprior=="uniform"}.
	\end{itemize}
	
    \item {\tt modelpriorvec} -- A vector of additional parameters. If {\tt
        mprior=="uniform"} this argument is ignored.
    If {\tt mprior=="beta-binomial"} this should be a postive vector of length
    2 corresponding to the shape parameters of a Beta distribution (the values
    $a$ and $b$ above). If {\tt mprior=="bernoulli"} this should be a vector of
    length $p$ with values on the interval $(0,1)$.
	
	\item {\tt cores} -- the number of computer cores to use.
\end{itemize}

\noindent 
The object returned is a 
list containing:
\begin{itemize}
	\item 
	{\tt vR2} -- the vector $R$-square values for each model; 
	
	\item 
	{\tt vp\_gamma} -- the vector of number of covariates for each model;
	
	\item 
	{\tt vlogp} -- the vector of logs of the marginal likelihoods of each model; and
	
	\item 
	{\tt vinclusion\_prob} -- the vector of posterior inclusion probabilities for each of the covariates. 
\end{itemize}

\noindent Note that we do not return the fitted values of
$\widehat{\vbeta}_{\vgamma}$ which should only be calculated for a subset of
models. We also do not return $\mGamma$, the Gray code matrix which we provide
a separate function to calculate. We made the decisions not to return these
quantities to reduce the memory overhead.

A short example fitting the {\tt USCrime} data described in Section
\ref{sec:BLMA} is found below.

\begin{verbatim}
library(blma); library(MASS)
dat <- UScrime
dat[,-c(2,ncol(UScrime))] <- log(dat[,-c(2,ncol(UScrime))])
vy <- dat$y
mX <- data.matrix(cbind(dat[1:15]))
colnames(mX) <- c("log(AGE)","S","log(ED)","log(Ex0)","log(Ex1)",
"log(LF)","log(M)","log(N)","log(NW)","log(U1)","log(U2)","log(W)",
"log(X)","log(prison)","log(time)") 
blma_result <- blma(vy, mX, prior="ZE")
\end{verbatim}

\noindent Results for the above example are summarised as part of the result
within Section \ref{sec:BLMA}.

\subsection{Bayesian linear model averaging on data}\label{sec:BLMA}

We considered several small datasets to illustrate our methodology. These
datasets can be found in the {\tt R} packages {\tt MASS} \citep{Venables2002}
and {\tt Ecdat} \citep{Croissant2016}. Table \ref{tab:g_prior_datasets}
summarizing the sizes,  sources, and response variable used for each dataset
used.  We chose {\tt USCrime} data because it is used in most papers in the
area and is small enough so that n\"aive implementations using special
functions will not lead to numerical issues. The 
%{\tt Hitters} nad
{\tt Kakadu} dataset is chosen to be large enough to begin to strain the
resources of a typical 2018 laptop so that relative differences in speeds
between different packages becomes apparent. Finally, the {\tt Kakadu} dataset
is chosen to lead numerical instability in the direct evaluation of Bayes
factors for some of the priors on $g$ considered in this paper.

\begin{table}[h]
	\begin{center}
		\begin{tabular}{l|r|r|l|l}
			Dataset	& $n$ & $p$ & Response & {\tt R} package \\ 
			\hline 
			UScrime 	& 47 & 15 & y & {\tt MASS} \\  
			%Bodyfat	& 244  & 13 &  \\ 
			%	\hline 
			%Hitters	& 263 & 19 & Salary & {\tt ISLR} \\ 
			%	\hline 
			%Wage	& 3000 & 17 &  {\tt ISLR}  \\
			VietNamI	& 27765 & 11 & lnhhexp & {\tt Ecdat}  \\ 
			Kakadu	& 1827 & 22 & income & {\tt Ecdat}   \\  
		\end{tabular} 
	\end{center}
	\caption{A summary of the datasets used in the paper and their respective {\tt R} packages.}
	\label{tab:g_prior_datasets}
\end{table}

For each of the datasets some minimal preprocessing was used.  We first used
the {\tt R} command {\tt na.omit()} to remove samples containing missing
predictors.  For {\tt USCrime} all variables except the predictor {\tt S} were
log-transformed. For all datasets the {\tt R} command {\tt model.matrix()} was
used to construct the design matrix using all variables except for the response
as predictors.

Tables \ref{tab:UScrime}, \ref{tab:VietNamI}, and \ref{tab:Kakadu} summarise
the times and variable inclusion probabilities, i.e., $\bE(\vgamma|\vy)$, for
all of the mixture $g$-prior structures we have considered here under a uniform
prior on the model space.  All times are based on running {\tt R} code on a
dedicated server with 48 cores, each running at 2.70GHz, with a total of 512GB
of RAM.  The {\tt BVS} package in the table refers to the {\tt BayesVarSelect}
{\tt R} package where we have used a this acronym to save space in the tables. 

\begin{sidewaystable}[h!]
	\begin{center}
		{\scriptsize 
			\begin{tabular}{c|r|r|rrrrrr|rrrr|rrr}
				Package & blma   & blma   & BAS    & BAS     & BVS    & BMS    & blma & blma & BAS & blma & blma & blma & BVS & blma & blma \\ 
				Prior   & BIC    & ZE     & $g$    & $g$     & $g$    & $g$    & $g$  & $g$ &  $g/n$ & $g/n$ & $g/n$ & $g/n$ & Robust & Robust & Robust \\ 
				Method  & (\ref{eq:marginalLikelihoodCake})  & (\ref{eq:marginalLikelihoodBetaPrime}) 
				& (\ref{eq:hyperGmarginal}) & Laplace & (\ref{eq:hyperGmarginal}) & (\ref{eq:hyperGmarginal}) & (\ref{eq:hyperGmarginal}) & (\ref{eq:hyperGmarginal2}) & Laplace & 
				{\tt appell} & quad. & (\ref{eq:hyperGonNmarginalApprox}) & (\ref{eq:yGivenGammaRobust}) & (\ref{eq:yGivenGammaRobust}) & (\ref{eq:yGivenGammaRobust2}) \\ 
				\hline
				1 & 70.87 & 65.51 & 65.93 & 65.99 & 64.74 & 65.93 & 65.93 & 65.93 & 65.14 & 65.10 & 65.10 & 65.72 & 64.74 & NaN & 64.74 \\ 
				2 & 19.06 & 22.88 & 25.52 & 25.54 & 24.51 & 25.52 & 25.52 & 25.52 & 22.93 & 22.91 & 22.91 & 22.47 & 24.51 & NaN & 24.51 \\ 
				3 & 92.07 & 86.91 & 86.23 & 86.28 & 85.59 & 86.23 & 86.23 & 86.23 & 86.54 & 86.51 & 86.51 & 87.24 & 85.59 & NaN &  85.59 \\ 
				4 & 72.53 & 69.65 & 69.20 & 69.22 & 69.02 & 69.20 & 69.20 & 69.20 & 69.52 & 69.51 & 69.51 & 69.89 & 69.02 & NaN &  69.02 \\ 
				5 & 37.01 & 42.36 & 44.61 & 44.61 & 44.08 & 44.61 & 44.61 & 44.61 & 42.53 & 42.52 & 42.52 & 41.88 & 44.08 & NaN &  44.08 \\ 
				6 & 15.82 & 20.18 & 23.06 & 23.08 & 22.04 & 23.06 & 23.06 & 23.06 & 20.27 & 20.26 & 20.26 & 19.73 & 22.04 & NaN &  22.04 \\ 
				7 & 27.06 & 32.43 & 34.55 & 34.55 & 34.08 & 34.55 & 34.55 & 34.55 & 32.59 & 32.59 & 32.59 & 32.00 & 34.08 & NaN &  34.08 \\ 
				8 & 60.64 & 56.91 & 57.34 & 57.39 & 56.47 & 57.34 & 57.34 & 57.34 & 56.66 & 56.63 & 56.63 & 57.07 & 56.47 & NaN &  56.47 \\ 
				9 & 36.92 & 35.81 & 37.66 & 37.71 & 36.35 & 37.66 & 37.66 & 37.66 & 35.64 & 35.61 & 35.61 & 35.71 & 36.35 & NaN &  36.35 \\ 
				10 & 21.92 & 24.35 & 27.06 & 27.10 & 25.78 & 27.06 & 27.06 & 27.06 & 24.31 & 24.29 & 24.29 & 24.00 & 25.78 & NaN &  25.78 \\ 
				11 & 55.84 & 50.19 & 51.25 & 51.32 & 49.66 & 51.25 & 51.25 & 51.25 & 49.79 & 49.75 & 49.75 & 50.38 & 49.66 & NaN &  49.66 \\ 
				12 & 17.39 & 21.57 & 24.46 & 24.48 & 23.40 & 24.46 & 24.46 & 24.46 & 21.65 & 21.63 & 21.63 & 21.12 & 23.40 & NaN &  23.40 \\ 
				13 & 99.92 & 99.69 & 99.50 & 99.51 & 99.54 & 99.50 & 99.50 & 99.50 & 99.66 & 99.66 & 99.66 & 99.72 & 99.54 & NaN &  99.54 \\ 
				14 & 90.27 & 84.92 & 83.87 & 83.92 & 83.45 & 83.87 & 83.87 & 83.87 & 84.57 & 84.55 & 84.55 & 85.32 & 83.45 & NaN &  83.45 \\ 
				15 & 17.63 & 22.55 & 25.49 & 25.51 & 24.52 & 25.49 & 25.49 & 25.49 & 22.67 & 22.65 & 22.65 & 22.05 & 24.52 & NaN &  24.52 \\ 
				\hline
				%Nan \% & 0.00 & 0.00 & 0.00 & 0.00     & 0.00      &  0.00 & 0.00 & 0.00 & 0.00         & 0.00 & 0.00 & 0.00    & 0.00       & 2.56 & 0.00 \\	
				%Time (s) & 0.10 & 0.18 & 0.62 & 0.31 &         & 24.60 & 3.50 & 0.37 & 0.21 & 126.41  & 47.85 & 0.22 & 397.39  &  183.18 & 0.31    \\	
				Time (s) & 0.11 & 0.10 & 1.07 & 0.51 & 1358.61 & 44.73 & 0.12 & 0.10 & 0.30 &  12.59  & 40.36 & 0.25 & 618.59  &   31.81 & 0.11  \\
				\hline		
			\end{tabular}
		}
	\end{center}
    \caption{Variable inclusion probabilities (as a percentage) and
        computational times (in seconds) for the {\tt UScrime} dataset.  The
        first to third line indicates the package, mixture $g$-prior and
        evaluation method used respectively. Bracketed terms refer to equations
    in the paper. NaN entries indicate numerical issues for the
prior/implementation pair. The acronym BVS refers to the {\tt BayesVarSelect}
package.}
	\label{tab:UScrime}
\end{sidewaystable}

\begin{sidewaystable}[h!]
	\begin{center}
		{\scriptsize 
			\begin{tabular}{c|r|r|rrrrrr|rrrr|rrr}
				Package & blma   & blma   & BAS    & BAS     & BVS    & BMS    & blma & blma & BAS & blma & blma & blma & BVS & blma & blma \\ 
				Prior   & BIC    & ZE     & $g$    & $g$     & $g$    & $g$    & $g$  & $g$ &  $g/n$ & $g/n$ & $g/n$ & $g/n$ & Robust & Robust & Robust \\ 
				Method  & (\ref{eq:marginalLikelihoodCake})  & (\ref{eq:marginalLikelihoodBetaPrime}) 
				& (\ref{eq:hyperGmarginal}) & Laplace & (\ref{eq:hyperGmarginal}) & (\ref{eq:hyperGmarginal}) & (\ref{eq:hyperGmarginal}) & (\ref{eq:hyperGmarginal2}) & Laplace & 
				{\tt appell} & quad. & approx. & (\ref{eq:yGivenGammaRobust}) & (\ref{eq:yGivenGammaRobust}) & (\ref{eq:yGivenGammaRobust2}) \\ 
				\hline
				1 & 100.00 & 100.00 & 100.00 & 100.00 & NaN & NaN & NaN & 100.00 & 100.00 & NaN & 100.00 & 100.00 & NaN & 100.00 & 100.00 \\ 
				2 & 100.00 & 100.00 & 100.00 & 100.00 & NaN & NaN & NaN & 100.00 & 100.00 & NaN & 100.00 & 100.00 & NaN & 100.00 &  100.00 \\ 
				3 & 1.21 & 3.16 & 8.65 & 8.65 & NaN & NaN & NaN & 8.64 & 7.17 & NaN & 7.16 & 7.65 & NaN  & 4.77 &  4.77 \\ 
				4 & 100.00 & 100.00 & 100.00 & 100.00 & NaN & NaN & NaN & 100.00 & 100.00 &NaN  & 100.00 & 100.00 & NaN  & 100.00 &  100.00 \\ 
				5 & 100.00 & 100.00 & 100.00 & 100.00 & NaN & NaN & NaN & 100.00 & 100.00 & NaN & 100.00 & 100.00 & NaN & 100.00 &  100.00 \\ 
				6 & 100.00 & 100.00 & 100.00 & 100.00 & NaN & NaN & NaN & 100.00 & 100.00 & NaN & 100.00 & 100.00 & NaN & 100.00 &  100.00 \\ 
				7 & 0.62 & 1.72 & 5.33 & 5.33 & NaN &NaN  & NaN & 5.32 & 4.30 & NaN & 4.29 & 4.63 & NaN & 2.70 &  2.70 \\ 
				8 & 96.07 & 98.32 & 99.35 & 99.35 & NaN & NaN & NaN & 99.35 & 99.21 & NaN & 99.20 & 99.26 & NaN & 98.86 &  98.86 \\ 
				9 & 3.28 & 8.16 & 20.69 & 20.69 & NaN & NaN & NaN & 20.66 & 17.46 & NaN & 17.42 & 18.52 & NaN & 12.02 & 12.02 \\ 
				10 & 100.00 & 100.00 & 100.00 & 100.00 & NaN & NaN & NaN & 100.00 & 100.00 & NaN & 100.00 & 100.00 & NaN & 100.00 &  100.00 \\ 
				11 & 100.00 & 100.00 & 100.00 & 100.00 & NaN & NaN & NaN & 100.00 & 100.00 & NaN & 100.00 & 100.00 & NaN & 100.00 &  100.00 \\ 
				\hline
				%Nan \% & 0.00 & 0.00 & 0.00 & 0.00   &       &  ?? & 75.00 & 0.00 & 0.00         & 44.29 & 0.00 & 0.00    & *        & 0.00 & 0.00 \\	\hline	
				%Time (s) & 0.03 & 0.03 & 1.01 & 0.27 &       &  6.94 & 0.05 & 0.04 & 0.30 & 65.19 & 8.79 & 0.03    & *        & 14.69 & 0.04 \\	\hline	
				Time (s) & 0.03 & 0.02 & 0.88 & 0.33  &       &  5.89 & 0.02 & 0.02 & 0.08 & 84.73 & 2.69 & 0.10    & *        & 2.18 & 0.01 \\	\hline	
			\end{tabular}
		}
	\end{center}
    \caption{Variable inclusion probabilities (as a percentage) and
        computational times (in seconds) for the {\tt VietNamI} dataset.  The
        first to third line indicates the package, mixture $g$-prior and
        evaluation method used respectively. Bracketed terms refer to equations
    in the paper. NaN entries indicate numerical issues for the
prior/implementation pair. The acronym BVS refers to the {\tt BayesVarSelect}
package.}
	\label{tab:VietNamI}
\end{sidewaystable}

\begin{sidewaystable}[h!]
	\begin{center}
		{\scriptsize 
			\begin{tabular}{c|r|r|rrrrrr|rrrr|rrr}
				Package & blma   & blma   & BAS    & BAS     & BVS    & BMS    & blma & blma & BAS & blma & blma & blma & BVS & blma & blma \\ 
				Prior   & BIC    & ZE     & $g$    & $g$     & $g$    & $g$    & $g$  & $g$ &  $g/n$ & $g/n$ & $g/n$ & $g/n$ & Robust & Robust & Robust \\ 
				Method  & (\ref{eq:marginalLikelihoodCake})  & (\ref{eq:marginalLikelihoodBetaPrime}) 
				& (\ref{eq:hyperGmarginal}) & Laplace & (\ref{eq:hyperGmarginal}) & (\ref{eq:hyperGmarginal}) & (\ref{eq:hyperGmarginal}) & (\ref{eq:hyperGmarginal2}) & Laplace & 
				{\tt appell} & quad. & approx. & (\ref{eq:yGivenGammaRobust}) & (\ref{eq:yGivenGammaRobust}) & (\ref{eq:yGivenGammaRobust2}) \\ 
				\hline
				1 & 11.96 & 20.36 & 34.62 & 34.64      &  NaN      & 34.69 & 34.69 & 34.69 & 31.98     &  NaN   & 32.04 & 32.96   &  NaN       & 26.46 & 26.46 \\ 
				2 & 43.60 & 47.24 & 50.36 & 50.34      &  NaN      & 50.34 & 50.34 & 50.34 & 49.79     &  NaN   & 49.78 & 49.98   &  NaN       & 48.73 & 48.73 \\ 
				3 & 3.00 & 7.49 & 16.97 & 16.99        &  NaN      & 17.10 & 17.10 & 17.10 & 15.02     &  NaN   & 15.13 & 15.80   &  NaN       & 11.20 & 11.20 \\ 
				4 & 37.14 & 42.02 & 46.85 & 46.88      &  NaN      & 46.87 & 46.87 & 46.87 & 46.02     &  NaN   & 46.01 & 46.31   &  NaN       & 44.28 & 44.28 \\ 
				5 & 81.87 & 86.11 & 90.49 & 90.50      &  NaN      & 90.41 & 90.41 & 90.41 & 89.92     &  NaN   & 89.85 & 90.07   &  NaN       & 88.59 & 88.59 \\ 
				6 & 16.83 & 26.67 & 41.70 & 41.69      &  NaN      & 41.83 & 41.83 & 41.83 & 38.98     &  NaN   & 39.10 & 40.05   &  NaN       & 33.27 & 33.27 \\ 
				7 & 3.22 & 8.89 & 21.41 & 21.43        &  NaN      & 21.53 & 21.53 & 21.53 & 18.86     &  NaN   & 18.95 & 19.83   &  NaN       & 13.82 & 13.82 \\ 
				8 & 4.30 & 11.09 & 23.57 & 23.59       &  NaN      & 23.66 & 23.66 & 23.66 & 21.20     &  NaN   & 21.26 & 22.09   &  NaN       & 16.34 & 16.34 \\ 
				9 & 2.62 & 7.19 & 16.97 & 16.98        &  NaN      & 17.09 & 17.09 & 17.09 & 14.97     &  NaN   & 15.07 & 15.76   &  NaN       & 11.04 & 11.04 \\ 
				10 & 52.53 & 77.78 & 90.97 & 90.99     &  NaN      & 90.81 & 90.81 & 90.81 & 89.59     &  NaN   & 89.44 & 89.98   &  NaN       & 85.92 & 85.92 \\ 
				11 & 92.51 & 93.73 & 94.75 & 94.79     &  NaN      & 94.58 & 94.58 & 94.58 & 94.68     &  NaN   & 94.49 & 94.53   &  NaN       & 94.35 & 94.35 \\ 
				12 & 99.82 & 99.94 & 99.97 & 99.97     &  NaN      & 99.97 & 99.97 & 99.97 & 99.97     &  NaN   & 99.97 & 99.97   &  NaN       & 99.96 & 99.96 \\ 
				13 & 2.45 & 6.60 & 15.70 & 15.72       &  NaN      & 15.84 & 15.84 & 15.84 & 13.81     &  NaN   & 13.92 & 14.57   &  NaN       & 10.13 & 10.13 \\ 
				14 & 8.10 & 19.91 & 38.61 & 38.63      &  NaN      & 38.66 & 38.66 & 38.66 & 35.36     &  NaN   & 35.39 & 36.55   &  NaN       & 28.37 & 28.37 \\ 
				15 & 8.17 & 18.51 & 35.17 & 35.19      &  NaN      & 35.24 & 35.24 & 35.24 & 32.19     &  NaN   & 32.24 & 33.29   &  NaN       & 25.87 & 25.87 \\ 
				16 & 62.99 & 75.30 & 83.41 & 83.42     &  NaN      & 83.30 & 83.30 & 83.30 & 82.39     &  NaN   & 82.29 & 82.68   &  NaN       & 79.98 & 79.98 \\ 
				17 & 3.27 & 8.53 & 19.41 & 19.43       &  NaN      & 19.54 & 19.54 & 19.54 & 17.20     &  NaN   & 17.31 & 18.07   &  NaN       & 12.85 & 12.85 \\ 
				18 & 54.75 & 74.93 & 86.65 & 86.65     &  NaN      & 86.55 & 86.55 & 86.55 & 85.31     &  NaN   & 85.22 & 85.74   &  NaN       & 81.95 & 81.95 \\ 
				19 & 100.00 & 100.00 & 100.00 & 100.00 &  NaN      & 100.00 & 100.00 & 100.00 & 100.00 &  NaN   & 100.00 & 100.00 &  NaN       & 100.00 & 100.00 \\ 
				20 & 26.63 & 44.11 & 62.58 & 62.60     &  NaN      & 62.56 & 62.56 & 62.56 & 59.88     &  NaN   & 59.83 & 60.83   &  NaN       & 53.58 & 53.58 \\ 
				21 & 100.00 & 100.00 & 100.00 & 100.00 &  NaN      & 100.00 & 100.00 & 100.00 & 100.00 &  NaN   & 100.00 & 100.00 &  NaN       & 100.00 & 100.00 \\ 
				22 & 4.95 & 13.22 & 29.03 & 29.05      &  NaN      & 29.12 & 29.12 & 29.12 & 26.04     &  NaN   & 26.09 & 27.14    & NaN        & 19.87 & 19.87 \\ 
				\hline
				%	Nan \% &   &   &   &       & *      &   &   &   &           &   &   &      & *        &   &   \\ 	
				%	\hline
				%Time(s) & 38.46 & 64.31 & 18.89  & 12.06  &       & 2114.16  & 319.58  &   96.230 &  17.56         &  28325.40 & 34798.648  &   75.11   & 4921.23       & 53107.45  & 77.86  \\ 	
				Time(s) & 15.43 & 16.18 & 14.85  &  9.53   &       & 1735.66  & 34.925  &   17.55 &  10.82         &  25008.93 & 
				5425.11  &   18.06   & 4606.92       &  4275.55  & 21.03  \\ 
				\hline
			\end{tabular}
		}
	\end{center}
    \caption{Variable inclusion probabilities (as a percentage) and
        computational times (in seconds) for the {\tt Kakadu} dataset.  The
        first to third line indicates the package, mixture $g$-prior and
        evaluation method used respectively. Bracketed terms refer to equations
    in the paper. NaN entries indicate numerical issues for the
prior/implementation pair. The acronym BVS refers to the {\tt BayesVarSelect}
package. Note that the {\tt BayesVarSelect} method ran out of RAM for this
example.}
	\label{tab:Kakadu}
\end{sidewaystable}


For Table \ref{tab:UScrime} we see that all of the ``exact'' methods agree with
one another to the first 2 decimal places. We note that the Laplace
approximation is quite accurate and appears superior to  the method
``(\ref{eq:hyperGonNmarginalApprox})'' for the mixture $g/n$-prior. However,
for both of these approximation methods the discrepancies to their exact
counterparts is roughly the same size, or perhaps even less, than the
differences between each of the choices of mixture $g$-priors. In terms of
speed, {\tt BAS} and {\tt BMLA} are the fastest packages and roughly comparable
in speed. Both {\tt BMS} and {\tt BayesVarSelect} are not as fast.  For the
mixture $g$-prior we suspect that the package {\tt BAS} relies on Laplace's
method for models where direct evaluation of (\ref{eq:hyperGmarginal}) becomes
numerically problematic, which would explain differences between the {\tt BAS}
and {\tt blma} packages for the {\tt Kakadu} dataset.
%Overall our package {\tt BLMA} offers mixture $g$-priors not offered by {\tt
%BAS}, is arguably more accurate when $n$ is large, and is of comparable speed. 

\section{Conclusion}
\label{sec:chapter_3_conclusion}

We have reviewed the prior structures that lead to closed form expressions for
Bayes factors for linear models. We have described ways that each of these
priors with the exception of the hyper-$g/n$ prior can be evaluated in a
numerically stable manner and have implemented a package \texttt{blma} for
performing full exact Bayesian model averaging using this methodology. Our
package is competitive with \texttt{BAS} and \texttt{BMS} in terms of
computational speed, is numerically more stable and accurate, and offers some
different priors structures not offered in \texttt{BAS}. Our package is much
faster than \texttt{BayesVarSelect} and is also numerically more stable and
accurate.

Our package is competitive with {\tt BAS} and {\tt BMS} in terms of
computational speed, is numerically more stable and accurate, and offers some
different priors structures not offered in {\tt BAS}. Our package is much
faster than {\tt BayesVarSelect} and is also numerically more stable and
accurate, and represents an advance in the implementation of exact Bayesian
linear model averaging.

%! TEX root = thesis.tex
\chapter{Particle Variational Approximation}

\setlength{\parindent}{0pt}

\section{Introduction}

Bayesian model selection is a powerful set of techniques for model selection.
These techniques are especially useful in problems of high-dimension, such as
bioinformatics problems where the model space is complex and the optimal model
is difficult for statisticians to manually specify. However, Bayesian model
selection is computationally expensive, and prone to getting stuck in local
minima if the posterior likelihood is multi- modal. This issue is particularly
acute if the spike-and-slab prior, popular for Bayesian model selection, is
used. We seek to address both problems by proposing a population non-parametric
Variational Bayes approximation algorithm - a population-based optimisation
strategy. Maintaining a population of models allows the posterior distribution
to be explored more thoroughly, finding multiple maxima. The variational
approximation's lower bound includes an entropy term which ensures diversity in
the population by penalising similarity,  having the particles in the
population repel each other. This ensures the high probability regions of the
posterior distribution is thoroughly explored, which better reflects model
selection uncertainty. In this chapter, we focus on the important case of model
selection for normal linear models with priors as described in Section
\ref{sec:model}.
% which are slightly altered from those introduced in Section \ref{sec:model}
% to incorporate a zero-centred normal prior on $\alpha$ with variance $g
% \sigma^2$. This allows shrinkage of the intercept as well as the
% non-intercept regression co-efficients, driven by data.

\cite{Mitchell1988} initially proposed the spike-and-slab prior distribution on
regression co-efficients not currently included in the model -- which places a
mixture of a point mass 'spike' at $0$ and a diffuse uniform distribution
'slab' elsewhere. The approach was further developed by \cite{Madigan1994} to
incorporate an alternative Bayesian approach that takes full account of the
true model uncertainty by averaging over a small subset of models, and an
efficient search algorithm for finding these models. \cite{George1997}
investigated computational methods for posterior evaluation and exploration in
this setting, and using Gray Code sequencing and Markov Chain Monte Carlo to
explore the model space in moderate and large-sized problems respectively.
More recently, \cite{Ishwaran2005} developed a rescaled spike-and-slab model
which improves effective variable selection in terms of risk misclassification
by using selective shrinkage.

% We further extend this approach, by
% using the Cake variant of the spike-and-slab prior for model selection \cite{OrmerodEtal2017}, as it avoids
% the Lindley and Bartlett paradoxes \cite{Lindley1957}, \cite{Bartlett1957}.

Existing approaches to the problem of model selection focus upon finding a
single best model as quickly as possible, using the least computational effort
(\cite{You2014}, \cite{Rockova2014}). Exploring the model space using only one
model at a time will provide a misleading view of the uncertainty in the
posterior, as it is typically highly multimodal.

Many computational schemes for Bayesian model selection exist, using Monte
Carlo Markov Chains techniques to compute the posterior distribution of
$\vgamma$. However, these schemes are both computationally intensive and can
become trapped in local maxima of the posterior distribution if the
distribution is high-dimensional and multi-modal, as is the case with popular
choices of prior for Bayesian model selection problems, such as spike-and-slab
priors. The difficulty of becoming trapped in local maxima can be partially
mitigated by using population-based MCMC schemes such as Jasra et al. 2007,
Bottolo and Richardson 2010, Hans et al 2007, Liang and Wong 2000. However,
this increases the computational cost of sampling from the posterior
distributions still further, especially in high-dimensional problems.

\cite{Rockova2017} introduced the notion of Particle EM. Rather than searching
for a single optimal model, Particle EM instead maintains a population of
models (particles). This allows the algorithm to explore more of the posterior
model space, gaining a better estimate of the variation of the model space than
an algorithm involving only a single model. It also allows the particles to
``interact'', searching for the essential posterior modes together. In Particle
EM, this is done by incorporating an ``entropy term'' in the variational lower
bound, which ensures diversity amongst the models in the population, preventing
all particles from simply seeking the global posterior modes. The algorithm is
determininistic.

We build upon this work by proposing a fixed-form parametric Variational Bayes
approximation of $\vgamma$. We adopt a prior structure incorporating the Cake
prior for variable selection, which avoids the Lindley and Bartlett's
paradoxes. The difficulties in implementing practical Bayesian model selection
schemes have been noted in \cite{Chipman2014}. As our marginal likelihood
expression is a function only of $n$, $p_\vgamma$ and $R^2_\vgamma$, our model
selection algorithm can be executed efficiently using rank-one updates and
downdates to compute $R^2_{\vgamma^*}$ for each of the models $\vgamma^*$ that
we consider. To ensure uniqueness of the $K$ models in the population, before a
new candidate model with a covariate added or removed is considered, the
population of existing models is checked to see if it already exists in the
population. If so, the addition or removal of the covariate is skipped and the
next candidate model considered.

% Particle Variational Bayes Particle Variational Bayes approaches have proved
% useful in Bayesian nonparametric settings such as Latent Dirichlet Allocation
% \cite{Teh2007}
Our variational approximation of the posterior model likelihood is a weighted
sum of the indicators of the covariates of the model, where the weights are
determined by the relative contribution of each covariate to the model fit in
all particles in the opulation, balanced against the diversity of the
particles.
\[
	q(\vgamma) = \sum_{k=1}^K w_k \I(\vgamma_k)
\]

% Population-based MCMC approaches

Our main contributions are:

\begin{enumerate}
    \item Our algorithm searches over the binary strings $\vgamma$ directly, as
        the estimates of $\vbeta$ are available in closed form once $\vgamma$
        is known.

    \item We make use of a population--based optimisation scheme to search the
        model space. We take advantage of the population of solutions by
        incorporating a penalty for lack of entropy, which ensures diversity in
        the population of solutions.

    \item The entire trajectory of particles gives far more information about
        variable selection than a single snapshot of the final best decision.

    \item Our model can incorporate different hyperpriors on $g$ and $\vgamma$.
Using a hyperprior on $g$ avoids Lindley's paradox and Bartlett's paradox, the
model selection paradoxes which arise when a fixed choice of $g$ is made.
\end{enumerate}

% This chapter is organised as follows. In Section 2, we detail the derivation
% of our approximation and fitting algorithms. In Section 3, we discuss
% computational issues with implementing our algorithm efficiently. In Section
% 4, we present the results of our numerical experiments. In Section 5, we
% present our conclusions and discuss our results.

% \section{Method}

% \subsection{Posterior distributions of $g$-priors in terms of special functions}
% % Things we derived


% \subsubsection{Calculating $p(\vy | \vgamma)$}
% We want to calculate $p(\vy | \vgamma)$ so that we can calculate $p(\vgamma |
% \vy)$ using Bayes' Rule allowing us to rank models against one another and
% choose which amongst many candidate models is best.  To do this, we need to
% calculate $$p(\vy | \vgamma) = \int p(\vy | \alpha, \vbeta, \sigma^2,
% \vgamma) p(\alpha) p(\vbeta | g, \vgamma) p(\sigma^2) p(g) d \alpha d \vbeta
% d \sigma^2 d g.$$ We choose our priors so that this integral is tractable,
% with help from \cite{Gradshteyn1988}.  The marginal likelihoods obtained from
% this integral behave like BIC.

% We found the following closed form expressions. Include derivations in the appendices.
% \small
% \begin{itemize}
% 	\item Liang et al. 2008 hyper-$g$ prior \cite{Liang2008}
% 		$p(\vy | \vgamma) = \frac{K(n) (a - 2)}{p_\vgamma + a  - 2} {}_2 F_1 \left( \frac{n-1}{2}, 1; \frac{p_\vgamma + a}{2}; R_\vgamma^2 \right)$
% 	\item Liang et al. 2008 hyper-$g/n$ prior \cite{Liang2008}
% 		$p(\vy | \vgamma) = \frac{K(n) (a - 2)}{n (p_\vgamma + a  - 2)} F_1 \left( 1, \frac{a}{2}, \frac{n-1}{2}; \frac{p_\vgamma + a}{2}; 1 - \frac{1}{n}, R_\vgamma^2 \right)$
% \end{itemize}
% These expressions are numerically stable to evaluate.

% We want to be able to calculate the marginal likelihood of the data given
% $\vgamma$, so that we can calculate the posterior probability of $\vgamma$
% and rank models against one another. To do this, we need to calculate the
% following integral.  We choose our priors so that this integral is tractable,
% with help from the book Table of Integrals and Series by Gradshteyn. The
% marginal likelihoods behave like BIC.  We found closed form expressions for
% the marginal likelihood for the hyper-$g$ and hyper-$g/n$ priors.

% \subsubsection{Calculating $p(\vy | \vgamma)$ for Bayarri's robust prior}
% \begin{itemize}
% 	\item Bayarri calculated the marginal likelihood given $\vgamma$ for the Robust Bayarri prior
% 	 	in \cite{Bayarri2012} 
% 		\tiny
% 		$p(\vy | \vgamma) = K(n) \left(\frac{n + 1}{1 + p_\vgamma}\right)^{-p_\vgamma/2} \frac{(1 - R^2_\vgamma)^{-(n-1)/2}}{p_\vgamma + 1} {}_2 F_1 \left[ \frac{n-1}{2}, \frac{p_\vgamma+1}{2}; \frac{p_\vgamma + 3}{2}; \frac{[1 - 1/(1 - R^2_\vgamma)](p_\vgamma + 1)}{1 + n} \right]$.
% 	\small
% 	\item But this expression is not numerically well-behaved, because the second argument of the
% 				${}_2 F_1$ function is greater than $1$.
% 	\item Using a trick that's available when ${}_2 F_1(., 1; .; .)$ ... is numerically
% 				well-behaved, we derived the new expression
% 		\tiny
% 		$p(\vy | \vgamma) = K(n) \left(\frac{n + 1}{1 + p_\vgamma}\right)^{(n-p_\vgamma-1)/2} \frac{[1 + L (1 - R^2_\vgamma)]^{-(n-1)/2}}{p_\vgamma + 1} {}_2 F_1 \left[ \frac{n-1}{2}, 1; \frac{p_\vgamma + 3}{2}; 
% 		\frac{R^2_\vgamma}{1 + L(1 - R^2_\vgamma)} \right]$ \\
% 		\small
% 		where $L = (1 + n)/(1 + p_\vgamma) - 1$.
% \end{itemize}

% Bayarri calculated the marginal likelihood given $\vgamma$ for the Robust
% Bayarri prior.  But this expression is not numerically well-behaved.  We used
% Euler's transformation and properties of the hypergeometric function to
% derive a new, numerically stable function.


\section{Bayesian linear model averaging}
\label{sec:blma}

% We will start with the following linear model for predicting $\vy$ with
% design matrix $\mX$ via
% \begin{equation}\label{eq:linearModel}
% \ds \vy | \alpha, \vbeta, \sigma^2 \sim N_n(\vone_n\alpha + \mX \vbeta, \sigma^2 \mI_n),
% \end{equation} 

% \noindent where $\vy = (y_1,\ldots,y_n)^T$ is a response vector of length
% $n$, $\mX$ is an $n \times p$ matrix of covariates, $\alpha$ is the model
% intercept, $\vbeta$ is a coefficient vector of length $p$, $\sigma^2$ is the
% residual variance, and $\mI_n$ is the $n \times n$ identity matrix.  We
% standardize $\vy$ and $\mX$ so that $\overline{y} = 0$, $\|\vy\|^2 = \vy^T\vy
% = n$, $\overline{\mX}_j = 0$,  and $\|\mX_j\|^2 = n$ where $\mX_j$ is the
% $j$th column of $\mX$ to simplify algebra and so that predictors are in the
% same scale.

% \subsection{Prior structure on the linear model parameters} 

% We will adopt a priors structure for $\alpha$, $\vbeta$ and $\sigma^2$
% conditional a model  vector $\vgamma\in \{0, 1\}^p$ and a prior
% hyperparameter $g$  given by
% \begin{equation}
% \label{eq:priorStructure}
% \begin{array}{c}
% \ds p(\alpha) \propto 1,  
% \qquad 
% \vbeta_\vgamma | \sigma^2, g, \vgamma \sim N_p(\vzero, g \sigma^2 (\mX_\vgamma^T \mX_\vgamma)^{-1}), \\[1ex]
% \ds p(\vbeta_{-\vgamma}|\vgamma) = \prod_{j=1}^p \delta(\beta_j;0)^{1-\gamma_j}
% \quad \text{ and }  \quad 
% \ds p(\sigma^2) \propto (\sigma^2)^{-1} I(\sigma^2 > 0),                      
% \end{array}
% \end{equation} 

% \noindent where $\mX_\vgamma$ denotes the design matrix formed by including
% only the $j$th column of $\mX$ when $\gamma_j = 1$, where $\delta(x;a)$ is
% the Dirac delta function with location $a$.  The prior on $\vbeta_\vgamma$ is
% Zellner's $g$-prior \citep[see for example,][]{Zellner1986} with prior
% hyperparameter $g$. This family of priors for a Gaussian regression model
% where the prior covariance matrix of $\vbeta_\vgamma$ is taken to be a
% multiple of $g$ with the Fisher information matrix for $\vbeta$, and makes
% the marginal likelihood of the model scale-invariant.  The prior on
% $\vbeta_{-\vgamma}$ in (\ref{eq:priorStructure}) is the spike in what is
% referred to as a spike and slab prior where the prior on $\vbeta_{\vgamma}$
% is assumed to be flat (the slab).  The above structure implies that
% $p(\vbeta_{-\vgamma}|\vy)$ is a point mass at $\vzero$ and leads to algebraic
% and computational simplifications for components of $\vbeta$ when elements of
% $\vgamma$ equal 0.  Thus, $\gamma_j=0$ is equivalent to excluding the
% corresponding predictor $\mX_j$ from the model.

% Integrating out $\alpha$, $\vbeta$, and $\sigma^2$ from
% $p(\vy,\alpha,\vbeta,\sigma^2|g,\vgamma)$ we obtain
% \begin{equation}\label{eq:yGivenG}
% \begin{array}{rl}
% \ds p(\vy|g,\vgamma)
% %& \ds = \int \exp\left[
% %- \tfrac{n-1}{2}\log(2\pi) 
% %- \tfrac{1}{2}\log(n)
% %- \tfrac{p}{2}\log(1 + g)
% %- \left( \tfrac{n-1}{2} + 1\right)\log(\sigma^2) 
% %- \left( \tfrac{n}{2} \tfrac{1 + g(1-R^2)}{1 + g} \right)\sigma^{-2} 
% %\right]  d\sigma^2
% %\\
% %& 
% \ds = K(n)
% (1 + g)^{(n - p_\vgamma - 1)/2}(1 + g (1 - R_\vgamma^2))^{-(n-1)/2},
% \end{array} 
% \end{equation}

% \noindent where $K(n) = [\Gamma( (n-1)/2 )]/[\sqrt{n}(n\pi)^{(n-1)/2}]$,
% $R_\vgamma^2 =
% \vy^T\mX_\vgamma^T(\mX_\vgamma^T\mX_\vgamma)^{-1}\mX_\vgamma^T\vy/n$ is the
% usual R-squared statistic for model $\vgamma$, and $p_\vgamma$ is the number
% of variables in the model $\vgamma$.

% % Justify choice of prior

% \cite{Liang2008} considered several approaches to specifying $g$, including
% letting $g$ diverge, setting $g$ to a constant, choosing $g$ adaptively via a
% local or global empirical Bayes procedure, and placing a hyperprior should
% on $g$. They argued that for desirable inferential properties that a
% hyperprior should be placed on $g$.  For a given hyperprior on $g$ we define
% the (null based) Bayes factor for a model $\vgamma$ given by
% $$
% \ds \mbox{BF}(\vgamma) = \frac{p(\vy|\vgamma)}{p(\vy|\vzero)}.
% $$

% \noindent Note that the Bayes factor is a statistic commonly used in Bayesian
% hypothesis testing \citep{Raftery1997,OrmerodEtal2017}.  Values of BF greater
% than 1 lead to preferring the model $\vgamma$ and values less than 1 lead to
% preferring the null model $\vgamma=\vzero$, i.e., the model containing only
% the intercept. Note that when $\vgamma = \vzero$, i.e., the null model, then
% $p_\vgamma = 0$, and $R_\vgamma^2 = 0$ leading to the simplification
% $p(\vy|g,\vzero) = K(n)$ for all $g$. Hence, $p(\vy|\vzero) = K(n)$ provided
% the hyperprior on $g$ is a proper density.  

% \subsection{Hyperpriors on $g$}

% Recently, Greenaway \& Ormerod (2018) examined several hyperpriors on $g$
% considered in the literature.  These included the hyper-$g$ and hyper-$g/n$
% priors of \cite{Liang2008}, the beta-prime prior of \cite{Maruyama2011}, the
% robust prior of \cite{Bayarri2012}, and the cake prior of
% \cite{OrmerodEtal2017}. They showed how all of these, except for the
% hyper-$g/n$ priors can be evaluated an efficient, accurate and numerically
% stable manner.  We summarise their results in the list below.
% \begin{itemize}
% 	\item The {\bf hyper-$g$} prior of \cite{Liang2008} given by
% 	$p_{g}(g) = \frac{a - 2}{2}(1 + g)^{-a/2}$,
% 	for $a>2$ and $g>0$ leads to
% 	\begin{equation}\label{eq:hyperGmarginal2}
% 	\ds \mbox{BF}_g(\vgamma)
% 	=  
% 	\frac{a - 2}{2 R_\vgamma^2(1 - R_\vgamma^2)} 
% 	\frac{\mbox{pbeta}\left(R_\vgamma^2,\tfrac{p_\vgamma + a - 2}{2},\tfrac{n-p_\vgamma - a+1}{2}\right)}{
% 		\mbox{dbeta}\left(R_\vgamma^2,\tfrac{p_\vgamma + a - 2}{2},\tfrac{n-p_\vgamma - a+1}{2}\right)},
% 	\end{equation}
	
% 	where $\mbox{pbeta}(x;a,b)$ and $\mbox{dbeta}(x;a,b)$  are the cdf and pdf
% 	of the beta distribution respectively. The value $a=3$ is typically used.
% 	Note that \cite{Liang2008} derived an equivalent expression in terms of the
% 	Gaussian hypergeometric function ${}_2
% 	F_{1}(\cdot,\cdot;\cdot,\cdot;\cdot)$. The above expression is numerically
% 	easier to work with.
		
% 		\item The {\bf robust} hyperprior for $g$ proposed by \cite{Bayarri2012} is designed to have several nice theoretical
% 		properties outlined there. Using their default parameter
% 		choices, the hyperprior 
% 		on $g$ used by \cite{Bayarri2012} corresponds to 
% 		$p_{{rob}}(g) = \tfrac{1}{2}r^{1/2} (1 + g)^{-3/2} I(g>L)$  where $L = r - 1$ and $r = (1 + n)/(1 + p_\vgamma)$. 
% 		Letting $\widetilde{R}_\vgamma^2 = R_\vgamma^2/(1 + L\widehat{\sigma}_\vgamma^2)$
% 		the Bayes factor can be written as
% 		\begin{equation}\label{eq:yGivenGammaRobust2}
% 		%\begin{array}{rl}
% 		\ds \mbox{BF}_{{rob}}(\vgamma)
% 		%& \ds = \left( \frac{1 + n}{1 + p_\vgamma} \right)^{(n - p_\vgamma - 1)/2} \frac{\left( 1 + L\widehat{\sigma}_\vgamma^2 \right)^{-(n - 1)/2}}{1 + p_\vgamma}
% 		%{}_2F_1\left(  
% 		%\frac{n-1}{2}, 1; \frac{p_\vgamma+3}{2}; \frac{1 - \widehat{\sigma}_\vgamma^2}{1 + L\widehat{\sigma}_\vgamma^2}
% 		% \right)
% 		%\\
% 		%& \ds 
% 		= r^{(n - p_\vgamma - 1)/2} \frac{\left( 1 + L\widehat{\sigma}_\vgamma^2 \right)^{-(n - 1)/2}}{2 \widetilde{R}_\vgamma^2(1 - \widetilde{R}_\vgamma^2)} 
% 		\frac{
% 			\mbox{pbeta}\left( 
% 			\widetilde{R}_\vgamma^2,
% 			\frac{p_\vgamma +1}{2},
% 			\frac{n - p_\vgamma - 2}{2} 
% 			\right)
% 		}{
% 			\mbox{dbeta}\left( 
% 			\widetilde{R}_\vgamma^2,
% 			\frac{p_\vgamma +1}{2},
% 			\frac{n - p_\vgamma - 2}{2} 
% 			\right)
% 		}.
% 		%\end{array}
% 		\end{equation}
		
% 		\noindent Again, \cite{Bayarri2012} derived an equivalent expression in terms of ${}_2 F_1$, however
% 		(\ref{eq:yGivenGammaRobust2}) is numerically easier to work with.
 
		
% 	\item The {\bf beta-prime} hyperprior on $g$ employed by \cite{Maruyama2011} is given by
% 	$p_{bp}(g) = g^{b}(1 + g)^{-(a+b+2)}/\mbox{Beta}(a+1,b+1)$,
% 	where $g>0$, $a>-1$ and $b>-1$, i.e.,
% 	$g\sim \mbox{Beta-Prime}(b+1,a+1)$ using the typical parametrization of 
% 	the beta-prime distribution \citep{Johnson1995}. When	 $b = (n - p_\vgamma - 5)/2 - a$ 
% 	the corresponding Bayes factor simplifies to
% 	\begin{equation}\label{eq:marginalLikelihoodBetaPrime}
% 	\begin{array}{rl}
% 	\ds \mbox{BF}_{bp}(\vgamma) 
% 	%& \ds =
% 	%\frac{K(n)}{\mbox{Beta}(a+1,b+1)}
% 	%\int_0^\infty g^{b} \left[ 1 + g(1-R^2) \right]^{-(n-1)/2}  
% 	%dg
% 	%\\ [2ex]
% 	& \ds 
% 	=   
% 	\frac{\mbox{Beta}(p/2 + a + 1,b + 1)}{\mbox{Beta}(a+1,b+1)} (1-R_\vgamma^2)^{-(b + 1)}
% 	%\\ [2ex]
% 	%& \ds = \widetilde{K}(n,a)
% 	%
% 	%\Gamma(p/2 + a + 1)\Gamma(a + b + 2)
% 	%(\widehat{\sigma}^2)^{-(b + 1)}
% 	\end{array}
% 	\end{equation}	
	
% 	\noindent where $\mbox{Beta}(u,v) = \Gamma(u)\Gamma(v)/\Gamma(u+v)$ is the beta function.
% 	The default value for $a$ chosen by 
% 	\cite{Maruyama2011} is $a = -3/4$. Note that the prior used in \cite{Maruyama2011} for
% 	$\vbeta_\vgamma$ is different than from what we use here. Greenaway \& Ormerod (2018) discuss
% 	these differences and justify our choice.
	
% 	\item The {\bf cake} prior of \cite{OrmerodEtal2017} 
% 	departs from the prior structure (\ref{eq:priorStructure}) and instead uses
% 	\begin{equation}\label{eq:proirs2}
% 	\begin{array}{c}
% 	\ds \alpha|\sigma^2,g \sim N(0,g\sigma^2), \quad 
% 	\ds \vbeta_\vgamma|\sigma^2,g \sim N\left( \vzero,g\sigma^2\left( \tfrac{1}{n}\mX_\vgamma^T\mX_\vgamma\right)^{-1}\right)
% 	\\ \mbox{and} \quad
% 	p(g|\vgamma_j) = \delta(g; h^{1/(1 + p_\vgamma)})
% 	\end{array} 
% 	\end{equation}
	
% 	\noindent where $h$ is a common prior hyperparameter for all models. This is slightly
% 	different from (\ref{eq:priorStructure}). The log   Bayes factor
% 	as a function of $h$ 
% 	for model $\vgamma$ is of the form
% 	$$
% 	\begin{array}{rl}
% 	\ds \log\mbox{BF}(\vgamma;h)
% 	=
% 	-\tfrac{n}{2}\log\left( 1 - \tfrac{h^{1/(1+p_\vgamma)}}{1+h^{1/(1+p_\vgamma)}} R_\vgamma^2 \right) 
% 	- \tfrac{p_\vgamma}{2}\log\left(n + h^{-1/(1+p_\vgamma)} \right).
% 	\end{array}
% 	$$
	
% 	\noindent Taking $h\to\infty$ we obtain a null based Bayes factor approaches
% 	\begin{equation}\label{eq:marginalLikelihoodCake}
% 	\ds \mbox{BF}_{\mbox{\scriptsize cake}}(\vgamma)
% 	=
% 	\exp\left[ \,
% 	-\tfrac{n}{2}\log\left( 1 - R_\vgamma^2 \right) 
% 	- \tfrac{p_\vgamma}{2}\log\left(n \right) \,
% 	\right] = \exp\left[ \, -\tfrac{1}{2}\mbox{BIC}(\vgamma) \,\right]
% 	\end{equation}
	
% 	\noindent where $\mbox{BIC}(\vgamma) = n\log\left( 1 - R_\vgamma^2 \right) + p_\vgamma \log(n)$ is the Bayesian 
% 	Information of \cite{Schwarz1978} modulo constants which only depend on the sample size (and so are common to all models). 
% \end{itemize}

% \noindent All of the above Bayes factors can be evaluated on the log-scale
% using standard software avoiding numerical overflow and stability issues.
% All of the above hyperpriors on $g$ lead to model selection consistency under
% mild assumptions, except for the hyper-$g$ prior which is only model
% selection consistent when the true model is not the null model. 

% \subsection{Prior on the model space/size}

% The last ingredient to a fully Bayesian model specification is the prior on
% $\vgamma$, sometimes referred to as a prior on the model space, or model
% size.  As the dimension of the problem increases the choice of model prior
% becomes increasingly important. The two most commonly used priors are to use
% a uniform prior on the model space and the beta-binomial probability on the
% model size.  The choices of prior for the model space/size has been discussed
% in detail by \cite{scott2010} and are also discussed in \cite{castillo2015}.

% A uniform prior on the space of all models puts an equal prior probability on
% each model, i.e., 
% $$
% \ds p(\vgamma) = 2^{-p} \qquad \mbox{for all $\vgamma\in\{0,1\}^p$},
% $$

% \noindent which is equivalent to $p(\gamma_j) = 1/2$, $1\le j\le p$. While
% this model prior has the appeal of being flat or uninformative, it is a poor
% choice as a default prior for high dimensional problems. \cite{scott2010}
% state that this model prior provides no multiplicity control in a multiple
% testing setting, and uses an a priori model size of $p/2$ with a standard
% deviation of $\sqrt{p}/2$ leading to an a priori large fraction of covariates
% being included when $p$ is large. We have found that this choice of prior
% works poorly on high dimensional problems in practice.

% The beta-binomial prior on the model space uses a prior of the form
% $$
% \ds p(\vgamma) = \prod_{j=1}^p \rho^{\gamma_j} (1 - \rho)^{1 - \gamma_j} \qquad \mbox{and} \qquad \rho \sim \mbox{Beta}(a,b),
% $$

% \noindent where $\rho$ is the prior probability a variable is included in the
% mode, and $a$ and $b$ are fixed prior hyperparameters. After marginalizing
% out $\rho$ we have
% \begin{equation}\label{eq:betabinomial}
% \ds p(\vgamma) = \frac{\mbox{Beta}(a + p_\vgamma,b + p - p_\vgamma)}{\mbox{Beta}(a,b)},
% \end{equation} 

% \noindent which is a beta-binomial distribution on the model size. Note
% $a=b=1$ corresponds to a uniform prior on the prior variable inclusion
% probability, and is quite different to placing a uniform prior on the set of
% all models.


% Recently, \cite{castillo2015} introduce the concept of ``complexity priors''
% of the form $p(\vgamma) \propto c^{-p_\vgamma} p^{-a p_\vgamma}$ for
% constants $a, c>0$. Such a prior implies an exponential decrease in the prior
% model size as the model dimension increases, and is required for desirable
% model selection properties. They note that a convenient complexity prior has
% the form 
% $$
% \ds p(\vgamma) = \prod_{j=1}^p \rho^{\gamma_j} (1 - \rho)^{1 - \gamma_j} \qquad \mbox{and} \qquad \rho \sim \mbox{Beta}(1,p^u),
% $$

% \noindent for some constant $u>1$. They call this prior universal because it
% is free of unknown smoothing parameters. The above theory motivates our
% choice of using a beta-binomial prior with $a=1$ and $b=p$, which we have
% found to work well on high-dimensional problems in practice.

\subsection{Bayesian model averaging}


The Bayes factors introduced in the previous chapter play a key role in
Bayesian linear model averaging.  Via Bayes theorem the posterior probability
of a model is given by
$$
\ds p(\vgamma|\vy) = \frac{p(\vy|\vgamma)p(\vgamma)}{\sum_{\vgamma'} p(\vy|\vgamma')p(\vgamma')} = \frac{p(\vgamma)\mbox{BF}(\vgamma)}{\sum_{\vgamma'} p(\vgamma')\mbox{BF}(\vgamma')}
$$

\noindent where $\sum_{\vgamma}$ denotes a combinatorial sum over all $2^p$
possible values of $\vgamma$, and $p(\vgamma)$ is the chosen prior on
$\vgamma$. Numerical overflow can be avoided by dividing through the numerator
and denominator of $p(\vgamma|\vy)$ by the largest product
$p(\vgamma)\mbox{BF}(\vgamma)$ and performing calculations  on the log scale.
The posterior expectation of $\vgamma$ is given by $\bE(\vgamma|\vy) =
\sum_{\vgamma} \vgamma \cdot p(\vgamma|\vy)$.  The median posterior model  is
obtained by rounding $\bE(\vgamma|\vy)$ to the nearest integer and has
desirable optimality properties \citep{Barbieri2004}.
 
% Why is this here?
% \section{Variational approximation}
% \label{sec:vb}

% Variational approximation is now a relatively commonly used technique used in
% Statistics and Machine Learning.  Variational approximation is a general set
% of techniques where a problem is augmented via the introduction of extra
% degrees of freedom known as {\it variational parameters} in such a way that
% the augmented problem has a closed form solution. Once such a closed form is
% found, optimizing the augmented problem over the variational parameters leads
% to an approximate solution to the original problem.  While these techniques
% can be applied in frequentist settings \citep{Ormerod2012}, they are
% typically used in Bayesian inference as fast, albeit approximate,
% alternatives to MCMC. Relatively accessible material to this area can be
% found in \cite{Bishop2006}, \cite{Ormerod2010}, and \cite{Grimmer2011}. For a
% recent review in the area see \cite{BleiEtal2017}.

% Mean field variational Bayes or simply variational Bayes (VB) is a particular
% varational approximation based on minimizing the Kullback-Leibler distance
% between the true posterior distribution and a factorized approximation to the
% posterior. If we let $\vtheta$ be the set of all model parameters and $\vy$
% be a vector of observed data then $p(\vtheta|\vy)$ is approximated by
% $q(\vtheta) = \prod_{k=1}^K q_k(\vtheta_k)$ where
% $(\vtheta_1,\ldots,\vtheta_K)$ is a partition of $\vtheta$. It can	be
% shown that the $q_k$ densities, called $q$-densities, which mimimize the
% Kullback-Leibler divergence between $p(\vtheta|\vy)$ and $q(\vtheta)$ satisfy
% \begin{equation}\label{eq:optimal_q}
% \ds q_k(\vtheta_k) \propto \exp[ \bE_{-q_k(\vtheta_k)} \{ \log p(\vy,\vtheta) \}], \qquad 1\le k\le K,
% \end{equation}

% \noindent where $\bE_{-q_k(\vtheta_k)}$ is the expectation with respect to
% all densities except $q_k(\vtheta_k)$. For any fixed $q(\vtheta)$ a lower
% bound for the marginal likelihood for $\vy$ can be obtained by 
% $$
% \ds \log \underline{p}_{\mbox{\scriptsize }}(\vy) = \bE_q\left[ \log \left\{ \frac{p(\vy,\vtheta)}{q(\vtheta)} \right\} \right],
% $$ 

% \noindent where the	underline is used to indicate the quantity is a lower
% bound on the true value of that quantity. It can be shown that updating $q_k$
% via (\ref{eq:optimal_q}) for fixed forms of the remaining $q$-densities
% results in an increase in the lower bound $\log
% \underline{p}_{\mbox{\scriptsize }}(\vy)$. Cycling through the updates	for
% each $q_k$ can be interpreted as a coordinate ascent method for maximizing
% $\log \underline{p}_{\mbox{\scriptsize }}(\vy)$ which, under mild regularity
% conditions,  will converge to a local maximizer of the lower bound
% \citep{LuenbergerYe2008}. 

% For particular models the normalizing constants associated with the densities
% (\ref{eq:optimal_q}) cannot be found analytically. In such situations an
% alternative to the above approach is to specify the $q$-densities  to have
% specific parametric forms, e.g., $q_k(\vtheta_k) = \phi_{\mSigma_k}(\vtheta -
% \vmu_k)$ (which denotes a multivariate Gaussian density with mean $\vmu_k$
% and covariance matrix $\mSigma_k$). In this case the  lower bound for the
% marginal likelihood is parametrized by the variational parameters $\vmu_k$
% and $\mSigma_k$. The lower bound can then be maximized with respect to the
% $\vmu_k$ and $\mSigma_k$ values, again tightening the bound between the true
% value of $\log p(\vy)$ and its lower bound, improving the approximation. 

% When the $q$-densities satisfy (\ref{eq:optimal_q}) the form of the
% $q$-densities are chosen through  (\ref{eq:optimal_q}) is a nonparametric VB
% approach, whereas when $q_k(\vtheta_k)$ is chosen to have a particular
% parametric form this can be viewed as a type of parametric VB.  % Relevance?
% Recently, \cite{Wand2017} considered a general framework for combining these
% methods and showed how a general message passing algorithm could be developed
% using these ideas.

%\subsection{Particle Variational Approximation}

%A variant of VB, called collapsed VB, was first coined by \cite{Teh2006} in
%the context of latent Dirichlet allocation \citep{Teh2006}. It has also been
%used in hierarchical Dirichlet process \citep{Teh2008} and hidden Markov
%models \cite{wang2013}.  The key idea behind these methods is to collapse, or
%integrate out analytically a subset of parameters before applying VB
%methodology. This often results in improved approximation, but is in a sense
%no different conceptually.  We now introduce a reverse form of the collapsed
%VB of \cite{Teh2006}. In this form the lower bound is calculated by using VB
%for one set of parameters, and the remaining set of parameters are collapsed
%over.

%To fix ideas suppose that
%$\vtheta_1$ and $\vtheta_2$ be a partition of the %parameter vector $\vtheta$. 
%Particle Variational Bayes (PVA) seeks to integrate %out either $\vtheta_1$ or 
%$\vtheta_2$ analytically. Suppose that we can %integrate out $\vtheta_1$ analytically.
%Then the first step of PVA is to calculate the lower bound
%$$
%\log p(\vy,\vtheta_1) 
%\ge \log \underline{p}(\vy,\vtheta_1)
%= \int q(\vtheta_2) \log\left\{ \frac{p(\vy,\vtheta_1,\vtheta_2)}{q(\vtheta_2)}\right\} d\vtheta_2
%$$

%\noindent The second step is to integrate out %$\vtheta_1$ to obtain
%$$
%\begin{array}{rl}
%\ds \log \underline{p}_{\mbox{\tiny PVA}}(\vy)
%& \ds \ge \log \int  \underline{p}(\vy,\vtheta_1) %d\vtheta_1
%\\
%& \ds = \log \int \exp\left[ \int q(\vtheta_2) \log\left\{ \frac{p(\vy,\vtheta_1,\vtheta_2)}{q(\vtheta_2)}\right\} d\vtheta_2 \right] d\vtheta_1.
%\end{array}
%$$

%\noindent 
%Note that it is easy to show that
%$\log \underline{p}_{\mbox{\tiny PVA}}(\vy) \ge 
%\log \underline{p}_{\mbox{\tiny VB}}(\vy)$ (due to %Jensen's inequality).

%{\color{red}
%Now suppose that we have partitioned $\vtheta$ into %three sets of parameters $\vtheta_1$, 
%$\vtheta_2$ and $\vtheta_3$, and we want to apply %VB-type approximations
%to $\vtheta_2$ and $\vtheta_3$ while integrating out %$\vtheta_1$
%analytically. The iterations for PVA algorithms
%would be to repeat the following two steps until convergence:
%\begin{enumerate}
%	\item 
%	$\ds
%	q(\vtheta_2) \propto \int \exp\left[ \int q(\vtheta_3) \log\left\{ \frac{p(\vy,\vtheta_1,\vtheta_2,\vtheta_3)}{q(\vtheta_3)}\right\} d\vtheta_3 \right] d\vtheta_1.
%	$
	
%	\item 
%	$\ds
%	q(\vtheta_3) \propto \int \exp\left[ \int q(\vtheta_2) \log\left\{ \frac{p(\vy,\vtheta_1,\vtheta_2,\vtheta_3)}{q(\vtheta_2)}\right\} d\vtheta_2 \right] d\vtheta_1.
%	$
%\end{enumerate}

%\noindent Upon convergence calculate
%$$
%\ds \begin{array}{rl}
%q(\vtheta_1) 
%& \ds \propto 
%\exp\left[ \int q(\vtheta_2)q(\vtheta_3) \log\left\{ \frac{p(\vy,\vtheta_1,\vtheta_2,\vtheta_3)}{q(\vtheta_2)q(\vtheta_3)}\right\} d\vtheta_2 d\vtheta_3  \right] \quad \mbox{and}
%\\ [2ex]
%\ds \log \underline{p}_{\mbox{\tiny PVA}}(\vy) 
%& \ds  = \log \int \exp\left[ \int %q(\vtheta_2)q(\vtheta_3) \log\left\{ \frac{p(\vy,\vtheta_1,\vtheta_2,\vtheta_3)}{q(\vtheta_2)q(\vtheta_3)}\right\} 
%d\vtheta_2 d\vtheta_3  \right] d\vtheta_1.
%\end{array}
%$$
%\noindent 
%	The above ideas are easily generalizable to an arbitrary 
%	partition size.}
%Finally, it is also possible to choose one of the above $q$-densities
%to have specific parametric forms. In the next section we employ such
%a strategy.

\section{Particle based Variational Approximation}
\label{sec:pb-pva}

We will now present a population based variational collapsed Bayes
approximation (PVA) approach to model selection which is more appropriate to
use when $p$ is larger than, say, around $30$. This approach is closely related
to the PEM  method of Ro\v{c}kov\'{a} (2017), where the main difference being
that here we work in a fully Bayesian framework and we consider a wider range
of prior structure specifications.
% A comparison between our approach and PEM will be given in Section
% \ref{sec:pem}.

The marginal likelihood for $\vy$ is given by
$$
\begin{array}{rl}
\ds p(\vy) 
& \ds = \sum_{\vgamma} \left[ \int p(\vy,\alpha,\vbeta,\sigma^2,g|\vgamma) p(\alpha,\vbeta,\sigma^2,g|\vgamma) \, d\alpha \, d\vbeta \, d\sigma^2 \, dg\right] p(\vgamma) 
\\
& \ds = \sum_{\vgamma} p(\vy,\vgamma),
\end{array} 
$$

\noindent which is generic to the  prior distribution specification for
$p(\alpha,\vbeta,\sigma^2,g|\vgamma)$ and $p(\vgamma)$.  Our approach is to
collapse over $\alpha$, $\vbeta$, $\sigma^2$, and $g$, and then use a
variational approximation to the posterior likelihood of the model $\vgamma$.
This can be done using any combination of the prior specifications described in
Section \ref{sec:blma}.  This is conceptually equivalent to the collapsed
variational approximation technique developed by \cite{Teh2006}  who used the
concept of collapsing over a subset of variables in the context of  Latent
Dirichlet Allocation models.

We specify the $q$-density for $\vgamma$ parametrically by
\begin{equation}\label{eq:qgamma} 
q(\vgamma) = \sum_{k=1}^K w_k I(\vgamma = \vgamma_k)
\end{equation} 

\noindent where $0 < w_k \le 1$, $\sum_{k=1}^K w_k = 1$, $\mGamma =
[\vgamma_1,\ldots,\vgamma_K]$ is a population of models (with individual
$\vgamma_k$ referred to as particles), and $I(\,\cdot\,)$ is the indicator
function. Here $\vw$ and $\mGamma$ are variational parameters of the
probability mass function $q(\vgamma)$.

Using $q(\vgamma)$ we derive the following variational lower bound on $\log
p(\vy)$ via
\begin{equation}
\label{eq:variationalLowerBound}
\begin{array}{rl}
\ds \log p(\vy) 
%& \ds = \log \left[ \sum_{\vgamma} p(\vy,\vgamma) \right]
%\\
& \ds = \log \left[ \sum_{\vgamma} q(\vgamma) \left\{  \frac{p(\vy,\vgamma)}{q(\vgamma) } \right\} \right]
\\
& \ds \ge \sum_{\vgamma} q(\vgamma) \log p(\vy,\vgamma) 
- \sum_{\vgamma} q(\vgamma) \log q(\vgamma)
\\
& \ds \equiv \log \underline{p}(\vy;\vw,\mGamma)
\end{array}
\end{equation}

\noindent where going from the first to the second line of
(\ref{eq:variationalLowerBound}) is obtained using Jensen's inequality.
Maximizing the right hand  of (\ref{eq:variationalLowerBound}) tightens the
bound improving the quality of the approximation
$\underline{p}(\vy;\vw,\mGamma)$ to $p(\vy)$. It can be shown that the
difference between $\log p(\vy)$ and $\log \underline{p}(\vy;\vw,\mGamma)$ is
the Kullback-Leibler divergence between $p(\vgamma|\vy)$ and $q(\vgamma)$.

The second term of (\ref{eq:variationalLowerBound}) is related to the entropy
of $q$. Following \cite{Rockova2017} the variational lower bound for $\log
p(\vy)$ is given by
\begin{equation}\label{eq:objective}
\ds \log \underline{p}(\vy;\vw,\mGamma) 
= \sum_{k=1}^K w_k\log p(\vy,\vgamma_k) - w_k \log w_k
\end{equation}

\noindent which has been simplified under the assumption that population of
particles $\vgamma_1,\ldots,\vgamma_K$ contains only unique particles.

Since $\log \underline{p}(\vy;\vw,\mGamma)$ is a lower bound we can maximize
this bound with respect to $\vw$ and $\mGamma$ to make the bound as tight as
possible.  The main body of the algorithm to optimize $\log
\underline{p}(\vy;\vw,\mGamma)$ is a two--stage process. 

In the first stage, we iterate through the population of bitstrings, using a
greedy search strategy in an attempt to alter each bit in the model bitstring
to increase the log likelihood. If the log likelihood for the new bitstring is
no higher than the previous bitstring, then the alteration is rejected and the
next alteration tried. The alterations are also rejected if the new bitstring
already exists within the population, ensuring that the constraint that all
models in the population are unique is maintained.

In the second stage, we re--calculate the weights for each individual in the
population, based on the likelihood of that model relative to the data $p(\vy;
\vbeta_\vgamma)$ and use this to re--calculate the probability--based weights
$w_i$ for each bitstring in the population. This is then used to re--calculate
the lower bound
\[
	\log \underline{p}(\vy; \vw, \Gamma) = \sum_{k=1}^K \vw_k \log p(\vy; \vbeta_{\vgamma_k}) - \vw_k \log \vw_k
\]

which is the sum of the weighted log-likelihood of the population and the
entropy of the probability weights.  These two stages repeat until the lower
bound converges.

Note that for fixed $\vw$ each of the $\vgamma_k$'s can be optimized
independently since (\ref{eq:objective}) is an additive function of
$\vgamma_k$'s. Hence, the first stage optimizes $\log
\underline{p}(\vy;\vw,\mGamma)$ with respect to $\mGamma$ in a greedy search
over each of the  $\vgamma_k$'s.  To be more concrete, let $\vgamma_{jk}^{(i)}
= (\gamma_{1k},\ldots,\gamma_{j-1,k},i,\gamma_{j+1,k},\ldots,\gamma_{pk})^T$.
We optimize $\vw$ and $\vgamma_1,\ldots,\vgamma_K$ by executing the algorithm
given in Algorithm \ref{alg:algorithm_pva} below. $\vp$ is the vector of
posterior probabilities for each of the models  in the population, while $H$ is
the entropy of the entire population. Thus $F_\lambda$ balances the weighted
posterior  probabilities of the particles in the population against the
diversity within that population.
% Re-type this using algorithmic environment

\begin{algorithm}\label{alg:updateGamma}
	\caption{The PVA algorithm}
	\label{alg:algorithm_pva}
	\begin{algorithmic}
		\WHILE{$F_\lambda$ is still different from the previous iteration}
			\FOR{$k = 1, \ldots, K$}
				\FOR{$j = 1, \ldots, p$}
					\IF {$p(\vy, \vgamma_{jk}^{(1)}) > p(\vy, \vgamma_{jk}^{(0)})$}
						\STATE $\vgamma_{jk} = 1$
					\ELSE
						\STATE $\vgamma_{jk} = 0$
					\ENDIF
				\ENDFOR
				\STATE $w_k = p(\vy, \vgamma_k) / \sum_{j=1}^K p(\vy, \vgamma_j)$
			\ENDFOR
			\STATE Calculate $F_\lambda$ = $\vw^\top \vp + \lambda H$
		\ENDWHILE
	\end{algorithmic}
\end{algorithm}

    \noindent 
    %{\color{red} where an update is only performed if the resulting population
    %is unique.} 
    Since only one component is modified during each iteration of the inner
    loop of the algorithm, model updates and downdates can be efficiently used
    to implement the Algorithm (\ref{alg:algorithm_pva}) (for details see
    Section 5.1 of Greenaway \& Ormerod, 2018).
    %{\color{red} Population uniqueness can be maintained at a average cost of
    %$O(K)$ whenever we try to change $\gamma_{jk}$ using hashtables.}

\noindent Convergence is declared for a particular particle when no element of
the particle is updated over $j=1,\ldots,p$. Convergence of the algorithm is
declared when all particles have been converged. Note that optimization over
each of the $\vgamma_k$'s can be performed independently and as such
implemented in an embarrassingly parallel manner.  There is no need to
re-optimize $\mGamma$ for different $\vw$ since  the optimal values of the
$\vgamma_k$'s are independent of $\vw$.

Once the matrix $\mGamma$ is fitted duplicate particles can be discarded.  Let
${\vgamma}_1^*,\ldots,{\vgamma}_{K^*}^*$ denote the selected set of $K^*$
unique particles. Then the optimal value of the $w_k$'s satisfy
\begin{equation}
\label{eq:updateW}
\begin{array}{rl}
\ds w_k 
& \ds = \frac{\ds p(\vy,\vgamma_k^*)}{\sum_{j=1}^{K^*} p(\vy,\vgamma_j^*)}
= \frac{p(\vgamma_k^*)\mbox{BF}(\vgamma_k^*)
}{\sum_{j=1}^{K^*}
	p(\vgamma_j^*)\mbox{BF}(\vgamma_j^*)
}, \qquad 1\le k\le K^*.
\end{array}
\end{equation}

\noindent The approximate posterior inclusion probabilities, which we will
denote $\vomega$ can be calculated using $\vomega = \sum_{k=1}^{K^*} w_k
\vgamma_k^*$.  The median posterior model can be obtained by rounding the
elements of $\vomega$.

The main requirement of the above strategy is that closed form expressions for
$\mbox{BF}(\vgamma_k)$, $1\le k\le K$ are needed, or at least approximated in
some way.  Different specifications of the prior distributions lead to
different approximations of exact Bayesian model averaging.

We implement the above algorithm in {\tt C++} which we developed into an {\tt
R} package we call  {\tt BLMA}.  The internals of {\tt BLMA} are implemented in
{\tt C++} and use the {\tt R} packages \texttt{Rcpp} and \texttt{RcppEigen} to
enhance computational performance. The library {\tt OpenMP} was used to exploit
parallel computation.
% and the {\tt Standard Template Library}
%implementation of hashtables was used.

\section{Numerical results}
\label{sec:numerical}

We will now assess the performance of PVA. In Section \ref{sec:exact} we
compare PVA against exact Bayesian model averaging for four small examples with
$p<30$ against exact Bayesian model averaging via the {\tt R} package {\tt
blma} using the implementation outlined in Greenaway \& Ormerod (2018). All of
the following results were obtained in the {\tt R} version 3.4.2 \citep{CiteR}
and all figures were developed using the {\tt R} package {\tt ggplot2}.  In
sections \ref{sec:highdimensional}--\ref{sec:QTL} we will consider examples
with $p>30$ where it is infeasible to perform Bayesian model averaging exactly.
Most simulations were run on the second author's laptop computer (64 bit
Windows 10 Intel i7-7600MX central processing unit at 2.8GHz with 2
hyperthreaded cores and 32GB of random access memory).  Multicore comparisons
were run on a dedicated server using E5-2697v2 processors with 24 hyperthreaded
cores and 512GB of RAM.

\subsection{Comparing PVA against exact results} 
\label{sec:exact}

We considered several small datasets to illustrate our methodology for
situations where we could compare PVA against a gold standard. These datasets
can be found in the {\tt R} packages {\tt MASS} \citep{Venables2002},  {\tt
ISLR} \cite{James:2014:ISL:2517747} and {\tt Ecdat} \citep{Croissant2016}.
Table \ref{tab:cva_datasets} summarizes the sizes,  sources, and response
variable for each dataset used.   For each of the datasets some minimal
preprocessing was used.  We first used the {\tt R} command {\tt na.omit()} to
remove samples containing missing predictors.  For {\tt USCrime} all variables
except the predictor {\tt S} were log-transformed. For all datasets the {\tt R}
command {\tt model.matrix()} was used to construct the design matrix using all
variables except for the response as predictors.

\begin{table}[ht!]
	\begin{center}
		\begin{tabular}{l|r|r|l|l}
			Dataset	& $n$ & $p$ & Response & {\tt R} package \\ 
			\hline 
			UScrime 	& 47 & 15 &  {\tt y} & {\tt MASS} \\  
			%Bodyfat	& 244  & 13 &  \\ 
			%	\hline 
			College &  777   & 17      &  {\tt Grad.Rate}      & {\tt ISLR} \\ 
			Hitters	& 263 & 19 & {\tt Salary} & {\tt ISLR} \\ 
			%	\hline 
			%Wage	& 3000 & 17 &  {\tt ISLR}  \\
			%VietNamI	& 27765 & 11 & lnhhexp & {\tt Ecdat}  \\ 
			Kakadu	& 1827 & 22 & {\tt income} & {\tt Ecdat}   \\  
		\end{tabular} 
	\end{center}
	\caption{A summary of the datasets used in the paper and their respective {\tt R} packages.}
	\label{tab:cva_datasets}
\end{table}

\noindent To measure the quality of approximation of PVA to BMA we will use two
metrics. The total posterior mass (TPM), and the mean marginal variable error
(MMVE). These are given by $$ \ds \mbox{TPM} = \sum_{k=1}^{K^*}
p(\vgamma_k^*|\vy) \qquad \mbox{and}  \qquad \mbox{MMVE} = \frac{1}{p}
\sum_{j=1}^p |\omega_j - \bE(\gamma_j|\vy) |.  $$

\noindent Note that the quantities $p(\vgamma_k^*|\vy)$ and $\bE(\gamma_j|\vy)$
are available as outputs of the function {\tt blma()} from the {\tt R} package
{\tt blma}.  The average values of TPM and MMVE over 100 random initial values
of $\mGamma$ for each of the datasets where independently $\gamma_{kj} \sim
\mbox{Bernoulli}(1/10)$, $1\le j\le p$, $1\le k\le K$ over a grid of $K$ values
from $K=25$ to $K=500$ are summarised in Figure
\ref{fig:kakadu_total_posterior_mass}.  From this figure we see that both TPM
and MMVE increase and decrease with $K$ respectively. For each of the dataset
at least 50\% of the total posterior mass is captured with less than $K=200$
particles.  The mean absolute error in posterior inclusion probability with
this value of $K$ is roughly $0.05$ which indicates that the median posterior
model is reasonably well approximated.

% \begin{figure}[h!]
	
	% \includegraphics[width=0.48\textwidth]{./plotUScrime.pdf} \includegraphics[width=0.48\textwidth]{./plotCollege.pdf}
	
	% \includegraphics[width=0.48\textwidth]{./plotHitters.pdf} \includegraphics[width=0.48\textwidth]{./plotKakadu.pdf}
	
	
	% \caption{The total posterior mass (TPM) and  mean marginal variable error (MMVE) for four datasets
	% averaged over a 100 random generations of $\mGamma$ with $\gamma_{kj} \sim \mbox{Bernoulli}(1/10)$, $1\le j\le p$,
	% $1\le k\le K$ over a grid of $K$ values from $K=25$ to $K=500$.}
	% \label{fig:PVA_posterior_models}
% \end{figure}

\subsection{Competing method settings}
\label{sec:settings} 

For datasets with $p>30$ it is not feasible to perform exact BMA.  For these
examples we instead compare the model selection performance of PVA against the
Lasso, SCAD and MCP penalized regression methods as implemented by the {\tt R}
package {\tt ncvreg} \citep{Breheny2011}, PEM using the {\tt R} package {\tt
PEM} (obtained via personal communication with Veronika Ro\v{c}kov\'a), and
Bayesian model averaging via MCMC using the {\tt R} package {\tt BAS}.  The
setting are implied by the {\tt R} commands below.
\begin{itemize}

    \item {\bf Penalized regression via {\tt ncvreg} package.} We used the
        following command.
	\begin{verbatim}
	ncvreg(mX,vy,penalty=penalty)
	\end{verbatim}
	
    \noindent where {\tt penalty} is {\tt "MCP"}, {\tt "SCAD"} or {\tt "lasso"}
    corresponding to the penalties of the same name as described in
    \cite{Breheny2011}.  For these methods we make use of the extended Bayesian
    information criteria (EBIC) \citep{Chen2008} to choose the tuning parameter
    $\lambda$. The EBIC minimizes
	$$
	\mbox{EBIC}(\lambda) = n\log(\mbox{RSS}_\lambda/n) + d_\lambda
	\left[ \log(n) + 2\log(p) \right],
	$$
	
    \noindent where $\mbox{RSS}_\lambda$ is the estimated residual sum of
    squares $\|\vy - \mX\widehat{\vbeta}_\lambda\|^2$,
    $\widehat{\vbeta}_\lambda$ is the estimated value of $\vbeta$ for a
    particular value of $\lambda$ and $d_\lambda$ is the number of non-zero
    elements of $\widehat{\vbeta}_\lambda$.  This differs from the regular BIC
    by an addition of a $2 d_\lambda \log(p)$ term.  \cite{Wang2007} showed
    that this criterion performs well in several contexts.
	
    \item {\bf Particle EM via the {\tt PEM} package.} We used the following
        command.
	\begin{verbatim}
	PEM(vy, mX, v0, v1, type="betabinomial", penalty="entropy",
	    epsilon=1.0E-5, theta=0.5, a=1, b=p, alpha=1, current=t(mGamma), 
	    weights="FALSE")
	\end{verbatim}
	
    \noindent where $\gamma_{kj} \sim \mbox{Bernoulli}(\rho)$, $1\le j\le p$,
    $1\le k\le K$ with $K=200$ (noting that the initial population matrix in
    PVA is the transpose of the initial population matrix used by PEM). The
    choices used for {\tt rho}, {\tt v0} and {\tt v1} are different for each
    dataset and are described in each of the sections below.
	
	\item {\bf MCMC via the {\tt BAS} package}: We used the following command.
	\begin{verbatim}
	bas.lm(vy~mX, prior="g-prior", modelprior=uniform(), 
    initprobs="uniform", MCMC.iterations=1e7)
	\end{verbatim}
	
    The estimated median posterior model is used for the purposes of model
    selection.
	
\end{itemize}

We used simulated data in each of the sections below in such a way that the
true data generating model was known. We used the $F_1$-score
\citep[see][]{Van_Rijsbergen1979} to assess the quality of model selection for
each of the above models, which is defined to be the harmonic mean between
precision and recall given by
$$
\mbox{F}_1 = \frac{2\cdot TP}{2\cdot TP + FP + FN}
$$

\noindent with $TP$, $FP$ and $FN$ being the number of true positives, false
positives and false negatives respectively. Note that $F_1$ is a value between
0 and 1 and higher values are being preferred. We use this measure avoid
preference of the two boundary models, that is selecting none or all of the
variables. 

\subsection{Simulated high-dimensional example}
\label{sec:highdimensional} 

We first present an example where $n > p$ and $p$ is relatively small ($p =
12$), to allow for the full enumeration of the model space. Later, we show an
example for the important $p > n$ case. We compare our results  against the
Lasso \citep{Tibshirani1996}, SCAD \citep{Fan2001}, MCP \citep{Zhang2010}, {\tt
BMS} \citep{Zeugner2015} and {\tt VARBVS} \citep{Carbonetto2011} algorithms.

Our first numerical experiment is designed to show that our algorithm
successfully finds the posterior models of high probability, overcoming the
difficulties of optimising over the multi-modal spike-and-slab posterior.  This
example is taken from \citep{Rockova2017}.  We consider a random sample of $n =
50$ observations on $p = 12$ predictors. $\mX_i \sim N_p(\vzero, \mSigma)$ for
$i = 1, \ldots, n$ where $\mSigma = \text{bdiag}(\mSigma_1, \mSigma_1,
\mSigma_1, \mSigma_1)$ with $\mSigma_1 = (\sigma_{ij})_{i, j = 1}^{3, 3}$ where
$\sigma_{ij} = 0.9$ for $i \ne j$ and $\sigma_{ii} = 1$.  The true model is
$\vbeta_0 = (1.3, 0, 0, 1.3, 0, 0, 1.3, 0, 0, 1.3, 0, 0)^\top$.  The responses
are then generated from $\vy = \mX \vbeta_0 + \vepsilon$, where $\vepsilon \sim
N_n(\vzero, \mI_n)$.

A comparison of the performance of PVA for the hyper-$g$, robust Bayarri,
Beta-prime and Cake priors on $g$ and the uniform, beta-binomial(1, 1) and
beta-binomial(1, p) priors on $\vgamma$ using $F_1$ score is presented in
Figure \ref{fig:highDimPVA_F1}. A comparison of the performance of the MCP,
SCAD, lasso, PVA, BMS, BAS and PEM  methods using $F_1$ score is given in
Figure \ref{fig:highDimPVA_F1_compare}.

\begin{figure}[h!]
	\begin{center}
		\includegraphics[width=0.95\textwidth]{./highDimPVA_F1.pdf}  
	\end{center}
	\caption{Comparison of the performance of the PVA method on the
						high-dimensional data set with different $g$ and $\vgamma$ priors using $F_1$ score.
						The hyper-$g$,
						robust Bayarri, Beta-prime and Cake priors on $g$ and the uniform, beta-binomial(1, 1) and beta-binomial(1, p) priors on $\vgamma$ are used.}
	\label{fig:highDimPVA_F1}
\end{figure}

\begin{figure}[h!]
	\begin{center}	
		\includegraphics[trim={0 4cm 0 0},width=0.95\textwidth]{./highDimPVA_F1_compare_edit.pdf}  
	\end{center}
	\caption{Comparison of the performance of the MCP, SCAD, lasso, PVA, BMS, BAS and PEM methods on the
						high-dimensional data set using $F_1$ score. For PVA, the
						robust Bayarri prior on $g$ and the uniform, beta-binomial(1, 1) and beta-binomial(1, p) priors
						on $\vgamma$ are used.}
	\label{fig:highDimPVA_F1_compare}
\end{figure}

\subsubsection{Exploration of the posterior model space}

If the covariates in a model selection problem are highly collinear then the
posterior distribution will be highly multi-modal when a spike-and-slab prior
structure is used. This can make seeking the optimal model very challenging,
due to the many local optima. In this section, we present a series of numerical
experiments which demonstrate the capability of our algorithm to successfully
find the models with high posterior probability in such situations.

Our population of bit strings $\mGamma^{(0)} = (\vgamma_1^{(0)}, \ldots,
\vgamma_K^{(0)})$ with $K = 20$ particles was randomly initialised from a
sequence of independent Bernoulli trials with probability of success $1/2$.
Figure \ref{fig:PVA_posterior_models} shows all $4096$ posterior model
probabilities ordered by the model's bit strings, represented by blue dots.
Superimposed over this are the models found by PVA, represented by red dots.
We can clearly see a few peaks in the full posterior distribution. Our
experiment aims to show that most of these posterior peaks are successfully
identified by our algorithm.

As Figure \ref{fig:PVA_posterior_models} shows, in the plots of the log
posterior probabilities of the models, the particles can be seen clustering at
the highest probability models first, then spreading through the medium and low
probability models. From these plots we can see that once $K$ is high enough,
there is a good variety of high, medium and low posterior probability models in
the population of particles. The coverage of the posterior probability
distribution by the population of particles is high, as the particles tend to
cluster towards the higher posterior probability models as PVA's greedy search
algorithm proceeds.

% We begin by using the setting $\lambda = 1$, allowing particle repulsion between each of the models within the
% population.

\begin{figure}	
	\includegraphics[width=0.95 \textwidth]{code/blma/cva_low_dimensional.pdf}
	\caption{Posterior model probabilities when $p = 12$. Red points denote models visited by the PVA
						algorithm, while blue points are models that were not visited. Note that the PVA algorithm
						visits the highest posterior probability points first}
	\label{fig:PVA_posterior_models}
\end{figure}

\subsection{Communities and crime dataset}
\label{sec:crime}

We use the {\tt Communities and Crime} dataset obtained from the UCI Machine
Learning Repository   \\

\url{http://archive.ics.uci.edu/ml/datasets/Communities+and+Crime}  \\

\noindent The data collected was part of a study by \cite{Redmond2002}
combining socio-economic data from the 1990 United States Census, law
enforcement data from the 1990 United States Law Enforcement Management and
Administrative Statistics survey, and crime data from the 1995 Federal Bureau
of Investigation's Uniform Crime Reports.

The raw data consists of 2215 samples of 147 variables the first 5 of which we
regard as non-predictive, the next 124 are regarded as potential covariates
while the last 18 variables are regarded as potential response variables.
Roughly 15\% of the data is missing. We proceed with a complete case analysis
of the data.  We first remove any potential covariates which contained missing
values leaving 101 covariates. We also remove the variables {\tt rentLowQ} and
{\tt medGrossRent} since these variables appeared to be nearly linear
combinations of the remaining variables (the matrix $\mX$ had two singular
values approximately $10^{-9}$ when these variables were included).  We use the
{\tt nonViolPerPop} variable as the response. We then remove any remaining
samples where the response is missing. The remaining dataset consist of 2118
samples and 99 covariates. Finally, the response and covariates are
standardized to have mean 0 and standard deviation 1. Empirical correlations
between variables range from $3.3\times10^{-5}$ to $0.999$.

A comparison of the performance of PVA for the hyper-$g$, robust Bayarri,
Beta-prime and Cake priors on $g$ and the uniform, beta-binomial(1, 1) and
beta-binomial(1, p) priors on $\vgamma$ using $F_1$ score is presented in
Figure \ref{fig:commPVA_F1}. A comparison of the performance of the MCP, SCAD,
lasso, PVA, BMS, BAS and PEM  methods using $F_1$ score is given in Figure
\ref{fig:commPVA_F1_compare}.

\begin{figure}
	\begin{center}	
		\includegraphics[width=0.95\textwidth]{./commPVA_F1.pdf}  
	\end{center}
	\caption{Comparison of the performance of the PVA method on the
						Communities and Crime data set with different $g$ and $\vgamma$ priors using $F_1$ score.
						The hyper-$g$,
						robust Bayarri, Beta-prime and Cake priors on $g$ and the uniform, beta-binomial(1, 1) and beta-binomial(1, p) priors on $\vgamma$ are used.}
	\label{fig:commPVA_F1}
\end{figure}

\begin{figure}
	\begin{center}	
		\includegraphics[trim={0 4cm 0 0},width=0.95\textwidth]{./commPVA_F1_compare_edit.pdf}  
	\end{center}
	\caption{Comparison of the performance of the MCP, SCAD, lasso, PVA, BMS, BAS and PEM methods on the
						Communities and Crime data set using $F_1$ score. For PVA, the
						robust Bayarri prior on $g$ and the uniform, beta-binomial(1, 1) and beta-binomial(1, p) priors
						on $\vgamma$ are used.}
	\label{fig:commPVA_F1_compare}
\end{figure}

\subsection{Quantitative trait loci dataset}
\label{sec:QTL}

    For our final $p>n$ simulation example we will use the design matrix based
    on an experiment on a backcross population of $n=600$ individuals for a
    single large chromosome of 1800 cM. This giant chromosome was covered by
    121 evenly spaced markers from \cite{Xu2007}. Nine of the markers
    overlapped with QTL ofthe main effects and 13 out of the ${121 \choose 2} =
    7260$ possible marker pairs had interaction effects. The $\mX$ matrix
    combines the main effects and interaction effects to make a $600\times
    7381$ matrix. The values of the true coefficients are listed in Table 1 of
    \cite{Xu2007} ranging from 0.77 to 4.77 in absolute magnitude and
    correlations range from 0 to 0.8 where most of the higher correlation
    occurs along the off-diagonal values of the correlation matrix of the
    covariates. Here we center the $\mX$ matrix and simulate new data from $\vy
    = \mX\vbeta_0 + \vvarepsilon$ where $\vvarepsilon =
    (\varepsilon_1,\ldots,\varepsilon_n)^\top$ and the $\vvarepsilon_i$ are
    independently drawn with $\vvarepsilon_i \sim N(0,20)$. Similar simulation
    studies were conducted in \cite{Xu2007} and \cite{Karkkainen2012}. This
    process was repeated $50$ times.  For this simulation setting PVA has the
    best model selection accuracy, smallest MSEs and smallest parameter biases
    of all the methods compared.  The Lasso, SCAD, MCP, EMVS, PVA and BMS
    methods took 1.5, 1.5, 1.8, 1229, 2011, 5327 seconds respectively.

A comparison of the performance of PVA for the hyper-$g$, robust Bayarri,
Beta-prime and Cake priors on $g$ and the uniform, $\text{beta-binomial}(1, 1)$
and $\text{beta-binomial}(1, p)$ priors on $\vgamma$ using $F_1$ score is
presented in Figure \ref{fig:qtlPVA_F1}. A comparison of the performance of the
MCP, SCAD, lasso, PVA, BMS, BAS and PEM  methods using $F_1$ score is given in
Figure \ref{fig:qtlPVA_F1_compare}.

\begin{figure}[h!]
	\begin{center}	
		\includegraphics[width=0.95\textwidth]{./qtlPVA_F1.pdf}  
	\end{center}
	\caption{Comparison of the performance of the PVA method on the
						QTL data set with different $g$ and $\vgamma$ priors using $F_1$ score.
						The hyper-$g$,
						robust Bayarri, Beta-prime and Cake priors on $g$ and the uniform, beta-binomial(1, 1) and beta-binomial(1, p) priors on $\vgamma$ are used.}
	\label{fig:qtlPVA_F1}
\end{figure}

\begin{figure}[h!]
	\begin{center}	
		\includegraphics[trim={0 4cm 0 0},width=0.95\textwidth]{./qtlPVA_F1_compare_edit.pdf}  
	\end{center}
	\caption{Comparison of the performance of the MCP, SCAD, lasso, PVA, BMS, BAS and PEM methods on the
						QTL data set using $F_1$ score. For PVA, the
						robust Bayarri prior on $g$ and the uniform, beta-binomial(1, 1) and beta-binomial(1, p) priors
						on $\vgamma$ are used.}
	\label{fig:qtlPVA_F1_compare}
\end{figure}


% Mine
% \subsection{$n < p$, High dimensional example, Particle EM}

% We first present an example where $n > p$ and $p$ is relatively small ($p = 12$), to allow for the full
% enumeration of the model space. Later, we show an example for the important $p > n$ case. We compare our
% results  against the Lasso (\cite{Tibshirani1996}), SCAD (\cite{Fan2001}), MCP (\cite{Zhang2010}), {\tt BMS}
% (\cite{Zeugner2015}) and {\tt VARBVS} (\cite{Carbonetto2012}) algorithms.


% \subsection{$p > n$}

% \begin{table}
% 	\caption{Simulation results including the average number of modes located, the average percentage of
% 	posterior coverage achieved, and the percentage of times that the global mode was located, for various
% 	population sizes $(K=20, 50, 100)$ and choices of repulsion $\lambda=0, 1, 2, 3$}
% 	\label{tab:result2}
% 	\begin{tabular}{l|llll|llll|llll|llll}
% 	\hline
% 	 					& \multicolumn{4}{c}{K=20} 	& \multicolumn{4}{c}{K=50} & \multicolumn{4}{c}{K=100} \\
% 	$\lambda$ & 0 & 1 & 2 & 3 & 0 & 1 & 2 & 3 & 0 & 1 & 2 & 3 & 0 & 1 & 2 & 3 \\
% 	\hline
% 	\# Modes & \\
% 	\% Posterior & 83.1 & 84.7 & 83.7 & 82.3 & 84.7 & 83.2 & 85.4 & 84.8 & 81.9 & 82.9 & 83.9 & 84.6 & 83.9 & 81.5 & 82.0 & 85.2 \\
% 	\% Global Mode & 91 & 90 & 87 & 82 & 86 & 85 & 80 & 90 & 90 & 90 & 86 & 91 & 88 & 91 & 76 & 88 \\
% 	\hline
% 	\end{tabular}

% \end{table}

% Show posterior probabilities on the log scale

\subsection{Comparison of PVA against other model selection methods on
simulated data sets}

	% TP <- sum(vgamma.hat[pos])
	% TN <- sum(1 - vgamma.hat[neg])
	% FP <- sum(vgamma.hat[neg])
	% FN <- sum(1 - vgamma.hat[pos])
	
	% sensitivity <- TP/length(pos)
	% specificity <- TN/length(neg)
	% precision <- TP/sum(vgamma.hat)
	% recall <- TP/(TP + FN)
	% accuracy <- (TP + TN)/(TP + TN + FP + FN)
		
	% F1 <- 2*precision*recall/(precision+recall)		

The method used to assess the quality of the variable selection was to generate
data from a known true model $\vgamma$, and then compare this against the model
$\widehat{\vgamma}$ found by each of the model selection methods that we
compared. We then calculated the $F_1$ score for $\widehat{\vgamma}$.

% ,

% where
% $$F_1 = 2 \times \text{precision} \times \text{recall} / (\text{precision} + \text{recall})$$ with
% $\text{precision} = \text{true positives} / (\text{true positives} + \text{false positives})$ and
% $\text{recall} = \text{true positives} / (\text{true positives} + \text{false negatives})$.

% When running PVA, three methods of initialising $\Gamma$ were tried.  A cold
% start where each $\gamma_{ij}$ in $\gamma_i$, $1 \leq i \leq K$ was
% initialised randomly from a $\text{Bernoulli}(p)$ distribution, with $p = 10
% / |\vgamma|$.

% Two methods of warm start were also tried, to explore whether the PVA
% algorithm could do better by being initialised from another method than by
% being initialised randomly: a warm start from SCAD where models with more
% covariates were preferred, and a warm start from SCAD where models with
% higher likelihood were preferred.

The experiments were repeated with 
the Cake prior, Maruyama's Beta-prime prior, Liang's hyper-g prior and Bayarri's robust prior.
The results of the algorithm were found to be insensitive to the choice of prior.
For each combination of population size, data set, and prior the experiment was repeated 50 times.

% \subsubsection{High-dimensional example $p > n$, Quantitative Trait Locus}
% The Quantitative Trait Locus (QTL) example is taken from \cite{Xu2007}. A BC population of $n=600$ was
% simulated for a single large chromosome of $1800$ cM. This chromosome was covered by 121 evenly spaced
% markers. Nine of the markers overlapped with QTL of the main effects and 13 out of the $\binom{121} 2 = 7,260$
% possible marker pairs had interaction effects.

% The PVA method was compared against Lasso, SCAD, Mcp, BMS and VARBVS. The {\tt BMS} method was run with
% $1,000,000$ iterations. While the $F_1$ scores from this method were excellent, the runtimes of the BMS method
% with this number of iterations were prohibively long - from 10 minutes to half an hour. When the number of
% iterations was reduced to $100,000$ less, the resulting $F_1$ scores were much poorer. The {\tt VARBVS} method
% was run with $\sigma = 10.0$ and $sa = 10.0$, which improved the $F_1$ score obtained.

% When initialising $\Gamma$ with a warm start from SCAD preferring models with more covariates or with higher
% likelihood, PVA performed poorly with a lower number of particles (K=$20$). Performance improved as the number
% of particles in the population was increased, as can be seen in Figures \ref{fig:QTL_warm_start_covariates}
% and \ref{fig:QTL_warm_start_likelihood}. Performance was better when $\Gamma$ was initialised randomly, for
% all population sizes $K=20$, $50$ or $100$.  PVA consistently achieved $F_1$ scores that were either
% competitive with or higher than the competing methods that we examined. This indicates that the PVA algorithm
% is sensitive to its' initialisation, and that as SCAD does poorly on these problems, PVA also does poorly
% comparatively with a warm start from SCAD, although still better than SCAD alone.

% \begin{figure}
% \includegraphics[width=0.95 \textwidth]{QTL_covariates_maruyama.pdf}
% \label{fig:QTL_warm_start_covariates}
% \caption{$F_1$ scores for PVA on the simulated QTL data with $\Gamma$ initialised from the SCAD models, with
% models with more covariates preferred. Here $K=20$, $50$ and $100$ respectively}
% \includegraphics[scale=0.33]{code/QTL/results/20_generate_data_QTL_warm_start_covariates_log_prob1.pdf}
% \includegraphics[scale=0.33]{code/QTL/results/50_generate_data_QTL_warm_start_covariates_log_prob1.pdf}
% \includegraphics[scale=0.33]{code/QTL/results/100_generate_data_QTL_warm_start_covariates_log_prob1.pdf}
% \end{figure}

% \begin{figure}
% \includegraphics[width=0.95 \textwidth]{QTL_likelihood_maruyama.pdf}
% \label{fig:QTL_warm_start_likelihood}
% \caption{$F_1$ scores for PVA on the simulated QTL data with $\Gamma$ initialised from the SCAD models, with
% models with higher likelihood preferred. Here $K=20$, $50$ and $100$ respectively}
% \end{figure}


% Initialising $\Gamma$ randomly, PVA performed poorly by comparison with either of the warm start methods
% above, as can be seen from Figure \ref{fig:QTL_cold_start}.
% \begin{figure}
% \includegraphics[width=0.95 \textwidth]{QTL_cold_maruyama.pdf}
% \label{fig:QTL_cold_start}
% \caption{$F_1$ scores for PVA on the simulated QTL data with $\Gamma$ initialised randomly.
% 					Here $K=20$, $50$ and $100$ respectively}
% \includegraphics[scale=0.33]{code/QTL/results/20_generate_data_QTL_cold_start_log_prob1.pdf}
% \includegraphics[scale=0.33]{code/QTL/results/50_generate_data_QTL_cold_start_log_prob1.pdf}
% \includegraphics[scale=0.33]{code/QTL/results/100_generate_data_QTL_cold_start_log_prob1.pdf}
% \end{figure}

% When initialising $\Gamma$ with a warm start from SCAD preferring models with higher likelihood,
% PVA performed less well on the QTL example than SCAD preferring models with more covariates, but still
% better than other competing methods except for MCP,
% as can be seen from Figure \ref{fig:QTL_warm_start_likelihood}

\section{Variable inclusion for small data sets}

We compared variable selection using PVA against exact variable selection on five small data sets,
Hitters, Bodyfat, Wage, College and US Crime. 
The variable inclusion probabilities were estimated by taking the sum of the columns of the population
of models selected $\mGamma$
weighted by marginal likelihood of each model.
The exact variable inclusion probabilities were calculated by summing the columns of the matrix of all possible
models $\mGamma$ weighted by the marginal likelihood of each model.
The mean relative error of the variable inclusion probabilities estimated by PVA was calculated,
and the results of these comparisons are presented in Table \ref{tab:variable_inclusion_rel_error}.
The number of particles in the population $K$ affected the
variable inclusion probability in the variables selected by PVA, while the marginal probability
$p(\vgamma | \vy)$ used to weight models in $\Gamma$ seemed to have only a very minor impact.
% Is this still true?
When the robust Bayarri prior is chosen to rank models in PVA,
the marginal probability $p(\vgamma | \vy)$ changes a lot as opposed to ranking models with other priors.
From the previous section, we see that changing the PVA model ranking
marginal had a large impact on the $F_1$ score obtained.
Variables with low posterior probability are truncated to 0, as PVA seeks higher posterior probability models,
ignoring the lower posterior probability models.

\begin{table}[!ht]
\begin{tabular}{|ll|rrr|rrr|}
	\hline
	Dataset & Prior & & $<=0.5$ & & & $>0.5$ &\\
	& & $K = 20$ & $K = 50$ & $K = 100$ & $K = 20$ & $K = 50$ & $K = 100$ \\
	\hline
	Bodyfat&BIC&$0.63$&$0.48$&$0.37$&$0.07$&$0.01$&$0.02$\\
	&Liang's hyper-$g$ prior&$0.66$&$0.52$&$0.42$&$0.07$&$0.01$&$0.02$\\
	&Bayarri's robust prior&$0.65$&$0.52$&$0.4$&$0.07$&$0.01$&$0.02$\\
	&ZE&$0.65$&$0.51$&$0.39$&$0.06$&$0.01$&$0.02$\\
	College&BIC&$0.7$&$0.58$&$0.49$&$0.03$&$0.02$&$0.03$\\
	&Liang's hyper-$g$ prior&$0.9$&$0.78$&$0.64$&$0.06$&$0.06$&$0.06$\\
	&Bayarri's robust prior&$0.88$&$0.78$&$0.63$&$0.06$&$0.06$&$0.06$\\
	&ZE&$0.82$&$0.66$&$0.57$&$0.03$&$0.06$&$0.06$\\
	Hitters&BIC&$0.74$&$0.64$&$0.5$&$0.12$&$0.07$&$0.06$\\
	&Liang's hyper-$g$ prior&NA&$0.81$&$0.83$&NA&$0.17$&$0.07$\\
	&Bayarri's robust prior&$0.84$&$0.81$&$0.75$&$0.29$&$0.17$&$0.07$\\
	&ZE&$0.79$&$0.72$&$0.67$&$0.27$&$0.13$&$0.05$\\
	USCrime&BIC&$0.82$&$0.7$&$0.64$&$0.47$&$0.15$&$0.12$\\
	&Liang's hyper-$g$ prior&$0.76$&$0.71$&$0.64$&$0.45$&$0.16$&$0.12$\\
	&Bayarri's robust prior&$0.79$&$0.7$&$0.61$&$0.35$&$0.14$&$0.08$\\
	&ZE&$0.76$&$0.7$&$0.64$&$0.45$&$0.16$&$0.13$\\
	Wage&BIC&$0.67$&$0.49$&$0.35$&$0$&$0$&$0$\\
	&Liang's hyper-$g$ prior&$0.69$&$0.47$&$0.33$&$0$&$0$&$0$\\
	&Bayarri's robust prior&$0.69$&$0.47$&$0.32$&$0$&$0$&$0$\\
	&ZE&$0.69$&$0.47$&$0.32$&$0$&$0$&$0$\\
	\hline
\end{tabular}
\label{tab:variable_inclusion_rel_error}
\caption{Relative error of the variable inclusion probability estimated by PVA to the
					exact variable inclusion probability, partitioned by exact probability under or equal to $0.5$ and
					over $0.5$}
\end{table}

% Small data set, Kakadu
\begin{figure}[ht!]
	\includegraphics[scale=0.5]{posterior_prob_Kakadu.pdf}
	\includegraphics[scale=0.5]{inclusion_error_Kakadu.pdf}

	\caption{PVA was run on the Kakadu data set. The total posterior model probability and error in posterior 
						variable inclusion probability were calculated using the exact posterior model and variable  
						inclusion probability from every possible sub-model. These were calculated for a range of 
						population sizes from 25 to 500, in 25 model increments.
						As the population increases, the total posterior model probability increases while the error in 
						posterior variable inclusion probabliity decreases.}
	\label{fig:kakadu_total_posterior_mass}
\end{figure}

% VietNam
% GradRate
% UScrime

% Interpretation of results
The same general trends were observed in all small data sets. Total posterior probability is higher for the
beta- binomial model prior than for the uniform model prior, while the variable inclusion error is lower. This
same general trend is seen regardless of $g$-prior. Total posterior probability increases with increased
population size $K$, while variable inclusion error decreases. Although the PVA algorithm is deterministic,
variation in the results amongst the trials is seen due to the random initialisation of $\mGamma$.

\section{Conclusion}
\label{sec:chapter_4_conclusion}
We have proposed a deterministic Bayesian model selection algorithm which is computationally efficient
and simple. Like Particle EM \citep{Rockova2017}, our algorithm maintains a population of solutions and ensures
diversity of that population to explore the uncertainty of the selected model. This gives far more information
about the model selection process than simply choosing one best model. However, whereas Particle EM uses
a spike-and-slab prior for the regression co-efficients, our approach uses a $g$-prior, which avoids 
the Bartlett's Paradox and Information Paradox. Importantly, both approaches can be implemented using rank-one
updates and downdates and the model selection posterior probabilities are available in closed form,
which allows the algorithm to be implemented in a computionally efficient manner.

While previously model selection algorithms using the Maruyama, Liang-$g$ and robust Bayarri priors have
typically been implemented using Monte Carlo Markov Chains, our algorithm allows the advantages of these
priors while using a deterministic algorithm. The PVA algorithm  presented in this chapter is implemented in
the {\tt blma} package in the {\tt cva} function.

%! TEX root = thesis.tex
\chapter{Conclusion and Future Directions}
\label{sec:chapter_2_conclusion}
		
We have described a Variational Bayes approximation to Zero-Inflated Poisson
regression models which allows such models to be fit with considerable
generality. We have also devised and extensively tested a number of alternative
approaches for fitting such models, and extended one of these alternative
approaches with a new parameterisation. Using MCMC methods as the gold standard
to test against, we have assessed the accuracy and computational speed of these
algoritms.
		
We applied our model fitting algorithms to a number of data sets to fit a range
of models. The Cockroaches model in Section \ref{sec:cockroaches} had few fixed
covariates, a random intercept for each apartment building and incorporated
zero-inflation. The Police stops model in Section \ref{sec:police_stops} was a
pure Poisson mixed model, with no zero-inflation and a random intercept for
precincts/locality. The Biochemists model in Section \ref{sec:biochemists} was
zero-inflated with fixed effects. The Owls model in Section \ref{sec:owls} was
zero-inflated,  with a random intercepts for each nest. There were a large
number of nests $(m=27)$. We were able to estimate the variance component for
this model very accurately.

The use of Mean Field Variational Bayes allows estimation of Bayesian ZIP
models in a fraction of the time taken to fit the same model using even the
best MCMC methods available, with only a small loss of accuracy. This is of
great utility in applications where speed matters, such as when applied
statisticians are comparing and choosing amongst many candidate models, as is
typical in practice.
		
The new parameterisation of Gaussian Variational Approximation using the
Cholesky factorisation of the inverse of $\mLambda$ presented in Section
\ref{sec:param} provides significant advantages when used to estimate mixed
models.

Mixed models have covariance matrices with a block structure, due to the
dependence structure of the random effects. The precision parameterisation
presented in this chapter is able to preserve this sparsity within the
structure of the Cholesky factors of the inverses of the covariance matrices
use in the variational lower bound by re-ordering the rows and columns of the
matrices so that the random effects blocks appear first. The Owls example
presented in this chapter shows the computational advantages of this approach
when the number of groups $m$ in the model is large (m=27 in this case) -- as
the covariance parameterisation takes 46 seconds to fit whereas the inverse
parameterisation only takes 3 seconds. This clearly demonstrates advantage of
using sparsity to reduce the dimension of the optimisation problem to be solved
when models are being fit -- as only the non-zero values in the covariance
matrices need to be optimised over. This allows models to be fit more quickly,
and with greatly improved numerical stability and without loss of accuracy.

While all of the fitting algorithms presented in this chapter except the
Laplace's approximation algorithm were able to fit ZIP random and fixed effects
models with high accuracy, and the  Gaussian inverse parameterisation and fixed
point algorithms were able to do so at high speed, they  could be numerically
unstable depending on the data the model was being fit to and their starting
points. In the case of the Gaussian inverse parameterisation algorithm, the
source of the problem was tracked down to the exponential function used in the
parameterisation of the diagonal of the Cholesky factor of the precision matrix
combined with the exponential that arises in the derivation of the Gaussian
variational lower bound for Poisson mixed models -- leading to frequent numeric
overflows during the fitting process. This problem, once discovered, was
mitigated by replacing the exponential parameterisation of the diagonal of the
Cholesky factor with a piecewise function which is exponential beneath a
threshold and quadratic above that threshold. This was shown to greatly
increase the numeric stability of the GVA inverse parameterisation for a range
of starting points.

Some of the algorithms which we experimented with were found to be very
sensitive to their starting points.  While these algorithms are typically
initialised with a starting point as close as possible to the final solution,
this gives some sense of the stability of each algorithm. We were able to
develop a variant of the algorithm that employs a parameterisation which is
much more numerically stable.

This thesis chapter presents the essential ideas necessary for a performant
implementation implementing model fitting for ZIP regression models.
%, but the performance would be even better if our algorithm was re-implemented
%in a compiled language with good numeric libraries such as C++ with Eigen.
The majority of the performance improvements over existing approaches come from
avoiding unneccessary matrix inversion, which is a computationally expensive
and numerically unstable process taking $\BigO(p^3)$ flops, and  from
constructing and calculating with sparse matrices. The gains of these
approaches, particularly from sparse  matrix techniques, can be difficult to
fully realise in R without expert knowledge of the underlying implementation
and libraries.
		
Our application of these ideas to Andrew Gelman's data showed that the new
parameterisation very effectively speeds up fitting zero-inflated mixed models
to real world data with a large number of groups, while still maintaining
excellent accuracy versus an MCMC approach. This demonstrates the applicability
of the ideas presented within this chapter to real world data sets.
\section{Zero-inflated models future directions}
		
The first directions for future research stemming from this chapter would be
generalising the approximation to other zero-inflated models which handle
overdispersion in the data without the need for a random intercept, such as the
zero-inflated negative binomial model.

Furthermore, much more exploration could be done on alternative
parameterisations of the covariance matrix in the Gaussian Variational
Approximation. The specific parameterisation of the diagonal of the Cholesky
factor as a piecewise exponential/quadratic polynomial function was chosen
largely for convenience.

The current mean field update and Gaussian Variational Approximation algorithms
use the entire sample. For large samples in the Big Data era, this may not be
computationally feasible. Other authors such as \cite{Tan2018} have used doubly
stochastic algorithms which both sub-sample the data and use noise to
approximate the integral expression for the expectation of the variational
lower bound. The sub-sampling in particular is very appealing in a Big Data
context. We wish to experiment with this class of algorithm, and compare the
performance and accuracy of this kind of doubly stochastic algorithm with the
more traditional mean field and Gaussian variational algorithms presented in
Chapter 2.

\section{$g$-prior future directions}

We have reviewed the prior structures that lead to closed form expressions for
Bayes factors for linear models. We have described ways that each of these
priors, except for the hyper-g/n prior can be evaluated in a numerically stable
manner and have implemented a package \texttt{blma} for performing full exact
Bayesian model averaging using this methodology. Our package is competitive
with \texttt{BAS} and \texttt{BMS} in terms of computational speed, is
numerically more stable and accurate, and offers some different priors
structures not offered in \texttt{BAS}. Our package is much faster than
\texttt{BayesVarSelect} and is also numerically more stable and accurate.

We are currently working on several extensions to this work. Firstly, we are
working on a parallel implementation of the package which will allow for exact
Bayesian inference for problems roughly the size $p\approx 30$.

Secondly, we are currently implementing  Markov Chain Monte Carlo (MCMC) and
population based MCMC methods for exploring the model space when $p>30$.
Lastly, we are deriving exact expressions for parameter posterior distributions
under some of the prior structures we have considered here.

\section{PVA future directions}

There are several planned future extensions to this work. Firstly, we would
like to generalise the PVA approach to generalised linear models as well as
linear models, to be able to perform model selection for regression models
applicable to a wider range of types of data.

Secondly, although the algorithm already runs in parallel on multicore CPUs
using \texttt{OpenMP}, we believe even greater gains in performance could be
achieved by porting the algorithm to run on GPUs, or by using distributed
computing such as \texttt{OpenMPI}.

Thirdly, and most excitingly, we could examine modifications to the PVA
algorithm itself. The way the algorithm currently ensures diversity amongst the
particles in the population is to reward increases in entropy, weighted by the
hyperparameter $\lambda$. The current version of the algorithm hardcodes
$\lambda$ to $1$, but it would be interesting to alter $\lambda$ and as in the
Population EM algorithm \cite{Tan2018} and observe the effect on model
selection performance. The algorithm also currently maintains diversity in the
population by maintaining uniqueness of every particle within the population.
It would be interesting to relax this constraint and compare the effect on
model performance.

% \include{Chapter_5_regression_posteriors}

\backmatter
\appendix
\section{Calculation of the Variational Lower bound} 
\label{sec:calculation_of_var_lb}
% TODO: Mean field updates?
% Where are the priors for \vbeta and \vu
		
The variational lower bound is equal to $\bE_q\{\log{p(\vy, \vtheta)} - \log{q(\vtheta)}\} = T_1 + T_2 + T_3$,
where
% This is the new T_1
\begin{equation*}
\begin{array}{rl}
	T_1 & = \quad \bE_q[\log{p(\vy, \vnu)} - \log{q(\vnu)}]\\
	    & = \quad \vy \mP \mC \vmu - \vp^\top \exp{\left\{ \mC \vmu + \frac{1}{2} \text{diag} (\mC \mLambda \mC^\top) \right\}} - \vone^\top\log \Gamma{(\vy + \vone)}\\
	    & \quad + \frac{p + m}{2} (1 + \log{2 \pi}) + \frac{1}{2} \log{|\mLambda|},\\
	T_2 & = \quad \bE_q \left\{ \log p(\mSigma_{\vu \vu}) - \log q(\mSigma_{\vu \vu}) \right\}\\
	    & = \quad \bE_q \big\{ v/2(\log |\Psi| - \log |\Psi + \vmu_\vu \vmu_\vu^\top + \mLambda_{\vu \vu}|) + \frac{1}{2} \log 2 + \frac{1}{2} \log|\mSigma_{\vu \vu}| \\
	    & \qquad \hphantom{\bE_q \{} + \log \Gamma_{p+1}(v/2) - \log \Gamma_{p}(v/2)    
	    + \frac{1}{2} \tr((\vmu_{\vu} \vmu_{\vu}^\top + \mLambda_{\vu \vu}) \mSigma_{\vu \vu}^{-1}) \big\} \\
	    & = \quad v/2\big(\log |\Psi| - \log |\Psi + \vmu_\vu \vmu_\vu^\top + \mLambda_{\vu \vu}|\big) + \frac{1}{2} \log 2 \\
	    & \quad + \frac{1}{2} \bE_q \log |\mSigma_{\vu \vu}| + \log \Gamma_{p+1}(v/2) - \log \Gamma_{p}(v/2) \\
	    & \quad + \frac{1}{2} \tr\big[\mI_m + \Psi(\Psi+ \vmu_\vu \vmu_\vu^\top + \mLambda_{\vu \vu})^{-1}/(v + p + 2)\big] \\
	T_3 & = - \vp^\top \log \vp - (\vone - \vp)^\top \log (\vone - \vp) - \log \Beta (\alpha_\rho, \beta_\rho) + \log \Beta (\alpha_q, \beta_q)                                                              
\end{array}
\end{equation*}
		
\noindent with $\bE_q \log |\mSigma_{\vu \vu}| = m \log 2 + \log \left | \Psi + \vmu_\vu \vmu_\vu^\top + \mLambda_{\vu \vu} \right | + \sum_{i=1}^m \Psi \left ( \frac{v - i + 1}{2} \right ).$

\section{Calculation of Derivatives for the Gaussian Variational Approximations}
\subsection{Derivatives for Laplace-Gaussian Variational Approximation}
\label{sec:appendix_derivatives_laplace}
\begin{equation*}
\begin{array}{ll}
	\frac{\partial \log p(\vmu, \mLambda; \vy)}{\partial \vmu}     & \approx \mP \mC (\vy - \exp{(\mC \vmu)}) - \mSigma^{-1} \vmu \text{ and} \\
	\frac{\partial \log p(\vmu, \mLambda; \vy)}{\partial \mLambda} & \approx - \mC^\top \text{diag}(\vp e^{(\mC \vmu)}) \mC - \mSigma^{-1}.   
\end{array}
\end{equation*}
		
\subsection{Derivatives for Gaussian Variational Approximation with parameterisation $\mLambda = \mR \mR^\top$}
\label{sec:appendix_derivatives_gva}
\begin{equation*}
\begin{array}{ll}
	\frac{\partial \log \underline{p}(\vmu, \mLambda; \vy)}{\partial \vmu}     & = \mP \mC (\vy - \mC^\top \exp(\mC \vmu + \frac{1}{2} \text{diag}{(\mC \mLambda \mC^\top)})) - \mSigma^{-1} \vmu \text{ and}                \\
	\frac{\partial \log \underline{p}(\vmu, \mLambda; \vy)}{\partial \mLambda} & = \left \{\mLambda^{-1} - \mP \mC^\top \exp(\mC \vmu + \frac{1}{2} \text{diag}(\mC \mLambda \mC^\top)) \mP \mC) - \mSigma^{-1} \right \} \mR. 
\end{array}
\end{equation*}

\subsection{Derivatives for Gaussian Variational Approximation fixed point}
\label{sec:appendix_derivatives_gva_fixed_point}
\begin{equation*}
\begin{array}{ll}
	\frac{\partial \log \underline{p}(\vmu, \mLambda; \vy)}{\partial \vmu}     & = \quad \mC^\top\vp \left [\vy - \mC\exp\{\mC \vmu + \tfrac{1}{2} \text{diag}(\mC \mLambda \mC^\top)\} \right ] - \mSigma^{-1} \vmu \text{ and} \\
	\frac{\partial \log \underline{p}(\vmu, \mLambda; \vy)}{\partial \mLambda} & = -\mC^\top \text{diag}[\vp^\top \exp\{\mC \vmu +\tfrac{1}{2} \text{diag}(\mC \mLambda \mC^\top)\}] - \mSigma^{-1}.                             
\end{array}
\end{equation*}

\bibliographystyle{elsarticle-harv}
\bibliography{references_mendeley}

\end{document}
