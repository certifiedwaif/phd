% Appendix.tex

% TODO: Does this need to be here?
\section{Hamiltonian Monte Carlo}

% Basic idea: Sample the typical set efficiently by following a contour/level set of the likelihood surface
% Hamiltonian is a mixture of the position and momentum
% If you can't write any more clearly than this, then you don't understand well enough yet

The Hamiltonian Monte Carlo algorithm was devised by Simon Duane, A. D. Kennedy, Brian Pendleton and Duncan
Roweth in 1987 
Duane, S., Kennedy,A. D., Pendleton, B. J., and Roweth, D. 1987. Hybrid Monte Carlo. Physics Letters B, 195:216–222.
Earl,
(found the reference in \cite{Neal2011}).

The Hamiltonian Monte Carlo method is a Metropolis-Hasting method that makes use of gradient information to
seek states of higher probability in the distribution being sampled from, reducing random walk behaviour. The method uses this information to spend a greater proportion of computational time sampling from the typical set of 
the distribution.

If the probability distribution that we are interested in sampling from can be written
\[
	P(\vx) = \frac{e^{E(\vx)}}{Z}
\]

then the state space $\vx$ can be considered the current position, and augmented by a momentum variable
$\vp$. Then the algorithm proceeds by alternating between two types of proposal. The first proposal generates
a sample for the momentum variable $\vp$, leaving the position variable $\vx$ unaltered. The second proposal
generates a new position sample for $\vx$ and a new momentum sample $\vp$ using simulated Hamiltonian dynamics
defined by the Hamiltonian
\[
	H(\vx, \vp) = E(\vx) + K(\vp),
\]

where $K(\vp)$ is the kinetic energy. These two proposals are used to sample from the joint density
\[
	P_H(\vx, \vp) = \frac{1}{Z_H} \exp{[-H(\vx, \vp)]} = \frac{1}{Z_H} \exp{[-E(\vx)]} \exp{[ -K(\vp)]}
\]

As this probability distribution factorises, the marginal distribution of $\vx$ is the distribution that we
wish to sample from. So, by retaining the $\vx$ samples and discarding the $\vp$ samples, we obtain a sequence
of samples $\{ \vx^{(t)} \}$ which asymptotically come from $P(x)$.

\subsection{Details of Hamiltonian Monte Carlo}

The first proposal, which can be considered a Gibbs sample, draws a new momentum from the Gaussian density
$\exp{-K(\vp)}/Z_K$. This proposal is always accepted. During the second, dynamical proposal, the momentum
variable determines the change in the state $\vx$, and the \emph{gradient} of $E(\vx)$ determines the change
in the momentum $\vp$, using the equations
\begin{align*}
	\dot{\vx} & = \vp                                    \\
	\dot{\vp} & = -\frac{\partial E(\vx)}{\partial \vx}. 
\end{align*}

As the change in $\vx$ is always in the direction of the momentum $\vp$ during each dynamical proposal, the
state of the system tends to move a distance that varies \emph{linearly} with the computational time, rather
than as the square root as with other sampling methods.

For an introduction to Hamiltonian Monte Carlo, see the chapter in \cite{MacKay:2002:ITI:971143}. A deeper
and more thorough introduction is provided in \cite{Betancourt2017}.

% TODO: I think this section is unnecessary.
\section{Matrix calculus}

Matrix calculus is a very elegant and compact notation for the derivatives of functions of vectors and matrices.
A useful introduction is \cite{Wand2002}, while a more comprehensive treatment can be found in
\cite{magnus99}.

\subsection{Notation}

Given a function $f:\R^{n \times n} \to \R$ of a matrix $\mLambda$, we denote the matrix derivative of $f$ with 
respect to $\mLambda$ as $\mD_\mLambda f$, and the Hessian of $f$ with respect to $\mLambda$ as
$\mH_\mLambda f$ the Hessian with respect to $\mLambda$.

The Hessian matrix is a $d \times d$ matrix whose $(i, j)$-th entry is equal to

\[
	\frac{\partial^2}{\partial x_i x_j} f(\vx)
\]

and is denoted by $\mH f(\vx)$.

% TODO: Rewrite.
\section{Optimisation}
Fitting a statistical model to data is often done by finding the parameters which maximise the likelihood
function for that data -- an optimisation problem. If the model is simple, and there are no interdependencies
between the parameters, then the parameters which maximise the likelihood may be available in closed form. In
general, however, this is not possible, and the likelihood must be maximised using an iterative optimisation
algorithm. The nature of the algorithm required depends on the difficult of the problem to be solved. If the
likelihood is convex in the parameters being maximised, then convex optimisation can be used --
see \citep{Boyd2010}. For likelihoods which are not convex in the parameters, more general optimisation
procedures must be employed -- see \citep{Nocedal2006}. These procedures often optimise non-convex functions
by repeatedly approximating them by convex functions and then using convex optimisation to optimise these.

\subsection{Newton-Raphson}

The Newton-Raphson algorithm is a second-order method where the function to be minimised is approximated
by a second order Taylor expansion of the function. The resulting quadratic form is convex, and thus can
be optimised with a Newton-Raphson iteration.

\subsubsection{Definition}

Consider the second order Taylor expansion of $f(\vtheta)$ around $\vtheta_{n}$
\[
	f(\vtheta) \approx f(\vtheta_{n}) + \nabla f(\vtheta_{n}) \Delta \vtheta_{n} + \half \Delta \vx^\top \mH_f(\vtheta_{n}) \Delta \vx + \sO((\Delta \vtheta_{n})^3).
\]
which is a quadratic form in $\vtheta_n$. This expression attains its' maxima with respect to $\Delta x$ when
\[
	\vzero = \mD_{\Delta \vtheta} [f(\vtheta_{n}) + \nabla f(\vtheta_{n}) \Delta \vtheta_{n} + \half \Delta \vtheta^\top \mH_f(\vtheta_{n}) \Delta \vtheta] = \nabla f(\vtheta_n) + \mH_f(\vtheta_n) \Delta \vtheta.
\]

Thus taking $\Delta \vtheta = -\nabla f(\vtheta_n) [\mH_f(\vtheta_n)]^{-1}$, we perform a Newton-Rapshon
Iteration by setting $\vtheta_{n + 1} = \vtheta_{n} - \nabla f(\vtheta_n) [\mH_f(\vtheta_n)]^{-1}$ as it will
be closer to the stationary point of the quadratic form, which will itself be closer to the stationary point
$\vtheta^*$ of $f$. This calculation of choice of $\vtheta_{n+1}$ is called a Newton-Raphson iteration. By
repeatedly calculating a sequence of these $\vtheta_n$ through Newton-Raphson iterations, each progressively
closer to the minima of $f$, we establish an iterative algorithm for finding the minima of $f$. The algorithm
continues until the step length $\| \vtheta_{n+1} - \vtheta_{n} \| < \epsilon$, for some suitably small choice
of $\epsilon$, such as $\epsilon = 10^{-8}$, indicating that convergence has been achieved.

\subsubsection{Time and memory complexity}

Execution of this algorithm requires the calculation of a vector addition, dot product and solution of a
linear system for each iteration. The vector addition and dot product can both be performed in $p$ flops,
where $p$ is the dimension of the problem. The most expensive part of the computation is the solution of the
linear system $\mH_f(\vtheta_n) \Delta \vtheta$, which takes $\sO(p^3)$ flops. Even on modern computer
systems, this can quickly become computationally intractable for problems of modest size. This motivates the
development of Quasi-Newton algorithms, such as BFGS, L-BFGS and L-BFGS-B, discussed in Section
\ref{sec:quasi_newton}.

\subsubsection{Radius of convergence and convergence to local solutions}

\subsection{Quasi-Newton Raphson algorithms: BFGS and L-BFGS-B}
\label{sec:quasi_newton}

\subsubsection{Definition}

% Wolfe conditions

As in Newton-Raphson we seek to find $\min_\vtheta f(\vtheta)$ for some smooth $f: \R^p \to \R$.  Again, we
calculate a sequence of steps $\theta_n$ with each step approximately solving the subproblem $\min_\alpha
f(\vx_k + \alpha \vp_k)$

where $\vtheta_n$ is the current best guess, $\vp_n \in \R^p$ is a search direction and $\alpha$ is the
step length.

The modification to the Newton-Raphson algorithm is that rather than use a true second-order Taylor expansion
of $f(\vtheta)$ around $\vtheta_n$, we instead approximate it using the model
\[
	m_n(\vtheta_n) = f(\vtheta_n) + \nabla f(\vtheta_n)^\top \vtheta_n + \tfrac{1}{2} \vtheta_n^\top \mB_n \vtheta_n
\]
where $\mB_n$ has taken the place of $\mH_f(\vtheta_n)$. This allows us to avoid having to calculate the complete 
Hessian matrix $\mH_f$ for each $\vtheta_n$ at every iteration, instead using $\mB_k$ % TODO: Define B_k

Denote a univariate function $\phi$ restricted to the direction $\vp_k$ as
$\phi(\alpha) = f(\vx_k + \alpha \vp_k)$. A step length $\alpha_k$ is said to satisfy the \emph{Wolfe conditions}
if the following two inequalities hold:

\begin{enumerate}
	\item[(i)] $f(\vx_k + \alpha_k \vp_k) \leq f(\vx_k) + c_1 \alpha_k \vp_k^\top \nabla f(\vx_k)$ \text{(Armijo rule)}
	\item[(ii)] $\vp_k^\top \nabla f(\vx_k + \alpha_k \vp_k) \geq c_2 \vp_k^\top \nabla f(\vx_k)$ \text{(curvature condition)}
\end{enumerate}

The first condition ensures that the step length $\alpha_k$ decreases $f$ sufficiently, while the second condition
ensures that the slope had been sufficiently reduced.

The rationale for imposing the Wolfe conditions in an optimisation algorithm where
$\vx_{k+1} = x_k + \alpha \vp_k$ is to ensure the convergence of the gradient to zero. In particular, if the
cosine of the angle between $\vp_k$ and the gradient,
$$
\cos \theta_k = \frac{\nabla f(\vx_k)^\top \vp_k}{\|\nabla f(\vx_k)\| \|\vp_k\|}
$$

is bounded away from zero and the Wolfe conditions hold, then $\nabla f(\vx_k) \to 0$.

An additional motivation, in the case of a quasi-Newton method is that if $\vp_k = - \mB_k^{-1} \nabla f(\vx_k)$,
where the matrix $\mB_k$ is updated by the BFGS or DFP formula, then if $\mB_k$ is positive definite condition ii
implies $\mB_{k+1}$ is also positive definite.

\subsubsection{Time and memory complexity}

\section{Approximations and Numerical Methods for Evaluating Integrals}
In many areas of mathematical statistics, and particularly in Bayesian statistics, many quantities of interest
are expressed as integrals such as expectations of parameters or posterior probabilities. Frequently, these
integrals cannot be evaluated analytically and so must be evaluated using other methods, such as by
approximating the integral or evaluating it numerically by repeatedly evaluating the integrand.

\subsection{Trapezoidal integration}

If an integral cannot be approximated well using Laplace's method or some other method of approximation, then
it can often be evaluated numerically using a quadrature scheme. Trapezoidal integration is a method of
numerically computing the integral of a function $f: \R \to \R$ over the interval $(a, b)$. First, the
interval is partitioned into $x_1, x_2, \ldots, x_{n-1}, x_n$, and then
% Diagram?
\[
	\int_a^b f(x) dx \approx \sum_{i=1}^{n-1} \frac{1}{2} \frac{f(x_i) + f(x_{i+1})}{x_{i+1} - x_i}.
\]

Note that the points $x_i$ can be chosen arbitrarily within the interval $(a, b)$, and do not need to be
evenly spaced.

\subsubsection{Error}
% See https://en.wikipedia.org/wiki/Trapezoidal_rule#Error_analysis
The error can be shown to be
\[
	\text{error} = -\frac{(b - a)^2}{12 n^2}[f'(b) - f'(a)] + \BigO{(n^{-3})}.
\]
See \cite{Cruz-Uribe2002}.

\section{Numerical computation}

% In this section, I'd like to talk about floating point representation and its' consequences, numerical
% analysis/error bounds and in particular their relevance to numerical linear algebra.

% But most of this is overkill, only interesting to you.

% \subsection{Floating point numbers}

% $$\text{sign } \text{significand} \times \text{base}^\text{exponent}$$

% IEEE 754: floating point in modern computers

% On modern computers, IEEE 754 is ubiquitous, and the base is $2$.

% Error bounds on $+, -, *, /, \sqrt{}$

% $$
% \text{fl}(a \text{ op } b) = (1 + \epsilon) (a \text{ op } b) \text{ provided $\text{fl}(a \text{ op } b)$ neither overflows nor underflows}
% $$

% where $\epsilon$ is the machine epsilon, the minimum difference between $1$ and the next representable
% floating point number.

% Exceptional values: $\pm \infty$, $\text{NaN}$

% When doing arithmetic on computers, it is important to understand the full ramifications of this.

% \begin{enumerate}
% \item All numerical values are in fact approximations within an interval.
% \item Every arithmetic operation has an associated amount of error, and these errors accumulate as more
% 			arithmetic operations are performed.
% \end{enumerate}

% The field of numerical analysis (references Classical and Modern Numerical Analysis,
% Demmel's course: http://www.cs.berkeley.edu/~demmel/cs267/lecture21/lecture21.html, Kahan) analyses numerical 
% algorithms to assess their error bounds, and attempt to mitigate these problems as much as possible to allow
% us to solve numerical problems on modern computers.

% \subsection{Truncation and round-off error}
% \subsection{Numerical stability}
% \subsection{Numerical differentiation and integration}

\subsection{Numerical linear algebra}
\subsubsection{Matrix norms}
We can define a norm on the space of matrices such that the matrix norms obey the following axioms
\begin{align*}
	\|\mA\|        & \geq 0                                                   \\
	\|\mA\|        & = 0 iff \mA = 0                                          \\
	\|\alpha \mA\| & = |\alpha| \|\mA\| \forall \alpha \in \R                 \\
	\|\mA + \mB\|  & \leq \|\mA\| + \|\mB\|\text{ for all matrices } \mA, \mB 
\end{align*}
Additionally, for square matrices
\[
	\|\mA\mB\| \leq \|\mA\| \|\mB\|
\]
\subsubsection{Examples}

\begin{enumerate}
	\item{The Frobenius Norm:}
	The Frobenius Norm (Euclidean norm) is a matrix norm on the matrix $\mA$ defined as the square root of the sum
	of the absolute squares of its elements,
	\[
		\|\mA\|_F = \sqrt{\sum_{i=1}^m \sum_{j=1}^n |a_{ij}|^2}.
	\]
	
	\item{The p norm:}
	The p norm is a matrix norm on the matrix $\mA$ defined as the supremum of the ratio of the norm of
	$\mA \vx$ divided by the norm of $\vx$ over the set of vectors $\vx \ne \vzero$.
	\[
		\|\mA\|_p = \sup_{\vx \ne 0} \frac{\|\mA \vx\|_p}{\|\vx\|_p}.
	\]
\end{enumerate}

The $p$-norm of a matrix and its' singular values from its' SVD are connected in the following way.

\cite{Golub:1996:MC:248979} Corollary 2.4.3 % Matrix Computations
If $\sigma_1$, $\sigma_2$, \ldots, $\sigma_p$ are the ordered descending singular values of the SVD of $\mA
\in \R^{m \times n}$ where $p = \min{(m, n)}$ then
\[
	\|\mA\|_F = \sqrt{\sigma_1^2 + \sigma_2^2 + \ldots + \sigma_p^2} \geq |\sigma_1|
\]

\subsubsection{Condition numbers}
In numerical linear algebra, a useful measure for quantifying the sensitivity of numerical procedures such as
solving linear systems is the \emph{condition number} of A. It depends on the matrix norm used and is defined by
\[
	\kappa(\mA) := \|\mA^{-1}\|.\|\mA\| = \frac{\sigma_{\text{max}}(\mA)}{\sigma_{\text{min}}(\mA)}.
\]

This is a useful tool in characterising the accuracy and stability of numerical linear algebra algorithms.
\cite{Golub:1996:MC:248979} \S 2.

\subsubsection{LDU decomposition}

If a matrix $\mA$ is square, it can be factored into the product of a lower triangular matrix, a diagonal matrix and
an upper triangular matrix -- $\mA = \mL \mD \mU$.

\subsubsection{Cholesky decomposition}

If a matrix $\mA$ is square $p \times p$, symmetric and positive definite then it can be factored into the
product of a lower triangular matrix and its' transpose
\[
	\mA = \mL \mL^\top
\]
by noting that
\begin{align*}
	\mA = \mL^\top \mL &= \begin{pmatrix}
	\mL_{11}          & 0                                     & 0                                    \\
	\mL_{21}          & \mL_{22}                              & 0                                    \\
	\mL_{31}          & \mL_{32}                              & \mL_{33}                             
	\end{pmatrix}
	\begin{pmatrix}
	\mL_{11}          & \mL_{21}                              & \mL_{31}                             \\
	0                 & \mL_{22}                              & \mL_{32}                             \\
	0                 & 0                                     & \mL_{33}                             
	\end{pmatrix}	\\
	&= \begin{pmatrix}
	\mL_{11}^2        &                                       & \text{symmetric}                     \\
	\mL_{21} \mL_{11} & \mL_{21}^2 + \mL_{22}^2               &                                      \\
	\mL_{31} \mL_{11} & \mL_{31} \mL_{21} + \mL_{32} \mL_{22} & \mL_{31}^2 + \mL_{32}^2 + \mL_{33}^2 
	\end{pmatrix}
\end{align*}
and so
\begin{align*}
	\mL_{jj} & = \sqrt{\mA_{jj} - \sum_{k=1}^{j-1} \mL_{jk}^2} \text{ and}                               \\
	\mL_{ij} & = \frac{1}{\mL_{jj}} (\mA_{ij} - \sum_{k=1}^{j-1} \mL_{ik} \mL_{jk}), \text{ for } i > j. 
\end{align*}

This can be computed in $\BigO(p^2)$ for a $p \times p$ matrix. Once this form is obtained, it has several
computational advantages over working with the original matrix $\mA$. Firstly, as $\mL$ is lower triangular
only the non-zero entries need to be stored. Secondly, the triangular form of $\mL$ allows matrix equations
involving $\mL$ to be solved in $\BigO{(p^2)}$ using back-substitution and forward-substitution, without using
Gaussian elmination or calculating a matrix inverse.

\subsection{Difficulty of inverting matrices which are nearly singular}
In computational statistics, the task of inverting matrices often arises when
attempting to solve a statistical problem such as fitting a model. While theoretically this task can always be
done successfully, in practice if the matrix or its' inverse being inverted is close to singular this can be
difficult or even impossible. This can arise in numerical procedures which are prone to overflow, such as
Poisson regression, which overflows due to the exponential link function in the formulation of the problem.

\subsubsection{Sensitivity of solving linear systems}

A precise measure of linear system sensitivity can be obtained by considering the parameterised system
\[
	(\mA + \epsilon \mF) \vx(\epsilon) = \vb + \epsilon + \vf,  \vx(0) = \vx
\]
where $\mF \in \R^{n \times n}$ and $\vf \in \R^n$. If $\mA$ is nonsingular, it is clear that $x(\epsilon)$ is
differentiable in a neighbourhood of zero. Moreover, $\dot{x}(0) = \mA^{-1} (\vf - \mF \vx)$ and so the
Taylor series expansion for $\vx$ has the form
\[
	x(\epsilon) = x + \epsilon \dot{\vx}(0) + \BigO(\epsilon^2).
\]

Using any vector norm and consistent matrix norm we obtain
\[
	\frac{\|\vx(\epsilon) - \vx\|}{\vx} \leq |\epsilon| \|\mA^{-1}\| \left\{\frac{\|\vf\|}{\|\vx\|} + \|\mF\|\right\} + \BigO(\epsilon^2)
\]

From \S 2.6.2 of \citep{Golub:1996:MC:248979}
when solving linear equations in the form $\mA \vx = \vb$
\[
	\frac{\|\vx(\epsilon) - \vx\|}{\|\vx\|} \leq \kappa(\mA) (\rho_A + \rho_b) + \BigO(\epsilon^2)
\]
where $\rho_A = |\epsilon| \frac{\|\mF\|}{\|\mA\|}$ and $\rho_B = |\epsilon| \frac{\|f\|}{\|b\|}$ represent
the relative errors in $\mA$ and $\vb$, respectively. Thus the relative error in $\vx$ can be $\kappa(\mA)$
times the relative error in $\mA$ and $\vb$. In this sense, the condition number $\kappa(\mA)$ quantifies the
sensitivity of the $\mA \vx = \vb$ problem.

% From \cite{trefethen97}'s section on Conditioning and Condition Numbers, p91

A problem is \emph{well-conditioned} if $\kappa$ is small (e.g. $1$, $10$, $10^2$) and \emph{ill-conditioned}
if $\kappa$ is large (e.g. $10^6$, $10^{16}$) \cite{trefethen97}.  If a matrix is either close to singular or
has very large singular values, then it will be ill-conditioned.

An immediate consequence of this is that if either $\mA$ or its' inverse are ill-conditioned, then inversion
or solution of the linear system will be highly numerically unstable. In our ZIP code, we routinely saw values of 
at least $10^5$ on the diagonals of our Cholesky factors until we changed the parameterisation to the safe
exponential parameterisation.

\subsection{The block inverse formula}

% General introduction, computational advantages, consequences
Consider a block matrix
\[
	\begin{pmatrix}
		\mA & \mB \\
		\mC & \mD 
	\end{pmatrix}.
\]
Then the inverse of this matrix is given by the block-matrix inverse formula
%Demonstrate this using
%the block-inverse formula. Inverting a matrix is an inherently unstable operation. The number of floating
%point operations involved leads to magnification of error.

%Block inverse formula \\
\[
	\begin{pmatrix}
		\mA & \mB \\
		\mC & \mD 
	\end{pmatrix}^{-1}
	=
	\begin{pmatrix}
		\mI           & 0 \\
		-\mD^{-1} \mC & I 
	\end{pmatrix}
	\begin{pmatrix}
		(\mA - \mB^{-1}\mC)^{-1} & 0        \\
		0                        & \mD^{-1} 
	\end{pmatrix}
	\begin{pmatrix}
		\mI          & 0   \\
		-\mB\mD^{-1} & \mI 
	\end{pmatrix}
\]

This is frequently useful if the blocks $\mA$ and $\mD$ are substantially easier to invert than the entire
matrix. This is usually the case because $\mA$ and $\mD$ are smaller in dimension than the original matrix,
and matrix inversion is in general a $\BigO(p^3)$ process, where $p$ is the dimension of the square matrix
being inverted. But in the case where $\mA$ or $\mD$ is of a special known form, such as a sparse, triangular
or banded matrix, then the computational advantage obtained from exploiting this knowledge may be even
greater.

\subsubsection{Sparsity of the inverse}
We often wish to work with sparse matrices due to the computational advantages and increased numerical
stability they provide. But sparse matrices do not necessarily have sparse inverses. 

Consider the block inverse formula presented in the previous section. An immediate consequence of this formula
is that if any of the blocks of the matrix being inverted is not sparse, then the inverse will not be sparse
in \emph{any} of the blocks in the computed inverse.

\subsection{Positive semi-definite matrices}
A matrix $\mA \in \mathcal{R}^{p \times p}$ is \emph{positive semidefinite} if for any $\vx \in \mathcal{R} \ \vzero$,
$\vx^\top \mA \vx > 0$. Equivalently, all eigenvalues of $\mA$ are strictly positive. Covariance matrices
$\mSigma = \E[X - \mu_X][X - \mu_X]^\top$ are positive definite, as
$\vx^\top \E[X - \mu_X][X - \mu_X]^\top \vx = \vu^\top \vu > 0$ for $\vu = \E[(X - \mu_X)^\top \vx]$.

A problem which arises frequently in fitting statistical models with dependence amongst the responses such as mixed
models is maximising a likelihood function over the set of covariance matrices. This problem can be difficult not
just because of the high dimension of the parameter space involved, but also due to the constraints upon the
parameters to ensure that the resulting matrix is still positive definite, and hence a valid covariance matrix. This
problem can be dealt with through several methods, such as semidefinite programming \citep{Boyd2010}, or parameterising the covariance
matrix $\mSigma$ in terms of a matrix factorisation such as
$\mSigma = \mB \mB^\top$ for any unconstrainted matrix $\mB$ or it's Cholesky factors $\mSigma = \mL \mL^\top$.
The Cholesky factor approach has the advantage that the Cholesky factors are lower triangular, and so quicker to
compute with.
