\documentclass[11pt]{amsart}
\usepackage{amsmath}
\usepackage{amsfonts}
\usepackage{amssymb}
\title{Summary of Bayesian Data Analysis}
\begin{document}
\maketitle
\section{Chapter 1}
The heart of Bayesian analysis is Bayes' Rule, which is
\[
	p(\theta|y) = \frac{p(\theta, y)}{p(y)} = \frac{p(y|\theta)p(\theta)}{p(y)}
\]
where $p(y) = \sum_\theta p(y|\theta) p(\theta)$ in the discrete case and
$p(y) = \int p(y|\theta)p(\theta) d\theta$ in the continuous case.

An equivalent form of this omits the factor $p(y)$, which does not depend on 
$\theta$ and, with fixed $y$, can thus be considered a constant, yielding the
\emph{unnormalised posterior density} which is the right side of the equation.

\[
p(\theta|y) \propto p(y|\theta)p(\theta)
\]

These simple expressions encapsulate the technical core of Bayesian inference:
the primary task of any specific application is to develop a model $p(\theta, y)$
and perform the necessary computations to summarise $p(\theta|y)$ in appropriate
ways.

\section{Chapter 2 - Single parameter models}

% FIXME: Learn how to define macros in LaTeX? You're going to be doing this
% a lot.

\[
	p(y|\theta) = \text{Bin}(y|n, \theta) = {n \choose y}\theta^y (1-\theta)^{n-y}
\]

Assume a uniform $[0, 1]$ prior distribution for $\theta$. Then

\begin{align*}
p(\theta|y) &=\frac{p(y|\theta)p(\theta)}{p(y)} \\
	& \propto \theta^y (1-\theta)^{n-y}
\end{align*}

This is a beta distribution $\theta|y \sim \text{Beta}(y+1, n-y+1)$.

\subsection{Conjugacy}
The property that the posterior distribution follows the same parametric form
as the prior distribution is called $\emph{conjugacy}$; the beta prior 
distribution is a \emph{conjugate family} for the binomial likelihood. The
conjugate famioly is mathematically convenient in that the posterior distribution
follows a known parametric form.

Formally, if $\mathcal{F}$ is a class of sampling distributions $p(y|\theta)$,
and $\mathcal{P}$ is a class of prior distributions for $\theta$, then the class
$\mathcal{P}$ is \emph{conjugate} for $\mathcal{F}$ if
\[
p(\theta|y) \in \mathcal{P}\text{ for all }p(.|\theta)\text{ and }p(.)\in \mathcal{P}
\]

\end{document}