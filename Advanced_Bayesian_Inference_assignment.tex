\documentclass{amsart}
\begin{document}
\section{Question 3}
\begin{align*}
\frac{f(x)}{g(x)} &= \frac{\exp{(-\frac{x^2}{2})}}{\frac{1}{1+x^2}} \\
&= (1+x^2) \exp{(-\frac{x^2}{2})}
\end{align*}

Is this bounded? Yes, plot the curves to see this.
For what value is the bound attained? To find this, we differentiate $f(x)/g(x)$.

\begin{align*}
\frac{d}{dx} \frac{f(x)}{g(x)} &= (1+x^2)(-x) \exp{(-\frac{x^2}{2})} + 2x \exp{(-\frac{x^2}{2})} \\
&= (-x -x^3 + 2x)  \exp{(-\frac{x^2}{2})} \\
&= (x - x^3) \exp{(-\frac{x^2}{2})}
\end{align*}

Setting this to 0, we see that as the exponential function is positive for any real argument,
x(1-x^2) must be 0. Hence, $x=0$ or $\pm 1$.

At $x = \pm 1$,
\begin{align*}
\frac{f(1)}{g(1)} &= 2 \exp{-\frac{1}{2}} \\
&= \frac{2}{\sqrt{e}}
\end{align*}

By appropriately normalising f(x) and g(x), \ldots
% Normal by sqrt of 2pi, Cauchy by pi

\begin{align*}
m &= \frac{\sqrt{pi}}{\sqrt{2 \pi}} \frac{2}{\sqrt{e}}\\
&= \frac{\sqrt{2\pi}}{\sqrt{e}}
\end{align*}

Remember that in importance sampling, $f(x) \leq m g(x)$.
\end{document}