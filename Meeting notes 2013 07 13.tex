\documentclass{amsart}
\begin{document}
Meeting with Dr John Ormerod- 15/7/2013

Attendees: Dr John Ormerod, Mark Greenaway

Dr John Ormerod and I informally mapped out the structure of my PhD thesis project. Broadly, the currently proposed structure will
be

1 Response
Year 1
	No mixtures
		Poisson
		Negative binomial, two main parameterisations
	Mixtures
		Zero-Inflated (Truncated Poisson) Mixture
		Mixtures of Poisson
		Hurdle Methods

2 Mean
Spline smoothing
AR(1)
Random effects
Spatial

Year 2
Model Selection
If we are considering p covariates, then if we just restrict ourselves to the inclusion and exclusion of those
p variables in any candidate model then there are $2^p$ candidate models.

Bryman 96/97 Stepwise selection methods are inherently numerically unstable.

Year 3
Measurement Error
Missing data

Develop an R package implementing these techniques

Algorithms
- Maximum likelihood (Hard)
- EM (``Posterior distribution'')
- variation approximation, which can be seen as a generalisation of the EM algorithm
- MCMC ``Exact''
- Expectation propagation
- Laplace-like methods
- Hybrid
- Interpolation

First task
Summarise Poisson
Neg. Bin. - Poisson-Gamma, Poisson-Bessel?
Zero inflated
Hurdle methods, in terms of a probabilistic model. Describe in terms of a probability hierarchy.

Summary of data set: Interesting variables. What are the models that we want to fit.

Advice
When programming, don't re-invent the wheel. Develop re-usable functions as much as possible.
\end{document}