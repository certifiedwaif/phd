\documentclass{amsart}[12pt]
% \documentclass{usydthesis}[12pt]

\addtolength{\oddsidemargin}{-.75in}%
\addtolength{\evensidemargin}{-.75in}%
\addtolength{\textwidth}{1.5in}%
\addtolength{\textheight}{1.3in}%
\addtolength{\topmargin}{-.8in}%
\addtolength{\marginparpush}{-.75in}%
%\setlength\parindent{0pt}
%\setlength{\bibsep}{0pt plus 0.3ex}

\usepackage[authoryear]{natbib}
\usepackage[doublespacing]{setspace}
\usepackage{graphicx}
\usepackage{algorithm,algorithmic}
\usepackage{color}

\title{Preliminaries}
\author{Mark Greenaway}
% include.tex
\newcommand{\Bernoulli}[1]{\text{Bernoulli} \left( #1 \right)}
\newcommand{\mydigamma}[1]{\psi \left( #1 \right)}
%\newcommand{\diag}[1]{\text{diag}\left( #1 \right)}
\newcommand{\tr}[1]{\text{tr}\left( #1 \right)}
\newcommand{\Poisson}[1]{\text{Poisson} \left( #1 \right)}
\def \half {\frac{1}{2}}
\def \R {\mathbb{R}}
\def \vbeta {\vec{\beta}}
\def \vy {\vec{y}}
\def \vmu {\vec{\mu}}
\def \vmuqbeta {\vmu_{q(\vbeta)}}
\def \vmubeta {\vmu_{\vbeta}}
\def \Sigmaqbeta {\Sigma_{q(\vbeta)}}
\def \Sigmabeta {\Sigma_{\vbeta}}
\def \va {\vec{a}}
\def \vtheta {\vec{\theta}}
\def \mX {\vec{X}}

\def\ds{{\displaystyle}}

\def\diag{{\mbox{diag}}}


\usepackage{latexsym,amssymb,amsmath,amsfonts}
%\usepackage{tabularx}
\usepackage{theorem}
\usepackage{verbatim,array,multicol,palatino}
\usepackage{graphicx}
\usepackage{graphics}
\usepackage{fancyhdr}
\usepackage{algorithm,algorithmic}
\usepackage{url}
%\usepackage[all]{xy}



\def\approxdist{\stackrel{{\tiny \mbox{approx.}}}{\sim}}
\def\smhalf{\textstyle{\frac{1}{2}}}
\def\vxnew{\vx_{\mbox{{\tiny new}}}}
\def\bib{\vskip12pt\par\noindent\hangindent=1 true cm\hangafter=1}
\def\jump{\vskip3mm\noindent}
\def\etal{{\em et al.}}
\def\etahat{{\widehat\eta}}
\def\thick#1{\hbox{\rlap{$#1$}\kern0.25pt\rlap{$#1$}\kern0.25pt$#1$}}
\def\smbbeta{{\thick{\scriptstyle{\beta}}}}
\def\smbtheta{{\thick{\scriptstyle{\theta}}}}
\def\smbu{{\thick{\scriptstyle{\rm u}}}}
\def\smbzero{{\thick{\scriptstyle{0}}}}
\def\boxit#1{\begin{center}\fbox{#1}\end{center}}
\def\lboxit#1{\vbox{\hrule\hbox{\vrule\kern6pt
      \vbox{\kern6pt#1\kern6pt}\kern6pt\vrule}\hrule}}
\def\thickboxit#1{\vbox{{\hrule height 1mm}\hbox{{\vrule width 1mm}\kern6pt
          \vbox{\kern6pt#1\kern6pt}\kern6pt{\vrule width 1mm}}
               {\hrule height 1mm}}}


%\sloppy
%\usepackage{geometry}
%\geometry{verbose,a4paper,tmargin=20mm,bmargin=20mm,lmargin=40mm,rmargin=20mm}


%%%%%%%%%%%%%%%%%%%%%%%%%%%%%%%%%%%%%%%%%%%%%%%%%%%%%%%%%%%%%%%%%%%%%%%%%%%%%%%%
%
% Some convenience definitions
%
% \bf      -> vector
% \sf      -> matrix
% \mathcal -> sets or statistical
% \mathbb  -> fields or statistical
%
%%%%%%%%%%%%%%%%%%%%%%%%%%%%%%%%%%%%%%%%%%%%%%%%%%%%%%%%%%%%%%%%%%%%%%%%%%%%%%%%

% Sets or statistical values
\def\sI{{\mathcal I}}                            % Current Index set
\def\sJ{{\mathcal J}}                            % Select Index set
\def\sL{{\mathcal L}}                            % Likelihood
\def\sl{{\ell}}                                  % Log-likelihood
\def\sN{{\mathcal N}}                            
\def\sS{{\mathcal S}}                            
\def\sP{{\mathcal P}}                            
\def\sQ{{\mathcal Q}}                            
\def\sB{{\mathcal B}}                            
\def\sD{{\mathcal D}}                            
\def\sT{{\mathcal T}}
\def\sE{{\mathcal E}}                            
\def\sF{{\mathcal F}}                            
\def\sC{{\mathcal C}}                            
\def\sO{{\mathcal O}}                            
\def\sH{{\mathcal H}} 
\def\sR{{\mathcal R}}                            
\def\sJ{{\mathcal J}}                            
\def\sCP{{\mathcal CP}}                            
\def\sX{{\mathcal X}}                            
\def\sA{{\mathcal A}} 
\def\sZ{{\mathcal Z}}                            
\def\sM{{\mathcal M}}                            
\def\sK{{\mathcal K}}     
\def\sG{{\mathcal G}}                         
\def\sY{{\mathcal Y}}                         
\def\sU{{\mathcal U}}  


\def\sIG{{\mathcal IG}}                            


\def\cD{{\sf D}}
\def\cH{{\sf H}}
\def\cI{{\sf I}}

% Vectors
\def\vectorfontone{\bf}
\def\vectorfonttwo{\boldsymbol}
\def\va{{\vectorfontone a}}                      %
\def\vb{{\vectorfontone b}}                      %
\def\vc{{\vectorfontone c}}                      %
\def\vd{{\vectorfontone d}}                      %
\def\ve{{\vectorfontone e}}                      %
\def\vf{{\vectorfontone f}}                      %
\def\vg{{\vectorfontone g}}                      %
\def\vh{{\vectorfontone h}}                      %
\def\vi{{\vectorfontone i}}                      %
\def\vj{{\vectorfontone j}}                      %
\def\vk{{\vectorfontone k}}                      %
\def\vl{{\vectorfontone l}}                      %
\def\vm{{\vectorfontone m}}                      % number of basis functions
\def\vn{{\vectorfontone n}}                      % number of training samples
\def\vo{{\vectorfontone o}}                      %
\def\vp{{\vectorfontone p}}                      % number of unpenalized coefficients
\def\vq{{\vectorfontone q}}                      % number of penalized coefficients
\def\vr{{\vectorfontone r}}                      %
\def\vs{{\vectorfontone s}}                      %
\def\vt{{\vectorfontone t}}                      %
\def\vu{{\vectorfontone u}}                      % Penalized coefficients
\def\vv{{\vectorfontone v}}                      %
\def\vw{{\vectorfontone w}}                      %
\def\vx{{\vectorfontone x}}                      % Covariates/Predictors
\def\vy{{\vectorfontone y}}                      % Targets/Labels
\def\vz{{\vectorfontone z}}                      %

\def\vone{{\vectorfontone 1}}
\def\vzero{{\vectorfontone 0}}

\def\valpha{{\vectorfonttwo \alpha}}             %
\def\vbeta{{\vectorfonttwo \beta}}               % Unpenalized coefficients
\def\vgamma{{\vectorfonttwo \gamma}}             %
\def\vdelta{{\vectorfonttwo \delta}}             %
\def\vepsilon{{\vectorfonttwo \epsilon}}         %
\def\vvarepsilon{{\vectorfonttwo \varepsilon}}   % Vector of errors
\def\vzeta{{\vectorfonttwo \zeta}}               %
\def\veta{{\vectorfonttwo \eta}}                 % Vector of natural parameters
\def\vtheta{{\vectorfonttwo \theta}}             % Vector of combined coefficients
\def\vvartheta{{\vectorfonttwo \vartheta}}       %
\def\viota{{\vectorfonttwo \iota}}               %
\def\vkappa{{\vectorfonttwo \kappa}}             %
\def\vlambda{{\vectorfonttwo \lambda}}           % Vector of smoothing parameters
\def\vmu{{\vectorfonttwo \mu}}                   % Vector of means
\def\vnu{{\vectorfonttwo \nu}}                   %
\def\vxi{{\vectorfonttwo \xi}}                   %
\def\vpi{{\vectorfonttwo \pi}}                   %
\def\vvarpi{{\vectorfonttwo \varpi}}             %
\def\vrho{{\vectorfonttwo \rho}}                 %
\def\vvarrho{{\vectorfonttwo \varrho}}           %
\def\vsigma{{\vectorfonttwo \sigma}}             %
\def\vvarsigma{{\vectorfonttwo \varsigma}}       %
\def\vtau{{\vectorfonttwo \tau}}                 %
\def\vupsilon{{\vectorfonttwo \upsilon}}         %
\def\vphi{{\vectorfonttwo \phi}}                 %
\def\vvarphi{{\vectorfonttwo \varphi}}           %
\def\vchi{{\vectorfonttwo \chi}}                 %
\def\vpsi{{\vectorfonttwo \psi}}                 %
\def\vomega{{\vectorfonttwo \omega}}             %


% Matrices
%\def\matrixfontone{\sf}
%\def\matrixfonttwo{\sf}
\def\matrixfontone{\bf}
\def\matrixfonttwo{\boldsymbol}
\def\mA{{\matrixfontone A}}                      %
\def\mB{{\matrixfontone B}}                      %
\def\mC{{\matrixfontone C}}                      % Combined Design Matrix
\def\mD{{\matrixfontone D}}                      % Penalty Matrix for \vu_J
\def\mE{{\matrixfontone E}}                      %
\def\mF{{\matrixfontone F}}                      %
\def\mG{{\matrixfontone G}}                      % Penalty Matrix for \vu
\def\mH{{\matrixfontone H}}                      %
\def\mI{{\matrixfontone I}}                      % Identity Matrix
\def\mJ{{\matrixfontone J}}                      %
\def\mK{{\matrixfontone K}}                      %
\def\mL{{\matrixfontone L}}                      % Lower bound
\def\mM{{\matrixfontone M}}                      %
\def\mN{{\matrixfontone N}}                      %
\def\mO{{\matrixfontone O}}                      %
\def\mP{{\matrixfontone P}}                      %
\def\mQ{{\matrixfontone Q}}                      %
\def\mR{{\matrixfontone R}}                      %
\def\mS{{\matrixfontone S}}                      %
\def\mT{{\matrixfontone T}}                      %
\def\mU{{\matrixfontone U}}                      % Upper bound
\def\mV{{\matrixfontone V}}                      %
\def\mW{{\matrixfontone W}}                      % Variance Matrix i.e. diag(b'')
\def\mX{{\matrixfontone X}}                      % Unpenalized Design Matrix/Nullspace Matrix
\def\mY{{\matrixfontone Y}}                      %
\def\mZ{{\matrixfontone Z}}                      % Penalized Design Matrix/Kernel Space Matrix

\def\mGamma{{\matrixfonttwo \Gamma}}             %
\def\mDelta{{\matrixfonttwo \Delta}}             %
\def\mTheta{{\matrixfonttwo \Theta}}             %
\def\mLambda{{\matrixfonttwo \Lambda}}           % Penalty Matrix for \vnu
\def\mXi{{\matrixfonttwo \Xi}}                   %
\def\mPi{{\matrixfonttwo \Pi}}                   %
\def\mSigma{{\matrixfonttwo \Sigma}}             %
\def\mUpsilon{{\matrixfonttwo \Upsilon}}         %
\def\mPhi{{\matrixfonttwo \Phi}}                 %
\def\mOmega{{\matrixfonttwo \Omega}}             %
\def\mPsi{{\matrixfonttwo \Psi}}                 %

\def\mone{{\matrixfontone 1}}
\def\mzero{{\matrixfontone 0}}

% Fields or Statistical
\def\bE{{\mathbb E}}                             % Expectation
\def\bP{{\mathbb P}}                             % Probability
\def\bR{{\mathbb R}}                             % Reals
\def\bI{{\mathbb I}}                             % Reals
\def\bV{{\mathbb V}}                             % Reals

\def\vX{{\vectorfontone X}}                      % Targets/Labels
\def\vY{{\vectorfontone Y}}                      % Targets/Labels
\def\vZ{{\vectorfontone Z}}                      %

% Other
\def\etal{{\em et al.}}
\def\ds{\displaystyle}
\def\d{\partial}
\def\diag{\text{diag}}
%\def\span{\text{span}}
\def\blockdiag{\text{blockdiag}}
\def\tr{\text{tr}}
\def\RSS{\text{RSS}}
\def\df{\text{df}}
\def\GCV{\text{GCV}}
\def\AIC{\text{AIC}}
\def\MLC{\text{MLC}}
\def\mAIC{\text{mAIC}}
\def\cAIC{\text{cAIC}}
\def\rank{\text{rank}}
\def\MASE{\text{MASE}}
\def\SMSE{\text{SASE}}
\def\sign{\text{sign}}
\def\card{\text{card}}
\def\notexp{\text{notexp}}
\def\ASE{\text{ASE}}
\def\ML{\text{ML}}
\def\nullity{\text{nullity}}

\def\logexpit{\text{logexpit}}
\def\logit{\mbox{logit}}
\def\dg{\mbox{dg}}

\def\Bern{\mbox{Bernoulli}}
\def\sBernoulli{\mbox{Bernoulli}}
\def\sGamma{\mbox{Gamma}}
\def\sInvN{\mbox{Inv}\sN}
\def\sNegBin{\sN\sB}

\def\dGamma{\mbox{Gamma}}
\def\dInvGam{\mbox{Inv}\Gamma}

\def\Cov{\mbox{Cov}}
\def\Mgf{\mbox{Mgf}}

\def\mis{{mis}} 
\def\obs{{obs}}

\def\argmax{\operatornamewithlimits{\text{argmax}}}
\def\argmin{\operatornamewithlimits{\text{argmin}}}
\def\argsup{\operatornamewithlimits{\text{argsup}}}
\def\arginf{\operatornamewithlimits{\text{arginf}}}


\def\minimize{\operatornamewithlimits{\text{minimize}}}
\def\maximize{\operatornamewithlimits{\text{maximize}}}
\def\suchthat{\text{such that}}


\def\relstack#1#2{\mathop{#1}\limits_{#2}}
\def\sfrac#1#2{{\textstyle{\frac{#1}{#2}}}}


\def\comment#1{
\vspace{0.5cm}
\noindent \begin{tabular}{|p{14cm}|}  
\hline #1 \\ 
\hline 
\end{tabular}
\vspace{0.5cm}
}


\def\mytext#1{\begin{tabular}{p{13cm}}#1\end{tabular}}
\def\mytextB#1{\begin{tabular}{p{7.5cm}}#1\end{tabular}}
\def\mytextC#1{\begin{tabular}{p{12cm}}#1\end{tabular}}

\def\jump{\vskip3mm\noindent}

\def\KL{\text{KL}}
\def\N{\text{N}}
\def\Var{\text{Var}}

\def \E {\mathbb{E}}
\def \BigO {\text{O}}
\def \IG {\text{IG}}
\def \Beta {\text{Beta}}



\newcommand{\mgc}[1]{{\color{blue}#1}}
\newcommand{\joc}[1]{{\color{red}#1}}

\begin{document}
\setlength{\parindent}{0pt}
\maketitle

\section{Introduction}

Linear and generalised linear regression models are the standard tools used by applied statisticians to
explain the relationship between an outcome variable and one or more explanatory variables. They provide a
general method  to analyse quantified relationships between variables within a data set in an easily
interpretable way. A standard assumption is that the outcomes are independent, and that the effect of the
explanatory variables on the outcome is fixed. But if the outcomes are dependent and this assumption is not
met, then linear and generalised linear models can be extended to linear mixed models. These allow us to
incorporate dependencies amongst the  observations via the assumption of a more complicated covariance
structure, including random effects for  different subgroups or longitudinal data and other extensions such as
splines, missing data and measurement error. This additional flexibility makes their application popular in
many fields, such as public health, psychology and agriculture.

While mixed models are very useful for gaining insight into a data set, fitting them can be computationally
challenging. For all but the simplest situations, fitting these models involves computing high-dimensional
integrals which are often analytically and computationally intractable. Thus, approximations must be used.

In the frequentist paradigm, model parameters are fixed and uncertainty enters the model through random
errors, which have associated variance. The data is modelled as a combination of these fixed parameters and
random errors. In the Bayesian paradigm, the uncertainty enters the model by assuming parameters are random
variables, while the data is fixed.

\section{Zero--inflated models}
Count data with a large number of zero counts arises in many areas of application, such as data arising from
physical activity studies, insurance claims, hospital visits or defects in manufacturing processes. Zero
inflation is a frequent cause of overdispersion in Poisson data, and not accounting for the extra zeroes
may lead to biased parameter estimates. These models have been used for many applications, including defects
in manufacturing in \citep{lambert1992}, horticulture in \citep{BIOM:BIOM1030}, length of
stay data from hospital admissions in \citep{BIMJ:BIMJ200390024}, psychology in \citep{JOFP:rethink},
pharmaceutical studies in \citep{Min01042005}, traffic accidents on roadways in \citep{Shankar1997829} and
longitudinal studies in \citep{LeeWangScottYauMcLachlan2006}.

The strength of this approach derives from modelling the zero and non-zero count data seperately as a mixture
of distributions for the zero and non-zero components, allowing analysis of both the proportion of zeroes in
the data set and the conditions for the transition from zero observations to non-zero observations. When
combined with a multivariate mixed model regression framework, an extremely rich class of models can be fit
allowing a broad range of applications to be addressed. Often the transition from zero to non-zero has a
direct interpretation in the area of application, and is interesting in its' own right.

Bayesian estimation methods for zero-inflated models was developed in \citep{Ghosh2006} using MCMC 
implemented with WinBUGS, and in \citep{Vatsa2014} using a Variational Bayes solution to the inverse
zero-inflated Poisson regression problem. While simple forms of these models are easy to fit with standard 
maximum likelihood techniques, more general models incorporating random effects, splines and missing data 
typically have no closed form solutions and hence present a greater computational challenge to fit.

Fitting these models is typically done with Monte Carlo Markov Chain techniques, but these can be slow and
prone to convergence problems. Other fitting methods such as the Variational Bayes approach above can be
inflexible, not allowing general models incorporating random effects and so on. 
We build upon a latent variable representation of these models to allow a
tractable semiparametric mean field Variational Bayes approximation to be derived. Semiparametric mean field
Variational Bayes is an approximate Bayesian inference method as detailed in \citep{Ormerod2010} and
\citep{Rohde2015}, which allows us to fit close approximations to these models using a deterministic algorithm
which converges much more quickly.

We allow a flexible regression modelling approach by using a Gaussian Variational Approximation as defined in
\citep{Ormerod2012} on the regression parameters to allow a non-conjugate Gaussian prior to be used, making the
resulting Gaussian posterior of the regression parameters easy to interpret. We adopt a Mean Field Variational
Bayes (VB) approach on the other parameters in the model to derive the rest of the approximation.

\section{Model selection}
% Given a full data matrix $\mX \in \R^{n \times p}$, we may wish to find the subset of variables $\gamma \in \{
% 1 \ldots p \}$ such that $\mX_\gamma \vbeta_\gamma$ describe the data in $\vy$ best.

The task of selecting a statistical model from a set of candidate models given a data set is one
of the fundamental tasks of scientific inquiry. It is one of the central tasks of applied statistical
analysis, and there is a correspondingly large literature on the subject. The problem of model selection for
normal linear models is particularly well studied, owing to the popularity and importance of normal linear
models in applications.
There are many approaches to model selection, including criteria based approaches,
variable selection,
penalised regression,
and Bayesian modelling approaches. Model selection is a difficult problem in high--dimensional space in general
because as the dimension of the space increases, the number of possible models increases combinatorially.
Many model selection algorithms use heuristics in an attempt to search the model space more efficiently but
still find an optimal or near-optimal model.

Many approaches to model selection have been investigated, with various model selection criteria
proposed to attempt to balance model likelihood against model complexity. One approach to model selection is
to select between entire models, using a model selection criterion. These criterion may have a fixed penalty
for model complexity, such as Akaike's Information Criterion \citep{Akaike1974}, the Risk Inflation Criterion
\citep{Foster1994}, the Schwarz criterion or Bayesian Information Criterion \citep{Schwarz1978}, the Deviance
Information Criterion \citep{Spiegelhalter2016} or the Principle of Minimum Description Length
\citep{Hansen2001}. Alternatively, the penalty may be adaptive/data--dependent as in \citep{George2000}.

An alternative to model selection criterion is to select models based on their posterior probability. In a Bayesian context, this can be
accomplished by comparing the posterior likelihoods of the candidate models to see which is most probable
given the observed data. This can be done, for example, by using Bayes Factors as in \citep{Kass1993}. Rather
than selecting one candidate model, several models can be combined together using Bayesian model  averaging,
as in \citep{Hoeting1999}, \citep{Raftery1997}, \citep{Fernandez2001} or \citep{Papaspiliopoulos2016}.

Model selection can be made implicit in the model fitting process itself, as in ridge regression
\citep{Casella1980}, of which the well-known lasso is a special case \citep{Tibshirani1996}. As shown in
papers by \citep{Breiman1996} and \citep{Efron2013} showed, while  the standard formulation of a linear model
is unbiased, the goodness of fit of these models is numerically  unstable. Breiman showed that by introducing
a penalty on the size of the regression co- efficients such as  in ridge regression, this numerical
instability can be avoided. This reduces the variances of the co-efficient estimates, at the expense of
introducing some bias --- the bias-- variance trade--off.

A special case of model selection is variable selection, where the focus is on selecting individual
covariates, rather than entire models. This approach can either be Fully Bayesian or Empirically Bayesian as
in \citep{Cui2008}. Variable selection approaches involve a stochastic search over the variables in the model
space. This search can be driven by posterior probabilities, as in \citep{Casella2006}, or by Gibbs sampling
approaches such as in \citep{George1993}. These two approaches of model selection and variable selection can
be combined, as in \citep{Geweke1996}. Variable selection can also be accomplished by selecting the median
probability model, consisting of those models whose posterior inclusion probability is at least $1/2$, as in
\citep{Barbieri2004}.


\mgc{You have so much work to do on this chapter.}

\section{Models and Assumptions}

\section{Frequentist Approaches to Model Selection}
\subsection{Information criteria}


In a frequentist context, there are many functions which can be used to judge which model is best,
such as AIC, BIC etc. These are functions $f: \gamma \to \R^+$ which allow the models under consideration to
be ranked, and the best model chosen from those available. Thus the optimal model under such a measure
f is
\[
	\gamma^* = \max_\gamma f(\gamma)
\]
These functions typically attempt to balance log likelihood against the complexity of the model, achieving
a compromise between each.

\mgc{AIC, BIC, DIC, Mallow's $C_p$}


Information criteria are frequently used to compare amongst models. Let $\vgamma$ denote the candidate model.
The Information Criteria are of the form
\[
	\text{Information Criteria} = -2 \log p(\vy | \hat{\vtheta_\vgamma}) + \text{complexity penalty}
\]

where $\log p(\vy | \hat{\vtheta_\gamma})$ is the log-likelihood of the model and the complexity penalty is
a function of the sample size $n$ and the number of parameters $p$ of the model. Information criteria attempt
to successfully compromise between goodness of fit and model complexity.

The most popular of the information criteria is the Akaike Information Criterion (AIC) \cite{Akaike1974}. AIC
calculates an estimate of the information lost when a given model is used to represent the process that
generates the data and so is an estimator of the Kullback-Leibler divergence of the true model from the fitted
model. The AIC of the model $\vgamma$ is defined as
\[
	\text{AIC}(\vgamma) = -2 \log p(\vy | \hat{\vtheta_\vgamma}) + 2 |\vgamma|
\]

where $|\vgamma|$ is the number of parameters in the model $\vgamma$. The model with the lowest AIC is selected
as the `best`.

Of a similiar form as the AIC, but derived via a more Bayesian framework is the Bayesian Information Criterion
or BIC. The BIC approximates the posterior probability of the candidate model $\vgamma$. The BIC is defined as
\[
	\text{BIC}(\vgamma) = -2 \log p(\vy | \hat{\vtheta_\vgamma}) + |\vgamma| \log(n)
\]

This is a more severe penalty for model complexity than in the Akaike's Information Criteria when $n$ is
greater than $8$. BIC can be shown to be approximately equivalent to model selection using Bayes Factors
in certain contexts \cite{Kass1993}.

\subsection{Existing literature}
The problem of model selection is one of the most important problems encountered in practice by applied
statistical practitioners. There are many approaches to model selection including approaches based on
functions of the model selection criterion, penalized regression and Bayesian modelling approaches. A major
motivation for this field of research is the need for a computationally feasible approach to performing model
selection on large scale problems where the number of covariates is large.

% Non-Bayesian
\subsection{Frequentist Approachs to Model Selection}
Many approaches to model selection have been investigated, with various model selection criteria proposed to
attempt to balance model likelihood against model complexity. One approach to model selection is to select
between entire models, using a model selection criterion. These criterion may have a fixed penalty for model
complexity, such as Akaike's Information Criterion \citep{Akaike1974}, the Risk Inflation Criterion
\citep{Foster1994}, the Schwarz criterion or Bayesian Information Criterion \citep{Schwarz1978}, the Deviance
Information Criterion \citep{Spiegelhalter2016} or the Principle of Minimum Description Length
\citep{Hansen2001}. Alternatively, the penalty may be adaptive/data--dependent as in \citep{George2000}.

Model selection can also be made implicit in the model fitting process itself, as in ridge regression
\citep{Casella1980}, of which the well-known lasso is a special case \citep{Tibshirani1996}. As shown in
papers by \citep{Breiman1996} and \citep{Efron2013} showed, while  the standard formulation of a linear model
is unbiased, the goodness of fit of these models is numerically  unstable. Breiman showed that by introducing
a penalty on the size of the regression co- efficients such as  in ridge regression, this numerical
instability can be avoided. This reduces the variances of the co-efficient estimates, at the expense of
introducing some bias -- the bias- variance trade--off.

\subsection{Bayesian Approaches to Model Selection}
An alternative to model selection criterion is to select models based on their posterior probability. In a
Bayesian context, this can be accomplished by comparing the posterior likelihoods of the candidate models to
see which is most probable given the observed data. This can be done, for example, by using Bayes Factors as
in \citep{Kass1993}. Rather than selecting one candidate model, several models can be combined together using
Bayesian model averaging, as in \citep{Hoeting1999}, \citep{Raftery1997}, \citep{Fernandez2001} or
\citep{Papaspiliopoulos2016}.

The problem of model selection can also be considered within a Bayesian framework.
Within this framework, candidate models can be selected by comparing their posterior likelihoods to see
which is most probable given the observed data, as in \citep{Kass1993}. Bayesian model selection approaches
can incorporate prior information, such as the prior probability of selecting each candidate model $\vgamma$.

A special case of model selection is variable selection, where the focus is on selecting individual
covariates, rather than entire models. Variable selection approaches search over the
variables in the model space for the best covariates to include in the candidate model. Due to the large
number of possible combinations of covariates -- typically $2^p$ where $p$ is the number of covariates, such
searches are often stochastic. This approach can either be Fully Bayesian or Empirically Bayesian as in
\citep{Cui2008}.  This search can be driven by posterior probabilities, as in \citep{Casella2006}, or by Gibbs
sampling approaches such as in \citep{George1993}. These two approaches of model selection and variable
selection can be combined, as in \citep{Geweke1996}. Variable selection can also be accomplished by selecting
the median probability model, consisting of those models whose posterior inclusion probability is at least
$1/2$, as in \citep{Barbieri2004}.

\section{Hamiltonian Monte Carlo}

% Basic idea: Sample the typical set efficiently by following a contour/level set of the likelihood surface
% Hamiltonian is a mixture of the position and momentum
% If you can't write any more clearly than this, then you don't understand well enough yet

The Hamiltonian Monte Carlo algorithm was devised by Simon Duane, A. D. Kennedy, Brian Pendleton and Duncan
Roweth in 1987 
Duane, S., Kennedy,A. D., Pendleton, B. J., and Roweth, D. 1987. Hybrid Monte Carlo. Physics Letters B, 195:216–222.
Earl,
(found the reference in \cite{Neal2011}).

The Hamiltonian Monte Carlo method is a Metropolis-Hasting method that makes use of gradient information to
seek states of higher probability in the distribution being sampled from, reducing random walk behaviour. The method uses this information to spend a greater proportion of computational time sampling from the typical set of 
the distribution.

If the probability distribution that we are interested in sampling from can be written
\[
	P(\vx) = \frac{e^{E(\vx)}}{Z}
\]

then the state space $\vx$ can be considered the current position, and augmented by a momentum variable
$\vp$. Then the algorithm proceeds by alternating between two types of proposal. The first proposal generates
a sample for the momentum variable $\vp$, leaving the position variable $\vx$ unaltered. The second proposal
generates a new position sample for $\vx$ and a new momentum sample $\vp$ using simulated Hamiltonian dynamics
defined by the Hamiltonian
\[
	H(\vx, \vp) = E(\vx) + K(\vp),
\]

where $K(\vp)$ is the kinetic energy. These two proposals are used to sample from the joint density
\[
	P_H(\vx, \vp) = \frac{1}{Z_H} \exp{[-H(\vx, \vp)]} = \frac{1}{Z_H} \exp{[-E(\vx)]} \exp{[ -K(\vp)]}
\]

As this probability distribution factorises, the marginal distribution of $\vx$ is the distribution that we
wish to sample from. So, by retaining the $\vx$ samples and discarding the $\vp$ samples, we obtain a sequence
of samples $\{ \vx^{(t)} \}$ which asymptotically come from $P(x)$.

\subsection{Details of Hamiltonian Monte Carlo}

The first proposal, which can be considered a Gibbs sample, draws a new momentum from the Gaussian density
$\exp{-K(\vp)}/Z_K$. This proposal is always accepted. During the second, dynamical proposal, the momentum
variable determines the change in the state $\vx$, and the \emph{gradient} of $E(\vx)$ determines the change
in the momentum $\vp$, using the equations
\begin{align*}
	\dot{\vx} & = \vp                                    \\
	\dot{\vp} & = -\frac{\partial E(\vx)}{\partial \vx}. 
\end{align*}

As the change in $\vx$ is always in the direction of the momentum $\vp$ during each dynamical proposal, the
state of the system tends to move a distance that varies \emph{linearly} with the computational time, rather
than as the square root as with other sampling methods.

For an introduction to Hamiltonian Monte Carlo, see the chapter in \cite{MacKay:2002:ITI:971143}. A deeper
and more thorough introduction is provided in \cite{Betancourt2017}.


\section{Variational approximations}

\subsection{Definition}

Consider a model $p(\vtheta|\vy)$ which is of interest to us but computationally difficult or intractable to 
fit. We may be able to gain much of the same insight from a given data set by fitting an accurate approximation 
of the model, allowing us to summarise the data and perform statistical inference.

Call the approximating model $q(\theta)$. Then we seek a $q(\theta)$ which approximates $p(\vtheta|\vy)$
as closely as possible in some sense. A useful measure of distance between probability distributions is the
Kullback-Leibler divergence, which is defined as
\[
	\KL(q || p) \equiv \int q(\vtheta) \log{\frac{q(\theta)}{p(\theta)}} d \vtheta.
\]

We may seek a best approximating distribution $q^*(\vtheta)$ to a given posterior distribution
$p(\vtheta|\vy)$ from a parameteric family of approximating distributions $Q$ by finding
\[
	q^*(\vtheta) = \argmin_{q \in Q} \KL \{ {q(\vtheta) || p(\vtheta|\vy)} \}.
\]

If $\vtheta$ is partitioned into $M$ partitions $\vtheta_1$, $\vtheta_2$, \ldots, $\vtheta_M$ then a 
simple form of approximation to adopt is the factored approximation of the form
\[
	q(\vtheta) = \Pi_{i=1}^M q(\vtheta_i)
\]

where each of the density $q(\vtheta_i)$ is a member of a parametric family of density functions. This form of
approximation is computationally convenient, but assumes that the partitions of $\vtheta$ are completely
independent of one another.

The optimal mean field update for each of the parameters $\vtheta_i$ can be shown to be
\[
	q^*(\vtheta_i) \propto \exp \{ \E [\log p(\vy; \vtheta)] \}.
\]

For details of the proof, and a more thorough introduction to the topic of variational approximations, see
\cite{Ormerod2010}.

\subsection{Gaussian Variational Approximation}

In cases where there is a strong dependence between partitions of $\vtheta$, such as between the parameters
$\vmu$ and $\mSigma$ in a hierarchical Gaussian model, a factored approximation may not approximate the true
distribution accurately. In this case, an alternate form of approximation may be used with the parameters
considered together to take their dependence into account. One such form of approximation is the Gaussian
Variational Approximation \cite{Ormerod2012}, which assumes that the distribution of the parameters being 
approximated is multivariate Gaussian. The covariance matrix of the Gaussian allows the approximation to
capture the dependence amongst the elements of $\vtheta$, which increases the accuracy of the variational
approximation relative to the factored approximation.

\section{Matrix calculus}

Matrix calculus is a very elegant and compact notation for the derivatives of functions of vectors and matrices.
A useful introduction is \cite{Wand2002}, while a more comprehensive treatment can be found in
\cite{magnus99}.

\subsection{Notation}

Given a function $f:\R^{n \times n} \to \R$ of a matrix $\mLambda$, we denote the matrix derivative of $f$ with 
respect to $\mLambda$ as $\mD_\mLambda f$, and the Hessian of $f$ with respect to $\mLambda$ as
$\mH_\mLambda f$ the Hessian with respect to $\mLambda$.

The Hessian matrix is a $d \times d$ matrix whose $(i, j)$-th entry is equal to

\[
	\frac{\partial^2}{\partial x_i x_j} f(\vx)
\]

and is denoted by $\mH f(\vx)$.

\section{Optimisation}
\subsection{Newton-Raphson}
\cite{Nocedal2006}

\subsubsection{Definition}

Consider the second order Taylor expansion of $f(\vtheta)$ around $\vtheta_{n}$

$$
f(\vtheta) \approx f(\vtheta_{n}) + \nabla f(\vtheta_{n}) \Delta \vtheta_{n} + \half \Delta x^\top \mH_f(\vtheta_{n}) \Delta x + \sO((\Delta \vtheta_{n})^3).
$$

This expression attains its' maxima with respect to $\Delta x$ when

$$
\vzero = \mD_{\Delta \vtheta} [f(\vtheta_{n}) + \nabla f(\vtheta_{n}) \Delta \vtheta_{n} + \half \Delta \vtheta^\top \mH_f(\vtheta_{n}) \Delta \vtheta] = \nabla f(\vtheta_n) + \mH_f(\vtheta_n) \Delta \vtheta.
$$

Thus taking $\Delta \vtheta = -\nabla f(\vtheta_n) [\mH_f(\vtheta_n)]^{-1}$, we hope that choosing
$\vtheta_{n + 1} = \vtheta_{n} - \nabla f(\vtheta_n) [\mH_f(\vtheta_n)]^{-1}$ will be closer to the stationary 
point $\vtheta^*$ of $f$. This choice of $\vtheta_{n+1}$ establishes an iterative algorithm for finding the 
maxima/minima of $f$.

Convergence is achieved when the step length $\| \vtheta_{n+1} - \vtheta_{n} \| < \epsilon$, for some
suitably small choice of $\epsilon$, such as $\epsilon = 10^{-8}$.

\subsubsection{Time and memory complexity}

Execution of this algorithm requires the calculation of a vector addition, dot product and solution of a
linear system for each iteration. The vector addition and dot product can both be performed in $p$ flops,
where $p$ is the dimension of the problem. The expensive part of the computation is the solution of the linear
system $\mH_f(\vtheta_n) \Delta \vtheta$, which takes $\sO(p^3)$ flops. Even on modern computer systems, this
can quickly become computationally intractable for problems of modest size. This motivates the development
of Quasi-Newton algorithms, such as BFGS, L-BFGS and L-BFGS-B, discussed in Section \ref{sec:quasi_newton}.

\subsubsection{Radius of convergence and convergence to local solutions}

\subsection{Quasi-Newton Raphson algorithms: L-BFGS-B}
\label{sec:quasi_newton}

\subsubsection{Definition}

% Wolfe conditions
The problem of interest is to find
\[
	\min_\vx f(\vx)
\]

for some smooth $f: \R^n \to \R$. Each step involves approximately solving the subproblem
\[
	\min_\alpha f(\vx_k + \alpha \vp_k)
\]

where $\vx_k$ is the current best guess, $\vp_k \in \R^n$ is a search direction and $\alpha$ is the
step length.

Denote a univariate function $\phi$ restricted to the direction $\vp_k$ as
$\phi(\alpha) = f(\vx_k + \alpha \vp_k)$. A step length $\alpha_k$ is said to satisfy the \emph{Wolfe conditions}
if the following two inequalities hold:

\begin{enumerate}
	\item[(i)] $f(\vx_k + \alpha_k \vp_k) \leq f(\vx_k) + c_1 \alpha_k \vp_k^\top \nabla f(\vx_k)$ \text{(Armijo rule)}
	\item[(ii)] $\vp_k^\top \nabla f(\vx_k + \alpha_k \vp_k) \geq c_2 \vp_k^\top \nabla f(\vx_k)$ \text{(curvature condition)}
\end{enumerate}

The first condition ensures that the step length $\alpha_k$ decreases $f$ sufficiently, while the second condition
ensures that the slope had been sufficiently reduced.

The rationale for imposing the Wolfe conditions in an optimisation algorithm where
$\vx_{k+1} = x_k + \alpha \vp_k$ is to ensure the convergence of the gradient to zero. In particular, if the
cosine of the angle between $\vp_k$ and the gradient,

$$
\cos \theta_k = \frac{\nabla f(\vx_k)^\top \vp_k}{\|\nabla f(\vx_k)\| \|\vp_k\|}
$$

is bounded away from zero and the Wolfe conditions hold, then $\nabla f(\vx_k) \to 0$.

An additional motivation, in the case of a quasi-Newton method is that if $\vp_k = - \mB_k^{-1} \nabla f(\vx_k)$,
where the matrix $\mB_k$ is updated by the BFGS or DFP formula, then if $\mB_k$ is positive definite condition ii
implies $\mB_{k+1}$ is also positive definite.

\subsubsection{Time and memory complexity}

\section{Integrals}
\subsection{Laplace's Method of approximation}

Laplace's method of approximation is used to approximate integrals of a unimodal function with negative second
derivative at the mode, indicating that the function is decreasing rapidly away from this point. The essential
idea is that if the function is decreasing rapidly away from the mode, the bulk of the area under the function
will be within a neighbourhood of the mode. Thus the integral of the function can be well approximated by an
integral over the neighbourhood of the mode. How large that neighbourhood needs to be is estimated using
how fast the function is changing at the mode, which is estimated by $|f''(x_0)|$.

Consider
\[
	\int_a^b e^{M f(x)} dx
\]

where $f(x)$ is twice differentiable and $f''(x_0) < 0$, $M \in \R$ and $a, b \in \R \cup \{-\infty, \infty\}$.

Let $f(x)$ have a unique global maximum at $x_0$. Then, Taylor expanding about $x_0$, we have

\[
	f(x) = f(x_0) + f'(x_0) (x - x_0) + \frac{1}{2} f''(x_0) (x - x_0)^2 + R
\]

where $R = \BigO((x - x_0)^3)$.

As $f$ has a global maximum at $x_0$, the first derivative of $f$ is zero at $x_0$. Thus, the function $f(x)$
may be approximated by
\[
	f(x) \approx f(x_0) - \frac{1}{2} |f''(x_0)| (x - x_0)^2
\]

for $x$ sufficiently close to $x_0$, as the second derivative is negative at $x_0$. This ensures the
accuracy of

\[
	\int_a^b e^{M f(x)} dx \approx e^{M f(x_0)} \int_a^b e^{-M |f''(x_0)|(x - x_0)^2} dx.
\]

This is a Gaussian integral, and thus we find that

\[
	\int_a^b e^{M f(x)} dx \approx \sqrt{\frac{2 \pi}{M |f''(x_0)|}} e^{M f(x_0)}
\]

Thus we have approximated our integral by a closed form expression.

\subsubsection{Relative error} % What to write about this? Include the justification/proof?
The approximation above is $O(1/\sqrt{M})$. % See the Relative error section of the Wikipedia page, for instance.
The approximation can be made more accurate by using a Taylor
expansion beyond second order.

\subsubsection{Extending to multiple dimensions}
This approach to approximating integrals extends naturally to multiple dimensions.
Consider the second order Taylor expansion of $\log f(\vtheta)$ around the mode $\vtheta_m$
$$
\log f(\vtheta) \approx f(\vtheta_m) + (\vtheta - \vtheta_m) \nabla \log f(\vtheta) + \half (\vtheta - \vtheta_m)^\top \mH_{\log f}(\vtheta_m) (\vtheta - \vtheta_m) + \sO(\|\vtheta - \vtheta_m\|^3).
$$

Assuming that $\vtheta$ is a stationary point of $\log f$, then $\nabla f(\vtheta) = \vzero$ and so
$$
\log f(\vtheta) \approx f(\vtheta_m) + \half (\vtheta - \vtheta_m)^\top \mH_{\log f}(\vtheta_m) (\vtheta - \vtheta_m) + \sO(\|\vtheta - \vtheta_m\|^3)
$$

at such a point. The quadratic form in $\vtheta$ in the approximate expression for the log likelihood above
leads to a Gaussian approximation for the likelihood $\N(\vtheta_m, \mH_{\log f}(\vtheta_m)^{-1})$.

The approximation is crude but can be quite accurate if the likelihood is symmetric and unimodal.

\subsection{Trapezoidal integration}

If an integral cannot be approximated well using Laplace's method or some other method of approximation, then
it can often be evaluated numerically using a quadrature scheme. Trapezoidal integration is a method of
numerically computing the integral of a function $f: \R \to \R$ over the interval $(a, b)$. First, the
interval is partitioned into $x_1, x_2, \ldots, x_{n-1}, x_n$, and then
\[
	\int_a^b f(x) dx \approx \sum_{i=1}^{n-1} \frac{1}{2} \frac{f(x_i) + f(x_{i+1})}{x_{i+1} - x_i}
\]

Note that the points $x_i$ can be chosen arbitrarily within the interval $(a, b)$, and do not need to be
evenly spaced. \mgc{What's the error bound? Error analysis using Taylor's theorem?}

\section{Numerical computation}

In this section, I'd like to talk about floating point representation and its' consequences, numerical
analysis/error bounds and in particular their relevance to numerical linear algebra.

% \subsection{Floating point numbers}

% $$\text{sign } \text{significand} \times \text{base}^\text{exponent}$$

% IEEE 754: floating point in modern computers

% On modern computers, IEEE 754 is ubiquitous, and the base is $2$.

% Error bounds on $+, -, *, /, \sqrt{}$

% $$
% \text{fl}(a \text{ op } b) = (1 + \epsilon) (a \text{ op } b) \text{ provided $\text{fl}(a \text{ op } b)$ neither overflows nor underflows}
% $$

% where $\epsilon$ is the machine epsilon, the minimum difference between $1$ and the next representable
% floating point number.

% Exceptional values: $\pm \infty$, $\text{NaN}$

% When doing arithmetic on computers, it is important to understand the full ramifications of this.

% \begin{enumerate}
% \item All numerical values are in fact approximations within an interval.
% \item Every arithmetic operation has an associated amount of error, and these errors accumulate as more
% 			arithmetic operations are performed.
% \end{enumerate}

% The field of numerical analysis (references Classical and Modern Numerical Analysis,
% Demmel's course: http://www.cs.berkeley.edu/~demmel/cs267/lecture21/lecture21.html, Kahan) analyses numerical 
% algorithms to assess their error bounds, and attempt to mitigate these problems as much as possible to allow
% us to solve numerical problems on modern computers.

% \subsection{Truncation and round-off error}
% \subsection{Numerical stability}
% \subsection{Numerical differentiation and integration}

\subsection{Numerical linear algebra}
\subsubsection{Matrix norms}
We can define a norm on the space of matrices such that the matrix norms obey the following axioms
\begin{align*}
	\|\mA\|        & \geq 0                                                   \\
	\|\mA\|        & = 0 iff \mA = 0                                          \\
	\|\alpha \mA\| & = |\alpha| \|\mA\| \forall \alpha \in \R                 \\
	\|\mA + \mB\|  & \leq \|\mA\| + \|\mB\|\text{ for all matrices } \mA, \mB 
\end{align*}
Additionally, for square matrices
\[
	\|\mA\mB\| \leq \|\mA\| \|\mB\|
\]
\subsubsection{Examples}

\begin{enumerate}
	\item{The Frobenius Norm:}
	The Frobenius Norm (Euclidean norm) is a matrix norm on the matrix $\mA$ defined as the square root of the sum
	of the absolute squares of its elements,
	\[
		\|\mA\|_F = \sqrt{\sum_{i=1}^m \sum_{j=1}^n |a_{ij}|^2}.
	\]
	
	\item{The p norm:}
	The p norm is a matrix norm on the matrix $\mA$ defined as the supremum of the ratio of the norm of
	$\mA \vx$ divided by the norm of $\vx$ over the set of vectors $\vx \ne \vzero$.
	\[
		\|\mA\|_p = \sup_{\vx \ne 0} \frac{\|\mA \vx\|_p}{\|\vx\|_p}.
	\]
\end{enumerate}

The $p$-norm of a matrix and its' singular values from its' SVD are connected in the following way.

\cite{Golub:1996:MC:248979} Corollary 2.4.3 % Matrix Computations
If $\sigma_1$, $\sigma_2$, \ldots, $\sigma_p$ are the ordered descending singular values of the SVD of $\mA
\in \R^{m \times n}$ where $p = \min{(m, n)}$ then
\[
	\|\mA\|_F = \sqrt{\sigma_1^2 + \sigma_2^2 + \ldots + \sigma_p^2} \geq |\sigma_1|
\]

\subsubsection{Condition numbers}
In numerical linear algebra, a useful measure for quantifying the sensitivity of numerical procedures such as
solving linear systems is the \emph{condition number} of A. It depends on the matrix norm used and is defined by
\[
	\kappa(\mA) := \|\mA^{-1}\|.\|\mA\| = \frac{\sigma_{\text{max}}(\mA)}{\sigma_{\text{min}}(\mA)}.
\]

This is a useful tool in characterising the accuracy and stability of numerical linear algebra algorithms.
\cite{Golub:1996:MC:248979} \S 2.

\subsubsection{LDU decomposition}

If a matrix $\mA$ is square, it can be factored into the product of a lower triangular matrix and an upper 
triangular matrix
\[
	\mA = \mL \mU
\]

\subsection{Cholesky decomposition}

If a matrix $\mA$ is square $p \times p$, symmetric and positive definite then it can be factored into the
product a lower triangular matrix and its' transpose
\[
	\mA = \mL \mL^\top
\]
by noting that
\begin{align*}
	\mA = \mL^\top \mL &= \begin{pmatrix}
	\mL_{11}          & 0                                     & 0                                    \\
	\mL_{21}          & \mL_{22}                              & 0                                    \\
	\mL_{31}          & \mL_{32}                              & \mL_{33}                             
	\end{pmatrix}
	\begin{pmatrix}
	\mL_{11}          & \mL_{21}                              & \mL_{31}                             \\
	0                 & \mL_{22}                              & \mL_{32}                             \\
	0                 & 0                                     & \mL_{33}                             
	\end{pmatrix}	\\
	&= \begin{pmatrix}
	\mL_{11}^2        &                                       & \text{symmetric}                     \\
	\mL_{21} \mL_{11} & \mL_{21}^2 + \mL_{22}^2               &                                      \\
	\mL_{31} \mL_{11} & \mL_{31} \mL_{21} + \mL_{32} \mL_{22} & \mL_{31}^2 + \mL_{32}^2 + \mL_{33}^2 
	\end{pmatrix}
\end{align*}
and so
\begin{align*}
	\mL_{jj} & = \sqrt{\mA_{jj} - \sum_{k=1}^{j-1} \mL_{jk}^2} \text{ and}                               \\
	\mL_{ij} & = \frac{1}{\mL_{jj}} (\mA_{ij} - \sum_{k=1}^{j-1} \mL_{ik} \mL_{jk}), \text{ for } i > j. 
\end{align*}

This can be computed in $\BigO(p^2)$ for a $p \times p$ matrix. Once this form is obtained, it has several
computational advantages over working with the original matrix $\mA$. Firstly, as $\mL$ is lower triangular
only the non-zero entries need to be stored. Secondly, the triangular form of $\mL$ allows matrix equations
involving $\mL$ to be solved in $\BigO{(p^2)}$ using back-substitution and forward-substitution, without using
Gaussian elmination or calculating a matrix inverse.

\subsection{Difficulty of inverting matrices which are nearly singular}
In computational statistics, the task of inverting matrices often arises when
attempting to solve a statistical problem such as fitting a model. While theoretically this task can always be
done successfully, in practice if the matrix or its' inverse being inverted is close to singular this can be
difficult or even impossible. This can arise in numerical procedures which are prone to overflow, such as
Poisson regression, which overflows due to the exponential link function in the formulation of the problem.

\subsubsection{Sensitivity of solving linear systems}

A precise measure of linear system sensitivity can be obtained by considering the parameterised system
\[
	(\mA + \epsilon \mF) \vx(\epsilon) = \vb + \epsilon + \vf,  \vx(0) = \vx
\]
where $\mF \in \R^{n \times n}$ and $\vf \in \R^n$. If $\mA$ is nonsingular, it is clear that $x(\epsilon)$ is
differentiable in a neighbourhood of zero. Moreover, $\dot{x}(0) = \mA^{-1} (\vf - \mF \vx)$ and so the
Taylor series expansion for $\vx$ has the form
\[
	x(\epsilon) = x + \epsilon \dot{\vx}(0) + \BigO(\epsilon^2).
\]

Using any vector norm and consistent matrix norm we obtain
\[
	\frac{\|\vx(\epsilon) - \vx\|}{\vx} \leq |\epsilon| \|\mA^{-1}\| \left\{\frac{\|\vf\|}{\|\vx\|} + \|\mF\|\right\} + \BigO(\epsilon^2)
\]

From \S 2.6.2 of \citep{Golub:1996:MC:248979}
when solving linear equations in the form $\mA \vx = \vb$
\[
	\frac{\|\vx(\epsilon) - \vx\|}{\|\vx\|} \leq \kappa(\mA) (\rho_A + \rho_b) + \BigO(\epsilon^2)
\]
where $\rho_A = |\epsilon| \frac{\|\mF\|}{\|\mA\|}$ and $\rho_B = |\epsilon| \frac{\|f\|}{\|b\|}$ represent
the relative errors in $\mA$ and $\vb$, respectively. Thus the relative error in $\vx$ can be $\kappa(\mA)$
times the relative error in $\mA$ and $\vb$. In this sense, the condition number $\kappa(\mA)$ quantifies the
sensitivity of the $\mA \vx = \vb$ problem.

% From \cite{trefethen97}'s section on Conditioning and Condition Numbers, p91

A problem is \emph{well-conditioned} if $\kappa$ is small (e.g. $1$, $10$, $10^2$) and \emph{ill-conditioned}
if $\kappa$ is large (e.g. $10^6$, $10^{16}$) \cite{trefethen97}.  If a matrix is either close to singular or
has very large singular values, then it will be ill-conditioned.

An immediate consequence of this is that if either $\mA$ or its' inverse are ill-conditioned, then inversion
or solution of the linear system will be highly numerically unstable. In our ZIP code, we routinely saw values of 
at least $10^5$ on the diagonals of our Cholesky factors until we changed the parameterisation to the safe
exponential parameterisation.

\subsubsection{Sparsity of the inverse}
We often wish to work with sparse matrices due to the computational advantages and increased numerical
stability they provide. But sparse matrices do not necessarily have sparse inverses as will be demonstrated
below. Consider a block matrix
\[
	\begin{pmatrix}
		\mA & \mB \\
		\mC & \mD 
	\end{pmatrix}.
\]
Then the inverse of this matrix is given by the block-matrix inverse formula
%Demonstrate this using
%the block-inverse formula. Inverting a matrix is an inherently unstable operation. The number of floating
%point operations involved leads to magnification of error.

%Block inverse formula \\
\[
	\begin{pmatrix}
		\mA & \mB \\
		\mC & \mD 
	\end{pmatrix}^{-1}
	=
	\begin{pmatrix}
		\mI           & 0 \\
		-\mD^{-1} \mC & I 
	\end{pmatrix}
	\begin{pmatrix}
		(\mA - \mB^{-1}\mC)^{-1} & 0        \\
		0                        & \mD^{-1} 
	\end{pmatrix}
	\begin{pmatrix}
		\mI          & 0   \\
		-\mB\mD^{-1} & \mI 
	\end{pmatrix}
\]

An immediate consequence of this formula is that if any of the blocks of the original matrix is not sparse,
then the inverse will not be sparse in \emph{any} of the corresponding blocks.

\section{Positive semi-definite matrices}
A positive semidefinite matrix $\mLambda$ is a square matrix of size $n \times n$ such that for any vector
$\vx \in \R^n$, $\vx^\top \mLambda \vx \geq 0$. Equivalently, all eigenvalues of $\mLambda$ are greater than or 
equal to 0. A typical example arising in statistics is the covariance matrix of a random vector, which is
positive definite by definition.

\section{Splines}
The most general form of the univariate regression problem is

\[
	y_i = f(x_i)
\]

where $f: \R \to \R$ is unknown, and we wish to estimate it as $\hat{f}$. We treat the functional form of
$f$ as unknown, and attempt to estimate it purely from the data available to us. Fully non-parametric regression
is a difficult problem to solve, but the problem can be simplified by pre-specifying the points at which the
function may change curvature, the so-called knots.

\subsection{B-Splines}
% This is taken from the Wikipedia page on the subject
A B-Spline is a piecewise polynomial function of degree $< n$ in a variable $x$. It is defined over a
domain $t_0 \leq x \leq t_m, m=n$. The points where $x = t_j$ are known as knots or break-points. The
number of internal knots is equal to the degree of the polynomial if there are no knot multiplicities.
The knots must be in ascending order. The number of knots is the minimum for the degree of the B-spline,
which has a non-zero value in the range between the first and last knot. Each piece of the function is a
polynomial of degree $< n$ between and including adjacent knots. A B-Spline is a continuous function at the
knots. When all internal knots are distinct its derivatives are also continuous up to the derivative of degree
$n - 1$. If internal knots are coincident at a given value of x, the continuity of derivative order is reduced
by 1 for each additional knot.

For any given set of knots, the B-spline is unique, hence the name, B being short for Basis. The usefulness
of B-splines lies in the fact that any spline function of order $n$ on a given set of knots can be expressed
as a linear combination of B-spline:

% B-Splines
$$
S_{n, t}(x) = \sum_i \alpha_i B_{i, n}(x)
$$

where

\begin{align*}
	B_{i, 0}(x) & := \begin{cases}                                                                                                        
	1           & \text{if } t_i \leq x < t_{i+1}                                                                                         \\
	0           & \text{otherwise}                                                                                                        
	\end{cases} \\
	B_{i, k}(x) & := \frac{x - t_i}{t_{i + 1} - t_i} B_{i, k-1} (x) + \frac{t_{i + k + 1} - x}{t_{i + k + 1} - t_{i + 1}} B_{i, k-1} (x). 
\end{align*}

\subsection{O'Sullivan Splines}

Let $B_1$, \ldots, $B_{K+4}$ be the cubic B-spline basis functions defined by the knots $\kappa_1$ to
$\kappa_{K+4}$

$B_{ik} = B_k (x)$
$B_x = [B_1(x), \ldots, B_K+4(x)]$

O'Sullivan splines are penalised splines which are penalised using the $\mOmega$ matrix.

Let $\mOmega$ be the $(K+4) \times (K+4)$ matrix where the $(k, k')-th$ element is
\[
	\mOmega_{k k'} = \int_a^b B''_k(x) B''_{k'}(x) dx.
\]
Then the O'Sullivan spline estimate of the true function $f$ at the point $x$ is
$\hat{f}_O(x; \lambda) = \mB_x \hat{\vnu}_O$, where
$\hat{\vnu}_O = (\mB^\top \mB + \lambda \mOmega)^{-1} \mB^\top \vy$.

$\mOmega$ is defined in this way to penalise oscillation, which is measured by the second derivative.

The penalty differs from P-splines in that the P-spline penalty matrix is $\mD_2^\top \mD_2$ where $\mD_2$ is
the second-order differencing matrix.

% Divided difference notation?
% Lagrange's interpolating polynomials?
% Semiparametric regression / Connection to mixed models
We follow the discussion of semiparametric regression in \cite{RuppertWandCarroll}.
Using a mixed models setup to fit spline models protects against overfitting.
Constructing an $\mZ$ matrix with the appropriate B-Spline function evaluations in each of the rows, where
each column corresponds to a knot.

\subsection{Degrees of freedom}

% From Section 2.5.2, Semiparametric regression

Consider the fixed effects model $\vy = \mX \vbeta$ where $\mX \in \R^{n \times p}$. If the hat matrix is
defined as $\mH = \mX (\mX^\top\mX)^{-1} \mX^\top$ then

\begin{align*}
	\tr(\mH) & = \tr[\mX(\mX^\top\mX)^{-1}\mX^\top]  \\
	         & = \tr[\mX^\top\mX (\mX^\top\mX)^{-1}] \\
	         & = \tr(\mI_p)                          \\
	         & = p                                   
\end{align*}

which is the number of parameters in the fixed effects model. Thus we use the trace of the hat matrix as
the definition of the effective degrees of freedom of the model. This definition then naturally extends to
more general smoother matrices which can be used to fit models involving splines or other kinds of smoothing.

\section{Generalised Linear Mixed models}

\subsection{Exponential family}

An exponential family is a set of probability distributions of the form
$$
f_X(x|\vtheta) = h(x) g(\theta) \exp(\eta(\theta) \dot T(x)).
$$

Most commonly used probability distributions in statistics such as normal, exponential, gamma,
chi-squared, beta, Dirichlet, Bernoulli, categorical, Poisson, Wishart and Inverse Wishart can
be represented in this form.

Canonical form
$$
\log p(\vnu; \vy) = \vy^\top \mC \vnu - b(\mC \vnu)
$$

Different choices of the function $b:\R \to \R$ lead to different generalised linear mixed models. A set
of available functions and corresponding distributions is presented in Table \ref{tab:b_functions}.

\begin{table}
	\caption{Table of b functions}
	\label{tab:b_functions}
	\begin{tabular}{|l|lll|}
		\hline
		Distribution                         & Parameter & Natural parameter       & Inverse parameter mapping \\
		\hline
		Bernoulli distribution               & $p$       & $\log{\frac{p}{1 - p}}$ & $\frac{1}{1 + e^\eta}$    \\
		Poisson distribution                 & $\lambda$ & $\log \lambda$          & $e^\eta$                  \\
		Normal distribution (known variance) & $\mu$     & $\frac{\mu}{\sigma}$    & $\sigma \eta$             \\
		\hline
	\end{tabular}
\end{table}

\subsection{Bias of MLE}

How does this relate to Bayesian estimation? Maybe it's just interesting background.
Bernstein-von Mises Theorem

\bibliographystyle{elsarticle-harv}
% \bibliography{Chapter_0_preliminary}
\bibliography{references_mendeley}

\end{document}
