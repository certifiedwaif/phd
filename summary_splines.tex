\documentclass{amsart}
\usepackage{amsmath}
\usepackage{amsfonts}
\usepackage{amssymb}
\begin{document}
\section{Splines summary}
From Elements of Statistical Learning

Assume we have bivariate data X and Y. Consider two random variables X and Y.
Suppose that we wish to construct a statisical model for the joint distribution 
P(X, Y), in order to study the relationships between X and Y in our data. We
could construct a statistical model
\begin{align*}
	Y = f(X) + \epsilon	
\end{align*}
where the random error has $E(\epsilon) = 0$ and is independent of X. What form
should the function f take? General functions f can be approximated by linear
combinations of basis functions of the form
\begin{align*}
	f_\theta(x) = \sum_{k=1}^K h_k(x) \theta_k
\end{align*}

Then we can fit such an $f_\theta$ to our data by minimising the residual
sum of squares
\begin{align*}
	RSS(\theta) = \sum_{i=1}^N (y_i - f_\theta(x_i))^2
\end{align*}

as a function of $\theta$. We can use least squares % you have this at home
to fit $\hat{f_{\theta}}$ to our function f in closed form.

\begin{align*}
	\hat{\theta} = (B^TB + \lambda \Omega)^{-1}B^T y
\end{align*}

Ormerod and Wand did this using O'Sullivan splines in their 2008 paper. This
form of spline apparently has numerical advantages. The co-efficient matrix
$B$ where $\{B\}_{ik} = B_{k}(x_i)$ is
almost banded, as splines have highly localised support.
$\Omega = \int_{-\infty}^{\infty} B''_{k}(x) B''_{k'}(x) dx$

\section{Things I don't understand}
\begin{enumerate}
	\item How did Wand and Ormerod change the basis from B Splines to O'Sullivan
	splines?
	\item Why was using Simpson's rule to exactly approximate the penalty matrix
	such a huge win versus the differencing matrix?
	\item Why do we fit splines as variance components?
	\item Numerics, OMG? Why does the Cholesky decomposition retain sparsity,
	LU decomposition, banded matrices
\end{enumerate}

Penalised splines
Fit using Least squares
Numerical advantages to O'Sullivan splines, using in Wand/Ormerod 2008.

\end{document}
