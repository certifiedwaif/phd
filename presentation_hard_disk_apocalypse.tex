\documentclass{beamer}
\usepackage{soul}
\begin{document}

\begin{frame}
\frametitle{Hard disk apocalypse}
What have I been doing? Buying and installing new hard disks. Deleting all of Shila's data.

Why? \st{I'm a terrible person. }It was an accident.

How did it happen? We installed and powered up two 4 TB Seagate drives. Then we formatted and made file systems on two 4 TB Seagate drives. Unfortunately, these were not the same two 4 TB Seagate drives that we had just installed.
\end{frame}

\begin{frame}
\frametitle{Lessons learned}
\begin{enumerate}
\item Know your hardware. Especially know your servers.
\item Always ensure that when you add new hard disks to a server, it's detected by the BIOS.
\item If there's a file system already on a hard disk, check what's on it before blowing it away.
\item Shila has impressive coping skills.
\item Data recovery tools actually work pretty well, when given a fighting chance.
\item The University of Sydney has an impressive Internet connection.
\item Backups? This is like doing a trapeze act with no safety net.
\end{enumerate}
\end{frame}

\begin{frame}
\frametitle{What else have I been doing?}
The melanoma data has been downloaded again, and is currently being converted from SRA to
gzipped FASTQ. If not for the gzipping, we would again run out of disk space.

It should all finish in a couple of days. Then we'll begin mapping.

The bottleneck is the external Seagate disks, which are currently connected with USB 2.0. 
They can write at about 20 meg./second each.

The disks support USB 3.0, but the server currently doesn't. If we installed a USB 3.0 PCI 
adapter, they could read and write at more like 150 meg./second each. These cost around 
\$40, so I think we should get one.

Happy to install it myself. What could possibly go wrong?
\end{frame}

\end{document}