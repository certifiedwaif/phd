% numerical_results_vb_normal_linear_models.tex

\subsection{Results}

To assess the accuracy of our variational approximations to the exact posteriors, we examined a number of
measures. All of the measures that we examined were functions of $\kappa$, $p$ and $R^2$, and so serve to
characterise the performance of our approximation on any possible data set with a number of observations
$n$, number of covariates $p$ and correlation $R$ between the response $\vy$ and the covariate matrix $\mX$.

\subsection{Shrinkage}

Exact posterior and approximate shrinkage $\left( \frac{g}{1 + g} \right)$ were calculated for a range of
values of $p$, $n$ and $R^2$ to compare their values. As can be seen from Figure $\ref{fig:shrinkage}$, the
values of the exact posterior shrinkage and approximate shrinkage are almost the same over most of the range
of these values, with deviation only noticeable in the $p=10$ and $p=20$ cases.

\begin{figure}[p]
	\includegraphics[width=17cm, height=17cm]{code/taug/Shrinkage.pdf}
	\caption{The ratio $\frac{g}{1 + g} | \vy$ controls the degree to which the model fit tends back towards
		the prior mean. This ratio is approximated in the variational approximation by $(1 + \tau_g^{-1})^{-1}$. We
		compare these two quantities in the figure above for a range of numbers of covariates, sample sizes and $R^2$
		values. Here $n = \kappa p$.}
	\label{fig:shrinkage}
\end{figure}

Coefficient posterior variance \\

We can see from Figure \ref{fig:variance} that as $p$ and $\kappa$ increase, the approximation to the
posterior variance of \ldots becomes more and more accurate.

\begin{figure}[p]
	\includegraphics[width=17cm, height=17cm]{code/taug/Variance.pdf}
	\caption{The posterior variance of the example model $p(\sigma^2 | \vy)$ is compared against the approximation
		to the posterior variance $q(\sigma^2)$ across a range of numbers of covariates, sample sizes and $R^2$
		values. Here $n = \kappa p$.}
	\label{fig:variance}
\end{figure}

Accuracy of approximation to $p(\sigma^2 | \vy)$ \\

We assessed the accuracy of the approximation to $p(\sigma^2 | \vy)$ by $q(\sigma^2)$ by numerically evaluating
the integral
\[
	1 - \frac{1}{2} \int_0^\infty |p(\sigma^2 | \vy) - q(\sigma^2)| d \sigma^2
\]
for a range of values of $p$, $\kappa$ and $R^2$. The results are presented in Figure \ref{fig:accuracy_sigma2}.

\begin{figure}[p]
	\includegraphics[width=17cm, height=17cm]{code/taug/Accuracy_sigma2.pdf}
	\caption{The accuracy of the approximation $q(\sigma^2)$ is assessed by computing the integral   $1 -
		\frac{1}{2} \int_0^\infty |p(\sigma^2 | \vy) - q(\sigma^2)| d \sigma^2$ and graphing the result over a range
		of numbers of covariates, sample sizes and $R^2$  values. Here $n = \kappa p$.}
	\label{fig:accuracy_sigma2}
\end{figure}

Accuracy of approximation to $p(g | \vy)$ \\

\begin{figure}[p]
	\includegraphics[width=17cm, height=17cm]{code/taug/Accuracy_g.pdf}
	\caption{The accuracy of the approximation $q(g)$ is assessed by computing the integral   $1 -
		\frac{1}{2} \int_0^\infty |p(g | \vy) - q(g)| d g$ and graphing the result over a range
		of numbers of covariates, sample sizes and $R^2$  values. Here $n = \kappa p$.}
	\label{fig:accuracy_g}
\end{figure}

We assessed the accuracy of the approximation to $p(g | \vy)$ by $q(g)$ by numerically evaluating the integral
\[
	1 - \frac{1}{2} \int_0^\infty |p(g | \vy) - q(g)| d g
\]
for a range of values of $p$, $\kappa$ and $R^2$. The results are presented in Figure \ref{fig:accuracy_g}.


$\log{p(\vy)}$ versus ELBO \\

We assessed the accuracy of the variational lower bound by calculating the relative error of
$\log{\underline{p}(\vy)}$ versus $\log{p(\vy)}$. The results are presented in Figure
\ref{fig:relative_error}.

\begin{figure}[p]
	\includegraphics[width=17cm, height=17cm]{code/taug/Relative_error_log_p.pdf}
	\caption{The accuracy of the variational lower bound was assessed by computing the relative error
		$\frac{\log p(\vy) - \log \underline{p}(\vy)}{p(\sigma^2 | \vy)}$ over a range
		of numbers of covariates, sample sizes and $R^2$  values. Here $n = \kappa p$.}
	\label{fig:relative_error}
\end{figure}

Precision \\

The accuracy of the precision was assessed by calculating the relative error of the approximate precision
versus the exact posterior precision. The results are presented in Figure \ref{fig:precision}. As $p$ and
$\kappa$ increase, we can see that the approximate precision is converging towards the exact precision for all
values of $R^2$.

\begin{figure}[p]
	\includegraphics[width=17cm, height=17cm]{code/taug/Precision.pdf}
	\caption{The accuracy of the approximation to the posterior precision was assessed by plotting
		$p(\sigma^{-2} | \vy)$ against $q(\sigma^{-2})$ over a range
		of numbers of covariates, sample sizes and $R^2$  values. Here $n = \kappa p$.}
	\label{fig:precision}
\end{figure}

\subsection{Marginal covariate inclusion probabilities}

The marginal covariate inclusion probabilities from the variational approximation match those from the the
exact posterior likelihood very closely. Comparing these marginal covariate inclusion probabilities to those
produced by weighting each model considered by AIC and BIC, we see that the probabilities produced by the
variational approximation and exact posterior likelihood are more conservative.

\begin{figure}[p]
	\includegraphics[width=17cm, height=17cm]{code/taug/Log_of_Relative_error_of_Variance_of_g.pdf}
	\caption{The accuracy of the approximation to the variance of $g$ was assessed by computing the relative
		error of $q(g)$ relative to $p(g | \vy)$ over a range
		of numbers of covariates, sample sizes and $R^2$  values. Here $n = \kappa p$.}
	\label{fig:rel_error_var_g}
\end{figure}


% vw1 <- read.csv("Hitters_vw1.csv", header=FALSE)
% r <- hist(as.matrix(vw1), breaks=19, axes=FALSE, prob=TRUE, main="", xlab="")
% axis(1, r$mid, c("1", "2", "3", "4", "5", "6", "7", "8", "9", "10", "11", "12", "13", "14", "15", "16", "17", "18"))

% bodyfat
% Major League Baseball Data from the 1986 and 1987 seasons.
% An Introduction to Statistical Learning with Applications in R

% \begin{tabular}{|l|lllllllllllllllllll|}
% 	\hline
% 	$\vp$ & 0.137 & 0.130 & 0.257 & 0.982 & 0.921 & 0.173 & 0.633 & 0.562 & 0.623 & 0.480 & 0.441 & 0.499 & 0.197 & 0.926 & 0.131 & 0.174 & 0.128 & 0.901 & 0.851 \\
% 	$\vq$ & 0.137 & 0.130 & 0.257 & 0.982 & 0.922 & 0.172 & 0.635 & 0.562 & 0.625 & 0.480 & 0.440 & 0.499 & 0.197 & 0.927 & 0.130 & 0.173 & 0.127 & 0.902 & 0.853 \\
% 	\hline
% \end{tabular}

\subsubsection{Hitters}

The Hitters data set was taken from the StatLib library maintained by Carnegie Mellon University, and used in
\citep{James:2014:ISL:2517747}. The data was gathered from the the performance of players during the Major
League Baseball seasons for 1986 and 1987, and includes the following covariates:

\begin{tabular}{|ll|}
	\hline
	Covariate & Description                                                                        \\
	\hline
	AtBat     & Number of times at bat in 1986                                                     \\
	Hits      & Number of hits in 1986                                                             \\
	HmRun     & Number of home runs in 1986                                                        \\
	Runs      & Number of runs in 1986                                                             \\
	RBI       & Number of runs batted in in 1986                                                   \\
	Walks     & Number of walks in 1986                                                            \\
	Years     & Number of years in the major leagues                                               \\
	CAtBat    & Number of times at bat during his career                                           \\
	CHits     & Number of hits during his career                                                   \\
	CHmRun    & Number of home runs during his career                                              \\
	CRuns     & Number of runs during his career                                                   \\
	CRBI      & Number of runs batted in during his career                                         \\
	CWalks    & Number of walks during his career                                                  \\
	League    & A factor with levels A and N indicating player’s league at the end of 1986       \\
	Division  & A factor with levels E and W indicating player’s division at the end of 1986     \\
	PutOuts   & Number of put outs in 1986                                                         \\
	Assists   & Number of assists in 1986                                                          \\
	Errors    & Number of errors in 1986                                                           \\
	Salary    & 1987 annual salary on opening day in thousands of dollars                          \\
	NewLeague & A factor with levels A and N indicating player’s league at the beginning of 1987 \\
	\hline
\end{tabular}

The marginal covariate inclusion probabilities are presented in Figure \ref{fig:Hitters_inclusion}.

\begin{figure}[p]
	\includegraphics[scale=.4]{Hitters_variable_selection.pdf}
	\caption{Hitters marginal inclusion probablities}
	\label{fig:Hitters_inclusion}
\end{figure}

The model ranking scatterplots are presented in Figure \ref{fig:Hitters_model_ranking}.



\begin{figure}[p]
	\includegraphics[scale=.4]{code/Model_selection_scatter_plot_1.pdf}
	\caption{Hitters model ranking scatterplot}
	\label{fig:Hitters_model_ranking}
\end{figure}

\subsubsection{Body Fat}
The Body Fat data set was taken from \citep{Tarr2015}, and is on the relationship between percentage of body
fat and simple body measurements. The data set consists of 128 observations, and the covariates in the data set
are:

\begin{tabular}{|ll|}
	\hline
	Covariate & Description                                                                  \\
	\hline
	Id        & Identifier                                                                   \\
	Bodyfat   & Bodyfat percentage                                                           \\
	Age       & Age (years)                                                                  \\
	Weight    & Weight (kg)                                                                  \\
	Height    & Height (inches)                                                              \\
	Neck      & Neck circumference (cm)                                                      \\
	Chest     & Chest circumference (cm)                                                     \\
	Abdo      & Abdomen circumference (cm) "at the umbilicus and level with the iliac crest" \\
	Hip       & Hip circumference (cm)                                                       \\
	Thigh     & Thigh circumference (cm)                                                     \\
	Knee      & Knee circumference (cm)                                                      \\
	Ankle     & Ankle circumference (cm)                                                     \\
	Bic       & Extended biceps circumference (cm)                                           \\
	Fore      & Forearm circumference (cm)                                                   \\
	Wrist     & Wrist circumference (cm) "distal to the styloid processes"                   \\
	\hline
\end{tabular}

The marginal covariate inclusion probabilities are presented in Figure \ref{fig:bodyfat_inclusion}.

% \begin{tabular}{|l|lllllllllllll|}
% 	\hline
% 	$\vp$ & 0.938 & 0.136 & 0.182 & 0.072 & 0.071 & 0.108 & 0.147 & 1.000 & 0.134 & 0.141 & 0.323 & 0.619 & 0.221 \\
% 	$\vq$ & 0.939 & 0.136 & 0.182 & 0.071 & 0.070 & 0.107 & 0.146 & 1.000 & 0.134 & 0.140 & 0.323 & 0.620 & 0.221 \\
% 	\hline
% \end{tabular}

\begin{figure}[p]
	\includegraphics[scale=.4]{bodyfat_variable_selection.pdf}
	\caption{bodyfat marginal inclusion probablities}
	\label{fig:bodyfat_inclusion}
\end{figure}

The model ranking scatterplots are presented in Figure \ref{fig:bodyfat_model_ranking}.

\begin{figure}[p]
	\includegraphics[scale=.4]{code/Model_selection_scatter_plot_2.pdf}
	\caption{bodyfat model ranking scatterplot}
	\label{fig:bodyfat_model_ranking}
\end{figure}

\subsubsection{Wage}

The Wage gap data set was taken from \citep{James:2014:ISL:2517747}, on 3000 workers in the Mid-Atlantic
region. The data set includes 12 covariates:

\begin{tabular}{|ll|}
	\hline
	Covariate        & Description                                                                           \\
	\hline
	Year             & Year that the wage information was recorded                                           \\
	Age              & Age of worker                                                                         \\
	Sex              & Gender                                                                                \\
	Marital status   & Marital status -- Never Married, Married, Widowed, Divorced or Seperated              \\
	Race             & White, Black, Asian or Other                                                          \\
	Education        & Education level -- Did not graduate high school, Graduated high school, Some College, 
	College Graduate \\
	                 & or Advanced College Degree                                                            \\
	Region           & Region of the country                                                                 \\
	Job class        & Type of job                                                                           \\
	Health           & Either Good or Less, or Very Good or Better                                           \\
	Health insurance & Yes or No                                                                             \\
	Log of wage      & The log of the worker's wage                                                          \\
	Wage             & The worker's wage                                                                     \\
	\hline
\end{tabular}

The marginal covariate inclusion probabilities are presented in Figure \ref{fig:Wage_inclusion}.

% \begin{tabular}{|l|lllllllllllllllll|}
% 	\hline
% 	$\vp$ & 1 & 1 & 0.010 & 0.024 & 1 & 0.054 & 0.083 & 0.019 & 0.011 & 0.010 & 0.013 & 0.014 & 0.011 & 0.014 & 0.057 & 0.042 & 0.033 \\
% 	$\vq$ & 1 & 1 & 0.010 & 0.024 & 1 & 0.054 & 0.083 & 0.019 & 0.011 & 0.010 & 0.013 & 0.014 & 0.011 & 0.014 & 0.057 & 0.042 & 0.033 \\
% 	\hline
% \end{tabular}

\begin{figure}[p]
	\includegraphics[scale=.4]{Wage_variable_selection.pdf}
	\caption{Wage marginal inclusion probablities}
	\label{fig:Wage_inclusion}
\end{figure}

The model ranking scatterplots are presented in Figure \ref{fig:Wage_model_ranking}.

\begin{figure}[p]
	\includegraphics[scale=.4]{code/Model_selection_scatter_plot_3.pdf}
	\caption{Wage model ranking scatterplot}
	\label{fig:Wage_model_ranking}
\end{figure}


\subsubsection{Graduation Rate}

The Graduation Rate data set was taken from \citep{James:2014:ISL:2517747}. The data is drawn from the 1995
issue of US News and World Report, and is on a large number of US Colleges. The data set consists of 777
observations, and includes the following covariates:

\begin{tabular}{|ll|}
	\hline
	Covariate   & Description                                                             \\
	\hline
	Private     & A factor with levels No and Yes indicating private or public university \\
	Apps        & Number of applications received                                         \\
	Accept      & Number of applications accepted                                         \\
	Enroll      & Number of new students enrolled                                         \\
	Top10perc   & Percentage of new students from top 10\% of H.S. class                  \\
	Top25perc   & Percentage of new students from top 25\% of H.S. class                  \\
	F.Undergrad & Number of fulltime undergraduates                                       \\
	P.Undergrad & Number of parttime undergraduates                                       \\
	Outstate    & Out-of-state tuition                                                    \\
	Room.Board  & Room and board costs                                                    \\
	Books       & Estimated book costs                                                    \\
	Personal    & Estimated personal spending                                             \\
	PhD         & Percentage of faculty with Ph.D.’s                                    \\
	Terminal    & Percentage of faculty with terminal degree                              \\
	S.F.Ratio   & Student/faculty ratio                                                   \\
	perc.alumni & Percentage of alumni who donate                                         \\
	Expend      & Instructional expenditure per student                                   \\
	Grad.Rate   & Graduation rate                                                         \\
	\hline
\end{tabular}

The marginal covariate inclusion probabilities are presented in Figure \ref{fig:GradRate_inclusion}.

% \begin{tabular}{|l|lllllllllllllllll|}
% 	\hline
% 	$\vp$ & 0.913 & 1.000 & 0.090 & 0.108 & 0.110 & 0.602 & 0.127 & 0.945 & 0.999 & 0.999 & 0.201 & 0.864 & 0.262 & 0.105 & 0.146 & 0.977 & 0.437 \\
% 	$\vq$ & 0.914 & 1.000 & 0.090 & 0.108 & 0.110 & 0.602 & 0.127 & 0.945 & 0.999 & 0.999 & 0.201 & 0.864 & 0.262 & 0.105 & 0.146 & 0.977 & 0.437 \\
% 	\hline
% \end{tabular}

\begin{figure}[p]
	\includegraphics[scale=.4]{GradRate_variable_selection.pdf}
	\caption{GradRate marginal inclusion probablities}
	\label{fig:GradRate_inclusion}
\end{figure}

The model ranking scatterplots are presented in Figure \ref{fig:GradRate_model_ranking}.

\begin{figure}[p]
	\includegraphics[scale=.4]{code/Model_selection_scatter_plot_4.pdf}
	\caption{GradRate model ranking scatterplot}
	\label{fig:GradRate_model_ranking}
\end{figure}

\subsubsection{US Crime}
The US Crime data set was taken from the \texttt{MASS} package \citep{Venables2002}, and is on the effect of
punishment regimes on crime rates. The data set includes 47 states of the United States of America. The
variables have been re-scaled for convenience, and include the following covariates:

\begin{tabular}{|ll|}
	\hline
	Covariate & Description                                                     \\
	\hline
	M         & percentage of males aged 14–24.                               \\
	So        & indicator variable for a Southern state.                        \\
	Ed        & mean years of schooling.                                        \\
	Po1       & police expenditure in 1960.                                     \\
	Po2       & police expenditure in 1959.                                     \\
	LF        & labour force participation rate.                                \\
	M.F       & number of males per 1000 females.                               \\
	Pop       & state population.                                               \\
	NW        & number of non-whites per 1000 people.                           \\
	U1        & unemployment rate of urban males 14–24.                       \\
	U2        & unemployment rate of urban males 35–39.                       \\
	GDP       & gross domestic product per head.                                \\
	Ineq      & income inequality.                                              \\
	Prob      & probability of imprisonment.                                    \\
	Time      & average time served in state prisons.                           \\
	y         & rate of crimes in a particular category per head of population. \\
	\hline
\end{tabular}

The marginal covariate inclusion probabilities are presented in Figure \ref{fig:USCrime_inclusion}.

% \begin{tabular}{|l|lllllllllllllll|}
% 	\hline
% 	$\vp$ & 0.226 & 0.849 & 0.997 & 0.216 & 0.502 & 0.244 & 0.358 & 0.569 & 0.324 & 0.202 & 0.424 & 0.696 & 0.869 & 0.229 & 0.655 \\
% 	$\vq$ & 0.220 & 0.856 & 0.997 & 0.210 & 0.507 & 0.240 & 0.358 & 0.573 & 0.318 & 0.196 & 0.418 & 0.699 & 0.876 & 0.224 & 0.661 \\
% 	\hline
% \end{tabular}

\begin{figure}[p]
	\includegraphics[scale=.4]{USCrime_variable_selection.pdf}
	\caption{USCrime marginal inclusion probablities}
	\label{fig:USCrime_inclusion}
\end{figure}

The model ranking scatterplots are presented in Figure \ref{fig:USCrime_model_ranking}.

\begin{figure}[p]
	\includegraphics[scale=.4]{code/Model_selection_scatter_plot_5.pdf}
	\caption{USCrime model ranking scatterplot}
	\label{fig:USCrime_model_ranking}
\end{figure}
