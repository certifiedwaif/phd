\documentclass{amsart}
\title{Things John asked me to think about}
\author{Mark Greenaway}
\begin{document}
\maketitle
\section{Multi-threading}
The multicore package is very easy to use. It supplies an mclapply function which
runs the function passed to apply on multiple cores, and returns the results of each
function in a list.

We could also link R to a multi-threaded version of BLAS. In fact, we should do this
as a matter of course.

\section{Distribution computing}
The two most popular approaches seem to be MPI and Hadoop. Both are supported by R
packages. Both of these can also be used to run programs across multiple cores on
a single machine. The advantage of these approaches is that they can also be used to
run programs across an entire cluster of machines.

Both of these solutions require programs to be written to distribute work across
the cluster.

\subsection{MPI}
MPI is based on a message-passing metaphor. It is mostly used by people who do
scientific computing, and is the standard for distributed scientific computing.
MPI is written in C.

There is an excellent tutorial for the Rmpi package here.
http://math.acadiau.ca/ACMMaC/Rmpi/index.html

\subsection{Hadoop}
Hadoop is an implementation of Google's MapReduce idea. This is what most of the 
``Big Data'' people are using. To run a Hadoop job, you break your program up into
a Map phase, which is run on every computer in the cluster, and a Reduce phase,
which is run on the Master and combines all of the results from the Map phase
run across the entire cluster into the final result.

Apache Hadoop is written in Java.

\section{Distributed file system}
The bioinformatics researchers are dealing with very large data sets, and the data
sets that they will be dealing with in the future will be even larger. A single very
fast, very large hard disk may not be able to store all the data, and furthermore may
not be able to read it fast enough.

A solution to this is to use multiple hard disks, with multiple controllers, possibly
spread across multiple machines. To ensure that data passes between the machines as
fast as possible, it is recommended that high speed interconnections are used e.g.
InfiniBand or 10 gigabit Ethernet.

All of the alternatives below have commercial support:

\subsection{Lustre}
Very mature, developed in 1999. Very high performance. Figures as high as
peak read performance of one terabyte a second are quoted.
\subsection{Ceph}
Relatively new, but made it into the Linux kernel recently.
\subsection{HDFS}
Hadoop file system. Can be mounted as a standard file system using FUSE.

\section{Cloud computing}
What's been done up to now is that we've bought and configured the servers ourselves.
But in the last few years, computing has become a commodity, and we're able to use
services such as Amazon Web Services EC2 to do our computing for us, and pay only
for what we use. They offer High Performance Computing instances with large numbers
of cores that are connected with 10 gigabit Ethernet.

Amazon also have a data centre in Sydney. I've started instances there and they were 
quite responsive. The whole process was very easy.

If nothing else, using cloud computing will allow us to quickly and cheaply try
ideas out without having to commit any money or spend any time configuring servers.
\end{document}