\documentclass[11pt]{amsart}
\usepackage{amsmath}
\usepackage{amsfonts}
\usepackage{amssymb}
\title{Summary of Bayesian Data Analysis}

\newcommand{\Poisson}[1] {\text{Poisson} ( #1 )}

\begin{document}
\maketitle
\section{Chapter 1}
The heart of Bayesian analysis is Bayes' Rule, which is
\[
	p(\theta|y) = \frac{p(\theta, y)}{p(y)} = \frac{p(y|\theta)p(\theta)}{p(y)}
\]
where $p(y) = \sum_\theta p(y|\theta) p(\theta)$ in the discrete case and
$p(y) = \int p(y|\theta)p(\theta) d\theta$ in the continuous case.

An equivalent form of this omits the factor $p(y)$, which does not depend on 
$\theta$ and, with fixed $y$, can thus be considered a constant, yielding the
\emph{unnormalised posterior density} which is the right side of the equation.

\[
p(\theta|y) \propto p(y|\theta)p(\theta)
\]

These simple expressions encapsulate the technical core of Bayesian inference:
the primary task of any specific application is to develop a model $p(\theta, y)$
and perform the necessary computations to summarise $p(\theta|y)$ in appropriate
ways.

\section{Chapter 2 - Single parameter models}

% FIXME: Learn how to define macros in LaTeX? You're going to be doing this
% a lot.

\[
	p(y|\theta) = \text{Bin}(y|n, \theta) = {n \choose y}\theta^y (1-\theta)^{n-y}
\]

Assume a uniform $[0, 1]$ prior distribution for $\theta$. Then

\begin{align*}
p(\theta|y) &=\frac{p(y|\theta)p(\theta)}{p(y)} \\
	& \propto \theta^y (1-\theta)^{n-y}
\end{align*}

This is a beta distribution $\theta|y \sim \text{Beta}(y+1, n-y+1)$.

\subsection{Conjugacy}
The property that the posterior distribution follows the same parametric form
as the prior distribution is called $\emph{conjugacy}$; the beta prior 
distribution is a \emph{conjugate family} for the binomial likelihood. The
conjugate famioly is mathematically convenient in that the posterior distribution
follows a known parametric form.

Formally, if $\mathcal{F}$ is a class of sampling distributions $p(y|\theta)$,
and $\mathcal{P}$ is a class of prior distributions for $\theta$, then the class
$\mathcal{P}$ is \emph{conjugate} for $\mathcal{F}$ if
\[
p(\theta|y) \in \mathcal{P}\text{ for all }p(.|\theta)\text{ and }p(.)\in \mathcal{P}
\]

\subsection{Conjugate prior distributions, exponential families and sufficient statistics}

$\mathcal{F}$ is an exponential family if
\[
	p(y_i|\theta) = f(y_i) g(\theta) e^{\phi(\theta)^T u(y_i)}
\]

The factors $\phi(\theta)$ and $u(y_i)$ are in general, vectors of equal
dimension to that of $\theta$. The vector $\phi(\theta)$ is called the
`natural parameter' of the family $\mathcal{F}$. The likelihood of the
corresponding sequence $y=(y_1, y_2, \ldots, y_n)$ of i.i.d. observations is
\[
p(y|\theta) = \Pi_{i=1}^n f(y_i) g(\theta)^n \exp{\left(\phi(\theta)^T \sum_{i=1}^n u(y_i)\right)}
\]

For all $n$ and $y$, this has a fixed form (as a function of $\theta$):
\[
p(y|\theta) \propto g(\theta)^n e^{\phi(\theta)^T t(y)}
\]
where $t(y) = \sum_{i=1}^n u(y_i)$.

The quantity $t(y)$ is said to be a \emph{sufficient statistic} for $\theta$,
because the likelihood for $\theta$ depends on the data $y$ only through
the value of $t(y)$. If the prior density is specified as
\[
p(\theta) \propto g(\theta)^\eta e^{\phi(\theta)^T \nu}
\]
then the posterior density is
\[
p(\theta|y) \propto g(\theta)^{\eta + n} e^{\phi(\theta)^T(\nu + t(y))}
\]

\subsection{p50, Normal distribution with known mean but unknown variance}
For $p(y|\theta, \sigma^2) = N(y|\theta, \sigma^2)$, with $\theta$ known and
$\sigma^2$ unknown, the likelihood for a vector $y$ of $n$ i.i.d observations is

\begin{align*}
	p(y|\sigma^2) & x \propto \sigma^{-n} \exp{\left(-\frac{1}{2\sigma^2} \sum_{i=1}^n (y_i - \theta)^2\right)} \\
	&= (\sigma^2)^{-n/2} \exp{\left(-\frac{n}{2\sigma^2}\nu \right)}
\end{align*}

The sufficient statistic is
\[
\nu = \frac{1}{n} \sum_{i=1}^n (y_i - \theta)^2
\]

The corresponding conjugate prior density is the inverse-gamma,
\[
p(\sigma^2) \propto {\sigma^2}^{-(\alpha+1)} e^{\beta/\sigma^2}
\]

\subsection{Poisson model}
The Poisson distribution arises naturally in the study of data taking the form
of counts.

If a data point y follows a Poisson distribution with rate $\theta$, then the
probability distribution of a single observation $y$ is
\[
p(y|\theta) = \frac{\theta^y e^{-\theta}}{y!}\text{, for y=0, 1, 2, \ldots}
\]

and for a vector $y=(y_1, y_2, \ldots, y_n)$ of i.i.d. observations, the
likelihood is
\begin{align*}
p(y|\theta) &= \Pi_{i=1}^n \frac{1}{y_i!} \theta^{y_i} e^{-\theta} \\
& \propto \theta^{t(y)} e^{-n\theta}
\end{align*}
where $t(y) = \sum_{i=1}^n y_i$ is the sufficient statistic. We can rewrite
the likelihood in exponential form as
\[
p(y|\theta) \propto e^{-n\theta} e^{t(y) \log{\theta}}
\]
revealing that the natural parameter is $\phi(\theta) = \log{\theta}$, and
the natural conjugate prior distribution is
\[
	p(\theta) \propto (e^{-\theta})^\eta e^{\nu \log{\theta}}
\]
This is a $\text{Gamma}(\nu+ 1, \frac{1}{\eta})$ distribution. With this
conjugate distribution \ldots the posterior distribution is
\[
\theta|y \sim \text{Gamma}(\alpha + n\bar{y}, \beta+n).
\]

There's more about negative binomial and Poisson w/rate of exposure/log
offsets etc.

\section{Chapter 14 - Introduction to regression models}

The simplest and most widely used version of this model is the
\emph{normal linear model}, in which the distribution of y given X is a normal
whose mean is a linear function of X:
\[
	E(y_i|\beta, X) = \beta_1 x_{i1} + \ldots + \beta_k x_{ik}
\]

for $i = 1, \ldots, n$. For an intercept term, fix $x_{i1} = 1$.

\subsection{Notation and basic model}

\[
	y|\beta, \sigma^2, X \sim N(X^\beta, \sigma^2 I)
\]

\subsection{The standard noninformative prior distribution}
A convenient noninformative prior distribution is uniform on $\beta, 
\log{\sigma}$, or equivalently,

\[
	p(\beta, \sigma^2|X) \propto \sigma^{-2}
\]

% Why the log of sigma?

\subsection{The posterior distribution}

The conditional posterior distribution of the (vector) parameter $\beta$,
given $\sigma^2$, is the exponential of a quadratic form in $\beta$ and hence
is normal. We use the notation
\[
	\beta|\sigma^2, y \sim N(\hat{\beta}, V_{\beta}\sigma^2)
\]
where, using the now familiar technique of completing the square, one finds,
\begin{align*}
	\hat{\beta} &= (X^T X)^{-1} X^T y\\
	V_{\beta} &= (X^T X)^{-1}
\end{align*}

% Exercise 14.3 goes through this process of completing the square.
Mark fills the detail in:
\[
p(y) &= (2\pi)^{-\frac{p}{2}} \sigma^{-1/2} \exp{(-1/2 (y - X^T\beta)^T \sigma^{-2} (y - X^T \beta))}
\]

Expanding the binomial argument to the exponential we obtain
\[
	y^T y -  2 y^T X^T \beta + X^T X \beta
\]

Consider this to be a quadratic in $\beta$. Then to complete the square in this quadratic
we add the constant term $(X^T X)^{-1} X^T y$ and allow the $y^T y$ term to be absorbed off
into a proportionality constant somewhere else. Here $X^T X$ takes the role of $a$ in the
quadratic $a x^2 + b x + c$ and $X^T y$ takes the role of $b$. So the constant that we need
is $c = \frac{-b}{2a}$.

\section{Chapter 16 - Generalised linear models}
\subsection{Poisson}
\[
	p(y|\beta) = \sum_{i=1}^n \frac{1}{y_i !} e^{- \exp{(\eta_i)}} (\exp{(\eta_i)})^{y_i}
\]
where $\eta_i = (X\beta)_i$ is the linear predictor in the i-th case.
\subsection{Example: Hierarchical Poisson regression for police stops}
This is not too dissimiliar from what we want to do.

\begin{align*}
y_{sp} & \sim \Poisson{n_{sp} e^{\alpha_e + \beta_p + \epsilon_{sp}}} \\
\beta_p & \sim N(0, \sigma_\beta^2) \\
\epsilon_{sp} & \sim N(0, \sigma_\epsilon^2)
\end{align*}


Some of my own summarising:
\section{Wishart distribution}
This sounds scary, because it's a distribution that's often used as a prior
for covariance matrices. But it's just a multivariate generalisation of the
chi-squared distribution.
\end{document}
