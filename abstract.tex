\documentclass{article}
\usepackage{natbib}
\title{Abstract}
\author{Mark Greenaway}
\begin{document}
\maketitle

In this thesis, we examine several complicated Bayesian models which applied statisticians wish to fit
to data. These models are typically fit with Monte Carlo Markov Chains, but these techniques are
computationally expensive and prone to convergence problems. These models can be fit much faster with Variational 
Bayes.

In the first chapter, we consider variational inference for zero--inflated Poisson regression models using a
latent   variable representation. The model is extended to include random effects which allow simple
incorporation of   spline and other modelling structures. Several variational approximations to the resulting
set of models are   presented, including a novel approach based on the inverse covariance matrix rather than
the covariance matrix   of the approximate posterior density for the random effects. This parameterisation
improves upon the   computational cost and numerical stability of previous methods. We demonstrate these
approximations on   simulated and real data sets.

In the second chapter, we develop mean field and structured variational Bayes approximations for Bayesian
model selection on linear models using a special case of the prior proposed in \cite{Maruyama2011}. We show
that both approaches only depend on a single variational parameter $\tau_g$ and the sample size, model
dimension and the $R^2$ value of the linear model. An algorithm is developed which allows these models to be
fit in parallel. Applications to a range of data sets are presented, showing  empirically that our method
performs comparably to other methods on real-world data. Our method is computationally more efficient  than
the exact Bayesian model.

In the third chapter, we approach model selection from a different, population-based perspective, by developed
a Collapsed Variational Approximation which uses a population-based approach to explore the model space for
a normal linear model selection problem. We compare the results of this approximation to the 
previously published EMVS approach in \cite{Rockova2014}.
\bibliographystyle{elsarticle-harv-nourl}
\bibliography{references_mendeley}

\end{document}