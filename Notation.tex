\subsection{Definitions}

Let $p$ be the dimension of the space of fixed effects, $m$ be the number of individuals in the random effects
and $b$ be the block size for each of those individuals. We use $\vone_p$ and $\vzero_p$ to denote the $p
\times 1$ column vectors with all entries equal to 1 or 0, respectively.

Let $\vy$ be the $n \times 1$ vector. The norm of a column vector $\vv$, defined to be $\sqrt{\vv^\top \vv}$,
is  denoted by $\|\vv\|$. For a $p \times 1$ vector $\va$, we let $\diag{(\va)}$ denote the $p \times p$
matrix with the elements of $\va$ along its' diagonal.

We denote the design matrix of fixed effects with dimensions $n \times p$ as $\mX$ is , and the design matrix
of random  effects with dimensions $n \times m b$ as $\mZ$. The combined design matrix $\mC$ is formed by
appending the columns of $\mX$ to the columns of $\mZ$, giving $\mC = [ \mX, \mZ ]$.

Let $\vtheta$ is the vector of all parameters.
Let $\vbeta$ be the $p \times 1$ column vector of fixed
effects, and $\vu$ the $m b \times 1$ column vector of random effects. $\vnu$ is the
concatenation of these vectors $[\vbeta^\top, \vu^\top]$.
% Let $\vp$ be the $n \times 1$ column vector of probabilities that each observation in $\vy$ is
% non-zero.

Let $\mSigma$ be the covariance matrix of the random effects $\vu$,
and 
$\mPsi$ the covariance matrix prior on $\mSigma$.
These matrices are all of dimension $(p + m b) \times (p + m b)$.


$\expit(x)$ denotes the function $\tfrac{1}{1 + \exp(-x)}$ which is the inverse of the logit
function.

$\text{Bernoulli}(\pi)$ denotes the probability distribution $\pi^k (1 - \pi)^{1-k}$ and
$\text{Inverse Wishart}(\mPsi, v)$ denotes the probability distribution
$$\tfrac{|\mPsi|^\frac{v}{2}}{2^{\frac{vp}{2}} \Gamma_p{(\tfrac{v}{2})}} |\mX|^{-\tfrac{v + p + 1}{2}}
\exp{[-\half \tr{(\mPsi \mX^{-1})}]}$$ where $\Gamma_p{(x)}$ denotes the multivariate gamma function and $\tr$
is the trace function.

\subsection{Bias of MLE}

How does this relate to Bayesian estimation? Maybe it's just interesting background.
Bernstein-von Mises Theorem
