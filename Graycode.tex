\documentclass{amsart}
\begin{document}
Graycode

The Graycode was originally developed to aid in detecting errors in analog to digital conversions in
communications systems by Frank Gray in 1947. It is a sequence of binary numbers whose key feature is that
one and only one binary digit is different between binary number in the sequence. As we index the space of
models $\Gamma$ using binary numbers, this sequence allows us to enumerate the entire model space in an order
which only adds or removes one covariate from the previous model at a time. We use this to perform rank
one updates and downdates in the calculation of the residual sum of squares for each model.

Special function evaluation

The Confluent Hypergeometric function is notoriously difficult to evaluate in general (see
\cite{Press:2007:NRE:1403886} Chapter 6), , due to issues such as numerical overflow/underflow and numerical
instability. As its' definition involves the ratio of two quantities which can diverge rapidly, an infinite
series
\[
	M(a, b, z) = \sum_{n=0}^\infty \frac{a^{(n) z^n}}{b^{(n)} n!} = {_1} F_1(a; b; z)
\]

No scheme for accurately numerically evaluating this function has been found which works across all ranges of
$a$, $b$ and $z$ values. Bill Gosper did a great deal of work into evaluating these functions using continued
fractions.

We tried several options. The fAsianOptions library only worked in a narrow range of cases. 
The hypergeo R package uses ideas originally due to Bill Gosper.
The GNU Scientific Library worked well for the range of cases that we tried.

\bibliographystyle{elsarticle-harv}
\bibliography{references_mendeley}

\end{document}