\documentclass{beamer}

\usetheme{Warsaw}
\usepackage{graphicx}
% include.tex
\newcommand{\Bernoulli}[1]{\text{Bernoulli} \left( #1 \right)}
\newcommand{\mydigamma}[1]{\psi \left( #1 \right)}
%\newcommand{\diag}[1]{\text{diag}\left( #1 \right)}
\newcommand{\tr}[1]{\text{tr}\left( #1 \right)}
\newcommand{\Poisson}[1]{\text{Poisson} \left( #1 \right)}
\def \half {\frac{1}{2}}
\def \R {\mathbb{R}}
\def \vbeta {\vec{\beta}}
\def \vy {\vec{y}}
\def \vmu {\vec{\mu}}
\def \vmuqbeta {\vmu_{q(\vbeta)}}
\def \vmubeta {\vmu_{\vbeta}}
\def \Sigmaqbeta {\Sigma_{q(\vbeta)}}
\def \Sigmabeta {\Sigma_{\vbeta}}
\def \va {\vec{a}}
\def \vtheta {\vec{\theta}}
\def \mX {\vec{X}}

\def\ds{{\displaystyle}}

\def\diag{{\mbox{diag}}}


\usefonttheme{serif}

\title{Variational approximations to zero-inflated count models 2: the optimiser strikes back}
\author{Mark Greenaway\\PhD candidate\\markg@maths.usyd.edu.au}

\mode<presentation>
{ \usetheme{boxes} }

\begin{document}
% 1. Front slide
\begin{frame}
\titlepage
% Details about myself here?
\end{frame}

% Details from last time: Gaussian variational approximation

% Detail what progress has been made, and what results have been obtained
% Algorithms:
% i) Optimise with L-BFGS, mLambda = R R^T
% ii) Optimise with L-BFGS, mLambda = (R R^T)^-1
% iii) Newton-Raphson optimisation on the variational lower bound
% iv) Newton-Raphson optimisation on the variational lower bound, using block inverses

% Motivations behind the algorithms
% ii) The Cholesky factor of a block matrix of the form
% diag for random effects, block for cross effects
% block for cross effects, diag for fixed effects
% is mostly diagonal
% Less parameters to optimise and store

% Work to date
% 2000 lines of R code
% 93 functions
% Debugging. So much debugging ...

% Accuracy results
% Accuracy verified using 1 million MCMC iterations from Stan
% Stan converts your model specification into C++ code
% A one million iteration run ``only'' takes ten CPU hours per chain
% Can be parallelised across multiple cores
% The need for approximate methods is obvious

\begin{frame}
\begin{tabular}{lllll}
\hline
	& Laplacian & GVA & GVA2 & GVA_nr \\
\hline
beta&0.95&0.95&0.95&0.95 \\
beta&0.81&0.90&0.85&0.90 \\
u&0.95&0.96&0.95&0.96 \\
u&0.94&0.96&0.94&0.96 \\
u&0.95&0.96&0.95&0.96 \\
u&0.95&0.96&0.95&0.96 \\
u&0.95&0.96&0.95&0.96 \\
u&0.95&0.96&0.95&0.96 \\
u&0.94&0.95&0.95&0.95 \\
u&0.95&0.96&0.95&0.96 \\
u&0.92&0.94&0.93&0.94 \\
u&0.95&0.97&0.95&0.97 \\
sigma2_u&0.96&0.95&0.96&0.95 \\
rho&0.90&0.90&0.90&0.90 \\
\hline
\end{tabular}
\end{frame}

% Further work
% All algorithms should arrive at exactly the same result, up to numerical error ~10^-8
% Algorithms i and iii work well
% Algorithm ii does not get exactly the right result
% Algorithm ii works on some data sets, but this isn't good enough
% That is, I have concerns about its stability
% Algorithm iv is yet to be implemented
% Splines should be ``easy'' to implement, but I've heard that before
% Shut up and write

\end{document}
