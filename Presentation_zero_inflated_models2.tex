\documentclass{beamer}

\usetheme{Warsaw}
\usepackage{graphicx}
% include.tex
\newcommand{\Bernoulli}[1]{\text{Bernoulli} \left( #1 \right)}
\newcommand{\mydigamma}[1]{\psi \left( #1 \right)}
%\newcommand{\diag}[1]{\text{diag}\left( #1 \right)}
\newcommand{\tr}[1]{\text{tr}\left( #1 \right)}
\newcommand{\Poisson}[1]{\text{Poisson} \left( #1 \right)}
\def \half {\frac{1}{2}}
\def \R {\mathbb{R}}
\def \vbeta {\vec{\beta}}
\def \vy {\vec{y}}
\def \vmu {\vec{\mu}}
\def \vmuqbeta {\vmu_{q(\vbeta)}}
\def \vmubeta {\vmu_{\vbeta}}
\def \Sigmaqbeta {\Sigma_{q(\vbeta)}}
\def \Sigmabeta {\Sigma_{\vbeta}}
\def \va {\vec{a}}
\def \vtheta {\vec{\theta}}
\def \mX {\vec{X}}

\def\ds{{\displaystyle}}

\def\diag{{\mbox{diag}}}


\usepackage{latexsym,amssymb,amsmath,amsfonts}
%\usepackage{tabularx}
\usepackage{theorem}
\usepackage{verbatim,array,multicol,palatino}
\usepackage{graphicx}
\usepackage{graphics}
\usepackage{fancyhdr}
\usepackage{algorithm,algorithmic}
\usepackage{url}
%\usepackage[all]{xy}



\def\approxdist{\stackrel{{\tiny \mbox{approx.}}}{\sim}}
\def\smhalf{\textstyle{\frac{1}{2}}}
\def\vxnew{\vx_{\mbox{{\tiny new}}}}
\def\bib{\vskip12pt\par\noindent\hangindent=1 true cm\hangafter=1}
\def\jump{\vskip3mm\noindent}
\def\etal{{\em et al.}}
\def\etahat{{\widehat\eta}}
\def\thick#1{\hbox{\rlap{$#1$}\kern0.25pt\rlap{$#1$}\kern0.25pt$#1$}}
\def\smbbeta{{\thick{\scriptstyle{\beta}}}}
\def\smbtheta{{\thick{\scriptstyle{\theta}}}}
\def\smbu{{\thick{\scriptstyle{\rm u}}}}
\def\smbzero{{\thick{\scriptstyle{0}}}}
\def\boxit#1{\begin{center}\fbox{#1}\end{center}}
\def\lboxit#1{\vbox{\hrule\hbox{\vrule\kern6pt
      \vbox{\kern6pt#1\kern6pt}\kern6pt\vrule}\hrule}}
\def\thickboxit#1{\vbox{{\hrule height 1mm}\hbox{{\vrule width 1mm}\kern6pt
          \vbox{\kern6pt#1\kern6pt}\kern6pt{\vrule width 1mm}}
               {\hrule height 1mm}}}


%\sloppy
%\usepackage{geometry}
%\geometry{verbose,a4paper,tmargin=20mm,bmargin=20mm,lmargin=40mm,rmargin=20mm}


%%%%%%%%%%%%%%%%%%%%%%%%%%%%%%%%%%%%%%%%%%%%%%%%%%%%%%%%%%%%%%%%%%%%%%%%%%%%%%%%
%
% Some convenience definitions
%
% \bf      -> vector
% \sf      -> matrix
% \mathcal -> sets or statistical
% \mathbb  -> fields or statistical
%
%%%%%%%%%%%%%%%%%%%%%%%%%%%%%%%%%%%%%%%%%%%%%%%%%%%%%%%%%%%%%%%%%%%%%%%%%%%%%%%%

% Sets or statistical values
\def\sI{{\mathcal I}}                            % Current Index set
\def\sJ{{\mathcal J}}                            % Select Index set
\def\sL{{\mathcal L}}                            % Likelihood
\def\sl{{\ell}}                                  % Log-likelihood
\def\sN{{\mathcal N}}                            
\def\sS{{\mathcal S}}                            
\def\sP{{\mathcal P}}                            
\def\sQ{{\mathcal Q}}                            
\def\sB{{\mathcal B}}                            
\def\sD{{\mathcal D}}                            
\def\sT{{\mathcal T}}
\def\sE{{\mathcal E}}                            
\def\sF{{\mathcal F}}                            
\def\sC{{\mathcal C}}                            
\def\sO{{\mathcal O}}                            
\def\sH{{\mathcal H}} 
\def\sR{{\mathcal R}}                            
\def\sJ{{\mathcal J}}                            
\def\sCP{{\mathcal CP}}                            
\def\sX{{\mathcal X}}                            
\def\sA{{\mathcal A}} 
\def\sZ{{\mathcal Z}}                            
\def\sM{{\mathcal M}}                            
\def\sK{{\mathcal K}}     
\def\sG{{\mathcal G}}                         
\def\sY{{\mathcal Y}}                         
\def\sU{{\mathcal U}}  


\def\sIG{{\mathcal IG}}                            


\def\cD{{\sf D}}
\def\cH{{\sf H}}
\def\cI{{\sf I}}

% Vectors
\def\vectorfontone{\bf}
\def\vectorfonttwo{\boldsymbol}
\def\va{{\vectorfontone a}}                      %
\def\vb{{\vectorfontone b}}                      %
\def\vc{{\vectorfontone c}}                      %
\def\vd{{\vectorfontone d}}                      %
\def\ve{{\vectorfontone e}}                      %
\def\vf{{\vectorfontone f}}                      %
\def\vg{{\vectorfontone g}}                      %
\def\vh{{\vectorfontone h}}                      %
\def\vi{{\vectorfontone i}}                      %
\def\vj{{\vectorfontone j}}                      %
\def\vk{{\vectorfontone k}}                      %
\def\vl{{\vectorfontone l}}                      %
\def\vm{{\vectorfontone m}}                      % number of basis functions
\def\vn{{\vectorfontone n}}                      % number of training samples
\def\vo{{\vectorfontone o}}                      %
\def\vp{{\vectorfontone p}}                      % number of unpenalized coefficients
\def\vq{{\vectorfontone q}}                      % number of penalized coefficients
\def\vr{{\vectorfontone r}}                      %
\def\vs{{\vectorfontone s}}                      %
\def\vt{{\vectorfontone t}}                      %
\def\vu{{\vectorfontone u}}                      % Penalized coefficients
\def\vv{{\vectorfontone v}}                      %
\def\vw{{\vectorfontone w}}                      %
\def\vx{{\vectorfontone x}}                      % Covariates/Predictors
\def\vy{{\vectorfontone y}}                      % Targets/Labels
\def\vz{{\vectorfontone z}}                      %

\def\vone{{\vectorfontone 1}}
\def\vzero{{\vectorfontone 0}}

\def\valpha{{\vectorfonttwo \alpha}}             %
\def\vbeta{{\vectorfonttwo \beta}}               % Unpenalized coefficients
\def\vgamma{{\vectorfonttwo \gamma}}             %
\def\vdelta{{\vectorfonttwo \delta}}             %
\def\vepsilon{{\vectorfonttwo \epsilon}}         %
\def\vvarepsilon{{\vectorfonttwo \varepsilon}}   % Vector of errors
\def\vzeta{{\vectorfonttwo \zeta}}               %
\def\veta{{\vectorfonttwo \eta}}                 % Vector of natural parameters
\def\vtheta{{\vectorfonttwo \theta}}             % Vector of combined coefficients
\def\vvartheta{{\vectorfonttwo \vartheta}}       %
\def\viota{{\vectorfonttwo \iota}}               %
\def\vkappa{{\vectorfonttwo \kappa}}             %
\def\vlambda{{\vectorfonttwo \lambda}}           % Vector of smoothing parameters
\def\vmu{{\vectorfonttwo \mu}}                   % Vector of means
\def\vnu{{\vectorfonttwo \nu}}                   %
\def\vxi{{\vectorfonttwo \xi}}                   %
\def\vpi{{\vectorfonttwo \pi}}                   %
\def\vvarpi{{\vectorfonttwo \varpi}}             %
\def\vrho{{\vectorfonttwo \rho}}                 %
\def\vvarrho{{\vectorfonttwo \varrho}}           %
\def\vsigma{{\vectorfonttwo \sigma}}             %
\def\vvarsigma{{\vectorfonttwo \varsigma}}       %
\def\vtau{{\vectorfonttwo \tau}}                 %
\def\vupsilon{{\vectorfonttwo \upsilon}}         %
\def\vphi{{\vectorfonttwo \phi}}                 %
\def\vvarphi{{\vectorfonttwo \varphi}}           %
\def\vchi{{\vectorfonttwo \chi}}                 %
\def\vpsi{{\vectorfonttwo \psi}}                 %
\def\vomega{{\vectorfonttwo \omega}}             %


% Matrices
%\def\matrixfontone{\sf}
%\def\matrixfonttwo{\sf}
\def\matrixfontone{\bf}
\def\matrixfonttwo{\boldsymbol}
\def\mA{{\matrixfontone A}}                      %
\def\mB{{\matrixfontone B}}                      %
\def\mC{{\matrixfontone C}}                      % Combined Design Matrix
\def\mD{{\matrixfontone D}}                      % Penalty Matrix for \vu_J
\def\mE{{\matrixfontone E}}                      %
\def\mF{{\matrixfontone F}}                      %
\def\mG{{\matrixfontone G}}                      % Penalty Matrix for \vu
\def\mH{{\matrixfontone H}}                      %
\def\mI{{\matrixfontone I}}                      % Identity Matrix
\def\mJ{{\matrixfontone J}}                      %
\def\mK{{\matrixfontone K}}                      %
\def\mL{{\matrixfontone L}}                      % Lower bound
\def\mM{{\matrixfontone M}}                      %
\def\mN{{\matrixfontone N}}                      %
\def\mO{{\matrixfontone O}}                      %
\def\mP{{\matrixfontone P}}                      %
\def\mQ{{\matrixfontone Q}}                      %
\def\mR{{\matrixfontone R}}                      %
\def\mS{{\matrixfontone S}}                      %
\def\mT{{\matrixfontone T}}                      %
\def\mU{{\matrixfontone U}}                      % Upper bound
\def\mV{{\matrixfontone V}}                      %
\def\mW{{\matrixfontone W}}                      % Variance Matrix i.e. diag(b'')
\def\mX{{\matrixfontone X}}                      % Unpenalized Design Matrix/Nullspace Matrix
\def\mY{{\matrixfontone Y}}                      %
\def\mZ{{\matrixfontone Z}}                      % Penalized Design Matrix/Kernel Space Matrix

\def\mGamma{{\matrixfonttwo \Gamma}}             %
\def\mDelta{{\matrixfonttwo \Delta}}             %
\def\mTheta{{\matrixfonttwo \Theta}}             %
\def\mLambda{{\matrixfonttwo \Lambda}}           % Penalty Matrix for \vnu
\def\mXi{{\matrixfonttwo \Xi}}                   %
\def\mPi{{\matrixfonttwo \Pi}}                   %
\def\mSigma{{\matrixfonttwo \Sigma}}             %
\def\mUpsilon{{\matrixfonttwo \Upsilon}}         %
\def\mPhi{{\matrixfonttwo \Phi}}                 %
\def\mOmega{{\matrixfonttwo \Omega}}             %
\def\mPsi{{\matrixfonttwo \Psi}}                 %

\def\mone{{\matrixfontone 1}}
\def\mzero{{\matrixfontone 0}}

% Fields or Statistical
\def\bE{{\mathbb E}}                             % Expectation
\def\bP{{\mathbb P}}                             % Probability
\def\bR{{\mathbb R}}                             % Reals
\def\bI{{\mathbb I}}                             % Reals
\def\bV{{\mathbb V}}                             % Reals

\def\vX{{\vectorfontone X}}                      % Targets/Labels
\def\vY{{\vectorfontone Y}}                      % Targets/Labels
\def\vZ{{\vectorfontone Z}}                      %

% Other
\def\etal{{\em et al.}}
\def\ds{\displaystyle}
\def\d{\partial}
\def\diag{\text{diag}}
%\def\span{\text{span}}
\def\blockdiag{\text{blockdiag}}
\def\tr{\text{tr}}
\def\RSS{\text{RSS}}
\def\df{\text{df}}
\def\GCV{\text{GCV}}
\def\AIC{\text{AIC}}
\def\MLC{\text{MLC}}
\def\mAIC{\text{mAIC}}
\def\cAIC{\text{cAIC}}
\def\rank{\text{rank}}
\def\MASE{\text{MASE}}
\def\SMSE{\text{SASE}}
\def\sign{\text{sign}}
\def\card{\text{card}}
\def\notexp{\text{notexp}}
\def\ASE{\text{ASE}}
\def\ML{\text{ML}}
\def\nullity{\text{nullity}}

\def\logexpit{\text{logexpit}}
\def\logit{\mbox{logit}}
\def\dg{\mbox{dg}}

\def\Bern{\mbox{Bernoulli}}
\def\sBernoulli{\mbox{Bernoulli}}
\def\sGamma{\mbox{Gamma}}
\def\sInvN{\mbox{Inv}\sN}
\def\sNegBin{\sN\sB}

\def\dGamma{\mbox{Gamma}}
\def\dInvGam{\mbox{Inv}\Gamma}

\def\Cov{\mbox{Cov}}
\def\Mgf{\mbox{Mgf}}

\def\mis{{mis}} 
\def\obs{{obs}}

\def\argmax{\operatornamewithlimits{\text{argmax}}}
\def\argmin{\operatornamewithlimits{\text{argmin}}}
\def\argsup{\operatornamewithlimits{\text{argsup}}}
\def\arginf{\operatornamewithlimits{\text{arginf}}}


\def\minimize{\operatornamewithlimits{\text{minimize}}}
\def\maximize{\operatornamewithlimits{\text{maximize}}}
\def\suchthat{\text{such that}}


\def\relstack#1#2{\mathop{#1}\limits_{#2}}
\def\sfrac#1#2{{\textstyle{\frac{#1}{#2}}}}


\def\comment#1{
\vspace{0.5cm}
\noindent \begin{tabular}{|p{14cm}|}  
\hline #1 \\ 
\hline 
\end{tabular}
\vspace{0.5cm}
}


\def\mytext#1{\begin{tabular}{p{13cm}}#1\end{tabular}}
\def\mytextB#1{\begin{tabular}{p{7.5cm}}#1\end{tabular}}
\def\mytextC#1{\begin{tabular}{p{12cm}}#1\end{tabular}}

\def\jump{\vskip3mm\noindent}

\def\KL{\text{KL}}
\def\N{\text{N}}
\def\Var{\text{Var}}

\def \E {\mathbb{E}}
\def \BigO {\text{O}}
\def \IG {\text{IG}}
\def \Beta {\text{Beta}}



\usefonttheme{serif}

\title{Variational approximations to ZIP models 2: the optimiser strikes back}
\author{Mark Greenaway\\PhD candidate\\markg@maths.usyd.edu.au}

\mode<presentation>
{ \usetheme{boxes} }

\begin{document}
% 1. Front slide
\begin{frame}
\titlepage
% Details about myself here?
\end{frame}

\begin{frame}
\frametitle{Gaussian variational approximation}
% Details from last time: Gaussian variational approximation
For generalised linear models, there is no tractable factored 
approximation which works well, so we attempt to approximate
the GLM with a multivariate Gaussian (reference relevant
Ormerod paper).
\end{frame}

\begin{frame}
\begin{align*}
\log p(\vbeta, \vu, \mLambda) &= \vy^T \mR \mC \vnu + \vone^T c(\vy) + \frac{m}{2} \log |\mLambda| - \frac{mK}{2} \log{(2 \pi)}\\
&\quad \log  \int_{\mathbb{R}^{m}} \exp \big[ \vy^T (\mR \mZ \vu - \vone^T b(\mR( \mX \vbeta + \mZ \vu))) \\
 &\quad \quad \quad \quad \quad \quad \quad - \half {\vu^T \mLambda^{-1} \vu } \big] d \vu \\
\log \overline{p}(\vmu, \mLambda) &= \vy^T \mR \mC \vmu - \vp^T \exp{\left(\mC \vmu + \half \diag{(\mC \mLambda \mC^T)}\right)} \\
& \quad - \vmu^T \mSigma^{-1} \vmu - \half \tr{(\mLambda \mSigma^{-1})} + \half \log{|\mSigma^{-1}\mLambda|} + \text{const.} \\
&= \mR\mC^T(\vy - B^{(1)}(\vmu, \sigma^2_u)) \\
& \quad - \vmu^T \mSigma^{-1} \vmu - \half \tr{(\mLambda \mSigma^{-1})} + \half \log{|\mSigma^{-1}\mLambda|} + \text{const.} \\
\end{align*}
%\end{align*}
\end{frame}

\begin{frame}
\frametitle{Algorithms to fit model}
% Detail what progress has been made, and what results have been obtained
There are several alternative algorithms we've implemented or will implement
to fit this model:
\begin{enumerate}
\item Optimise variational lower bound with L-BFGS, $\mLambda = \mR \mR^T$
\item Optimise variational lower bound with L-BFGS, $\mLambda = (\mR \mR^T)^{-1}$
\item Newton-Raphson optimisation on the variational lower bound
\item Newton-Raphson optimisation on the variational lower bound, using block inverses
\end{enumerate}

These algorithms should all fit the same model to the same data
within numerical tolerances.
\end{frame}

\begin{frame}
\frametitle{Cholesky factors}
Any symmetric matrix can be written $\mSigma = \mR \mR^T$
where $\mR$ is lower triangular.

\begin{align*}
&\begin{pmatrix}
R_{11} \\
R_{21} & R_{22} \\
R_{31} & R_{32} & R_{33}
\end{pmatrix}
\begin{pmatrix}
R_{11} & R_{21} & R_{31} \\
& R_{22} & R_{32} \\
& & R_{33}
\end{pmatrix}
\\
=& \begin{pmatrix}
R_{11}^2 & & \text{symmetric}\\
R_{21}L_{11} & R_{21}^2 + R_{22}^2 \\
R_{31} R_{11} & R_{31}R_{21} + R_{32}L_{22} & R_{31}^2 + R_{32} ^2 + R_{33}^2
\end{pmatrix}
\end{align*}
\end{frame}

\begin{frame}
% Motivations behind the algorithms
\frametitle{Motivations behind the algorithms}
% ii) The Cholesky factor of a block matrix of the form
% diag for random effects, block for cross effects
% block for cross effects, diag for fixed effects
% is mostly diagonal
% Less parameters to optimise and store
If we re-arrange the covariance matrix to have random effects before
fixed effects, our posterior covariance matrix is of the form
\begin{align*} \mLambda =
\begin{pmatrix}
\sigma_\vu^2 & & & \sigma_{\vbeta \vu} & \sigma_{\vbeta \vu} \\
& \ddots & & \vdots & \vdots \\
& & \sigma_\vu^2 & \sigma_{\vbeta \vu} & \sigma_{\vbeta \vu} \\
\sigma_{\vbeta \vu} & \ldots & \sigma_{\vbeta \vu} & \sigma_\vbeta^2 \\
\sigma_{\vbeta \vu} & \ldots &\sigma_{\vbeta \vu}  & & \sigma_\vbeta^2
\end{pmatrix}
\end{align*}

As the $\sigma_\vu^2$ block is diagonal, the Cholesky factor is 
sparse, with only $m + \frac{p(p-1)}{2}$ non-zero elements. If $p$ is small
and $m$ is large, as will typically be the case, then this algorithm will
be much faster.

\end{frame}

\begin{frame}
\frametitle{Implementation}
% Work to date
\begin{itemize}
\item 2000 lines of R code, about enough for an R package
\item 93 functions
\item Debugging. So much debugging ...
\item I've had to learn some numerics/computational statistics along the
way
\end{itemize}
\end{frame}

\begin{frame}
\frametitle{Accuracy results using MCMC}
\begin{itemize}
\item Accuracy verified using 1 million MCMC iterations from Stan
\item Stan converts your model specification into C++ code
\item A one million iteration run ``only'' takes ten CPU hours per chain
\item Can be parallelised across multiple cores
\item Pro tip: Checkpoint every few thousand iterations
\item The need for approximate methods is obvious, this is simply not
practical for many types of applied work
\end{itemize}
\end{frame}

\begin{frame}
\frametitle{Accuracy results}
\begin{tabular}{ccccc}
\hline
	& Laplacian & GVA & GVA2 & GVA\_nr \\
\hline
$\vbeta_1$&0.95&0.95&0.95&0.95 \\
$\vbeta_2$&0.81&0.90&0.85&0.90 \\
$\vu_1$&0.95&0.96&0.95&0.96 \\
$\vu_2$&0.94&0.96&0.94&0.96 \\
$\vu_3$&0.95&0.96&0.95&0.96 \\
$\vu_4$&0.95&0.96&0.95&0.96 \\
$\vu_5$&0.95&0.96&0.95&0.96 \\
$\vu_6$&0.95&0.96&0.95&0.96 \\
$\vu_7$&0.94&0.95&0.95&0.95 \\
$\vu_8$&0.95&0.96&0.95&0.96 \\
$\vu_9$&0.92&0.94&0.93&0.94 \\
$\vu_{10}$&0.95&0.97&0.95&0.97 \\
$\sigma_\vu^2$&0.96&0.95&0.96&0.95 \\
$\rho$&0.90&0.90&0.90&0.90 \\
\hline
\end{tabular}
\end{frame}

% Pretty pictures
\begin{frame}
\frametitle{Plot the MCMC-estimated and approximating densities}
\end{frame}

\begin{frame}
\frametitle{Further work}
\begin{itemize}
\item All algorithms should arrive at exactly the same result, up to numerical error i.e. around $10^{-8}$. But they currently don't
\item More checking. These algorithms should produce the same results on
a wide range of randomly generated data sets
\item Algorithms i and iii work well
\item Algorithm ii is accurate on some data sets, but this isn't good enough
\item I have concerns about its stability
\item Algorithm iv is yet to be implemented, but should be straightforward
\item Splines should be ``easy'' to implement. Doesn't require any major changes to the existing code
\item Finish writing this all up into a paper, and release an R package
\end{itemize}
\end{frame}

\end{document}
